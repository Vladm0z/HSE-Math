\newpage
%\href{Books/book#page.}{[], гл. $\S$}
\begin{enumerate}
\item[1.] 
	Теоремы о неявном и обратном отображении(формулировки). Регулярные поверхности в $\mathbb{R}^n$, примеры и эквивалентность трех определений. (\href{Books/book2.pdf#page.165}{[2], гл. XII, $\S$1};)
\item[2.] 
	Гладкие многообразия (топологическое многообразие, карта, атлас, эквивалентность атласов, гладкая структура, примеры гл. структур) (\href{Books/book2#page.}{[2], гл. XV, $\S$2.1})
\item[3.] 
	Ориентация на многообразии. Ориентирующие атласы и их эквивалентность. Примеры ориентирующего и неориентирующего атласов одного многообразия (\href{Books/book2#page.}{[2], гл. XV, $\S$2.3})
\item[4.] 
	Существование на ориентируемом многообразии ровно двух различных ориентаций. (\href{Books/book2#page.}{[2], гл. XV, $\S$2.3})
\item[5.] 
	Формулировка критерия ориентируемости многообразия с помощью цепочки карт. Пример неориентируемого многообразия (\href{Books/book2#page.}{[2], гл. XV, $\S$2.3})
\item[6.] 
	Многообразия с краем: определение. Край многообразия с краем является многообразием без края той же гладкости и на единицу меньшей размерности, чем само многообразие (\href{Books/book2#page.}{[2], гл. XII, $\S$ 3.2})
\item[7.] 
	Ориентация края, согласованная с ориентацией многообразия. Пример для поверхности в $\mathbb{R}^n$ (\href{Books/book2#page.}{[2], гл. XII, $\S$3.2})
\item[8.] 
	Гладкие функции на многообразии и гладкие отображения многообразий. Индуцированные ими гомоморфизмы алгебр гладких функций на многообразиях. (\href{Books/book#page.}{[], гл. $\S$})
\item[9.] 
	Различные определения касательного вектора (класс эквивалентности кривых, дифференцирование) и их эквивалентность. (\href{Books/book4#page.}{[4], гл. 1})
\item[10.] 
	Касательное пространство к многообразию в точке. Формула преобразования при переходе из одной карты в другую. Дифференциал и отображение $f^*$ (\href{Books/book4#page.}{[4], гл. 1})
\item[11.] 
	Касательное расслоение к многообразию в точке. Устройство аталаса тотального пространства. (\href{Books/book4#page.}{[4], гл. 1, $\S$ 25})
\item[12.] 
	Кокасательное пространство в точке, кокасательное расслоение как многообразие. (\href{Books/book4#page.29}{[4], гл. 1, $\S$25 (стр. 29)}) (\href{Books/book#page.}{[], гл. $\S$})
\item[13.] 
	Определение векторного расслоения. Эквивалентные и тривиальные расслоения. Касательное расслоение. Векторное поле как сечение касательного расслоения. (\href{Books/book4#page.}{[4], гл. 1, $\S$25})
\item[14.] 
	Гладкое разбиение единицы, подчиненное покрытию. (\href{Books/book2#page.}{[2], гл. XV, $\S$2.4}; \href{Books/book5#page.}{[5], $\S$ 1.2})
\item[15.] 
	Вложение, погружение и подмногообразие. Вложение произвольного компактного многообразия в $\mathbb{R}^N$ при достаточно большом $N$. (\href{Books/book2#page.}{[2], гл. XV, $\S$2.4})
\item[16.] 
	Векторные поля на многообразии. Скобка Ли (коммутатор). Основные свойства коммутатора. (\href{Books/book#page.}{[], гл. $\S$})
\item[17.] 
	Отображение потока векторного поля. Множество $D_t$, на котором определен поток векторного поля, свойства множества $D_t$. Поток векторного поля на компактном многообразии. (\href{Books/book#page.}{[], гл. $\S$})
\item[18.] 
	Производная Ли векторного поля и её основные свойства (Свойство $L_X Y = [X,Y]$ без доказательства) (\href{Books/book#page.}{[], гл. $\S$})
\item[19.] 
	Тензоры и внешние формы: выражение через базис и их интерпретация как полилинейных функций. (\href{Books/book#page.}{[], гл. $\S$})
\item[20.] 
	Дифференциальные формы на многообразии. Определение через расслоения или в координатах. Отображения перехода в координатах. Структура алгебры $\Omega(M)$. Внешний дифференциал формы. (\href{Books/book#page.}{[], гл. $\S$})
\item[21.] 
	Отображение $f^*$: действие на дифференциальных формах. Коммутирование отображения $f^*$ на формах внешнего дифференциала (\href{Books/book#page.}{[], гл. $\S$})
\item[22.] 
	Интегрирование дифференциальных форм в области $\mathbb{R}^n$ и на многообразии. Формула Стокса. Формулировки формул Грина, Граусса-Остроградского, трехмерной формулы Стокса и их физический смысл. (\href{Books/book#page.}{[], гл. $\S$})
\item[23.] 
	Когомологии де Рама: определение, примеры (\href{Books/book#page.}{[], гл. $\S$})
\item[24.] 
	Теорема Пуанкаре. Отображение $f^*$: действие на пространстве когомологий ($f^*$ на когомологиях де Рама совпадают для гомотопных отображений, пространства когомологий де Рама гомотопически эквивалентных многообразий изоморфны) (\href{Books/book#page.}{[], гл. $\S$})
\end{enumerate}
\section{2021}

\begin{prob}
Найдите экстремаль функционала
$$
S[y(x)]=\int_0^1\left(2\left(y^{\prime}\right)^2+y^2 / 2+e^x\left(2 y^{\prime}-y\right)\right) d x
$$
заданного на пространстве дважды непрсрыпно дифференцируемых функций $y(x) \in C^2[0,1]$ с фиксированным граничным значением $y(1)=\sqrt{e}-e$ (число $e-$ основание натуральных логарифмов) и произвольным значением $y(0)$.
\end{prob}

\begin{proof}

\end{proof}
\vskip 0.6in




\begin{prob}
Найдите экстремаль функционала
$$
S[y(x)]=\int_0^{\pi / 2}\left(\left(y^{\prime \prime}\right)^2-y^2+8 y^{\prime \prime} e^{x-\pi / 2}\right) d x
$$
заданного на пространстве функций $y(x) \in C^4[0, \pi / 2]$ с фиксированными граничными значениями:
$$
y^{\prime}(0)=0, \quad y^{\prime}(\pi / 2)=-1, \quad y(\pi / 2)=e^{-\pi / 2}-\pi / 2
$$
и произвольным значением $y(0)$.
\end{prob}

\begin{proof}

\end{proof}
\vskip 0.6in




\begin{prob}
Точечная частица массы $m$ движется без трения по компоненте двухполостного гиперболоида, отвечающей положительным значениям координаты $z$ :
$$
z^2-x^2-y^2=a^2, \quad z>0
$$
где $x, y$ и $z$ - декартовы прямоугольные координаты в евклидовом пространстве $\mathbb{R}^3$ и $a>0$ - заданная константа. На частицу действует однородное поле тяжести, направленное против оси Оз с ускорением свободного падения $\vec{g}$.
\begin{itemize}
\item[]
\item[(a)] Составьте лагранжиан этой механической системы и выпишите ее уравнения движения.
\item[(б)] Приведите формулы для всех интегралов движения (законов сохранения).
\item[(в)] Найдите решения уравнений движения, отвечающие постоянному значению координаты $z$.
\end{itemize}
\end{prob}

\begin{proof}

\end{proof}
\vskip 0.6in




\begin{prob}
Однородный обруч массы М и радиуса $R$ катится без проскальзывания по оси $O x$, все время оставаясь в плоскости $x O y$. К геометрическому центру обруча шарнирно прикреплен жесткий невесомый тонкий стержень длины $\ell$, на свободном конце которого закреплена точечная частица массы $m$. Стержень $\ell$ может свободно вращаться в плоскости $x O y$ вокруг центра обруча $M$. На систему действует однородное поле тяжести, направленное против оси $O y$ с ускорением свободного падения $\vec{g}$. Обруч при движении не отрывается от оси $O x$.
\begin{itemize}
\item[]
\item[(а)] Составьте лагранжиан этой механической системы и выпишите ее уравнения движения.
\item[(б)] Найдите все интегралы движения (сохраняющиеся величины).
\item[(в)] Найдите период малых колебаний системы возле положения устойчивого равновесия, когда обруч неподвижен, а частица $m$ находится в своем низшем положении (стержень $\ell$ неподвижно висит параллельно осит $O y)$.
\end{itemize}
\end{prob}

\begin{proof}

\end{proof}
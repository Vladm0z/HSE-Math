\section{Лекция 10 (02.22.2021)}

\begin{defn}
Формула $A\left(b_{1}, \ldots, b_{n}\right)$ сигнатуры $\sum$ общезначима, если для любой модели $(M ; \Sigma)$ и любых констант $c_{1}, \ldots, c_{n} \in M$ $M \vDash A\left[b_{1} / \underline{c}_{1}, \ldots, b_{n} / \underline{c}_{n}\right]$
\end{defn}

\begin{defn}
Формулы $A$ и $B$ сигнатуры $\sum$ равносильны (обозначение $A \equiv B),$ если в любой модели $(M ; \Sigma)$ они определяют один и тот же предикат, то есть если $A_{M}=B_{M}$.
\end{defn}

Утверждение.\\
$A \equiv B \Longleftrightarrow$ формула $A \leftrightarrow B$ общезначима.\\
$A \leftrightarrow B$ есть сокращение для $(A \rightarrow B) \wedge(B \rightarrow A) .$
\vskip 0.1in
Равносильности логики высказываний
$$
\begin{array}{rl|rl}
A \wedge B & \equiv B \wedge A & A \vee B & \equiv B \vee A \\
A \wedge(B \wedge C) & \equiv(A \wedge B) \wedge C & A \vee(B \vee C) & \equiv(A \vee B) \vee C \\
A \wedge A & \equiv A & A \vee A & \equiv A \\
A \wedge(B \vee C) & \equiv(A \wedge B) \vee(A \wedge C) & A \vee(B \wedge C) & \equiv(A \vee B) \wedge(A \vee C) \\
A \vee(A \wedge B) & \equiv A & A \wedge(A \vee B) & \equiv A \\
\neg(A \wedge B) & \equiv \neg A \vee \neg B & \neg(A \vee B) & \equiv \neg A \wedge \neg B \\
\perp & \equiv A \wedge \neg A & T & \equiv A \vee \neg A \\
\neg \neg A & \equiv A & A \rightarrow B & \equiv \neg A \vee B
\end{array}
$$
Кванторы
$$
\begin{aligned}
\forall x A[a / x] & \equiv \forall y A[a / y] \\
\exists x A[a / x] & \equiv \exists y A[a / y] \\
(\forall x A[a / x] \vee B) & \equiv \forall x(A[a / x] \vee B) \\
(\exists x A[a / x] \vee B) & \equiv \exists x(A[a / x] \vee B) \\
(\forall x A[a / x] \wedge B) & \equiv \forall x(A[a / x] \wedge B) \\
(\exists x A[a / x] \wedge B) & \equiv \exists x(A[a / x] \wedge B) \\
\neg \forall x A[a / x] & \equiv \exists x \neg A[a / x] \\
\neg \exists x A[a / x] & \equiv \forall x \neg A[a / x]
\end{aligned}
$$

Стандартные факты:\\
(1) Допустимость правил подстановки и замены подформулы
на эквивалентную\\
(2) Переименование связанных переменных\\
(3) Теорема о предварённой нормальной форме
\vskip 0.2in

(1) Обогатим язык логики первого порядка пропозициональными переменными. Можно считать переменную $P$ нульместным предикатным символом\\
(2) Распространим на расширенный язык все синтаксические понятия, включая понятие формулы.\\
(3) Пропозициональные переменные считаются атомарными формулами.
\vskip 0.2in

\begin{defn}
$C[P / A]$ означает результат замены всех вхождений $P$ в формулу $C$ на формулу $A .$ (т
\end{defn}
Замечание.\\
$C[P / A]$ не всегда является формулой. Если $C=\forall x(Q(x) \wedge P)$ и $A=\exists x R(x),$ то $C[P / A]=\forall x(Q(x) \wedge \exists x R(x))$
\vskip 0.2in

\begin{lem}
$C[P / A]$ - формула, если и только если любое вхождение $P$ в формулу С не находится в области действия квантора по переменной $x \in$ BdVar, входящей в А.
\end{lem}
\begin{defn}
Говорим, что разрешена подстановка формулы А вместо $P$ в С, если выполнено условие предыдущей леммы.
\end{defn}

\begin{theo}
Если $A \equiv B$ и пазрешена подстановка формул $A, B$ вместо $P$ в $C,$ то $C[P / A] \equiv C[P / B]$
\end{theo}
\begin{proof}
индукция по построению формулы С. Шаг индукции на основе леммы:
\begin{lem}
Если $A \equiv A^{\prime}$ и $B \equiv B^{\prime},$ то\\
(1) ( $A \wedge B \equiv A^{\prime} \wedge B^{\prime}, \quad A \vee B \equiv A^{\prime} \vee B^{\prime}, \quad \neg A \equiv \neg A^{\prime}$,\\
(2) $\forall \times A[a / x] \equiv \forall \times A^{\prime}[a / x]$ (если $х$ не входит в $A$ и $A^{\prime}$)\\
(3) ( $\exists x A[a / x] \equiv \exists x A^{\prime}[a / x]$ (если $x$ не входит в $A$ и $A^{\prime}$).\\
\end{lem}
\end{proof}

(1) Пропозициональная переменная $P$ в модели $M$
интерпретируется как логическая константа, то есть $P_{M}$ is $\mathbb{B}$.\\
(2) Считается $M \models P_{M},$ если $P_{M}= \text{И}$ и $M \not \models P_{M}$, если $P_{M}=\text{Л}$.\\
(3) Понятие общезначимой формулы распространяется на формулы расширенного языка.
\vskip 0.2in

\begin{theo} 
Пусть формула $A$ общезначима и разрешена подстановка формулы $C$ вместо $P$ в $A$, тогда общезначима формула $A[P / C]$
\end{theo}
\begin{proof}
(1) Допустим, $M \not \models f(A[P / C])$ при некоторой оценке $f$.\\
(2) Расширим $M$ до модели $(M, P)$ сигнатуры с переменной $P$ :
$P_{M}= \text{И} \Longleftrightarrow M \vDash f(C)$\\
(3) Индукцией по построению формулы $B$ проверим, что
$$
(M, P) \vDash B \Longleftrightarrow M \vDash B[P / C]
$$
для любой замкнутой формулы $B$, в которую разрешена подстановка $C$ вместо $P$.\\
(4) Отсюда получаем $(M, P) \not \models f(A)$.
\end{proof}

\begin{corol}
Если $A \equiv B$ и разрешена подстановка $C$ вместо $P$ в $A$ и $B$, то $A[P / C] \equiv B[P / C]$
\end{corol}

\begin{lem}
Пусть $y \in \mathrm{BdVar}$ не входит в формулу $B$. Тогда $B[x / y]$ есть формула и $B[x / y] \equiv B$.
\end{lem}
\begin{proof}
Применяем индукцию по числу вхождений кванторов по переменной $\times$ в В. Каждая подформула $\forall \times C[a / x]$ или $\exists \times C[a / x]$ заменяется на эквивалентную $\forall y C[a / y]$ или $\exists y C[a / y]$.
\end{proof}

\begin{defn}
Формула А называется предварённой, если А имеет вид $\mathrm{Q} x_{1} \mathrm{Q} x_{2} \ldots \mathrm{Q} x_{n} A_{0}\left[b_{1} / x_{1}, \ldots, b_{n} / x_{n}\right],$ где $\mathrm{Q}$ означает квантор $\forall$ или $\exists$, а формула $A_{0}$ бескванторная.
\end{defn}
\begin{theo}
Для каждой формулы $А$ можно указать эквивалентную ей предварённую формулу $A^{\prime}$ от тех же свободных переменных.
\end{theo}
\begin{proof}
Последовательно выносим кванторы наружу, используя основные эквивалентности и леммы о замене связанных переменных и о подстановке. Разбор алгоритма на семинарских занятиях.
\end{proof}
\vskip 0.8in

\begin{defn}
Теорией сигнатуры $\sum$ называем произвольное множество $T$ замкнутых формул языка $\mathcal{L}_{\Sigma} .$ Элементы $A \in T$ называем нелогическими аксиомами $T$.
\end{defn}
Пример.\\
Теория отношения эквивалентности:\\
(1) $\forall x R(x, x)$\\
(2) $\forall x, y(R(x, y) \rightarrow R(y, x))$\\
(3) $\forall x, y, z(R(x, y) \wedge R(y, z) \rightarrow R(x, z))$
\vskip 0.2in

Пример\\
Модель $(M ;<)$ есть строгий частичный порядок, если в $(M ;<)$ истинны следующие предложения:\\
(1) $\forall x, y, z(x<y \wedge y<z \rightarrow x<z)$\\
(2) $\forall x \neg x<x$
\vskip 0.2in

Пример\\
Простой граф - это модель вида $(V ; E),$ где $E-$ бинарный предикат смежности, причём отношение $E$ симметрично и иррефлексивно:\\
(1) $\forall x \neg E(x, x)$\\
(2) $\forall x, y(E(x, y) \rightarrow E(y, x))$
\vskip 0.2in

Пример.\\
$(M ;=, \cdot, 1)$ есть группа, если $M$ есть модель следующей теории (при условии, что `$=$' в $M$ понимается как равенство):\\
(1) $\forall x, y, z \quad x \cdot(y \cdot z)=(x \cdot y) \cdot z$\\
(2) $\forall x(1 \cdot x=x \wedge x \cdot 1=x)$\\
(3) $\forall x \exists y(x \cdot y=1 \wedge y \cdot x=1)$
\vskip 0.3in

Пусть $\Sigma$ - сигнатура, содержащая выделенный предикатный символ $=$
\begin{defn}
Нормальной моделью называем модель $(M ; \Sigma),$ в которой $=$ интерпретируется как равенство $\{\langle x, x\rangle \chi \mid x \in M\}$.
\end{defn}
\vskip 0.2in

\begin{defn}
Аксиомы равенства для $\Sigma$ - универсальные замыкания следующих формул:\\
(1) аксиомы отношения эквивалентности для $=$\\
(2)
$$
a_{1}=b_{1} \wedge a_{2}=b_{2} \wedge \cdots \wedge a_{n}=b_{n} \rightarrow
\left(P\left(a_{1}, \ldots, a_{n}\right) \leftrightarrow P\left(b_{1}, \ldots, b_{n}\right)\right)
$$
(3)
$$
a_{1}=b_{1} \wedge a_{2}=b_{2} \wedge \cdots \wedge a_{n}=b_{n} \rightarrow
\left(f\left(a_{1}, \ldots, a_{n}\right)=f\left(b_{1}, \ldots, b_{n}\right)\right)
$$
для всех $f \in \operatorname{Func}_{\Sigma}$ and $P \in \operatorname{Pred}_{\Sigma}$
\end{defn}
\vskip 0.3in

Предложсение.\\
Если $(\mathrm{M} ; \Sigma)$ - нормальная модель, то в $\mathrm{M}$ истинны все
аксиомы равенства.
\begin{defn}
Теорией с равенством называем теорию сигнатуры $\sum$ с
равенством, содержащую все аксиомы равенства.
\end{defn}
\vskip 0.2in

\begin{theo}
Пусть $T$ - теория с равенством. Если $T$ выполнима, то $T$ имеет нормальную модель.
\end{theo}
\begin{proof}
Пусть $M \vDash T .$ Предикат $=_{M}$ есть отношение эквивалентности на М. Положим $M^{\prime} \rightleftharpoons M /=_{M}$ - множество классов эквивалентности и $\varphi: M \rightarrow M^{\prime}$ сопоставляет любому $x \in M$ его класс $\varphi(x) \in M^{\prime}$\\
Интерпретируем предикатные и функц. символы в $M^{\prime}:$ 
$$P_{M^{\prime}}\left(\varphi\left(x_{1}\right)^{\left\{\mathfrak{m}_{2}\right.} \ldots, \varphi\left(x_{n}\right)\right) \stackrel{\text { def }}{\Longleftrightarrow} P_{M}\left(x_{1}, \ldots, x_{n}\right)\\
f_{M^{\prime}}\left(\varphi\left(x_{1}\right), \ldots, \varphi\left(x_{n}\right)\right):=\varphi\left(f_{M}\left(x_{1}, \ldots, x_{n}\right)\right)
$$
В силу аксиом равенства в $M$, определение корректно и $M^{\prime}$ -- нормальная модель.\\
Индукцией по построению формулы $A$ проверяем
$$
M \vDash A\left[x_{1}, \ldots, x_{n}\right] \Longleftrightarrow M^{\prime} \vDash A\left[\varphi\left(x_{1}\right), \ldots, \varphi\left(x_{n}\right)\right]
$$
Отсюда следует $M^{\prime} \vDash T$.
\end{proof}
\vskip 0.4in

Формальная арифметика Пеано\\
Сигнатура $\Sigma=\{0, S,+, \cdot,=\}$\\
(1) аксиомы равенства для $\Sigma$\\
(2) $\neg S(a)=0, \quad S(a)=S(b) \rightarrow a=b$\\
(3) $a+0=a, \quad a+S(b)=S(a+b)$\\
(4) $a \cdot 0=0, \quad a \cdot S(b)=a \cdot b+a$\\
(5) (Схема аксиом индукции)
$A[a / 0] \wedge \forall x(A[a / x] \rightarrow A[a / S(x)]) \rightarrow \forall x A[a / x]$
для любой формулы А.
\vskip 0.3in

Теория множеств $\text{ZFC}$\\
Сигнатура $\Sigma=\{=, \in\}$\\
(1) (Аксиомы равенства)\\
(2) (Экстенсинальность) $a=b \leftrightarrow \forall x(x \in a \leftrightarrow x \in b)$\\
(3) (Пара) $\exists z \forall x(x \in z \leftrightarrow(x=a \vee x=b))$\\
(4) (Объединение) $\exists z \forall x(x \in z \leftrightarrow \exists y(x \in y \wedge y \in a))$\\
(5) (Степень) $\exists z \forall x(x \in z \leftrightarrow \forall y(y \in x \rightarrow y \in a))$\\
(6) (Схема выделения) $\exists z \forall x(x \in Z \leftrightarrow(x \in a \wedge \varphi[b / x]))$ для всех формул $\varphi$ сигнатуры $\Sigma$\\
(7) (Бесконечность) $\exists z(\varnothing \in z \wedge \forall x(x \in z \rightarrow x \cup\{x\} \in z))$\\
(8) (Регулярность) $\exists z(z \in a \wedge \forall x(x \in a \rightarrow x \notin z))$\\
(9) (Схема подстановки)\\
(10) (Аксиома выбора)
\vskip 0.3in

Аксиоматика Тарского:\\
G1. $a b \cong b a$\\
G2. $a b \cong p q \wedge a b \cong r s \rightarrow p q \cong r s$\\
G3. $a b \cong c c \rightarrow a=b$\\
G4. $Babd \wedge Bbcd \rightarrow Babc$\\
G5. $\exists x( Bqax \wedge a x \cong b c)$
G6. $(a \neq b \wedge B a b c \wedge B a^{\prime} b^{\prime} c^{\prime} \wedge a b \cong a^{\prime} b^{\prime} \wedge b c \cong b^{\prime} c^{\prime} \wedge a d \cong a^{\prime} d^{\prime} \wedge b d \cong b^{\prime} d^{\prime}) \rightarrow c d \cong c^{\prime} d^{\prime}$ (Аналог равенства треугольников)\\
G7. (аксиома Паша)\\
$Bapc \wedge B q c b \rightarrow \exists x(\operatorname{Baxq} \wedge B b p x)$\\
G8. $\exists x, y, z(\neg B x y z \wedge \neg B y z x \wedge \neg B z x y)$\\
G9. $(\operatorname{dim} \leq 2)$
$\left(p_{1} \neq p_{2} \wedge a p_{1} \cong a p_{2} \wedge b p_{1} \cong b p_{2} \wedge c p_{1} \cong c p_{2}\right) \rightarrow$
$a \in b c$\\
G10. (аксиома Евклида)\\
G11. (схема аксиом непрерывности)
$$
\begin{aligned}
\exists u \forall x, y(C[a / x] \wedge D[a / y] \rightarrow B u x y) & \rightarrow \\
\exists v \forall x, y^{\prime} &(C[a / x] \wedge D[a / y] \rightarrow B x v y)
\end{aligned}
$$
Здесь $x, y, u, v$ не входят в $C, D .$\\
G11'. (аксиома непрерывности 2-го порядка)
$$
\begin{array}{r}
\forall X, Y(\exists u \forall x, y(x \in X \wedge y \in Y \rightarrow B u x y) \rightarrow \\
\left.\exists v \forall x, y\left(x \in X^{\prime} \wedge y \in Y \rightarrow B x v y\right)\right)
\end{array}
$$
\vskip 0.2in
\begin{theo}[Тарксого о полноте] 
Для любого предложения А языка элементарной геометрии, если $\left(\mathbb{R}^{2} ;=, B, \cong\right) \vDash A,$ то А логически следует из аксиом $G 1-G 11$
\end{theo}
\begin{theo}
Существует алгоритм проверки формулы А на выполнимость в $\mathbb{R}^{2}$
\end{theo}
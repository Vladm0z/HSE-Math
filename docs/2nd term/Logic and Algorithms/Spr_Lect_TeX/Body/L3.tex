
\section{Лекция 3 (01.19.2021)}
Идея: $n - \{\text{натуральные числа, меньшие } n\}$\\
$$
0 = \emptyset,\ 1 = \{0\} = \{\emptyset\},\ 2 = \{0,1\} = \{\emptyset, \{\emptyset\}\}
$$
Обозначения: $0 := \emptyset.\ x+1 = S(x) := x \cup \{x\}$

\begin{defn}[Индуктивное множество]
Множество $Y$ называется \textit{индуктивным}, если $0 \in Y,\ \forall x:\ (x \in Y \to x+1 \in Y)$\\
\end{defn}

\begin{defn}[Множество натуральных чисел]
Наименьшее по включению (- наименьшее) индуктивное множество называется множеством натуральных чисел и обозначается $\mathbb{N}$\\
\end{defn}

\myuline{Утверждение} такое множество существует\\

\begin{theo}[Принцип математической индукции]
Дано некоторое множество $A$. Если $0 \in A$ и $\forall n \in \mathbb{N}\ (n \in A \to n+1 \in A)$, то $n+1 \in A$
\end{theo} 

\myuline{Обозначение} $x < y\ :\Leftrightarrow\ x \in y$\\

\begin{theo}[Принцип порядковой индукции]
Дано некоторое множество $A$. Если $\forall n \in \mathbb{N} (\forall m < n\ m \in A \to n \in A)$, то $\mathbb{N} \subset A$
\end{theo}

\begin{theo}[Принцип минимального элемента]
Пусть $A$ -- некоторое непустое множество $\mathbb{N}$
\end{theo}

\begin{defn}[Вполне упорядоченное множество]
Линейно упорядоченное множество называется вполне упорядоченным, если любое его непустое подмножество $Y$ имеет наименьший элемент (обозначается $\min Y$)
\end{defn}

\begin{theo}
Отношение $<$ на $\mathbb{N}$ линейно упорядочивает $\mathbb{N}$. Более того, этот порядок является полным
\end{theo}

\myuline{Замечание}\\
Поскольку $x < y +1\ \Leftrightarrow\ x < y \text{ или } x = y$, для любого натурального $n$, число $n+1$ является непосредственно следующим за $n$ в смысле порядка $<$

\begin{defn}[Последовательность элементов]
Последовательность элементов множества $A$ -- это функция $\mathbb{N} \to A$
\end{defn}

\begin{theo}[О рекурсии]
Пусть $Y$ -- некое множество $y_0 \in Y$ и $h: Y \to Y$ -- любая функция. Тогда существует единственная функция $f: \mathbb{N} \to Y$, удовлетворяющая для всех $n \in \mathbb{N}$ условию
$$
\begin{cases}
	f(0) = y_0\\
	f(n+1) = h(f(n))
\end{cases}
$$
\end{theo}

\begin{lem}
$$
\forall n \in \mathbb{N}\ (n = 0 \vee \exists m \in \mathbb{N}\ n = m+1)
$$
\end{lem}

\begin{proof}
Даны множество $Y$, элемент $y_0 \in Y$ и функция $h: Y \to Y$. Пусть $F$ -- множество всех функций $g:\ m \to Y,\ m \in \mathbb{N}$, удовлетворяющих условиям на $\operatorname{dom} g$\\
\\
Любые две функции $g_0, g_1 \in F$ совпадают на пересечении своих областей определения. В противном случае рассмотрим минимальный $k \in \mathbb{N}$ такой, что $g_0(k) \ne g_1(k)$. Поскольку $g_0(0) = y_0 = g_1(0)$, имеем $k \ne 0$. Следовательно $k = s+1$, причем $g_0(s) = g_1(s)$, поскольку $k$ -- минимальный. Отсюда $g_0(k) = g_0(s+1) = h(g_0(s)) = h(g_1(s)) = g_1(s+1) = g_1(k)$, противоречие
\end{proof}

\begin{defn}[функциональное множество]
Соотв $R \subset A \times B$ функционально, если $\forall a \in A\ \forall b_1, b_2 \in B\ ((a, b_1) \in R \wedge (a, b_2) \in R\ \Leftrightarrow\ b_1 = b_2)$\\
\end{defn}

Каждая $g: m \to Y$ есть подмножество $m \times Y \subset \mathbb{N} \times Y$. Рассмотрим множество $f := \bigcup F \subset \mathbb{N} \times Y$ и докажем, что $f$ является искомой функцией $\mathbb{N} \to Y$\\
\\
Отношение $f = \bigcup F$ функционально, поскольку любые два элемента совпадают на общей области определения. Свойства (1) очевидно выполняются для $f$\\
\\
Докажем тотальность, рассуждая от противного. Рассмотрим минимальное $k$ такое, что $k \notin \operatorname{dom} f$. Имеем $f: k \to Y$. Можно продолжить $f$ до функции $f_0: k+1 \to Y$, определив $f_0(k) := y_0$, если $k = 0$ и $f_0(k) := h(f(s))$, если $k = s+1$. Очевидно, что $f_0 \in F$, поэтому $k \in \operatorname{dom} f$, противоречие. Тем самым доказано существование $f$\\
\\
Единственность $f$, как в рассуждении выше, легко следует по принципу наименьшего числа\\
Теорема доказана


\begin{defn}[Сложение]
Пусть $s: \mathbb{N} \to \mathbb{N}$, где $s(n) = n+1$. Сложение ($+$) определяется как (единственная) функция $\mathbb{N} \times \mathbb{N} \to \mathbb{N}$, удовлетворяющая рекурсивным условием для всех $n,m \in \mathbb{N}$:
$$
\begin{cases}
	m + 0 = m\\
	m + s(n) = s(m+n)
\end{cases}
$$
Также заметим, что
\begin{gather*}
	m + n \ne m \oplus n\\
	l = m\ \{k \in \mathbb{N}\ |\ m+k \ne m \oplus k\}\\
	l = 0\ \Leftrightarrow\ m + 0 = m = m \oplus 0\\
	l = s(t) = t+1\ \Leftrightarrow m+l = m + s(t) = s(m+t) = s(m \oplus t) = m \oplus s(t) = m \oplus l
\end{gather*}
\end{defn}

\begin{defn}[Умножение]
Умножение ($\cdot$) определяется как (единственная) функция $\mathbb{N} \times \mathbb{N} \to \mathbb{N}$, удовлетворяющая рекурсивным условиям для всех $m,n \in \mathbb{N}$:
$$
\begin{cases}
	m \cdot 0 = 0\\
	m \cdot s(n) = m \cdot n + m
\end{cases}
$$
\end{defn}

\myuline{Доказательство существования функции сложения}\\
Рассмотрим функцию $H: \mathbb{N}^\mathbb{N} \to \mathbb{N}^\mathbb{N}$ такую, что $H(G) = s \circ G$. По теореме о рекурсии существует функция $F: \mathbb{N} \to \mathbb{N}^\mathbb{N}$, удовлетворяющая условиям
$$
\begin{cases}
	F(0) = \operatorname{id}_\mathbb{N}\\
	F(n+1) = H(F(n))
\end{cases}
$$
Положим $m+n = F(n)(m)$, тогда
\begin{gather*}
m+0 = F(0)(m) = \operatorname{id}_\mathbb{N}(m) = m\\
m+s(n) = F(s(n))(m)\\
=F(n+1)(m)\\
=H(F(n))(m)\\
=(s \circ F(n))(m)\\
=s(F(n)(m))\\
= s(m+n)
\end{gather*}
\myuline{Идея}: целое число можно представить разностью двух натуральных числе $m-n$. При этом некоторые пары задают одно и то же число\\
\\
Множество целых чисел $\mathbb{Z}$ можно ввести, как фактормножество $\mathbb{Z} := (\mathbb{N} \times \mathbb{N})\backslash =_{\mathbb{Z}}$, где отношение эквивалентности $=_{\mathbb{Z}}$ задается следующим образом:
\begin{gather*}
	(m_1,n_1) =_{\mathbb{Z}} (m_2, n_2) \Leftrightarrow m_1 + n_2 = n_1 + m_2
\end{gather*}
\myuline{Идея}: рациональное число $q= \frac{m}{n}$ можно рассматривать как пары $(m,n)$, где $m \in \mathbb{Z}$ и $n \in \mathbb{N} \backslash \{0\}$. Однако, некоторые пары задают одно и то же рациональное число.\\
\\
Множество рациональных чисел $\mathbb{Q}$ можно ввести, как фактормножество $\mathbb{Q} := (\mathbb{Z} \times (\mathbb{N} \backslash \{0\}))\backslash =_{\mathbb{Q}}$, где отношение эквивалентности $=_{\mathbb{Q}}$ задается следующим образом:
\begin{gather*}
	(m_1, n_1) =_{\mathbb{Z}} (m_2, n_2) \Leftrightarrow m_1n_2 = n_1m_2
\end{gather*}
Простые свойства вполне упорядоченных множеств:\\
1) всякое непустое вполне упорядоченное множество имеет наименьший элемент\\
2) всякий отличный от наибольшего элемент $x \in X$ имеет непосредственного последователя, то есть $\exists y \in X\ \forall z \in X\ (x < z \to y \leqslant z)$\\
3) всякое ограниченное сверху подмножество имеет точную верхнюю грань
\vskip 0.2in

\begin{lem}
Даны вполне упорядоченное множеств $(X, <)$ и функция $f: X \to X$, сохраняющая порядок. Тогда $x \leqslant f(x)$ для любого $x \in X$
\end{lem}
\begin{proof}
Предположим, что $Y = \{x \in X\ |\ f(x) < x\}$ не является пустым, и рассмотрим $a = \min Y$. Имеем $f(a) < a$, поскольку $a \in Y$. Следовательно $f(f(a)) < f(a)$ по монотонности $f$. Тогда $f(a) \in Y$ и $f(a) < a$, что противоречит минимальности $a$. Заключаем, что $Y$ пусто.
\end{proof}

\begin{defn}[Начальный отрезок]
Начальным отрезком множества $(X, <)$ называется такое подмножество $Y \subset X$, для которого
\begin{gather*}
	\forall x,y \in X\ (y < x \wedge x \in Y \Rightarrow y \in Y)
\end{gather*}
\end{defn}

\myuline{Обозначение}\\
Для $a \in X$ обозначим $[0,a) = \{x \in X\ |\ x < a\}$
\myuline{Наблюдение}\\
Любой собственный начальный отрезок $(X, <)$ имеет вид $[0,a)$ для некоторого $a \in X$\\
\\
\begin{lem}
Вполне упорядоченное множество не изоморфно никакому собственному начальному отрезку
\end{lem}

\begin{proof}
Допустим, что существуют собственный начальный отрезок $Y \subset X$ и изоморфизм $f: X \to Y$. Рассмотрим $a \in X \backslash Y$. Имеем $f(a) < a$, поскольку $a \notin Y,\ f(a) \in Y$ и $Y$ -- начальный отрезок $X$. Противоречие с предыдущей леммой.
\end{proof}

\begin{theo}[Кантор]
Для любых двух вполне упорядоченных множеств одно изоморфно начальному отрезку другого
\end{theo}
\begin{proof}
Возьмем два вполне упорядоченных множества $A, B$, рассмотрим $R = \{(x,y) \in A \times B\ |\ [0,x)_A \cong [0,y)_B\}$. Соответствие $R$ функционально и инъективно, то есть является биекцией из $\operatorname{dom} R$ в $\operatorname{ran} R$
\end{proof}
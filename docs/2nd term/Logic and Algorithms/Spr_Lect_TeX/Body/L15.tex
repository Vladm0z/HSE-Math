\section{Лекция 15 (03.23.2021)}

\begin{defn}
Фильтр на множестве $I-$ это непустое $\mathcal{F} \subset \mathcal{P}(I)$ со свойствами
$\bullet X, Y \in \mathcal{F} \Rightarrow(X \cap Y) \in \mathcal{F}$
$\bullet X \in \mathcal{F} \& X \subset Y \Rightarrow Y \in \mathcal{F}$
Фильтр $\mathcal{F}$ собственный, если $\varnothing \notin \mathcal{F}$ Ультрафильтр - максимальный по включению собственный фильтр.
\end{defn}
\begin{lem}
Свойства ультрафильтров:\\
- $X \in \mathcal{F} \& Y \in \mathcal{F} \Leftrightarrow(X \cap Y) \in \mathcal{F}$\\
- $X \notin \mathcal{F} \Leftrightarrow(I \backslash X) \in \mathcal{F}$
\end{lem}
\begin{lem}
Любой собственный фильтр можно расширить до ультрафильтра,
\end{lem}

\begin{defn}
Фильтр $\mathcal{F}$ главный, если $\bigcap \mathcal{F} \neq \varnothing .$
\end{defn}
\begin{lem}
Ультрафильтр $\mathcal{U}$ главный, если и только если существует конечное $J \in \mathcal{U}$. 
\end{lem}
\begin{defn}
Пусть задан ультрафильтр $\mathcal{U}$ на $I .$ Рассмотрим свойства элементов $I$ (одноместные предикаты). Свойство Ф верно почти всегда (относительно $\mathcal{U})$, если
$$
\{i \mid \Phi(i)\} \in \mathcal{U}
$$
Обозначение: $\forall^{\infty} i \Phi(i) .$
\end{defn}
\begin{lem}
Свойства квантора $\forall^{\infty} .$\\
- $\forall^{\infty} i(\Phi(i) \wedge \Psi(i)) \Leftrightarrow \forall^{\infty} i \Phi(i) \wedge \forall^{\infty} i \Psi(i)$\\
- $\forall^{\infty} i \neg \Phi(i) \Leftrightarrow \neg \forall^{\infty} i \Phi(i) .$
\end{lem}

\begin{lem}
Пусть $\left(M_{i}\right)_{i \in I}-$ семейство моделей сигнатуры $\Omega, \mathcal{U}$ ультрафильтр на $I .$ Тогда
$$
\alpha \approx_{\mathcal{U}} \beta:=\forall^{\infty} i\left(\alpha_{i}=\beta_{i}\right)
$$
задает отношение эквивалентности на множестве $\prod M_{i} .$
\end{lem}
Класс элемента $\left(\alpha_{i}\right)_{i \in I}$ обозначается $\left[\alpha_{i}\right]_{i \in I} .$

\begin{defn}
Пусть $\left(M_{i}\right)_{i \in I}-$ семейство моделей сигнатуры $\Omega .$ $\mathcal{U}-$ ультрафильтр на $I .$ Ультрапроизведение семейства $\left(M_{i}\right)_{i \in I}$ по ультрафильтру $\mathcal{U}$ задается следующим образом.
- Носитель $M$ это $\prod_{i \in I} M_{i} / \approx_{\mathcal{U}} .$
$c_{M}:=\left[c_{M_{i}}\right]_{i \in I}$
$f_{M}\left(\left[m_{i}^{1}\right], \ldots,\left[m_{i}^{k}\right]\right):=\left[f_{M_{i}}\left(m_{i}^{1}, \ldots, m_{i}^{k}\right)\right]$
$M \vDash P\left(\left[m_{i}^{1}\right], \ldots,\left[m_{i}^{k}\right]\right) \Leftrightarrow \forall^{\infty} i M_{i} \vDash P\left(m_{i}^{1}, \ldots, m_{i}^{k}\right)$
Обозначение: $\prod_{\mathcal{U}} M_{i} .$
\end{defn}
\begin{theo}[Лось]
$$
\prod_{\mathcal{U}} M_{i} \vDash A\left(\left[m_{i}^{1}\right], \ldots,\left[m_{i}^{k}\right]\right) \Leftrightarrow \forall^{\infty} i M_{i} \vDash A\left(m_{i}^{1}, \ldots, m_{i}^{k}\right)
$$
\end{theo}
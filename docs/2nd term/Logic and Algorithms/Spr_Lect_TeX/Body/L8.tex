\section{Лекция 8 (02.09.2021)}

Философские вопросы математики:\\
1) Что значит доказать теорему?\\
2) Что значит дать определение понятию?\\
3) Правомерно ли рассуждать об актуально бесконечных множествах?\\
4) Когда говорим об истинности и доказуемости, имеются ли ввиду одно и то же?\\
5) Противоречива ли математика? Возможно ли установить непротиворечивость?
\vskip 0.3in
Направения Философии математики: Логицизм (Фреге, Рассел, Уайтхед), Интуиционизм (Брауэр, Вейль), Формализм (Гильберт), Платонизм (Гедель) и т.д.\\
Подробнее в: С.Клини -- ``Введение в математику''
\vskip 0.2in
Фксиоматический аетод Гильберта предполагает явную формулировку всех предположений теории и допускает лишь чисто логические выводы из этих посылок.
\vskip 0.1in
Логический вывод может быть записан в символьном виде, что превращает его в вычислительный процесс. Это привело к созданию формальных аксиоматических теорий.
\vskip 0.2in
Компьютерные реализации: Coq, HOL, Mizar
\vskip 0.2in
Программа Гильберта:\\
1) Формализовать математику (теорию множеств) в рамках аксиоматической теории $T$\\
2) Формальные доказательства в $T$ представляют собой конечные объекты (строки символов), строящиеся по вполне определенным правилам\\
3) Из следует проанализировать элементарными комбинаторными средствами и установить, что противоречие в $T$ не доказуемо\\
4) Тем самым мы сведем использование теоретико-множественных методов к заведомо надежным элементарным методам
\vskip 0.2in
Математическая логика -- построение и исследование формальных языков математическими методами.\\
Метаязык (описывание объекта), Метатеория (теория в рамках которой мы рассуждаем об исследуемой теории), Синтаксис (правила построения выражений), Семантика (значение/смысл выражений)
\vskip 0.4in
Предикаты и функции\\
Пусть $M$ -- непустое множество\\
\begin{defn}
$n$-арный предикат на $M$: функция $Q:\ M^n \to \{0,1\}$ (Интуитивно: $Q(x_1, \ldots ,x_n)$ есть высказывание, зависящее от выбора параметров $x_1, \ldots, x_n \in M$. Предикаты можно также понимать как $n$-арные отношения на $M$, то есть подмножества $M^n$)
\end{defn}
\begin{defn}
$n$-арная функция на $M$: функция $f:\ M^n \to M$
\end{defn}
\begin{defn}
константа: элемент $M$
\end{defn}
\begin{defn}
Сигнатурой называется некоторая совокупность имен функций, предикатов и констант. Сигнатура $\Sigma$ задается:\\
$\operatorname{Pred_{\Sigma}}$ предикатные символы\\
$\operatorname{Func_{\Sigma}}$ функциональные символы\\
$\operatorname{Const_{\Sigma}}$ символы констант\\
Функция валентности $\operatorname{Pred_{\Sigma}} \cup \operatorname{Func_{\Sigma}} \to \mathbb{N}\backslash \{0\}$
\end{defn}
\begin{defn}
Модель сигнатуры $\Sigma$ есть непустое множество $M$ вместе с отображением, сопоставляющим:\\
Каждому $P \in \operatorname{Pred_{\Sigma}}$ некоторый предикат $P$ на $M$ той же валентности\\
Каждому $f \in \operatorname{Func_{\Sigma}}$ функцию $f_M$ на $M$ той же валентности\\
Каждому $c \in \operatorname{Const_{\Sigma}}$ константу $c_M \in M$.
\end{defn}

Синтаксис Логики первого порядка
\vskip 0.1in
Алфавит $\mathcal{L}_{\Sigma}$ содержит:\\
Символы сигнатуры: $\Sigma$\\
Свободные переменные: $\operatorname{FrVar} = \{a_0, a_1, \ldots\}$\\
Связанные переменные: $\operatorname{BdVar} = \{v_0, v_1, \ldots\}$\\
Булевы связки: $\rightarrow, \neg, \wedge, \vee ;$\\
Кванторы $\forall,\ \exists$\\
Знаки пунктуации: $\ll ( \gg,\ \ll )\gg,\ \ll , \gg$

\begin{defn}
Множество термов $\operatorname{Tm_{\Sigma}}$ есть наименьшее множество, замкнутое относительно следующих двух правил:\\
1) Свободные переменные и константы суть термы
2) Если $f \in \operatorname{Func_{\Sigma}}$ валентности $n$ и $t_1, \ldots, t_n$ -- термы, то выражение $f(t_1, \ldots, t_n)$ есть терм
\end{defn}

\begin{defn}
Множество формул $\operatorname{Fm_{\Sigma}}$ есть наименьшее множество, замкнутое относительно следующих правил:\\
Если $P \in \operatorname{Pred_{\Sigma}}$ валентности $n$ и $t_1, \ldots, t_n$ -- термы, то $P(t_1, \ldots, t_n)$ есть формула (называемая атомарной формулой)\\
Если $A,B$ -- формулы, то формулыми являются $(A \to B),\ \neg A,\ (A \wedge B),\ (A \vee B)$\\
Если $A$ -- формула, и $a$ -- свободная переменная, то для любой связанной переменной $x$, не входящей в $A$, выражения $(\forall x\ A[a \backslash x])$ и $(\exists x\ A[a \backslash x])$ -- формулы ($A[a \backslash x]$ -- замена $a$ на $x$).
\end{defn}
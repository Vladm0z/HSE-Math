\section{Лекция 6 (02.02.2021)}

\begin{defn}
Множество $g$ называется трансфинитнй последовательностью, если $g:\ \zeta \to X$ для некоторого ординала $\zeta$ и некоторого множества $X$
\end{defn}

\begin{theo}
Пусть $\varphi(x,y)$ -- некоторое свойство множеств, причем для любой трансфинитной последовательности $x$ существует не более одного $y$, удовлетворяющего $\varphi(x,y)$
\vskip 0.1in
Тогда выполнено следующее:
1) либо для любого ординала 
\end{theo}

\begin{theo}[Цорн]
Пусть $(P,<)$ -- частично упорядоченное множество, в котором всякая цепь имеет верхнюю грань. Тогда $(P,<)$ содержит максимальный элемент
\end{theo}
\begin{proof}
Пусть $f$ -- функция выбора на семействе всех непустых подмножеств $P$, она существует по аксиоме выбора
\vskip 0.1in
Назовем трансинитную последовательность $g$ хорошей, если $\operatorname{ran} g \subset P$ и $g(a) < g(b)$ для любых $a < b \in \operatorname{dom} g$

\begin{defn}
Назовем строгой верхней грань. множества $A \subset P$ такой элемент $z \in P$, что $a < z$ для любого $a \in A$. Через $b(A)$ обозначим множество всех строгих верхних граней $A$
\end{defn}

Рассмотрим условие $\varphi(x,y):\ x$ -- хорошая трансфинитная последовательность, для которой $b(\operatorname{ran} x) \ne \varnothing$, и $y = f(b(\operatorname{ran} x))$
\vskip 0.1in
Видим, что для любой трансинитной последовательноти $x$ существует не более одного множества $y$, удовлетворяющего $\varphi(x,y)$
\vskip 0.1in
Допустим, что для любого ординала $\alpha$ существует единственная $\alpha$-последовательность, удовлетворяющая $\varphi$
\vskip 0.1in
Придем к противоречию, рассмотрев условие $\psi(c,d):\ c \in P,\ d$ -- ординал, и для некоторой трансфинитной последовательности $g$, удовлетворяющей рекурсивному условию, $g(d) = c$
\vskip 0.1in
Видим, что для любого множества $c$ существуею не более одного множества $d$, удовлетворяющего $\psi(c,d)$. По аксиоме подстановки существует множество $D = \{d\ |\ \exists c \in P\ \psi(c,d)\}$. В нашем предположении $D$ является множеством всех ординалов, противоречие.
\vskip 0.1in
Следовательно существует такая последовательность $g$, удовлетворяющая $\varphi$, которую нельзя продолжить, то есть нет такого $y$, что $\varphi(g,y)$
\vskip 0.1in
Поскольку $g$ является хорошей и $g$ нельзя продолжить, получаем, что $b(\operatorname{ran} g) = \varnothing$. Кроме того, $\operatorname{ran} g$ является цепью в $P$. Замечаем, что $a$ (верхнаяя грань для $\operatorname{ran} g$) -- искомый максимальный элемент $P$, поскольку иначе $b(\operatorname{ran} g) \ne \varnothing$ и $g$ можно было бы продолжить
\end{proof}

\begin{theo}[Кантора-Бернштейна]
Если $A \lesssim B$ и $B \lesssim A$, то $A \sim B$
\end{theo}
\begin{corol}
1) Всякое бесконечное множество содержит счетное подмножество\\
2) Мощность бесконечного множества не меняется при объединении с конечным\\
3) $\mathbb{N} \sim \mathbb{N} \times \{0,1\} \sim \mathbb{N} \times \mathbb{N}$\\
$$
	(m_1, m_2) < (n_1, n_2) \Leftrightarrow
	\begin{cases}
	\max\{m_1,m_2\} < \max\{n_1, n_2\}\\
	\max\{m_1,m_2\} = \max\{n_1, n_2\} \text{ и } m_1 < n_1\\
	\max\{m_1,m_2\} = \max\{n_1, n_2\} \text{ и } m_1 = n_1,\ m_2 < n_2
	\end{cases}
$$
\end{corol}

\begin{lem}
Если множество $A$ бесконечно, то $A \sim A \times \{0,1\}$
\end{lem}
\begin{proof}
Рассмотрим $P$, состоящее из пар вида $(B,f)$, где $B$ -- бесконченое подмножество, $f:\ B \to B \times 2$ -- биекция. Зададим на $P$ частичный порядок
$$
	(B_1,f_1) \leqslant (B_2,f_2) \leftrightarrow B_1 \subset B_2,\ f_1 = f_2 \uparrow_{B_1}
$$
Пусть $C$ -- произвольная цепь в $P$ является верхней гранью $C$. Проверим, что $P$ непусто. Бескончное множество $A$ содержит счетное подмножество $D$. Поскольку $D$ счетно, ято существует биекция $g:\ D \to D \times 2$. Получаем, что пара $(D,g) \in P$ и является верхней гранью $C$.
\vskip 0.1in
Предположим что $C \ne \varnothing$\\
Рассмотрим $D = \bigcup\{B\ |\ \exists f\ (B,f) \in C\}$, то есть объединение всех первых компнент элементов $C$, и $g = \bigcup\{f\ |\ \exists B\ (B,f) \in C\}$, то есть объединение всех вторых компонент
\vskip 0.1in
Соответствие $g \subset D \times (D \times 2)$ функционально в силу того, что все вторын компоненты элеметов $C$ попарно совпадают на общих областях определения. Очевидно, что $g$ тотально, а следовательно $g$ -- функция.
\vskip 0.1in
Функция $g$ инъективна: для различных $d_1, d_2 \in D$ возьмем большее из множеств, которым принадлежат $d_1$ и $d_2$. На нем $g$ является инъекцией по предположению
\vskip 0.1in
Кроме того, $g$ -- сюръекция, для любой пары возьмем $(d, i) \in D \times 2$ возьмем множество $B$ из которого произошло $d$ и вспомним, что мы имели взаимно однознечное соответствие между ним и $B \times 2$
\vskip 0.1in
Мы нахлдимся в условиях леммы Цорна и знаем, что $P$ содержит максимальный элемент $(E,h)$
\vskip 0.1in
Рассмотрим дополнение $E$ до $A$. Если множество $A \backslash E$ конечно, то $A = (A \backslash E) \cup E \sim E$. Получим $A \sim E \sim E \times 2 \sim A \times 2$, все доказано.
\vskip 0.1in
Если множество $A \backslash E$ бесконечно, то оно содержит счетное подмножество $E'$. Кроме того, существует биекция $h':\ E'\to E'\times 2$
\vskip 0.1in
Тогда $h \cup h'$ -- биекция из $E \cup E'$ в $(E \cup E') \times 2 = E \times 2 \cup E'\times 2$. Полчим пару $(E \cup E', h \cup h')$ из $P$, которая больше пары $(E,h)$, что противоречит максимальности $(E,h)$, этот случай невозможен.
\end{proof}

\begin{theo}
Объединение двух бесконечных множеств $A$ и $B$ равномощно большему из них 
\end{theo}
\begin{proof}
Поскольку любые два множества сравнимы по мощности, можно считать без ограничения общености, что $A \lesssim B$. Тогда $B \lesssim A \cup B \lesssim B \times \{0,1\} \sim B$
По теореме кантора-Бернштейна получаем что $B \sim A \cup B$
\end{proof}

\begin{lem}
Если $A$ бескончено, то $A \sim A \times A$
\end{lem}
\begin{proof}
Рассмотрим множество $P$, состоящее из пар вида $(B,f)$, где $B$ -- бесконечное подмножество $A$, $f:\ B \to B \times B$ -- биекция. Зададим на $P$ частичный порядок 
$$
(B_1, f_1) \leqslant (B_2, f_2) \Leftrightarrow B_1 \subset B_2,\ f_1 = f_2 \uparrow_{B_1}
$$
Пусть $C$ -- произвольная цепь в $P$. Убедимся что $C$ имеет верхнюю грань $(D,g)$\\
Если $C = \varnothing$, то любой элемент $P$ является верхней гранью $C$. Проверим, что $P$ непусто. Бесконченое множество $A$ содержит счетное подмножество $D$. Поскльку $D$ счетно, существует биекция $g: D \to D \times D$. Получаем, что пара $(D,g) \in P$ и является верхней гранью $C$.
\vskip 0.1in
Теперь предположим, что $C \ne \varnothing$\\
Рассмотрим $D = \bigcup\{B\ |\ \exists f\ (B,f) \in C\}$, то есть объединение всех первыъ компонент элементов $C$, и $g = \bigcup\{f\ |\ \exists B\ (B,f) \in C\}$, то есть объединение всех вторых компонент.
\vskip 0.1in
Как и предыдущем доказательстве, замечаем, что соответствие $g \subset D \times (D \times D)$ является функцией.
\vskip 0.1in
Функция $g$ инъективна: для различных $d_1, d_2 \in D$ возьмем большее из множеств, которым принадлежат $d_1, d_2$, на нем $g$ является индукцией по предположению
\vskip 0.1in
Кроме того, $g$ является сюръекцией: для люой пары $(d_1,d_2) \in D \times D$ возьмем множества $B_1, B_2$, из которых произошли $d_1, d_2$ выберем из этих множеств большее и вспомним, что мы имели взаимно однозначное соответствие между ним и его квадратом
\vskip 0.1in
Мы находимся в условиях леммы Цорна и знаем, что $P$ содержит максимальный элемент $(E,h)$.
\vskip 0.1in
Рассмотрим дополнение $E$ до $A$. Если $A \backslash E \lesssim E$, то $A = (A \backslash E) \cup E \lesssim E$. Получаем, что $A \sim E \sim E \times E \sim A \times A$ и все доказано.
\vskip 0.1in
Если $E \lesssim A \backslash E$, то $A \backslash E$ содержит подмножество $E'$, которое равномощно $E$.
\vskip 0.1in
Если $E \lesssim A \backslash E$, то $A \backslash E$ содержит подмножество $E'$, которое равномощно $E$. 
\vskip 0.1in
Биекцию $h$ из $E$ в $E \times E$ можно продолжить до биекции из $E \cup E'$ в $S = (E \cup E') \times (E \cup E')$, поскольку $E' \sim S \backslash (E \times E)$, действительно
$$
S \backslash (E \times E) = (E \times E') \cup (E'\times E') \cup (E'\times E) \sim E \times E \sim E \sim E'
$$
Получаем пару из $P$, которая больше пары $(E,h)$, что противоречит максимальности $(E,h)$. Таким образом, этот случай невозможен.
\end{proof}

\begin{corol}
Если множество $A$ бесконечно, то множество всех последовательностей длины $n > 0$, составленных из элементов $A$, равномощно $A$, то есть $A^n \sim A$
\end{corol}
\begin{corol}
Если множество $A$ бесконечно, то множество всех конечных последовательностей, составленных из элементов $A$, равноможно $A$, то есть $A^* \sim A$
\end{corol}
\begin{proof}
$$
A* = \bigcup\{A^n\ |\ n \in \mathbb{N}\} \sim A \times \mathbb{N} \sim A
$$
\end{proof}
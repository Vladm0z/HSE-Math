\section{Лекция 12 (03.09.2021)}

\begin{defn}
Пусть $T$ -- теория, $A$ -- замкнутая формула в ее сигнатуре. $A$ логически следует из $T$ (обозначение: $T \vDash A$), если любая модель теории $T$ является моделью формулы $A$.
\end{defn}

\begin{theo}[о корректности для исчисления предикатов]
Если $T \vdash A$, то $T \vDash A$.
\end{theo}
\begin{theo}[Гёделя о полноте для исчисления предикатов]
Если $T \vDash A$, то $T \vdash A$.\\
\\
Версия для теорий с равенством:\\
$T \vdash A$ означает выводимость с использованием аксиом равенства. $T \vDash A$ означает логическое следование на нормальных моделях.
\end{theo}

\begin{defn}
Пусть $M, M^{\prime}$ -- модели сигнатуры $\Omega$. Отображение носителей $\alpha: M \longrightarrow M^{\prime}$ называется изоморфизмом $M$ на $M^{\prime}$, если\\
- $\alpha$ -- биекция,\\
- $\alpha\left(c_{M}\right)=c_{M^{\prime}}$ для всех констант $c($ из $\Omega)$\\
- $\alpha\left(f_{M}\left(m_{1}, \ldots, m_{k}\right)\right)=f_{M^{\prime}}\left(\alpha\left(m_{1}\right), \ldots, \alpha\left(m_{k}\right)\right)$ для любого
$k$ -местного функционального символа $f$ и $m_{1}, \ldots, m_{k} \in M$,\\
- $P_{M}\left(m_{1}, \ldots, m_{k}\right)=P_{M^{\prime}}\left(\alpha\left(m_{1}\right), \ldots, \alpha\left(m_{k}\right)\right)$ для любого
$k$ -местного предикатного символа $P$ и $m_{1}, \ldots, m_{k} \in M$. Запись $\alpha: M \cong M^{\prime}$ означает, что $\alpha$ -- изоморфизм $M$ на $M^{\prime}$. 
\end{defn}

\begin{lem}
(1) Если $\alpha: M \cong M^{\prime}$ и $\beta: M^{\prime} \cong M^{\prime \prime}$, то $\beta \alpha: M \cong M^{\prime \prime}$.\\
(2) Если $\alpha: M \cong M^{\prime}$, то $\alpha^{-1}: M^{\prime} \cong M$.
\end{lem}

\begin{defn}
Модели $M, M^{\prime}$ называются изоморфными (обозначение: $\left.M \cong M^{\prime}\right)$ если существует изоморфизм $\alpha: M \cong M^{\prime}$.. $\cong$ задает отношение эквивалентности на классе всех моделей данной сигнатуры.
\end{defn}

\begin{defn}
Терм, оцененный в $M$, -- это замкнутый терм расширенной сигнатуры $\Omega(M)$. Из обычного терма $t\left(a_{1}, \ldots, a_{n}\right)$, получаются оцененные термы
$$
t\left(m_{1}, \ldots, m_{n}\right):=t\left[a_{1}, \ldots, a_{n} / \underline{m_{1}}, \ldots, m_{n}\right]
$$
$|r|_{M}$ - значение оцененного терма $r$ в модели $M$; это элемент из $M$.
\end{defn}
\begin{defn}
Формула, ощененная в $M$, - это замкнутая формула сигнатуры $\Omega(M),|A|_{M}$ -- значение оцененной формулы $A$ в $M$ ($0$ или $1$)
\end{defn}


Пусть $M, M^{\prime}$ -- модели сигнатуры $\Omega, \alpha: M \cong M^{\prime}$. Для терма $t$, оцененного в $M$, обозначим через $\alpha \cdot t$ терм, полученный заменой всех констант $m$ из $M$ на их образы $\alpha(m)$. (Формально $\alpha \cdot t$ определяется по индукции)
-- Аналогично по формуле $A$, оцененной в $M$, строится формула $\alpha \cdot A$, оцененная в $M^{\prime}$.

\begin{theo}
Пусть $M, M^{\prime}-$ модели сигнатуры $\Omega, \alpha: M \cong M^{\prime} .$\\
- Если $t-$ терм, оцененный в $M,$ то $|\alpha \cdot t|_{M^{\prime}}=\alpha\left(|t|_{M}\right) .$\\
- Если $A-$ формула, оцененная в $M,$ то $|\alpha \cdot A|_{M^{\prime}}=|A|_{M}$ Определение

Пусть $M-$ модель сигнатуры $\Omega .$ Элементарная теория модели $M$ - это множество всех замкнутых формул сигнатуры $\Omega,$ которые истинны в $M .$
$$
T h(M):=\{A \mid M \models A\}
$$
Модели $M_{1}, M_{2}$ одной сигнатуры называются элементарно
эквивалентными, если в них истинны одни и те же замкнутые формулы, т.е. $T h\left(M_{1}\right)=T h\left(M_{2}\right) ;$ обозначение: $M_{1} \equiv M_{2} .$
\end{theo}


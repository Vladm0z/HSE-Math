\section{Лекция 5 (01.26.2021)}

\begin{defn}
 Ординал -- транзитивное множество, все элементы которого транзитивны
 \end{defn}

 Обозначение: $x < y \Leftrightarrow x \in y$

 \begin{theo}
 Класс всех ординалов линейно упорядочен с помощью $<$. Более того, всякое множество содержит $<$-наименьший элемент
 \end{theo}

 \begin{theo}[Парадокс Бурали-Форти 1897]
 Не существует множества, состоящего в точности из всех ординалов
 \end{theo}

\begin{defn}
Ординалы вида $\beta + 1$ называются ординалами-последователями. Все остальные ординалы, кроме $0$, называются предельными
\end{defn}

Замечание: Не существует ординала $\gamma$ такого, что $\alpha < \gamma < \alpha +1$

\begin{defn}
Пусть $\zeta$ -- некоторый ординал. Множество $g$ называется $\zeta$-последовательностью, если $g: \zeta \to X$ для некоторого $X$.\\
Такие последовательности также обозначают $(X_{\nu})_{\nu < \zeta}$
\end{defn}

\begin{defn}
Пусть $\varphi(x,y)$ -- некоторое свойство множеств, причем для любой трансинитной последовательности $x$ существует не более одного множества $y$, удовлетворяющего $\varphi(x,y)$\\
Будем говорить, что трансинитная последовательность $g$ (длины $\zeta$) удовлетворяет рекурсивному условию, заданному $\varphi$, если для всякого ординала $\nu < \zeta$ имеет место $\varphi(g \uparrow_{\nu}, g(\nu))$
\end{defn}

\begin{theo}
Пусть $\varphi(x,y)$ -- некоторое свойство множеств, причем для любой трансинитной последовательности $x$ существует не более одного множества $y$, удовлетворяющего $\varphi(x,y)$\\
Тогда выполнено следующее:\\
1) либо для любого ординала $\alpha$ существует единственная $\alpha$-последовательность, удовлетворяющая рекурсивному условию, заданному $\varphi$\\
2) либо существует единственная трансфинитная последовательность $g$, удовлетворяющая рекурсивному условию, заданному $\varphi$, для которой не существует такого $y$, что $\varphi(g,y)$
\end{theo}

\begin{proof}
Будем говорить, что трансинитная последовательность $g$ (длины $\zeta$) удовлетворяет рекурсивному условию, если для всякого ординала $\nu < \zeta$ имеет место $\varphi(g \uparrow_{\nu}, g(\nu))$
\vskip 0.1in
Любые две трансинитные последовательности $g_1, g_2$ удовлетворяющие рекурсивному условию, совпадают на пересечении своих областей определения. В противном случае рассмотрим $\in$-минимальный ординал $\lambda$ такой, что $g_1(\lambda) \ne g_2(\lambda)$. В силу минимальности $\lambda$ получаем, что $g \uparrow_{\lambda}$
\vskip 0.1in
Предположим, что не для всякого ординала $\alpha$ существует $\alpha$ - последовательность, удовлетворяющая рекурсивному условию. Рассмотрим минимальный ординал $\lambda$, для которого не существует соответствующей $\lambda$-последовательности
\vskip 0.1in
Видим, что $\lambda \ne 0$. Проверим, что $\lambda$ не является предельным ординалом. Рассмотрим условие $\psi(u,v)$: $u$ -- ординал, $v$ -- $u$-последовательность, удовлетворяющая рекурсивному условию.
\vskip 0.1in
Для всякого $u$ существует не более одного $v$ такого, что верно $\psi(u,v)$. По акиоме подстановки существует множество $V = \{v\ |\ \exists u \in \lambda\ \psi(u,v)\}$. Тогда $\bigcup V$ -- $\lambda$-последовательность, удовлетворяющая рекурсивному условию. Противоречие.
\vskip 0.1in
Следовательно $\lambda = \lambda_0 + 1$ для некоторого ординала $\lambda_0$. По минимальности $\lambda$, найдется $\lambda_0$-последовательность $g$, удовлетворяющая рекурсивному условию. Видим, что не существует такого $y$, что 
\end{proof}

\begin{theo}[Цермело]
Для всякого множества $X$ существует бинарное отношение $<$ на $X$ такое, что $(X, <)$ -- вполне упорядоченное множество
\end{theo}
\begin{proof}
Пусть $f$ -- функция выбора на семействе всех непустых подмножеств $X$. Такая функция существует по аксиоме выбора.
\vskip 0.1in
Назовем трансфинитную последовательность $g$ хорошей, если $\operatorname{ran} g \subset X$ и $g(a) \ne g(b)$ для любых $a \ne b$ из $\operatorname{dom} g$. Другими словами, хорошая последовательность -- трансфинитная последовательность из различных элементов $X$.
\vskip 0.1in
Рассмотрим условие $\varphi(x,y)$: $x$ -- хорошая трансфинитная последовательность, для которой $X \backslash \operatorname{ran} x \ne \emptyset$ и $y = f(X \backslash \operatorname{ran} x)$
\vskip 0.1in
Видим, что для любой трансфинитной последовательности $x$ существует не более одного множества $y$, удовлетворяющего $\varphi(x,y)$
\vskip 0.1in
Мы находимся в условиях теоремы о трансфинитной рекурсии. Заметим, что любая трансфинитная последовательность, удовлетворяющая рекурсивному условию, заданному $\varphi$, является хорошей
\vskip 0.1in
Допустим, что для любого ординала $\alpha$ существует единственная $\alpha$-последовательность, удовлетворяющая рекурсивному условию, заданному $\varphi$.
\vskip 0.1in
Придем к противоречию, рассмотрев условие $\psi(c,d):\ c \in X$, $d$ -- ординал, и для некоторой трансифнитной последовательности $g$, удовлетворяющей рекурсивному условию, $g(d) = c$
\vskip 0.1in
Видим, что для любого множества $c$ существует не более одного множества $d$, удовлетворяющего условию $\psi(c,d)$. По аксиоме подстановки существует множество $D = \{d\ |\ \exists c \in X\ \psi(c,d)\}$. В нашем предположении $D$ является множеством всех ординалов, но такое множество не существует.
\vskip 0.1in
Следовательно, существует трансфинитная последовательность $g$, удовлетворяющая рекурсивному условию, заданному $\varphi$, которую нельзя продолжить, то есть не существует такого $y$, что $\varphi(g,y)$
\vskip 0.1in
Поскольку $g$ является хорошей и ее нельзя продолжить, то $X \backslash \operatorname{ran} g = \emptyset$. Другими словами, $g$ является биекцией из некоторого ординала $\alpha$ в $X$. Тогда полный порядок на $X$ определяется как $\{(a,b) \in X \times X\ |\ g^{-1}(a) \in g^{-1}(b)\}$
\end{proof}

\begin{defn}
Кардинал -- это такой ординал, который неравномощен накакому меньшему ординалу
\end{defn}

\begin{lem}
Для любого множества $A$ существует единственный кардинал, который равномощен $A$.
\end{lem}
\begin{proof}
$(A, <) \cong \alpha$, рассмотрим ординалы, равномощные $\alpha:\ K = \min\{\beta \leqslant \alpha\ |\ \beta \sim \alpha \sim A\}$ (тогда $K$ -- кардинал и он единственен по определению).
\end{proof}

\begin{defn}
Кардинал $k$ называется мощностью множества $A$, если он равномощен $A$
\end{defn}

\begin{lem}
Любые два множества $A, B$ сравнимы по мощности, то есть $A \lesssim B$ или $B \lesssim A$
\end{lem}
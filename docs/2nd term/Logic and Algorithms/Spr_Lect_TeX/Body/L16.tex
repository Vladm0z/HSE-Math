\section{Лекция 16 (04.05.2021)}

\begin{theo}[Гёделя - Мальцева]
Пусть $T-$ теория в некоторой сигнатуре. Если каждое конечное подмножество $T$ выполнимо, то $T$ выполнима.
\end{theo}
\begin{proof}
Рассмотрим
$$
I:=\{S \subset T \mid I \text { конечно }\} \text { . }
$$
Для каждого $S \in I$ существует модель $M_{S} \vDash S .$\\
Для $A \in T$ пусть
$$
J_{A}:=\{S \in I \mid A \in S\}
$$
\end{proof}
\begin{lem}
Существует ультрафильтр на $I$, содержащий все $J_{A} .$
\end{lem}
\begin{proof}
$J_{A_{1}} \cap \ldots \cap J_{A_{k}} \neq \varnothing$, т.к. содержит $\left\{A_{1}, \ldots, A_{k}\right\} .$
Поэтому найдется фильтр, содержащий все такие пересечения.
Пусть $\mathcal{U}$ содержит все $J_{A}$ для $A \in T .$ Тогда
$$
\prod_{\mathcal{U}} M_{S} \vDash T .
$$
Действительно,
$$
J_{A} \in \mathcal{U} \Leftrightarrow \forall^{\infty} S A \in S \text { . }
$$
Тогда
$$
\forall^{\infty} S M_{S} \vDash A
$$
По теореме Лося,
$$
\prod_{\mathcal{U}} M_{S} \vDash A .
$$
\end{proof}

\begin{theo}
Если теория имеет конечные модели неограниченной мощности, то она имеет и бесконечную модель.
\end{theo}
\begin{theo}[Лёвенгейма - Сколема о подъеме]
Если теория в сигнатуре $\Omega$ имеет бесконечную модель, то она имеет модели любой бесконечной мощности $k \geq|\Omega| .$
\end{theo}
\section{Лекция 11 (03.01.2021)}

Исчисление предикатов сигнатуры $\sum$ задаётся след. аксиомами и правилами вывода.
\vskip 0.1in
Аксиомы:\\
A1. Подстановочные примеры тавтологий,\\
A2. $\forall x A[a / x] \rightarrow A[a / t]$\\
A3. $A[a / t] \rightarrow \exists x A[a / x]$
\vskip 0.2in

Подстановочным примером тавтологии $A$ мы называем результат замены всех пропозициональных переменных $A$ на некоторые формулы сигнатуры $\sum$.
\vskip 0.1in
Пример:\\
$B \vee \neg B,$ где $B-$ любая формула. В А2 и АЗ $A$ - любая формула сигнатуры $\sum$ и $t-$ любой терм $(\times$ не входит в $A)$.
\vskip 0.2in

Правила вывода:\\
R1. $\frac{A \quad A \rightarrow B}{B}$ (modus ponens)\\
R2. $\frac{A \rightarrow B}{A \rightarrow \forall x B[a / x]}$\\
R3. $\frac{B \rightarrow A}{\exists x B[a / x] \rightarrow A}$
\vskip 0.1in
Здесь а не входит в $A$(и $x$ не входит в $B$).\\
Правила R2 и R3 называются правилами Бернайса.
\vskip 0.2in

\begin{defn}
Выводом в исчислении предикатов называется конечная последовательность формул, каждая из которых либо является аксиомой, либо получается из предыдущих формул по одному из правил вывода $R_1, R_2, R_3$
\end{defn}

Пример
$\forall x A[A / x] \to A\qquad (A2)$\\
$\forall x A[A / x] \to \forall y A[a / y]\qquad (R2)$
\vskip 0.2in

\begin{defn}
Формула $A$ называется выводимой в исчислении предикатов или теоремой исчисления предиеатов если существует вывод, в котором последняя формула есть $A$
\end{defn}

\begin{defn}
Вывод в теории $T$ называется конечная последовательность формул, каждая из которых либо приндлежит множеству $T$, либо является логической аксиомой вида $A_1, A_2, A_3$, либо получается из предыдущих формул по одному из правил $R_1, R_2, R_3$
\end{defn}

\begin{defn}
Формула $A$ называется выводимой (доказуемой) в теории $T$ или теоремой $T$ (обозначение $T \vdash A),$ если существует вывод в $T$, в котором последняя формула есть $A$.
\end{defn}
\begin{defn}
Формула $\mid A$ опровержима в $T$, если $T \vdash \neg А$.
\end{defn}
\begin{defn}
Формула $A$ независима от $T$, если $T \not \models A$ и $T \not \vdash A$.
\end{defn}
\vskip 0.1in
Если $T \subseteq U$ и $T \vdash A$, то $U \vdash A$ (монотонность)\\
Если $T \vdash A$, то существует такое конечное множество $T_0 \subseteq T$, что $T_0 \vdash A$ (компактность)\\
Если $T \vdash A$ и для каждой аксиомы $B \in T$ имеет место $U \vdash B$, то $U \vdash A$ (транзитивность)
\vskip 0.2in

\begin{defn}
Теорией сигнатуры $\sum$ называем произвольно множество $T$ замкнутых формул языка $\zeta_{\sum}$. Теорию $T \cup \{A\}$ обозначают также $T,A$ или $T + A$
\end{defn}

\begin{theo}
Для любой теории $T$ и любой замкнутой формулы $A$
$$
T, A \vdash B \Leftrightarrow T \vdash A \to B
$$
\end{theo}

\begin{proof}
Индукция по длине вывода $T, A \vdash B$\\
Если $B$ является логической аксиомой или $B \in T$, то в $T$ выводимо:
$$
\begin{array}{ll}
	B\\
	B \to (A \to B)& \text{(тавтология)}\\
	A \to B& \text{(MP)} 
\end{array}
$$
Если $A = B$, то используем $A \to A$
\vskip 0.2in

Пусть $B$ получена из $C$ и $C \rightarrow B$ по modus ponens.
Имеем $T \vdash(A \rightarrow C)$ и $_{\chi} T \vdash(A \rightarrow(C \rightarrow B))$ по предположению
индукции.
Соединяем эти два вывода и достраиваем так:
$$
\begin{array}{ll}
(A \rightarrow(C \rightarrow B)) \rightarrow((A \rightarrow C) \rightarrow(A \rightarrow B)) & (\text { тавтология }) \\
(A \rightarrow C) \rightarrow(A \rightarrow B) & (\mathrm{MP}) \\
A \rightarrow B & (\mathrm{MP})
\end{array}
$$
\vskip 0.1in

Допустим $B=(C \rightarrow \forall \times D[a / x])$ получена из $C \rightarrow D$ по $R 2 .$ По пр. индукции
$$
T \vdash A \rightarrow(C \rightarrow D)
$$
Надо построить вывод
$$
T \vdash A \rightarrow(C \rightarrow \forall x D[a / x])
$$
\vskip 0.1in
Достраиваем вывод $A \rightarrow(C \rightarrow D)$ в $T:$
$$
\begin{array}{ll}
A \rightarrow(C \rightarrow D) \\
(A \rightarrow(C \rightarrow I D)) \rightarrow(A \wedge C \rightarrow D) & (\text { тавтология }) \\
(A \wedge C) \rightarrow D & (\mathrm{MP}) \\
(A \wedge C) \rightarrow \forall \times D[a / x] & (\mathrm{R} 2, A \text { замкнута })
\end{array}
$$
$$
A \rightarrow(C \rightarrow \forall \times D[a / x])
$$
(аналогично)\\
Правило $R3$ рассматривается аналогично.
\end{proof}

\begin{defn}
Теория $T$ противоречива, если существует $A$ такая, что $T \vdash A$ и $T \vdash \neg A$. В противном случае теория $T$ называется непротиворечивой.
\end{defn}

\begin{corol}
$T \cup \{A\}$ противоречива $\Leftrightarrow T \vdash \neg A$.
\end{corol}
\vskip 0.2in

\begin{theo}[О корректности]
Если $M \models T$ и $T \vdash A$, то $M \models A$
\end{theo}
\begin{proof}
Индукция по длине вывода $A$ в $T$
\end{proof}

\begin{corol}
Если $\vdash A$, то $A$ общезначима.
\end{corol}
\vskip 0.2in

\begin{corol}
Если теория $T$ имеет модель, то $T$ непротиворечива
\end{corol}

\begin{corol}
Следующие теории непротиворечивы:\\
Исчесление предикатов(пустая теория)\\
Теория групп\\
Элементарная геометрия\\
Формальная арифметика
\end{corol}

\begin{corol}Если существует модель $M$ теории $T$ для которой $M \not \models A,$ то $T \not \models A$
\end{corol}
Пример\\
Модель Пуанкаре $\mathrm{H}^{2}$ показывает, что аксиома Евклида независима от остальных аксиом элементарной геометрии.

\begin{theo}[Геделя о полноте]
(1) Всякая непротиворечивая теория Т выполнима, то есть имеет модель $M \vDash T$.\\
(2) Если $T \not$ А, то найдётся модель $M \vDash T$ для которой $M \not \models A$
\end{theo}

Покажем равносильность этих утверждений.\\
$(1 \Rightarrow 2):$ Если $T \not \models A,$ то $T \cup\{\neg A\}$ непротиворечива. Действительно, если $T, \neg A$ противоречива, то $T \vdash \neg \neg A,$ а Значит $T \vdash A($ используем тавтологию $\neg \neg A \rightarrow A) .$ Следовательно, $T \cup\{\neg A\}$ имеет модель $M .$\\
$(2 \Rightarrow 1):$ Пусть $T$ непротиворечива. Возьмём $A=(B \wedge \neg B) .$ Тогда $T \not \models$ А, следовательно у теории $T$ дблжна быть модель (опровергающая $A$ ).
\vskip 0.3in

\begin{theo}[Геделя-Мальцева о компактности]
Теория Т выполнима $\Longleftrightarrow$ любое конечное подмножество $T_{0} \subseteq T$ выполнимо.
\end{theo}
\begin{proof}
Если $T$ невыполнима, то существует вывод противоречия в $T$, использующий лишь конечное число аксиом $T$.
\end{proof}

Пример.\\
Пусть $(\mathbb{N} ;=, S,+, \cdot, 0)$ - стандартная модель арифметики и Th $(\mathbb{N})$ есть множество всех истинных в $\mathbb{N}$ предложеній.
Добавим к сигнатуре новую константу с и рассмотрим теорию
$$
T \rightleftharpoons T h(\mathbb{N}) \cup\{\neg c=0, \neg c=S 0, \neg c=S S 0, \ldots\}
$$

Терм $\bar{n} \rightleftharpoons S S \ldots S 0(n$ раз $)$ называем нумералом. Нумераль служат именами натуральных чисел.
\begin{lem}
Каждая конечная подтеория $T_{0} \subseteq T$ выполнима.
Доказательство. $T_{0}$ содержит лишь конечное число аксиом вида $c \neq \bar{n}_{1}, \ldots,$ $c \neq \bar{n}_{k} .$ Интерпретируем константу с в стандартной модели как любое число $m>n_{1}, \ldots, n_{k} .$
\end{lem}

По теореме о компактности существует (нормальная) модель $M \vDash$ Т. Модель $M$ обладает следующими свойствами:\\
- $\mathbb{N}$ изоморфна начальному сегменту $M$; вложение $\mathbb{N} \rightarrow M$ задаётся функцией $\varphi: n \longmapsto \bar{n}_{M}$.\\
- $M \vDash T h(\mathbb{N})$\\
- $M \not \models \mathbb{N},$ в частности $c_{M} \in M$ есть «бесконечно большое число», поскольку $c_{M}$ отлично от всякого $n \in \mathbb{N}$.
\vskip 0.2in

Следовательно, те же аксиомы выполнены и в М. Поэтому предикат $<_{M}$ на $M$ представляет собой строгий линейный порядок с наименьшим элементом 0. При этом каждый элемент имеет последователя, и каждый элемент, кроме 0,
имеет непосредственного предшественника.

\begin{proof}
Если $G_{1}<G_{2},$ возьмём чётные $x_{1} \in G_{1}$ и $x_{2} \in G_{2}$ и рассмотрим $y=\left(x_{1}+x_{2}\right) / 2($ функция $g(x)=x / 2$ определима в $\mathbb{N},$ а значит и в $M$ ).
Если $y \in G_{1},$ то $\left(x_{1}+x_{2}\right) / 2=x_{1}+\bar{n}$ для некоторого $n \in \mathbb{N}$ Тогда $2 x_{1}+2 \bar{n}=x_{1}+x_{2},$ откуда $x_{1}+2 \bar{n}=x_{2},$ то есть $x_{2} \in G_{1}$
Аналогично показываем $y \notin G_{2}$.
\end{proof}
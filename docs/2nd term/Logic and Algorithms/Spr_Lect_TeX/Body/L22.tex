\section{Лекция 22 (05.11.2021)}

- Процесс применения алгоритма $\mathcal{A}_{\mathrm{K}}$ данным $x \in X$ происходит по шагам.\\
- Процесс или заканчивается после конечного числа шагов с результатом $y \in Y$, или останавливается без результата или продолжается бесконечно.\\
- Таким образом, с алгоритмом $\mathcal{A}$ связывается частичная функция $f: X \rightarrow Y$\\
Мы будем говорить: Алгоритм $\mathcal{A}$ вычисляет функщию $f$.\\
\begin{defn}
Частичной функцией $f: X \rightarrow Y$ называется подмножество $f \subseteq X \times Y$ такое, что из $\left\langle x, y_{1}\right\rangle \in f$ и $\left\langle x, y_{2}\right\rangle \in f$ следует $y_{1}=y_{2}$.
\end{defn}
Пишем $f(x)=y$ вместо $\langle x, y\rangle \in f ;$\\
$! f(x)$ вместо $\exists y f(x)=y$
\begin{defn}
Областью определения частичной функции $f$ называется множество $\operatorname{dom}(f):=\{x \in X: \exists y \in Y\langle x, y\rangle \in f\}$
\end{defn}
\begin{defn}
Областью значений частичной функшии $f$ называется множество $\operatorname{rng}(f):=\{y \in Y: \exists x \in X\langle x, y\rangle \in f\} .$
\end{defn}
\begin{defn}
Частичная функция $f: X \rightarrow Y$ вычислима, если она вычисляется некоторым алгоритмом.
\end{defn}
B частности, можно говорить о вычислимых функциях $f: \Sigma^{*} \rightarrow \Sigma^{*}, f: \mathbb{N}^{k} \rightarrow \mathbb{N}$
и т.д.
\begin{theo}
Каждая из вышеперечисленных моделей определяет один и тот же к.ласс вычислимых частичных функций $f: \Sigma^{*} \rightarrow \Sigma^{*} .$
\end{theo}
Такие модели (языки программирования) называются полными по Тьюрингу.\\
\\
\begin{theo}[Тезис Черча-Тьюринга]
Любая вычислимая в интуитивном смыс.ле частичная функция $f: \Sigma^{*} \rightarrow \Sigma^{*}$ вычислима на машине Тьюринга.
\end{theo}
Замечание\\
Это утверждение не является математическим, но говорит об адекватности математической модели (вычислимости по Тьюрингу) реальному явлению (вычислимости).\\
\\
\begin{theo}
Всякая функция $f: \Sigma^{*} \rightarrow \Sigma^{*}$, вычислимая на (идеализированном) физически реализуемом устройстве, вычислима на машине Тьюринга.
\end{theo}
Замечание\\
Физический тезис предполагает возможность ана.логового вычисления, квантово-механические эффекты и т.д.
\vskip 0.2in
Машина Тьюринга задаётся конечными\\
- рабочим алфавитом $\Sigma$, содержатим символ $\#$ (пробел);\\
- множеством состояний $Q$, содержашим состояния $q_{1}$ (начальное) и $q_{0}$ (конечное);\\
- набором команд (программой).$P$
\vskip 0.2in
- Команды имеют вид $q a \rightarrow r b \nu .$ где $q, r \in Q, a, b \in \Sigma$ и $\nu \in\{L, N, R\}$. «прочтя символ $a$ в состоянии $q$ перейти в состояние $r$, заменить содержимое ячейки на $b$ и сместиться влево (L), остаться на месте (N) и.ли сместиться вправо
$(\mathrm{R})$ на одну ячейку, в зависимости от значения $\nu »$
- Требуется, чтобы в программе $P$ была ровно одна команда с левой частью $qa$ для каждого $q \in Q \slash \{q_0\}$ и $a \in \Sigma$

\begin{defn}
Машина Тьюринга есть набор $M = \langle Q, \Sigma, P ,q_0, q_1 \rangle$.
\end{defn}

\begin{defn}
Конфигураџия машины $M$ определяется содержимым ленты, состоянием и положением головки. Конфигурация записывается словом вида $X q a Y$, где\\
- $X a Y \in \Sigma^{*}$ есть содержимое ленты,\\
- $q \in Q$ есть состояние $M$.\\
- головка обозревает символ $a .$
\end{defn}

\begin{defn}
$M$ вычисляет частичную функцию $f: \Delta^{*} \rightarrow \Delta^{*}$, если для каждого $x \in \Delta^{*}$\\
- если $x \in \operatorname{dom}(f)$, то начав работу в конфигурации $q_{1} \# x_{3}$ машина $M$ останавливается в конфигуращии $q_{0} \# f(x) ;$\\
- если $x \notin d o m(f)$, то машина $M$ не останавливается.
\end{defn}
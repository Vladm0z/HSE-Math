\section{Лекция 4 (01.25.2021)}

\begin{theo}[Кантор]
Для любых двух вполне упорядоченных множеств одно изоморфно начальному отрезку другого
\end{theo}

\begin{proof}
Возьмем два вполне упорядоченных множества $A, B$. Рассмотрим $R = \{(x,y) \in A \times B\ |\ [0,x)_A \cong [0,y)_B\}$.\\
Проверим инъективность соответствия $R$, если $(x_1, y), (x_2, y) \in R$ то $[0,x_1)_A \cong [0,y)_B \cong [0,x_2)$. Так как ни одно из множеств не может являтся собственным начальным отрезком другого, то $x_1 \not<_A x_2,\ x_2 \not< x_1$, откуда $x_1 = x_2$\\
Аналогично проверяется функциональность соответсвия $R$
\vskip 0.1in
Множество $\operatorname{dom} R$ является начальным отрезком $A$. Действительно, если $x <_A x'$ и $x'\in \operatorname{dom} R$, то $[0, x)_A \subset [0, x')_A$ и существует изоморфизм $g:[0,x')_A \to [0,y')_B$. Ограничение изоморфизма $g$ на $[0, x)_A$ -- $[0,x)_A \cong [0,g(x))_B$, тогда $(x,g(x)) \in R$ и $x \in \operatorname{dom} R$
\vskip 0.1in
Соответсвие $R$ сохраняет порядок, так как если $x <_A x',\ (x,y) \in R,\ (x',y') \in R,\ y <_B y'$, иначе $[0,x')_A$ изоморфен своему начальному отрезку
нало
\vskip 0.1in
Аналогично $\operatorname{ran} R = \{y \in B\ |\ \exists x \in A(x,y) \in R\}$ является начальным отрезком в $B$, и отношение $R^{-1}$ сохраняет порядок
\vskip 0.1in
Получается что $R$ -- изоморфизм из $\operatorname{dom} R$ в $\operatorname{ran} R$, проверим что $\operatorname{dom} R = A,\ \operatorname{ran} R = B$\\
Если это не так, рассмотрим $x_0 = \min(A \backslash \operatorname{dom} R)$ и $y_0 = \min(B \backslash \operatorname{ran} R)$, тогда $\operatorname{dom} R = [0,x_0)_A,\ \operatorname{ran} R = [0, y_0)_B$ и $(x_0, y_0) \in R$ -- противоречие
\end{proof}

Идея: трансфинитно продолжим ряд натуральных чисел так, чтобы всякий член ряда был равен множеству предшествующих членов ряда
\vskip 0.1in
Обозначение: $x + 1 := x \cap \{x\}$

\begin{defn}
Множество $T$ называется транзитивным, если $\bigcup T \subset T$, или экваивалентно $\forall x,y\ (x \in y \in T \to x \in T)$\\
Ординал -- транзитивное множество, все элементы которого также транзитивны
\end{defn}

\begin{lem}
Всякий элемент ориданала -- ординал
\end{lem}

\begin{proof}
Пусть $\alpha$ -- ординал, то есть транзитивное множество, каждый элемент которого транзитивен\\
Тогда $\beta \subset \alpha$ в силу транзитивности, тогда каждый элемент $\beta$ транзитивен и сам тоже транзитивен, а следовательно он ординал
\end{proof}

\begin{lem}
Для любых ординалов $\alpha, \beta, \gamma$ имеем $\alpha \notin \alpha,\ \alpha \in \gamma$ если $\alpha \in \beta,\ \beta \in \gamma$
\end{lem}
\begin{proof}
Предположим что $\alpha \in \alpha$ и рассмотрим $\{\alpha\}$. В нем по аксиоме регулярности должен найтись элемент без $\alpha$, но такого нет
\vskip 0.1in
Рассмотрим ординалы $\alpha, \beta, \gamma$, тогда $\alpha \in \beta \in \gamma$ и по транзитивности $\alpha \in \gamma$
\end{proof}

\begin{lem}
всякое непустое множество ординалов $X$ содержит $\in$-минимальный элемент
\end{lem}
\begin{proof}
Через аксиому регулярности
\end{proof}

\begin{lem}
Для любых ординалов $\alpha, \beta$ верно, что $\alpha \in \beta$, или $\alpha = \beta$, или $\beta \in \alpha$
\end{lem}
\begin{proof}
Допустим, что это не так, то есть существует ординал $\alpha$, который несравним с некоторым ординалом.\\
Рассмотрим ординал $\alpha + 1 = \alpha \cup \{\alpha\}$ и его подмножество $X$, состоящее из тех элементов, которые несравнимы с некоторым ординалом. По предыдущей лемме множество $X$ содержит минимальный $\alpha_0$ элемент
\vskip 0.1in
Пусть $\beta$ -- некоторый ординал, с которым несравним $\alpha_0$, рассмотрим $\beta + 1 = \beta \cup \{\beta\}$ и его подмножество $Y$, состоящее из тех элементов, которые насравнимы с ординалом $\alpha_0$. По лемме $Y$ содержит минимальный $\beta_0$, проверим что $\alpha_0 = \beta_0$
\vskip 0.1in
Установим вкоючение $\alpha_0$ в $\beta_0$, если $\gamma \in \alpha_0,\ \gamma \in \alpha + 1$ и $\gamma \notin X$, тогда так как $\alpha_0$ минимальный, то $\gamma$ сравним со всеми ординалами, а тогда $\gamma \in \beta_0$, так как иначе $\beta_0 \in \alpha_0$, и это противоречие
\vskip 0.1in
Установим включение $\beta_0$ в $\alpha_0$. Если $\delta \in \beta_0$, то $\delta \in \beta + 1$, тогда $\delta$ сравним с $\alpha_0$ и $\delta \in \alpha_0$, так как иначе будет противоречие с несравнимостью $\alpha_0$ и $\beta_0$
\end{proof}

Обозначение $x < y :\Leftrightarrow x \in y$

\begin{theo}
Класс всех ординалов линейно упорядочен через $<$, а также всякое непустое множество содержит $<$-наименьший элемент
\end{theo}

\begin{corol}
Любой ординал $\alpha$ сам как множество вполне упорядочен с помощью $<$ и является начальным отрезком в классе всех ординалов
\end{corol}

\begin{theo}[Трансфинитная индукция]
Пусть $\varphi$ -- некоторое свойство множеств. Допустим что для всякого ординала $\alpha$ имеет место $\forall \beta < \alpha\ \varphi(\beta) \to \varphi(\alpha)$. Тогда для всех ординалов $\gamma$ верно $\varphi(\gamma)$
\end{theo}
\begin{proof}
Допустим, что $\varphi(\gamma)$ не выполнено для некоторого ординала $\gamma$. Рассмотрим подмножество $X$ множества $\gamma + 1 = \gamma \cup \{\gamma\}$, состоящее из ординалов, которые не удовлетворяют свойству $\varphi$. Поскольку множество $X$ непусто, оно содержит $<$-минимальный элемент $\alpha$. Получаем, что $\varphi(\alpha)$ верно, поскольку $\varphi(\beta)$ верно для любого $\beta < \alpha$, противоречие
\end{proof}

\begin{theo}[парадокс Бурали-Форти 1897]
Класс всех ординалов не является множеством
\end{theo}
\begin{proof}
Допустим, что существует $O$, которое в точности содержит все ординалы\\
Тогда $O$ является транзитивным множеством транзитивных множетсв, то есть ординалом\\
Следовательно множество $O \in O$, что противоречит иррефлексивности $\in$.
\end{proof}

Утверждение\\
Каждое натуральное число и все множество $\mathbb{N}$ -- ординалы
\vskip 0.2in
Схема аксиом подстановки\\
Пусть свойство $\varphi(x,y)$ -- такое, что для любого множества $x$ найдется не более одного множества $y$, для которого $\varphi(x,y)$. Тогда для любого $X$ найдется множество $Y = \{u\ |\ \exists x \in X\ \varphi(x,y)\}$
\vskip 0.1in

\begin{theo}[Кантор]
Пусть $(X, <)$ -- вполне упорядоченное множество. Тогда существуе единственный ординал $\alpha$, изоморфный множеству $(X, <)$
\end{theo}
\begin{proof}
Рассмотрим свойство $\varphi(x,y):\ x \in X,\ y$ -- ординал, и $[0,x)_X \cong y$.\\
Видим, что для любого множества $x$ найдется не более одного множества $y$, для которого имеет место $\varphi(x,y)$.\\
По аксиоме подстановки найдется множество $Y = \{y\ |\ \exists x \in X\ \varphi(x,y)\}$, содержащее те и только те ординалы, которые изоморфны собственным начальным отрезкам $(X, <)$.
\vskip 0.1in
Поскольку не существует множества всех ординалов, то найдется ординал $\alpha$, не лежащий в $Y$
\vskip 0.1in
По теореме Кантора о сравнении вполне упорядоченных множеств, множество $(X, <)$ изоморфно некоторому начальному отрезку $\alpha$. поскольку $\alpha$ и все его собственные начальные отрезки являются ординалами, получаем, что $(X, <)$ изоморфно ординалу
\vskip 0.1in
Единственность следует из того, что для двух различных ординалов, один является собственным начальным отрезком другого. Следовательно разные ординалы неизоморфны как вполне упорядоченные множества.
\end{proof}

\begin{defn}
Ординал $\alpha$ называется порядковым типом вполне упорядоченного множества $(X, <)$, если он изоморфен $(X, <)$.
\end{defn}

\begin{lem}
Всякий ординал $\alpha$ либо имеет вид $\beta + 1$ для некоторого ординала $\beta$, либо равен объединению всех предшествующих ординалов $\bigcup \alpha$
\end{lem}

Замечание: ординалы вида $\beta + 1$ называются ординалами-последвателями; се остальные ординалы, кроме 0, называются предельными.
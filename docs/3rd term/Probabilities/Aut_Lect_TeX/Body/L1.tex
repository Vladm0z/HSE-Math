\section{Lecture 1 (09.07.2021)}
\vskip 0.1in
\href{https://sites.google.com/view/terver2122}{site}
\vskip 0.1in
Books:\\
\href{https://www.springer.com/gp/book/9781447153610}{Achim Klenke - Probability Theory, A Comprehensive course}
\begin{gather*}
	\text{Test 1} \cdot 0.25 + \text{Colloquium} \cdot 0.15 + \text{Test 2} \cdot 0.25 + \text{Exam(block)} \cdot 0.35 + \text{Bonus} \cdot 0.1
\end{gather*}
\vskip 0.2in
\begin{exmp}
	24 pairs of 2d6, find probability of (6,6)\\
	\begin{gather*}
		1 - \left(\frac{35}{36}\right)^{24} \sim 0.49 < \frac{1}{2}
	\end{gather*}
\end{exmp}

\href{https://terrytao.wordpress.com/2010/01/01/254a-notes-0-a-review-of-probability-theory/}{T.Tao}:\\
Probability theory is only “allowed” to study concepts and perform operations which are preserved with respect to extension of the underlying sample space. (This is analogous to how differential geometry is only “allowed” to study concepts and perform operations that are preserved with respect to coordinate change, or how graph theory is only “allowed” to study concepts and perform operations that are preserved with respect to relabeling of the vertices, etc.)
\vskip 0.2in

\begin{defn}[Naive prob space (we'll never use this definition)]
	A naive prob space is a couple $(\Omega, \mathbb{P})$ where $\Omega$ is a nonempty set and $\mathbb{P}: 2^{\Omega}\to [0,1]$ is a map from the power ser $\Omega$ to $[0,1]$ such that
	\begin{gather*}
		\mathbb{P}(\Omega) = 1\\
		\mathbb{P}(A \cup B) = \mathbb{P}(A) + \mathbb{P}(B) \text{, for all } A,B \subset \Omega \text{ such that } A \cap B = \emptyset
	\end{gather*}
\end{defn}

$\bullet$ On the one hand, in math we want often to discuss limits and continuous object, so we would like to replace the second condiition with a countable union of set
\vskip 0.1in
$\bullet$ However, as you maybe seen in measure theory, in this case the requirement that $\mathbb{P}$ can be evaluated on any subset of $\Omega$ is very strong, and not useful. See the \href{https://bit.ly/2Z3L8px}{Banach-Tarski paradox}.

\begin{defn}
	If $\Omega$ is a non-empty set, a $\sigma$-algebra on $\Omega$ is a subset $\mathcal{F}$ of the power set of $\Omega$ such that
	\begin{enumerate}
	\item[(a)] $\emptyset \in \mathcal{F}$
	\item[(b)] If $A \in \mathcal{F}$ its complement is in $\mathcal{F}$, namely $A \in \mathcal{F}$ implies $A^c \in \mathcal{F}$
	\item[(c)] If $(A_i)_{i \geqslant 0}$ is a countable family of sets in $\mathcal{F}$, then $\bigcup_{i \geqslant 0} A_{i} \in \mathcal{F}$
	\end{enumerate}
\end{defn}

\begin{defn}[Measurable space]
	A measurable space is a couple $(\Omega, \mathcal{F})$
\end{defn}

\begin{defn}
	A probability space is a triple $(\Omega, \mathcal{F}, \mathbb{P})$ where $(\Omega, \mathcal{F})$ is measurable space, and $\mathbb{P}: \mathcal{F} \to [0,1]$ is such that
	\begin{enumerate}
	\item $\mathbb{P}(\Omega) = 1$
	\item If $(A_i)_{i \geqslant 0}$ is a countable family of events, that are pairwise disjoint, that is $A_i \cup A_j = \emptyset$ for $i \ne j$ then \begin{gather*}
	\mathbb{P} (\cup_{i \geqslant 0} A_i) = \sum\limits_{i \geqslant 0} \mathbb{P}(A_i)
	\end{gather*}
	\end{enumerate}
\end{defn}

\begin{exmp}
	If we want to model the throw of a die, we can choose $\Omega = \{1,2,3,4,5,6\},\ \mathcal{F} = 2^{\Omega}$, and $\mathbb{P}$ to be defined as $\mathbb{P}(A) = |A|/6$
\end{exmp}

\begin{exmp}
	We want to model the flip of two coins, each of them giving either H or T. Then we can take $\Omega = \{H, T\} \times \{H, T\}$ and again $\mathbb{P}(A) = |A|/4$. However if we are only interested inthe number of heads, we can also simply take $\Omega = \{0,1,2\}$ and $\mathbb{P}(A) = \sum_{i\in A} p_i$, where $p_i$ is defined by $p_0 = 1/4,\ p_1 = 1/2,\ p_2 = 1/4$
\end{exmp}


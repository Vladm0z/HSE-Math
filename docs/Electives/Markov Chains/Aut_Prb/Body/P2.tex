\section{Лист 2}
    \begin{prob}
        Докажите, что следующие утверждения эквивалентны:
        \begin{itemize}
        \item[(а)] квадратная $n \times n$ матрица $A$ стохастическая
        \item[(б)] $Af \geqslant 0$ для всех неотрицательных векторов-столбцов.\\
        $A1 = 1$, где $1 = (1, \ldots, 1)^{t}$, а $t$ обозначает транспонирование
        \item[(в)] Если $A$ вектор-строка $\mu$ распределение, то $\mu A$ тоже распределение
        \end{itemize}
    \end{prob}
    \begin{proof}
        \begin{itemize}
        \item[(а $\Rightarrow$ б)] очевидно
        \item[(а $\Rightarrow$ в)]
            \begin{gather*}
                \mu = (x_1, \ldots, x_k)\quad \sum x_i = 1\\
                \mu A = (x_1, \ldots, x_k)
                \begin{pmatrix}
                    a_{11} & \ldots & a_{k1}\\
                    \vdots & \ddots &\\
                    a_{1k} & & a_{kk}
                \end{pmatrix}
                = \left(
                    \sum\limits_{i = 1}^{k} x_i a_{1i},
                    \sum\limits_{i = 1}^{k} x_i a_{2i},
                    \ldots,
                    \sum\limits_{i = 1}^{k} x_i a_{ki}
                \right)\\
                \sum\limits_{i = 1}^{k} x_i a_{1i}
                + \sum\limits_{i = 1}^{k} x_i a_{2i}
                + \ldots
                + \sum\limits_{i = 1}^{k} x_i a_{ki}
                = x_1 (a_{11} + a_{21} + \ldots + a_{k1})
                + x_2 (a_{12} + a_{22} + \ldots + a_{k2})
                + \ldots
                + x_k (a_{1k} + a_{2k} + \ldots + a_{kk})
                = \sum x_i
                = 1
            \end{gather*}
        \item[(в $\Rightarrow$ а)]
            Пусть $\alpha_i = \sum\limits_{j=1}^{k} a_{ji}$, тогда
            \begin{gather*}
            \begin{cases}
                \sum\limits_{i} a_i x_i = 1\\
                \sum\limits_{i} x_i = 1
            \end{cases}
            \end{gather*}
            Пусть $\mu_1 = (1, 0, \ldots, 0)$, по условию $\mu_1 A$ тоже распределение, откуда $\sum \alpha_i x_i = \alpha_1 \cdot 1 = 1$ и $\alpha_1 = a_{11} + a_{21} + \ldots + a_{k1} = 1$. Аналогично $\mu_2 = (0, 1, \ldots, 0), \ldots, \mu_k = (0, \ldots, 0, 1)$, то есть $\alpha_1 = \alpha_2 = \ldots = \alpha_{k} = 1$ и матрица $A$ -- стохастическая 
        \item[(б $\Rightarrow$ в)]
            \begin{gather*}
                Af \geqslant 0, f \geqslant 0\\
                A
                \begin{pmatrix}
                    1 \\ \vdots \\ 1
                \end{pmatrix}
                =
                \begin{pmatrix}
                    1 \\ \vdots \\ 1
                \end{pmatrix}
                \\
                \begin{pmatrix}
                    a_{11} & \ldots & a_{k1}\\
                    \vdots & \ddots &\\
                    a_{1k} & & a_{kk}
                \end{pmatrix}
                \begin{pmatrix}
                    1 \\ \vdots \\ 1
                \end{pmatrix}
                =
                \begin{pmatrix}
                    \sum a_{i1} \\ \vdots \\ \sum a_{ik}
                \end{pmatrix}
                =
                \begin{pmatrix}
                    1 \\ \vdots \\ 1
                \end{pmatrix}                
            \end{gather*}
            Откуда $\sum a_{i1} = \ldots = \sum a_{ik} = 1$
        \end{itemize}
    \end{proof}
\vskip 0.6in


    \begin{prob}
        Докажите, что произведение стохастических матриц одинакового размера также является стохастической матрицей
    \end{prob}
    \begin{proof}
        Пусть
        \begin{gather*}
            \sum\limits_{i=1}^{n} A_{ji}
            = 1
            = \sum\limits_{i=1}^{n} B_{ji}
        \end{gather*}
        Тогда для строк
        \begin{gather*}
        \sum\limits_{i=1}^{n} (AB)_{ji}
        = \sum\limits_{i=1}^{n} \left( \sum\limits_{k=1}^{n} A_{jk} B_{kj} \right)
        = \sum\limits_{k=1}^{n} \left( A_{jk} \left( \sum\limits_{i=1}^{n} B_{ki} \right)\right)
        = \sum\limits_{k=1}^{n} A_{jk}
        = 1
        \end{gather*}
        Аналогично для столбцов
    \end{proof}
\vskip 0.6in


    \begin{prob}
    \end{prob}
    \begin{proof}
    \end{proof}
\vskip 0.6in


    \begin{prob}
    \end{prob}
    \begin{proof}
    \begin{itemize}
    \item[(а)]
        \begin{gather*}
            p^{(2)}
            \begin{pmatrix}
                \frac{1}{3} & \frac{1}{6} & \frac{1}{2}
            \end{pmatrix}
            \begin{pmatrix}
                0 & \frac{3}{4} & \frac{1}{4}\\
                \frac{2}{3} & 0 & \frac{1}{3}\\
                1 & 0 & 0
            \end{pmatrix}^2
            =
            \begin{pmatrix}
                \frac{1}{3} & \frac{1}{6} & \frac{1}{2}
            \end{pmatrix}
            \begin{pmatrix}
                \frac{3}{4} & 0 & \frac{1}{4}\\
                \frac{2}{3} & \frac{1}{2} & \frac{1}{6}\\
                0 & \frac{3}{4} & \frac{1}{4}
            \end{pmatrix}
            = 
            \begin{pmatrix}
                \frac{11}{36} & \frac{11}{24} & \frac{17}{72}
            \end{pmatrix}
        \end{gather*}
    \item[(б)]
        \begin{gather*}
            p^{(3)} =
            \begin{pmatrix}
                \frac{11}{36} & \frac{11}{24} & \frac{17}{72}
            \end{pmatrix}
            \begin{pmatrix}
                0 & \frac{3}{4} & \frac{1}{4}\\
                \frac{2}{3} & 0 & \frac{1}{3}\\
                1 & 0 & 0
            \end{pmatrix}
            =
            \begin{pmatrix}
                \frac{13}{24} & \frac{11}{48} & \frac{11}{48}
            \end{pmatrix}\\
            p_{2}^{(3)} = \frac{13}{48}\\
            p^{(1)} =
            \begin{pmatrix}
                \frac{1}{3} & \frac{1}{6} & \frac{1}{2}
            \end{pmatrix}
            \begin{pmatrix}
                0 & \frac{3}{4} & \frac{1}{4}\\
                \frac{2}{3} & 0 & \frac{1}{3}\\
                1 & 0 & 0
            \end{pmatrix}
            =
            \begin{pmatrix}
                \frac{11}{18} & \frac{1}{4} & \frac{5}{36}
            \end{pmatrix}\\
            \mathbb{P}(\xi_1 = 3, \xi_3 = 2)
            = p_3^{(1)} p_2^{(3)}
            = \frac{13}{48} \cdot \frac{5}{36}
            = \frac{5 \cdot 13}{2^6 \cdot 3^3}
        \end{gather*}
    \end{itemize}
    \end{proof}
\vskip 0.6in


    \begin{prob}
    \end{prob}
    \begin{proof}
        Заметим, что вероятность перейти из состояния $i$ в $i+1$ равна $\frac{m-i}{m}$ и вероятность остаться в том же состоянии $\frac{i}{m}$, в другие состояния из состояния $i$ он перейти не может.
        \begin{gather*}
            \mathbb{E}[r_m]
            = \mathbb{E}[\{\min n: \varepsilon_n = m\}]\\
            = \sum\limits_{i = 1}^{m} i \cdot \left(\frac{m-1}{m} \cdot \frac{m-2}{m} \cdot \ldots \cdot \frac{m-i+1}{m}\right)\\
            = 1 + 2 \cdot \frac{m-1}{m} + 3 \cdot \frac{m-1}{m} \cdot \frac{m-2}{m} + \ldots\\
            = 1 + \frac{2 (m-1)!}{m (m-2)!} + \frac{3(m-1)!}{m^2 (m-3)!} + \ldots + \frac{m (m-1)!}{m^{i-1} (m-i)!}\\
            = (m-1)! \sum\limits_{j=1}^{i} \frac{j}{m^{j-1} (m-j)!}
        \end{gather*}
    \end{proof}

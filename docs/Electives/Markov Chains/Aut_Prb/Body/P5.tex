\section{Лист 5}
    \begin{prob}
    \end{prob}
    \begin{proof}
    (a) - нет, (b) - да, в 5 степени, (с) - нет, (d) - нет
    \end{proof}
\vskip 0.6in


 
    \begin{prob}
        Верно ли, что всякая марковская цепь с конечным числом состояний, имеющая единственное стационарное состояние, эргодична? Если да - докажите, если нет - приведите контрпример.
    \end{prob}
    \begin{proof}
        Цепь эргодична: $\exists ! \pi,\ p^{(n)} \to \pi,\quad p^{(n)} = p^{(0)} \Pi^{n}$\\
        $\exists ! \pi$ - стационарное состояние: $\pi \Pi = \pi$
        \vskip 2in
        \begin{gather*}
            A = 
            \begin{pmatrix}
                0 & 1 & 0\\
                0 & 0 & 1\\
                1 & 0 & 0
            \end{pmatrix}\\
            y_{A} = t^3 - 1
            = (t-1)(t^2+t+1)\\
            \left(\frac{1}{3}, \frac{1}{3}, \frac{1}{3}\right)
            \to \left(\frac{1}{3}, \frac{1}{3}, \frac{1}{3}\right)
        \end{gather*}
        $\Rightarrow \exists !$ стационарное состояние\\
        но $p^{(n)}$ не сходится, а следовательно цепь не эргодична 
    \end{proof}
\vskip 0.6in


 
    \begin{prob}
        Рассмотрим случайное блуждание на множестве состояний $\{1, \ldots, L\}$, заданное веролтностями перехода $p_{u+1}=p$ и $p_{u-1}=1-p$ пля $2 \leq i \leq L-1, p_{12}=a$, $p_{11}=1-a$ и $p_{L L-1}=b, p_{L L}=1-b$ д.ля каких-нибудь $0<p<1$ и $0<a, b \leq 1, a$ д.ля оста.пных $i, j$ вынолнено $p_{i j}=0$
        \begin{itemize}
        \item [(a)]
            Докажите, что соответствующая матрица переходньх веролтностей перемешивает тогда и только тогда, когда выполнено хотя бы одно из неравенств $a<1$ и.ш $b<1$.
        \item [(б)]
            Найдите все пары $0<a, b \leq 1$, при которых случайное блуждание эргодично при произвольном $0<p<1$
        \item [(в)]
            Для пронзвольных $a, b, p$ удовлетворяощих $0<a, b \leq 1$ и $0<p<1$ найдите стащионарное состояние. Единственно ли оно? Нашлись ли такие $a, b, p$ при которых есть стационарное состояние единственно, а эргодичности нет?
        \end{itemize}
    \end{prob}
    \begin{proof}
        \begin{itemize}
        \item [(a)]
            ($\Leftarrow$): можем отсидеться в 1 или L нужное число шагов\\
            ($\Rightarrow$): Если $a = 1 = b$, то каждый шаг меняет четность (так как идем в соседнюю), то есть не перемешивающая
        \item [(б)]
            \begin{gather*}
            P =
            \begin{pmatrix}
                1-a & 1-p & 0 & \ldots & &\\
                a & 0 & 1-p & 0 & &\\
                0 & p & 0 & & &\\
                \vdots & \vdots & p & \ddots & & \\
                & & & & &\\
                & & & & 0 & b\\
                0 & & & & p & 1-b\\
            \end{pmatrix}\\
            p = (p_{1] \ldots p_{L}})\\
            pP = \left((1-a)p_{1} + ap_{2}, (1-p)p_{1} + pp_{3}, \ldots, bp_{L_{1}} + (1-b)p_{L}\right)\\
            \\
            \det A_{2}
            = \det 
            \begin{pmatrix}
                0 & b\\
                p & 1-b
            \end{pmatrix}
            = -pb\\
            \det A_{3}
            = \det 
            \begin{pmatrix}
                0 & 1-p & 0\\
                p & 0 & b\\
                0 & p & 1-b
            \end{pmatrix}
            = -(1-p) p (1-b)\\
            \det 
            \begin{pmatrix}
                0 & 1-p & 0 & \ldots & &\\
                p & 0 & 1-p & \ldots & &\\
                0 & & 0 & 1-p & \ldots &\\
                \vdots & & p & & &\\
                & & & & & b\\
                0 & & & & p & 1-b\\
            \end{pmatrix}
            = (1-a) \cdot \det A_{l-1} - (1-p) \cdot a \cdot \det A_{l-2}\\
            \end{gather*}
            Пусть $L$ - четное, преставим его в виде $L = 2k + 2$
            \begin{gather*}
                \det A_{2k + 1} = (-p(1-p)) \cdot \det A_{2k-1}
                = (-p(1-p))^{k-1} \cdot \det A_{3}\\
                = (-1)^{k-1} \cdot p^{k-1} \cdot (1-p)^{k-1} \cdot -(1-p) p (1-b)\\
                = (-1)^{k} p^{k} (1-p)^{k} (1-b)\\
                \\
                \det A_{2k} = (-1)p(1-p) \cdot A_{2k-2}
                = (-1)^{k-1} p^{k-1} (1-p)^{k-1} \cdot \det A_{2}\\
                = (-1)^{k} p ^{k} (1-p)^{k-1} b\\
                \\
                \det P
                = (1-a)(-1)^{k} p^{k} (1-p)^{k} (1-b) - (1-p) a (-1)^{k} p^{k} (1-p)^{k-1} b\\
                = (-1)^{k} p^{k} (1-p)^{k} ((1-a)(1-b) - ab)
                = (-1)^{k} p^{k} (1-p)^{k} (1 - a - b)
            \end{gather*}
            Пусть $L$ - нечетное, преставим его в виде $L = 2k + 1$
            \begin{gather*}
                \det A_{2k} = (-1)^{k} p^{k} (1-p^{k-1}) b\\
                \det A_{2k-1} = (-1)^{k-1} p^{k-1} (1-p)^{k-1} (1-b)\\
                (1-a) \det A_{2k} - (1-p) a \det A_{2k-1}\\
                = (1-a) (-1)^{k} p^{k} (1-p)^{k-1} b - (1-p) a (-1)^{k-1} p^{k-1} (1-p)^{k-1} (1-b)\\
                = p^{k-1} (-1)^{k} (1-p)^{k-1} \left((1-a) b p + (1-p) a (1-b)\right)\\
                = p^{k-1} (-1)^{k} (1-p)^{k-1} (bp - abp + a - ap - ab + apb)\\
                = p^{k-1} (-1)^{k} (1-p)^{k-1} (p(b-a) + a - ab)
            \end{gather*}
            $\det \ne 0,\ a \ne 1$ или $b \ne 1$, тогда $\exists !\ \pi$ - собств (из системы уравнений), то есть мы попали в условия эргодической теоремы\\
            $\det = 0$, следовательно существует более одного собвтсвенного вектора и цепь не эргодична\\
            $a = b = 1\ \Rightarrow$ при неч $d = 0$\\
            $a = b = 1$ - эргодичности нет, так как зависит от начального состояния (чет/нечет)  
        \item [(в)]
        \end{itemize}
    \end{proof}
\vskip 0.6in



 
    \begin{prob}

    \end{prob}
    \begin{proof}
    \begin{itemize}
    \item[(a)]
        Пусть $\xi_{n}$ обозначает число шаров в первом ящике после $n$ вытаскиваний. Случайные величины $\xi_{n}$ можно постронть следующим образом: $\xi_{0}$ - задано, а
        \begin{gather*}
            \xi_{n+1}(\omega)=\xi_{n}(\omega)+\eta_{n+1}^{\xi_{n}(\omega)}(\omega)
        \end{gather*}
        где случайные величины $\eta_{j}^{m}$ определены для $0 \leq m \leq N$ и всех $j \geq 1^{1}$, независимы друг от друга и от $\xi_{0}$, и распределены следующим образом:
        \begin{gather*}
            \mathbb{P}\left(\eta_{n+1}^{m}=1\right)=1-m / N \text { and } \mathbb{P}\left(\eta_{n+1}^{m}=-1\right)=m / N
        \end{gather*}
        Следовательно,
        \begin{gather*}
            \mathbb{P}\left(\xi_{n+1}=i_{n+1} \mid \xi_{n}=i_{n}, \ldots, \xi_{0}=i_{0}\right)\\
            =\mathbb{P}\left(\xi_{n}+\eta_{n+1}^{\xi_{n}}=i_{n+1} \mid \xi_{n}=i_{n}, \ldots, \xi_{0}=i_{0}\right)\\
            =\mathbb{P}\left(i_{n}+\eta_{n+1}^{i_{n}}=i_{n+1} \mid \xi_{n}=i_{n}, \ldots, \xi_{0}=i_{0}\right)\\
            =\mathbb{P}\left(\eta_{n+1}^{i_{n}}=i_{n+1}-i_{n}\right)
        \end{gather*}
        так как $\eta_{n+1}^{i_{n}}$ не зависит от $\xi_{0}$ и $\eta_{j}^{m}$ с $j \leq n$, а значнт, по построению $\xi_{k}$, и от $\xi_{k}$ с $k \leq n$ (так как $\xi_{k}$ строятся через $\xi_{0}$ и $\eta_{j}^{m}$ с $j \leq n$ ). Здесь мы предполагали, что $\mathbb{P}\left(\xi_{n}=i_{n}, \ldots, \xi_{0}=i_{0}\right) \neq 0$. Совершенно ан алогично находим, что
        \begin{gather*}
            \mathbb{P}\left(\xi_{n+1}=i_{n+1} \mid \xi_{n}=i_{n}\right)
            = \mathbb{P}\left(\eta_{n+1}^{i_{n}}=i_{n+1}-i_{n}\right)
        \end{gather*}
        Последнне две формулы влекут, что последовательность $\xi_{n}$ образует марковскую цепь. Чтобы най ти переходные вероятности, достаточно вычислить пх правую часть, что делается тривиально из определения случайных величин $\eta_{j}^{m}$; мы находим переходные вероятности как на приложенной ниже картинке. Матрица переходных вероятностей имеет вид
        \begin{gather*}
        \Pi=
        \begin{pmatrix}
            0 & 1 & 0 & 0 & \ldots & 0 & 0 \\
            1 / N & 0 & 1-1 / N & 0 & \ldots & 0 & 0 \\
            0 & 2 / N & 0 & 1-2 / N & \ldots & 0 & 0 \\
            \vdots & \vdots & \vdots & \vdots & \ldots & \vdots & \vdots \\
            0 & 0 & 0 & 0 & \ldots & 1 & 0
        \end{pmatrix}
        \end{gather*}
        где первая строк а соответствует состоянию 0, а последняя - состоянию $N$
    \item[(б)]
        По определению, стацион арное состояния $\pi$ является левым собственным вектором матрицы П с собственным значением единица:
        \begin{gather*}
            \pi \Pi=\pi
        \end{gather*}
        Запишем явно полученную систему уравнений :
        \begin{gather*}
        \begin{aligned}
            \frac{\pi_{1}}{N} &=\pi_{0} \\
            \frac{\pi-1}{N}+\frac{2}{N} \pi_{2} &=\pi_{1} \\
            \frac{N-k+1}{N}+\frac{3}{N} \pi_{3} &=\pi_{2} \\
            \frac{k+1}{N} \pi_{k+1} &=\pi_{k} \\
            \frac{\pi_{N-1}}{N} &=\pi_{N}
        \end{aligned}
        \end{gather*}
        Последовательно выражаем $\pi_{j}$ через $\pi_{0}$ н находим $\pi_{1}=N \pi_{0}, \pi_{2}=\frac{N(N-1)}{2} \pi_{0}, \ldots$ Замеч аем закономерность $\pi_{j}=C_{N}^{j} \pi_{0}$, верность которой легко проверяется по индукции для всех $j$. Последнее уравнение до сих пор не пспользовалось, подставляем полу ченный результат в него:
        \begin{gather*}
            \frac{C_{N}^{N-1}}{N} \pi_{0}=C_{N}^{N} \pi_{0}
        \end{gather*}
        ч то верно при любом $\pi_{0}$. Зн ачение $\pi_{0}$ н аходим из условия $\sum_{j} \pi_{j}=1 .$ Так к ак $\sum_{j} C_{N}^{j}=$ $2^{N}$, получ аем $\pi_{0}=2^{-N}$. Итого,
        \begin{gather*}
            \pi_{j}=\frac{C_{N}^{j}}{2^{N}}
        \end{gather*}
        Стационарное состояние единственно, так как решение системы линейных уравнений выше - единственно.
    \item[(в)]
        Нет. Если бы МПВ перемешивала, то существовало бы такое $k$, что переходные вероятности $p_{i j}^{(k)}$ за $k$ шагов положительны для любых $i, j .$ Одн ако, $p_{00}^{(k)}$ может быть положительно лишь в том случае, если $k$ - четно, а $p_{01}^{(k)}$ может быть положительно лишь в случае, когда $k$-нечетно (смотри н а граф цепи!)
    \item[(г)]
        Нет. Если бы МЦ была бы эргодична, то переходные вероятности $p_{i j}^{(k)}$ сходились бы при $k \rightarrow \infty$ к компонентам стацион арного состояния $\pi_{j}$, независимо от $i$. Но $p_{00}^{(k)}$ может быть положительно лишь в том случае, если $k$-четно, и поэтому не может сходиться к $\pi_{0}=2^{-N}$ (на самом деле, здесь распределенне в определенном смысле таки приближается к $\pi$, но сходиться не может).
    \end{itemize}
    \end{proof}
\vskip 0.6in


 
    \begin{prob}
        Рассмотрим (однородную) МЦ $\xi_{0}, \xi_{1}, \ldots$ с конечным множеством состояний $\{1, \ldots, L\}$, $L \geq 2$, такую что $p_{11}=1$. Допустим, что для каждого состояния $2 \leq i \leq L$ сушествует $k_{i} \geq 1$, такое что $p_{i 1}^{\left(k_{i}\right)}>0$ - вероятность перейти из состояния $i$ в состояние $1$ за $k_{i}$ шагов положительна.
        \begin{itemize}
        \item[(а)] Покажите, что МПВ такой МЦ не может быть перемешиваюшей.
        \item[(б)] Докажите, что последовательность $\left(p_{i 1}^{(m)}\right)_{m \geq 0}$ не убывает.
        \item[(в)] Покажите, что $\mathbb{P}\left(\xi_{m k} \neq 1\right) \leq(1-\delta)^{m}$ для любого $m \geq 1$, где $k=\max_{i} k_{i}$, a $\delta=\min_{i} p_{i 1}^{\left(k_{i}\right)}>0$
        \item[(г)] Покажите, что $\mathbb{P}\left(\exists k \geq 0: \xi_{n}=1 \forall n \geq k\right)=1$ (т.е. с вероятностью единица мы приезжаем в состояние 1 и там живем.)
        \item[(д)] Покажите, что такая МЦ эргодична и $\pi=(1,0, \ldots, 0)$ - ее единственное стационарное состояние.
        \end{itemize}
    \end{prob}
    \begin{proof}
    \begin{itemize}
    \item[(а)] Не является перемешивающей так как невозможно покинуть вершину 1
    \item[(б)] $\left(p_{i 1}^{(m)}\right)_{m \geq 0}$ - не убывает: $P(\text{\{добрались за n шагов\}}) = P(\text{\{добрались за n-1 шаг\}}) + P(\text{\{первый раз попали на шагу n\}}) \geq P(\{n-1 \text{ шаг}\})$
    \item[(в)] $m = 1$, $P(\{\text{попали изначально находясь в точке } i\}) \geq p_{i}^{(k_i)}$ (потому что на $k_i$ шагу попали с такой вероятностью и не вышли)\\
    $P(\{\text{не попали, изначально находяст в } i\}) \leq 1 - p_{i}^{k^{i}} \leq 1 - \delta\quad \forall i$, тогда $P(\{\xi_{k} \ne 1\}) \leq 1 - \delta$\\
    Шаг:\\
    Пусть верно для всех меньших, докажем для $m$
    \begin{gather*}
        P(\xi_{mk} \ne 1)
        \leq P(\xi_{(m-1)k} \ne 1) \cdot P(\eta_{k} \ne 1)
        \leq (1 - \delta)^{m-1} (1 - \delta)
    \end{gather*}
    $P(\xi_{(m-1)k} \ne 1)$ - Независимые события поскольку $\eta_{k}$ - МЦ, зависящая только от $\xi_{k}$, а не от происходящего\\
    $P(\eta_{k} \ne 1)$ - где $\eta_{k}$ - см. величины для той же самой МЦ, но с начальным распределением $\xi_{(m-1)k}$\\
    То есть ``Продублировали цепь'' $\Rightarrow$ все независимо и можно рассматривать произведение
    \item[(г)] $\mathbb{P}(\exists k \geq 0: \epsilon_{n} = 1\ \forall n \geq k) = 1$: по пункту (В)
    \item[(д)] Все стационарные состояния сходятся к одному $\mathbb{P}(\xi_{mk} \ne 1) \leq (1 - \delta)^{m} = 1 - \Gamma$, следовательно выглядит так: $(\Gamma, p_2, \ldots, p_n)$ где $\sum \Gamma + p_2 + \ldots + p_n = 1$, то есть $p_2 \ldots p_n < 1 - \Gamma = (1 - \delta)^m$ и $p_i \to 0\ \forall i$, то есть все сходится к $(1,0,0, \ldots)$.\\
    Допустим есть еще какой-то стационарный вектор вида $(p_1, p_2, \ldots)$, будем рассматривать его как начальное распредление, тогда мы снова придем в $(1, 0, 0, \ldots)$
    \end{itemize}
    \end{proof}
\vskip 0.6in


 
    \begin{prob}
    \end{prob}
    \begin{proof}
    \end{proof}
\vskip 0.6in



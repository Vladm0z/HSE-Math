%если что-то сломалось, стучать сюда: vlad270136@gmail.com, t.me/mvr27

%кастомный класс документа, чтобы все было красиво
\documentclass[fleqn]{article}
\usepackage[utf8]{inputenc} 
\usepackage[left = 2cm, right = 1.5cm, top = 1cm, bottom = 2cm, bindingoffset = 0cm]{geometry} 
\geometry{a4paper}
\usepackage{times}
\usepackage[english,russian]{babel}

%бибилиотеки ams
\usepackage{amsmath, amsfonts, amssymb, amsthm}

%многострочный текст над/под строкой, скопировано с https://tex.stackexchange.com/questions/346990/text-inside-equation-with
\usepackage{newtxtext, ragged2e}

%tikz
\usepackage{tikz}

%нумерация, названия и вот это все
%\newtheorem{код LaTeX, который пишется в begin/end}[к какому элементу/группе привязать нумерацию]{отображающееся название}[к чему глобально привязать нумерацию]
\theoremstyle{plain}
\newtheorem{thm}{Теорема}[section]
\newtheorem{lem}[thm]{Лемма}
\newtheorem{prop}[thm]{Предложение}
\newtheorem*{cor}{Следствие}
\newtheorem{prob}{Задача}[section]

\theoremstyle{definition}
\newtheorem{defn}{Определение}[section]
\newtheorem{conj}{Гипотеза}[section]
\newtheorem{exmp}{Пример}[section]

\theoremstyle{remark}
\newtheorem*{rem}{Замечание}
\newtheorem*{note}{Отметим}

%\usepackage{tempora}  % Times for numbers in math mode
%\usepackage{newtxmath}  % Times in math mode

%работа с картиночками
\graphicspath{{images/}}
\DeclareGraphicsExtensions{.pdf, .png, .tif, .eps, .tiff, .psd, .jpg}

%настройки hyperref, можно поменять цвета, так как в своих вкусах я не уверен
\usepackage[
breaklinks=true,colorlinks=true,
%linkcolor=blue,urlcolor=blue,citecolor=blue,% PDF VIEW
linkcolor=black,urlcolor=black,citecolor=black,% PRINT
bookmarks=true,bookmarksopenlevel=2]{hyperref}


\title{Цепи Маркова \\ Задачи}
\date{}

\begin{document}

\section{Лист 1}
    \begin{prob}
        Вычислите число неубываюпих путей на целочистенной репетке $\mathbb{Z}^{2}$ (то есть таких, что за один шаг происходит сдвиг на единицу либо вправо, либо вверх), ведунцих из точки $(0,0)$ в точку $(n, n)$, и не пересеканопих диагональ (то есть, проходяпих только через такие точки $(i, j)$, для которых $j \leq i$). Затем почитать в Википедии про числа Каталана.
    \end{prob}
    \begin{proof}
        Всего путей из $(0,0)$ в $(n,n)$ ровно ${{2n}\choose{n}}$, так как нужно сделать всего $2n$ шагов, среди которых $n$ в одном направлении и $n$ в другом. Посчитаем пути выше диагонали, замтим что можно построить биекцию между этими путями и путями из $(-1, 1)$ (или $(1, -1)$) в $(n,n)$. Заметим что таких путей ровно ${{2n}\choose{n-1}}$, тогда путей ниже диагонали ${{2n}\choose{n}} - {{2n}\choose{n-1}} = \frac{1}{n+1} {{2n}\choose{n}}$
    \end{proof}
\vskip 0.6in


    \begin{prob}
        Имеется код длины $n$, состояцций из цифр от $0$ до $9$. Найдите вероятность того, что цифры в коде расположены в неубывающем порядке.
    \end{prob}
    \begin{proof}
        Заметим, что если последовательность неубывающая, то она имеет вид $0, \ldots, 0, 1 ,\ldots, 9$ (какие-то цифры могут отсутствовать), то есть это количество способов расставить $9$ перегородок на $(n-1) + 9 = n+8$ позиций, то есть ${{n+8}\choose{9}}$
    \end{proof}
\vskip 0.6in


    \begin{prob}
        Из мешка, в котором лежат все кости набора домино (их $28$ штук, соответствующих неупорядоченным парам чисел от $0$ до $6$), извлекают две костяшки. Найдите вероятность того, что их можно приложить друг к другу по правилам игры в домино.
    \end{prob}
    \begin{proof}
        Всего пар костяшек $\frac{28 \cdot 27}{2}$, посчитаем число тех, которые можно сложить вместе. Каждая костяшка с 2 одинаковыми значениями может быть в паре с $6$ другими, а с $2$ разными значениями --  в паре с $12$, откуда
        \begin{gather*}
            \frac{\frac{7 \cdot 6 + (28-7) \cdot 12}{2}}{\frac{28 \cdot 27}{2}}
            = \frac{7 \cdot 6 + (28 - 7) \cdot 12}{28 \cdot 27}
            = \frac{7 \cdot 2 + 21 \cdot 4}{28 \cdot 9}
            = \frac{7 \cdot 14}{28 \cdot 9}
            = \frac{7}{18}
        \end{gather*}
    \end{proof}
\vskip 0.6in


    \begin{prob}
        На экзамене студенты по очереди вытягиванот билеты (и, естественно, не возвращают их на место). Всего имеется $n$ билетов, из которых $k$ - простые, и $m \leq n$ студентов. Чему равна вероятность того, что первый студент вытянет простой билет? А последний? Меняется ли вероятность вытянуть простой билет в зависимости от того, каким номером студент тянет билет?
    \end{prob}
    \begin{proof}
        Добавим еще $n-m$ студентов, чтобы сткдентов было столько же, сколько и билетов, далее пронумеруем студентов от 1 до $n$ в соответствии с очередью. Также пронумеруем билеты от 1 до $n$, тогда вероятностное пространство:
        \begin{gather*}
            \Omega=\left\{\left(i_{1}, i_{2}, \ldots i_{n}\right) \mid 1 \leqslant i_{a} \leqslant n,\ i_{b} \neq i_{c},\ b \neq c\right\}
        \end{gather*}
        Элементарным событием является перестановка $\left(i_{1}, i_{2} \ldots i_{n}\right)$, где $i_j$ -- номер билета у студента с номером $j$.
        \vskip 0.1in
        $A_{i\: 1}$ -- студент под номером $i$ получил билет номер $1$, равновероятно событию $A_{i\ 2}$, так как между множествами перестановок, в которых $i$ студент получил билет номер $1$ и $2$, существует биекция. Аналогичном, события $\left\{A_{i\: 1}, A_{i\: 2}, \ldots A_{i\: n}\right\}$ равновероятны.
        \vskip 0.1in
        Событие $B_{j}$ студент $j$ вытянул простой билет, $\mathcal{P}(B_{j})=\frac{k}{n}$ для любого $j$, откуда вероятность вытягивания простого билета у всех студентов равна $\frac{k}{n}$.
    \end{proof}
\vskip 0.6in


    \begin{prob}
        Пассажиры заходят в самолет и рассаживаются случайным образом, не обрашая внимания на свои билеты. С какой вероятностью ни один пассажир не сядет на свое место?
    \end{prob}
    \begin{proof}
        То есть мы хотим рассмотреть отношение числа перестановки из $n$ элементов без неподвижной точки к общему числу перестановок, для этого мы вычтем из общего числа перестановок $n!$, число перестановок с $k$ неподвижными точками ${{n}\choose{k}} (n-k)!$
        \begin{gather*}
            \mathcal{P} = \frac{|A|}{|\Omega|}
            = \frac{n! - {{n}\choose{1}} (n-1)! + {{n}\choose{2}} (n-2)! - \ldots}{n!}\\
            = \frac{n! \cdot \left(1 - \frac{1}{1!} + \frac{1}{2}! - \ldots + (-1)^n \frac{1}{n!}\right)}{n!}\\
            = \frac{n! \cdot \sum\limits_{k = 0}^{n} \frac{(-1)^k}{k!}}{n!}
            = \sum\limits_{k = 0}^{n} \frac{(-1)^k}{k!}
        \end{gather*}
    \end{proof}
\vskip 0.6in


    \begin{prob}
        Профессор разбирает на семинаре $k$ задач, вызывая для каждой из них одного из $n$ присутствуюпих студентов случайным образом. Какова вероятность того, что к концу семинара каждый из студентов побывает у доски не более двух раз?
    \end{prob}
    \begin{proof}
        \begin{gather*}
            \Omega = \{(\omega_1, \ldots, \omega_k)\ |\ \omega_j = \{1, \ldots, n\} \}\\
            \Omega = \{(\omega_1, \ldots, \omega_k)\ |\ \omega_j = \{1, \ldots, n\} \text{ не более 2 повторений} \}
        \end{gather*}
        Тогда посчитаем число элементов в каждом из этих множеств, $|\Omega|$ -- каждую из $k$ задач может разобрать один из $n$ студентов
        \begin{gather*}
            |\Omega| = n^k
        \end{gather*}
        Подсчет числа искомых варинатов можно представить как способ выборать $k$ элементов из $2n$ (то есть двойной набор студенов)
        \begin{gather*}
            |A| = \frac{\# \{1, \ldots, n, \tilde{1}, \ldots, \tilde{n}\}}{(2!)^n}
            = \frac{2n \cdot (2n-1) \cdot \ldots \cdot (2n-k+1)}{(2!)^n}
            = \frac{(2n)!}{(2n-k)! 2^n}
        \end{gather*}
        Откуда
        \begin{gather*}
            \mathcal{P}(A) = \frac{|A|}{|Q|} = \frac{(2n)!}{(2n-k)! \cdot 2^n \cdot n^k}
        \end{gather*}
    \end{proof}
\vskip 0.6in


    \begin{prob}
        Приведите пример $n$ случайных событий, таких что любые $n-1$ из них независимы (в совокупности), а все они вместе - зависимы.
    \end{prob}
    \begin{proof}
        Пусть события $A_i,\ i \in [1, n-1]$ -- выпадение орла (при броске монеты), а $A_n$ -- выпало четное число орлов, тогда $\{A_1, \ldots, A_n\}$ зависимо, но любое подмножество независимо
    \end{proof}
\vskip 0.6in


    \begin{prob}
        Тест на ковид имеет чувствитетьность $50\%$ (т.е. верно диагностирует больного в $50\%$ случаев) и специфичность $70\%$ ($30\%$ здоровых людей обььявляет больными). Известно, что на данной территории ковидом болеет 1 человек из $300$. Какова вероятность того, что житель этой территории, обьявленный больным по результатам теста, действителыно болен?
    \end{prob}
    \begin{proof}
        Обозначим через ``+'' положительный результат теста и через ``-'' отрицательный, тогда
        \begin{gather*}
            P(+|\text{болен}) = 0.5\qquad
            P(+|\text{здоров}) = 0.3\\
            P(-|\text{болен}) = 1 - 0.5 = 0.5\qquad
            P(-|\text{здоров}) = 1 - 0.3 = 0.7\\
            P(\text{болен}) = \frac{1}{300}\\
            P(\text{здоров}) = \frac{299}{300}\\
        \end{gather*}
        Чтобы решить задачу, воспользуемся формулой Байеса:
        \begin{gather*}
            P(\text{болен}| +)
            = \frac{P(+|\text{болен}) P(\text{болен})}{P(+|\text{болен}) P(\text{болен}) + P(+|\text{здоров}) P(\text{здоров})}
            = \frac{\frac{1}{300} \cdot \frac{1}{2}}
            {\frac{1}{300} \cdot \frac{1}{2} + \frac{299}{300} \cdot \frac{3}{10}} = \frac{5}{2 \cdot 11 \cdot 41} \approx 0.006
        \end{gather*}  
    \end{proof}
\vskip 0.6in


    \begin{prob}
        Треугольник $A B C$ - прямоугольный равнобедренный с прямым углом $C$ и катетами, равными 1. В нём случайным образом выбирается точка $Z$. Пусть $X$ и $Y$-основания перпендикуляров из $Z$ на катеты. Обозначим буквами $P$ и $S$ соответственно периметр и площадь прямоугольника $C X Z Y$.
        \begin{itemize}
            \item[(a)] Найдите вероятность того, что $S < \frac{8}{81}$.
            \item[(b)] Найдите вероятность того, что $P < \frac{4}{3}$.
            \item[(c)] Найдите условную вероятность того, что $S < \frac{8}{81}$ при условии, что $P < \frac{4}{3}$
            \item[(d)] Найдите условную вероятность того, что $P < \frac{4}{3}$ при условии, что $S < \frac{8}{81}$.
            \item[(e)] Существуют ли такие числа $S_{0} \in (0, \frac{1}{4})$ и $P_{0} \in (0, 2)$, что события $\left\{ S < S_{0} \right\}$ и $\left\{ P < P_{0} \right\}$ будут независимы?
        \end{itemize}
    \end{prob}
    \begin{proof}
        Заметим, что все это вероятности являются отношением площади ГМТ подходящих под условие точек к площади всего трегуольника. Заранее посчитаем площади зеленой и синих частей
        \begin{figure}[h]
            \center{\includegraphics[width=0.8\linewidth]{IMG_1}}
        \end{figure}
        \begin{gather*}
            \begin{cases}
                y + x - \frac{2}{3} = 0\\
                y - \frac{8}{81 x} = 0
            \end{cases}\\
            \frac{8}{81 x} = -x + \frac{2}{3}\\
            81x^2 - 54x + 8 = 0\\
            x_1 = \frac{4}{9}\qquad y_1 = \frac{2}{9}\\
            x_2 = \frac{2}{9}\qquad y_2 = \frac{4}{9}\\
            \int\limits_{\frac{2}{9}}^{\frac{4}{9}} \frac{8}{81x} dx
            = \frac{8}{81} \ln x \bigg|_{\frac{2}{9}}^{\frac{4}{9}}
            = \frac{8}{81} (\ln \frac{4}{9} - \ln \frac{2}{9})
            = \frac{8}{81}\ln 2\\
            \frac{4}{27} + \frac{8}{81} \ln 2 
        \end{gather*}
        И
        \begin{gather*}
            \int\limits_{\frac{4}{9}}^{\frac{8}{9}} \frac{8}{81 x} dx
            = \frac{8}{81} \ln x \bigg|_{\frac{4}{9}}^{\frac{8}{9}}
            = \frac{8}{81} \left(\ln \frac{8}{9} - \ln \frac{4}{9}\right)
            = \frac{8}{81} \ln 2\\
            \frac{8}{81} \ln 2 - \frac{2}{81} = \frac{2}{81} (4\ln 2 - 1)
        \end{gather*}
        \begin{itemize}
            \item[(a)] Заметим что подходящие точки ограничены сторонами треугольника и кривой $y = \frac{8}{81 x}$, то есть ответ является отношением площади этой части к площади всего треугольника, то есть
            \begin{gather*}
                \frac{2 \cdot \frac{2}{81}(4 \ln 2 - 1) + \left(\frac{4}{27} + \frac{8}{81} \ln 2\right)}{\frac{1 \cdot 1}{2}}
                = \frac{\frac{8}{81}\left(3 \ln 2 + 1\right)}{\frac{1}{2}}
                = \frac{16}{81}\left(3 \ln 2 + 1\right)
                \approx 0.6
            \end{gather*}
            \item[(b)] Заметим что $P = CX + XZ + ZY + ZY = 2(CX + CY) < \frac{4}{3}$, тогда подходящие точки ограничены сторонами треугольника и прямой $y + x - \frac{2}{3} = 0$  вероятность равна отношению этой части к площади всего треугольника
            \begin{gather*}
                \frac{\frac{\frac{2}{3} \cdot \frac{2}{3}}{2}}{\frac{1 \cdot 1}{2}}
                = \frac{4}{9}
                \approx 0.44
            \end{gather*}
            \item[(c)] Вероятность равна отношению площади зеленой части к площади под прямой $y + x - \frac{2}{3} = 0$, ограниченной сторонами треугольника
            \begin{gather*}
                \frac{\frac{4}{27} + \frac{8}{81} \ln 2}{\frac{\frac{2}{3} \cdot \frac{2}{3}}{2}}
                = \frac{2}{9}\left(3 + 2 \ln 2\right)
                \approx 0.97
            \end{gather*}
            \item[(d)] Вероятность равна отношению площади зеленой части к площади под $y = \frac{8}{81 x}$, ограниченной сторонами треугольника
            \begin{gather*}
                \frac{\frac{4}{27} + \frac{8}{81} \ln 2}{\frac{4}{27} + \frac{8}{81} \ln 2 + 2 \cdot \frac{2}{81} (4\ln 2 - 1)}
                = \frac{\frac{4}{27} + \frac{8}{81} \ln 2}{\frac{8}{81}(3 \ln 2 + 1)}
                = \frac{3 + 2 \ln 2}{6 \ln 2 + 2}
                \approx 0.71
            \end{gather*}
            \item[(e)] Независимость событий говорит о том, что $P(A \cap B) = P(A) \cdot P(B)$, то есть в нашем случае это значит, что вероятность того, что выбранная точка $Z$ будет соответствовать сразу всем условиям, равна произведению вероятности того, что она будет иметь периметр $P_0$ и вероятности того, что площадь $S_0$. Тогда мы можем записать формулу $P(A \cap B) = P(A) \cdot P(B)$ через площади, получив условие на $P_0, S_0$.
            \begin{gather*}
                \frac{1 \cdot 1}{2}
                \cdot \left(
                    \int\limits
                    _{\frac{P_0 - \sqrt{P_0^2 - 8S_0}}{2}}
                    ^{\frac{P_0 + \sqrt{P_0^2 - 8S_0}}{2}}
                    S_0 \frac{1}{x} dx
                    + \frac{P_0 - \sqrt{P_0^2 - 8S_0}}{2} \cdot \frac{P_0}{2}
                \right)
                = \frac{P_0^2}{8}
                \cdot \left(
                    \int\limits
                    _{\frac{1 - \sqrt{1 - 4S_0}}{2}}
                    ^{\frac{1 + \sqrt{1 - 4S_0}}{2}}
                    S_0 \frac{1}{x} dx
                    + \frac{1 - \sqrt{1 - 4S_0}}{2} \cdot 1
                \right)\\
                4
                \cdot \left(
                    \int\limits
                    _{\frac{P_0 - \sqrt{P_0^2 - 8S_0}}{2}}
                    ^{\frac{P_0 + \sqrt{P_0^2 - 8S_0}}{2}}
                    S_0 \frac{1}{x} dx
                    + \frac{P_0 - \sqrt{P_0^2 - 8S_0}}{2} \cdot \frac{P_0}{2}
                \right)
                = P_0^2
                \cdot \left(
                    \int\limits
                    _{\frac{1 - \sqrt{1 - 4S_0}}{2}}
                    ^{\frac{1 + \sqrt{1 - 4S_0}}{2}}
                    S_0 \frac{1}{x} dx
                    + \frac{1 - \sqrt{1 - 4S_0}}{2}
                \right)\\
                8\int\limits
                _{\frac{P_0 - \sqrt{P_0^2 - 8S_0}}{2}}
                ^{\frac{P_0 + \sqrt{P_0^2 - 8S_0}}{2}}
                S_0 \frac{1}{x} dx
                + 2 \left(P_0^2 - P_0 \sqrt{P_0^2 - 8S_0}\right)
                = 2 P_0^2
                \int\limits
                _{\frac{1 - \sqrt{1 - 4S_0}}{2}}
                ^{\frac{1 + \sqrt{1 - 4S_0}}{2}}
                S_0 \frac{1}{x} dx
                + P_0^2 (1 - \sqrt{1 - 4S_0})\\
                8\int\limits
                _{\frac{P_0 - \sqrt{P_0^2 - 8S_0}}{2}}
                ^{\frac{P_0 + \sqrt{P_0^2 - 8S_0}}{2}}
                S_0 \frac{1}{x} dx
                - 2 P_0^2
                \int\limits
                _{\frac{1 - \sqrt{1 - 4S_0}}{2}}
                ^{\frac{1 + \sqrt{1 - 4S_0}}{2}}
                S_0 \frac{1}{x} dx
                = P_0^2 \left(2\sqrt{1 - 8\frac{S_0}{P_0^2}} - \sqrt{1 - 4S_0} -1\right)\\
            \end{gather*}
        \end{itemize}
    \end{proof}
\vskip 0.6in


    \begin{prob}
        Вася и Петя ездят в школу на автобусе. Вася приходит на автобуснуго остановку в момент времени, равномерно распределённый на отрезке между $8:00$ и $8:17$, и садится в первый подопедииии автобус; если же в $8:17$ Вася всё ещё не уехал, то он бежит в школу бегом. Его одноклассник Петя поступает ровно таким же образом. Автобусы подьезжанот к остановке каждые $5$ минут (время прибытия первого из них равномерно распределено на отрезке от $8:00$ до $8:05$). Kaкова вероятность того, что Вася и Петя поедут в пколу на одном и том же автобусе, если времена появления на остановке Васи, Пети и первого автобуса независимы?
    \end{prob}

    \begin{proof}
        Вероятностное пространство имеет вид $\Omega = \{\omega_n\ |\ n \in [0,17]\}$, события $\omega_n$ = прийти на остановку в $8:n$, $A_1 := (\text{будет 3 автобуса})$, то есть первый автобус приехал в $\left(2, 5\right]), \mathcal{P}(A_1) = \frac{3}{5}$, $A_2 := (\text{приедет 4 автобуса})$, то есть первый в $\left[0, 2\right]), \mathcal{P}(A_2) = \frac{2}{5}$. $\omega_{P} = t_1$ -- Петя пришел в $t_1$, $\omega_{V} = t_2$ -- Вася пришел в $t_2$. Петя и Вася будут в одном автобусе только в том случае, если они придут в один временной промежуток, так как события $\{A_1, \omega_{P}, \omega_{V}\}$ независимы, то
        \begin{gather*}
            \frac{x_1}{17} \cdot \frac{x_1}{17} + 2 \cdot \frac{5}{17} \cdot \frac{5}{17}
            = \frac{9 + 2 \cdot 5^4}{17^2 \cdot 5^2}
            = \frac{1259}{7225}
        \end{gather*}
        Аналогично $\{A_2, \omega_{P}, \omega_{V}\}$ независимы
        \begin{gather*}
            \frac{x_2}{17} \cdot \frac{x_2}{17} + 3 \cdot \frac{5}{17} \cdot \frac{5}{17}
            = \frac{2^2 + 3 \cdot 5^4}{17^2 \cdot 5^2}
            = \frac{1879}{7225}
        \end{gather*}
        Тогда итоговая вероятность
        \begin{gather*}
            \frac{1259}{7225} + \frac{1879}{7225}
            = \frac{3138}{7225}
            \approx 0.43
        \end{gather*}
    \end{proof}
\vskip 0.6in


    \begin{prob}
        На гранях двух кубиков расставлены натуральные числа $k_{1}, \ldots, k_{6}$. $l_{1}, \ldots, l_{6}$ cooтветственно, причём неверно, что $\left\{k_{1}, \ldots, k_{6}\right\}=\left\{l_{1}, \ldots, l_{6}\right\} = \{1, \ldots, 6\}$. Можно ли подобрать числа $k_{i}, l_{j}$ так, что для любого $s=2, \ldots, 12$ вероятность того, что при броске этих кубиков выпадет сумма очков, равная $s$, совпадает с той же вероятностью для «классических» кубиков (с цифрами от $1$ до $6$ на гранях)?
    \end{prob}
    \begin{proof}
        Рассмотрим кости с числами $\{1, 2_1, 2_2, 3_1, 3_2, 4\}$ (в данном случае индекс лишь обозначает несколько одинаковых значений) и $\{1, 3, 4, 5, 6, 8\}$, заметим, что вероятность выпадения каждой из сумм совпадает с вероятностью выпадения на классических кубиках. В следующей таблицы приведены все возможные исходы, причем они все равновероятны
        \begin{center}
            \begin{tabular}{ |c|cccccc| } 
                \hline
                2 & $\{1, 1\}$ &&&&& \\ 
                \hline
                3 & $\{2_1, 1\}$ & $\{2_2, 1\}$ &&&& \\ 
                \hline
                4 & $\{3_1, 1\}$ & $\{3_2, 1\}$ & $\{1, 3\}$ &&& \\ 
                \hline
                5 & $\{1, 4\}$ & $\{2_1, 3\}$ & $\{2_2, 3\}$ & $\{4, 1\}$ && \\ 
                \hline
                6 & $\{1, 5\}$ & $\{2_1, 4\}$ & $\{2_2, 4\}$ & $\{3_1, 3\}$ & $\{3_2, 3\}$ & \\ 
                \hline
                7 & $\{1, 6\}$ & $\{2_1, 5\}$ & $\{2_2, 5\}$ & $\{3_1, 4\}$ & $\{3_2, 4\}$ & $\{4, 3\}$ \\ 
                \hline
                8 & $\{2_1, 6\}$ & $\{2_2, 6\}$ & $\{3_1, 5\}$ & $\{3_2, 5\}$ & $\{4, 4\}$ & \\ 
                \hline
                9 & $\{1, 8\}$ & $\{3_1, 6\}$ & $\{3_2, 6\}$ & $\{4, 5\}$ && \\ 
                \hline
                10 & $\{2_1, 8\}$ & $\{2_2, 8\}$ & $\{4, 6\}$ &&& \\ 
                \hline
                11 & $\{3_1, 8\}$ & $\{3_2, 8\}$ &&&& \\ 
                \hline
                12 & $\{4, 8\}$ &&&&& \\ 
                \hline
            \end{tabular}
        \end{center}
    \end{proof}
\section{Лист 2}

\begin{prob}
\begin{gather*}
	-x_{1} + x_{2} - 2x_{3} + 3x_{4} + x_{5} \to \max \\
	x_{1} + 2x_{2} - x_{3} - 2x_{4} + x_{5} \leqslant 3 \\
	-x_{1} - x_{2} + x_{3} + 2x_{4} + x_{5} \leqslant 1 \\
	2x_{1} + x_{2} + x_{3} - x_{4} \leqslant 1 \\
	x_{i} \geqslant 0
\end{gather*}
\end{prob}
\begin{proof}
	\begin{gather*}
	\begin{array}{c|ccccc|c} 
		& x_{1} & x_{2} & x_{3} & x_{4} & x_{5} & \\
		\hline 
		y_{1} & 1 & 2 & -1 & -2 & 1 & -3\\
		y_{2} & -1 & -1 & 1 & 2 & 1 & -1 \\
		y_{3} & 2 & 1 & 1 & -1 & 0 & -1\\
		\hline 
		& -1 & 1 & -2 & 3 & 1 & 0
	\end{array}
	\end{gather*}
	Откуда
	\begin{gather*}
		x_{1} = y_{1} - y_{2} + 2y_{3} - 1\\
		x_{2} = 2y_{1} - y_{2} + y_{3} + 1\\
		x_{3} = -y_{1} + y_{2} + y_{3} - 2\\
		x_{4} = -2y_{1} + 2y_{2} - y_{3} + 3\\
		x_{5} = y_{1} + y_{2} + 1\\
	\end{gather*}
	В зависимости
	\begin{gather*}
		y_{1} = x_{1} + 2x_{2} - x_{3} - 2x_{4} + x_{5} - 3\\
		y_{2} = -x_{1} - x_{2} + x_{3} + 2x_{4} + x_{5} - 1\\
		y_{3} = 2x_{1} + x_{2} + x_{3} - x_{4} - 1
	\end{gather*}
	Выразим через $x_{1}$, $y_{1}$ остальные переменные
	\begin{gather*}
		x_{1} = y_{1} - 2x_{2} + x_{3} + 2x_{4} - x_{5} + 3
	\end{gather*}
	Получим
	\begin{gather*}
		y_{2}
		= -(y_{1} - 2x_{2} + x_{3} + 2x_{4} - x_{5} + 3) - x_{2} + x_{3} + 2x_{4} + x_{5} - 1
		= -y_{1} + x_{2} + 2x_{5} - 4\\
		y_{3}
		= 2(y_{1} - 2x_{2} + x_{3} + 2x_{4} - x_{5} + 3) + x_{2} + x_{3} - x_{4} - 1
		= 2y_{1} - 3x_{2} + 3x_{3} + 3x_{4} - 2x_{5} + 5
	\end{gather*}
	Выразим через $y_{1}, x_{2}, x_{3}, x_{4}, x_{5}$ функцию стоймости в двойственной задаче
	\begin{gather*}
		-x_{1} + x_{2} - 2x_{3} + 3x_{4} + x_{5}\\
		= -(y_{1} - 2x_{2} + x_{3} + 2x_{4} - x_{5} + 3) + x_{2} - 2x_{3} + 3x_{4} + x_{5}\\
		= -y_{1} + 3x_{2} - 3x_{3} + x_{4} + 2x_{5} - 3
	\end{gather*}
	Получим
	\begin{gather*}
	\begin{array}{c|ccccc|c} 
		& y_{1} & x_{2} & x_{3} & x_{4} & x_{5} & \\
		\hline 
		x_{1} & 1 & -2 & 1 & 2 & -1 & -3\\
		y_{2} & -1 & 1 & 0 & 0 & 2 & 4\\
		y_{3} & 2 & -3 & 3 & 3 & -2 & -5\\
		\hline 
		& -1 & 3 & -3 & 1 & 2 & 3
	\end{array}
	\end{gather*}
	Выразим через $x_{2}$, $y_{2}$ остальные переменные
	\begin{gather*}
		y_{2} = -y_{1} + x_{2} + 2x_{5} - 4\\
		x_{2} = y_{1} + y_{2} - 2x_{5} + 4\\
	\end{gather*}
	Получим
	\begin{gather*}
		x_{1} 
		= y_{1} - 2(y_{1} + y_{2} - 2x_{5} + 4) + x_{3} + 2x_{4} - x_{5} + 3
		= -y_{1} - 2y_{2} + x_{3} + 2x_{4} + 3x_{5} - 5\\
		y_{3}
		= 2y_{1} - 3(y_{1} + y_{2} - 2x_{5} + 4) + 3x_{3} + 3x_{4} - 2x_{5} + 5
		= -y_{1} - 3y_{2} + 3x_{3} + 3x_{4} + 4x_{5} - 7
	\end{gather*}
	Выразим через $y_{1}, y_{2}, x_{3}, x_{4}, x_{5}$ функцию стоймости в двойственной задаче
	\begin{gather*}
		-x_{1} + x_{2} - 2x_{3} + 3x_{4} + x_{5}\\
		= -y_{1} + 3(y_{1} + y_{2} - 2x_{5} + 4) - 3x_{3} + x_{4} + 2x_{5} - 3\\
		= 2y_{1} + 3y_{2} - 3x_{3} + x_{4} - 4x_{5} + 9
	\end{gather*}
	Получим
	\begin{gather*}
	\begin{array}{c|ccccc|c} 
		& y_{1} & y_{2} & x_{3} & x_{4} & x_{5} & \\
		\hline 
		x_{1} & -1 & 2 & 1 & 2 & 3 & 5\\
		x_{2} & 1 & -1 & 0 & 0 & -2 & -4\\
		y_{3} & 2 & -3 & 0 & 7 & 4 & 7\\
		\hline 
		& 2 & 3 & -3 & 1 & -4 & 9
	\end{array}
	\end{gather*}
	Откуда максимум равен $9$ и достигается при $x = (0, 3, 0, 2, 0)$
\end{proof}
\vskip 0.6in



\begin{prob}
\begin{gather*}
	3 x_{1}+4 x_{2}+5 x_{3} \to \max \\
	x_{1}+2 x_{2}+2 x_{3} \leqslant 1 \\
	-3 x_{1}+x_{3} \leqslant -1 \\
	-2 x_{1}-x_{2} \leqslant -1 \\
	x_{j} \geqslant 0
\end{gather*}
\end{prob}
\begin{proof}
	\begin{gather*}
	\begin{array}{c|ccc|c} 
		& x_{1} & x_{2} & x_{3} & \\
		\hline 
		y_{1} & 1 & 2 & 2 & -1\\
		y_{2} & -3 & 0 & 1 & 1 \\
		y_{3} & -2 & -1 & 0 & 1\\
		\hline 
		& 3 & 4 & 5 & 0
	\end{array}
	\end{gather*}
	Откуда
	\begin{gather*}
		x_{1} = y_{1} - 3y_{2} - 2y_{3} - 3\\
		x_{2} = 2y_{1} - y_{3} - 4\\
		x_{3} = 2y_{1} + y_{2} - 5
	\end{gather*}
	В зависимости
	\begin{gather*}
		y_{1} = x_{1} + 2x_{2} + 2x_{3} - 1\\
		y_{2} = -3x_{1} + x_{3} + 1\\
		y_{3} = -2x_{1} - x_{2} + 1
	\end{gather*}
	
	Выразим через $x_{1}$, $y_{2}$ остальные переменные
	\begin{gather*}
		y_{2} = -\frac{1}{3} x_{1} + \frac{1}{3} y_{1} - \frac{2}{3}y_{3} - 1
	\end{gather*}
	Получим
	\begin{gather*}
		x_{2}
		= 2y_{1} - y_{3} - 4\\
		x_{3}
		= 2y_{1} + y_{2} - 5
		= 2y_{1} -\frac{1}{3} x_{1} + \frac{1}{3} y_{1} - \frac{2}{3}y_{3} - 1 - 5
		= -\frac{1}{3} x_{1} + \frac{7}{3} y_{1} - \frac{2}{3} y_{3} - 6
	\end{gather*}
	Выразим через $y_{1}, x_{1}, y_{3}$ функцию стоймости в двойственной задаче
	\begin{gather*}
		y_{1} - y_{2} - y_{3}
		= y_{1} + \frac{1}{3} x_{1} - \frac{1}{3} y_{1} + \frac{2}{3}y_{3} + 1 - y_{3}
		= \frac{2}{3} y_{1} + \frac{1}{3} x_{1} - \frac{1}{3} y_{3} + 1
	\end{gather*}
	Получим
	\begin{gather*}
	\begin{array}{c|ccc|c} 
		& y_{2} & x_{2} & x_{3} & \\
		\hline 
		y_{1} & \frac{1}{3} & 2 & \frac{7}{3} & \frac{2}{3}\\
		x_{1} & -\frac{1}{3} & 0 & -\frac{1}{3} & \frac{1}{3}\\
		y_{3} & -\frac{2}{3} & -1 & -\frac{2}{3} & -\frac{1}{3}\\
		\hline 
		& -1 & -4 & -6 & 1
	\end{array}
	\end{gather*}

	Выразим через $x_{2}$, $y_{3}$ остальные переменные
	\begin{gather*}
		y_{3} = 2y_{1} - x_{2} - 4
	\end{gather*}
	Получим
	\begin{gather*}
		y_{2}
		= \frac{1}{3} y_{1} - \frac{1}{3} x_{1} - \frac{1}{3} (2 y_{1} - x_{2} - 4) - 1
		= -y_{1} - \frac{1}{3} x_{1} + \frac{2}{3} x_{2} + \frac{5}{3}\\
		x_{3}
		= \frac{7}{3} y_{1} - \frac{1}{3} x_{1} - \frac{2}{3} (2 y_{1} - x_{2} - 4) - 6
		= y_{1} - \frac{1}{3} x_{1} + \frac{2}{3} x_{2} - \frac{10}{3}
	\end{gather*}
	Выразим через $y_{1}, x_{1}, x_{2}$ функцию стоймости в двойственной задаче
	\begin{gather*}
		y_{1} - y_{2} - y_{3}
		= \frac{2}{3} y_{1} + \frac{1}{3} x_{1} - \frac{1}{3} (2y_{1} - x_{2} - 4) + 1
		= \frac{1}{3} x_{1} + \frac{1}{3} x_{2} + \frac{7}{3}
	\end{gather*}
	Получим
	\begin{gather*}
	\begin{array}{c|ccc|c} 
		& y_{2} & y_{3} & x_{3} & \\
		\hline 
		y_{1} & -1 & 2 & 1 & 0\\
		x_{1} & -\frac{1}{3} & 0 & -\frac{1}{3} & \frac{1}{3}\\
		x_{2} & \frac{2}{3} & -1 & \frac{2}{3} & \frac{1}{3}\\
		\hline 
		& \frac{5}{3} & -4 & -\frac{10}{3} & \frac{7}{3}
	\end{array}
	\end{gather*}

	Выразим через $y_{1}$, $y_{3}$ остальные переменные
	\begin{gather*}
		y_{3} = 2y_{1} - x_{2} - 4
	\end{gather*}
	Получим
	\begin{gather*}
		y_{2}
		= -\frac{1}{2} y_{3} + \frac{1}{6} x_{2} - \frac{1}{3} x_{1} - \frac{1}{3}\\
		x_{3}
		= \frac{1}{2} y_{3} + \frac{7}{6} x_{2} - \frac{1}{3} x_{1} - \frac{4}{3}
	\end{gather*}
	Получим
	\begin{gather*}
	\begin{array}{c|ccc|c} 
		& y_{2} & y_{3} & x_{3} & \\
		\hline 
		x_{1} & -\frac{1}{3} & 0 & -\frac{1}{3} & \frac{1}{3}\\
		x_{2} & \frac{2}{3} & -1 & \frac{2}{3} & \frac{1}{3}\\
		y_{1} & -1 & 2 & 1 & 0\\
		\hline 
		& \frac{5}{3} & -4 & -\frac{10}{3} & \frac{7}{3}
	\end{array}
	\end{gather*}

	Выразим через $x_{2}$, $y_{2}$ остальные переменные
	\begin{gather*}
		x_{2} = 6 y_{1} + 3 y_{3} + 2 x_{1} + 2
	\end{gather*}
	Получим
	\begin{gather*}
		y_{1}
		= \frac{1}{2} y_{3} + \frac{1}{2} (6 y_{2} + 3 y_{3} + 2 x_{1} + 2) + 2
		= 2 y_{3} + 3 y_{2} + x_{1} + 3\\
		x_{3}
		= 4 y_{3} + 2 x_{1} + 7 y_{2} + 1
	\end{gather*}
	Выразим функцию стоймости
	\begin{gather*}
		\frac{1}{3} x_{1} + \frac{1}{3} (6 y_{2} + 3 y_{3} + 2 x_{1} + 2) + \frac{7}{3}
		= \frac{1}{3} x_{1} + 2 y_{2} + y_{3} + \frac{2}{3} x_{1} + \frac{2}{3} + \frac{7}{3}
		= x_{1} + 2 y_{2} + y_{3} + 3
	\end{gather*}
	Получим
	\begin{gather*}
	\begin{array}{c|ccc|c} 
		& y_{1} & x_{2} & x_{3} & \\
		\hline 
		x_{1} & 1 & 2 & 2 & 1 \\
		y_{2} & 3 & 6 & 7 & 2\\
		y_{3} & 2 & 3 & 4 & 1\\
		\hline 
		& 3 & 2 & 1 & 3
	\end{array}
	\end{gather*}

	Откуда максимум равен $3$ и достигается при $x = (1, 0, 0)$	
\end{proof}
\vskip 0.6in



\begin{prob}
Найти все вершины многогранника в $\mathbb{R}^{4}$
\begin{gather*}
	x_{1} - 2x_{2} + 4x_{3} - x_{4} \leqslant 1 \\
	2x_{1} + 3x_{2} + x_{3} + 2x_{4} \leqslant 3 \\
	x_{i} \geqslant 0, i \in\{1,2,3,4\}
\end{gather*}
\end{prob}
\begin{proof}
	Вершина в $\mathbb{R}^4$ -- решение системы, в котором 4 неравенства обращены в равенства.
	\begin{itemize}
	\item
		В случае, когда зануляются неравенства $x_{i} \geqslant 0$, вершиной является $(0,0,0,0)$
	\item
		Если зануляются 3 неравенства $x_{i} \geqslant 0$ и $x_{1} - 2x_{2} + 4x_{3} - x_{4} = 1$, то вершинами являются $(0,0, \frac{1}{4}, 0)$ и $(1,0,0,0)$, а точки $(0,-\frac{1}{2},0,0), (0,0,0,1)$ вершинами не являются в силу $x_{2} < 0$ и $x_{4} < 0$
	\item
		Если зануляются 3 неравенства $x_{i} \geqslant 0$ и $2x_{1} + 3x_{2} + x_{3} + 2x_{4} = 3$, то вершины это $(0,0,0,\frac{3}{2}), (0,0,3,0), (0,1,0,0), (\frac{3}{2},0,0,0)$
	\item
		Если зануляются 2 неравенства $x_{i} \geqslant 0$, а также выполнено $x_{1} - 2x_{2} + 4x_{3} - x_{4} = 1$, $2x_{1} + 3x_{2} + x_{3} + 2x_{4} = 3$, то есть задачу можно представить в виде 6 систем:
	\begin{itemize}
	\item
		\begin{gather*}
		\begin{cases}
			x_1 - 2x_2 = 1\\
			2x_1 + 3x_2 = 3
		\end{cases}\\
		x_1 = \frac{9}{7}\quad
		x_2 = \frac{1}{7}
		\end{gather*}
	\item
		\begin{gather*}
		\begin{cases}
			x_1 + 4x_3 = 1\\
			2x_1 + x_3 = 3
		\end{cases}\\
		x_1 = \frac{11}{7}\quad
		x_3 = -\frac{1}{7}\quad \text{не вершина, так как не выполнено } x_i \geqslant 0
		\end{gather*}
	\item
		\begin{gather*}
		\begin{cases}
			x_1 - x_4 = 1\\
			2x_1 + 2x_4 = 3
		\end{cases}\\
		x_1 = \frac{5}{4}\quad
		x_4 = \frac{1}{4}
		\end{gather*}
	\item
		\begin{gather*}
		\begin{cases}
			-2x_2 + 4x_3 = 1\\
			3x_2 + x_3 = 3
		\end{cases}\\
		x_2 = \frac{11}{14}\quad
		x_3 = \frac{9}{14}
		\end{gather*}
	\item
		\begin{gather*}
		\begin{cases}
			-2x_2 - x_4 = 1\\
			3x_2 + 2x_4 = 3
		\end{cases}\\
		x_2 = -5\quad
		x_4 = 9\quad \text{не вершина, так как не выполнено } x_i \geqslant 0
		\end{gather*}
	\item
		\begin{gather*}
		\begin{cases}
			4x_3 - x_4 = 1\\
			x_3 + 2x_4 = 3
		\end{cases}\\
		x_3 = \frac{5}{9}\quad
		x_4 = \frac{11}{9}
		\end{gather*}
	\end{itemize}
	\end{itemize}
	То есть у данного многогранника 11 вершин:
	\begin{itemize}
		\item $(0,0,0,0)$
		\item $(0,0, \frac{1}{4}, 0)$
		\item $(1,0,0,0)$
		\item $(0,0,0,\frac{3}{2})$
		\item $(0,0,3,0)$
		\item $(0,1,0,0)$
		\item $(\frac{3}{2},0,0,0)$
		\item $(\frac{9}{7},\frac{1}{7},0,0)$
		\item $(\frac{5}{4},0,0,\frac{1}{4})$
		\item $(0,\frac{11}{14},\frac{9}{14},0)$
		\item $(0,0,\frac{5}{9},\frac{11}{9})$
	\end{itemize}
	Однако, так как по условию $x_{1} - 2x_{2} + 4x_{3} - x_{4} = 1$, $2x_{1} + 3x_{2} + x_{3} + 2x_{4} = 3$, то из этих вершин подходят только
	\begin{itemize}
		\item $(\frac{9}{7},\frac{1}{7},0,0)$
		\item $(\frac{5}{4},0,0,\frac{1}{4})$
		\item $(0,\frac{11}{14},\frac{9}{14},0)$
		\item $(0,0,\frac{5}{9},\frac{11}{9})$
	\end{itemize}
\end{proof}
\vskip 0.6in



\begin{prob}
Дано число $n$. Найти
\begin{gather*}
	\sum\limits_{i=1}^{n} u_i + \sum\limits_{j=1}^{n} v_j \to \max
\end{gather*}
при условии
\begin{gather*}
	u_i + v_j \leqslant 2^{i+j},\quad \forall 1 \leqslant i, j \leqslant n
\end{gather*}
\end{prob}
\begin{proof}
	Заметим что $u_i + v_j \leqslant 2^{i+j} \Rightarrow u_i \leqslant 2^{i+j},\ v_j \leqslant 2^{i+j}$, откуда $u_i + v_1 \leqslant 2^{i+1} \Rightarrow u_i \leqslant 2^{i+1}$, аналогично $v_j \leqslant 2^{j+1}$. Пусть $\{u_1', u_2', \ldots, u_n', v_1', v_2', \ldots, v_n'\}$ -- значения коэффициентов при максимуме суммы, заметим, что если $u_1$ или $v_1 > 0$, то сделав замену $u_1' = v_1' = 0, v_i' = v_i + v_1\ \forall i \geqslant 2$, сумма всех элементов не уменьшится. Так как $u_1 = v_1 = 0$, то $u_i = v_i = 2^{i+1}\ \forall i \geqslant 2$ является решением, и максимальная сумма элементов $2 \cdot (0 + 2^3 + \ldots + 2^{n+1}) = 2 \cdot (2^{n+2} - 2^{3})$
\end{proof}
\vskip 0.6in



\begin{prob}
Матрица размера $m \times n$ называется латинским прямоугольником, если элементы каждой строки этой матриџы образуют перестановку чисел от 1 до $n$, и в каждом столбще все числа разные.
\noindent
Докажите, что латинский прямоугольник $m \times n$ всегда можно дополнить до латинского квадрата.
\end{prob}
\begin{proof}
	Рассмотрим двудольный граф, где вершины одной доли соответствуют колонкам, назовем их $c_{1}, \ldots, c_{n}$, а другой доли числам $n_{1}, \ldots, n_{n}$. Пусть ребро соединяет 2 вершины $c_{i}, n_{j}$, если в колонке $i$ не стоит число $j$, заметим, что если рассмотреть латинский прямоугольник с $m$ колонками и $n$ строками, то вершины будут иметь степень $n-m$, тогда, убирая ребра, можно ставить числа в квадрат, дополняя его. (по факту решение задачи эквивалентно лемме Холла, где одна доля соответствет колонкам, а другая числам)
\end{proof}
\vskip 0.6in



\begin{prob}
Докажите теорему, "двойственную" к теореме Дилуорса. В конечном, частично упорядоченном множестве мощность длиннейшей цепи равна мощности наименьшего разбиения на антицепи
\end{prob}
\begin{proof}
	Пусть $L(a)$ -- длина длинной цепи с началом в $a$, заметим, что если $a_1 > a_2$, то $L(a_1) < L(a_2)$, и если $L(a_1) = L(a_2)$, то $a_1, a_2$ несравнимы и $A_k = \{a| L(a) = k\}$ -- антицепь. Пусть длина наибольшей цепи $b$, тогда в ней есть все значения $L$ от $1$ до $b$ (больше $b$ быть не может, так как в таком случае рассматриваемое множество не является самой длинной цепью), тогда заметим, что $A_{1}, \ldots, A_{b}$ -- является наименьшим разбиением на атицепи.
\end{proof}
\vskip 0.6in



\begin{prob}
В последоватетьности из $n m + 1$ различных действитетьных чисел П.Эрдёш ищет "длинную цепь" - $n + 1$ элемент, идущие слева направо в порядке возрастания. Д.Секерёш, напротив, ищет "длинную антицепь" - $m + 1$ элемент, идущие слева направо в порядке убывания. Докажите, что хотя бы один из них преуспеет.
\end{prob}
\begin{proof}
	Рассмотрим функцию $f(x),\ x \in [1, mn+1]$, значения которой равны длинам возрастающих последовательностей, начинающихся с числа на позиции $x$. Допустим цепи длины $n+1$ нет, то есть значения $f(x)$ лежат в $[0, n]$. Тогда по принципу Дирихле для какого-то значения существует хотя бы $m+1$ число $x_i$, такое что $f(x_1) = \ldots = f(x_{m+1})$, заметим, что эта последовательность является убывающей, так как иначе, если $\exists i,j: x_{i} \leqslant x_{j}$, то возрастающую последовательность можно продолжить на 1 элемент ($x_{j}$), а следовательно для какого-то $x_{k}$ равенство $f(x_1) = \ldots = f(x_{m+1})$ не будет выполнено.
\end{proof}



\newpage
	\section{ЛИСТ 3}
		\subsection{1}
		А)\\
		\begin{gather*}
			\lim\limits_{x \to \infty} x(\frac{\pi}{2} - \arctan(x)) = 
			\lim\limits_{x \to \infty} x(\text{arccot}(x)) = 
			\lim\limits_{x \to \infty} \frac{\text{arccot}(x)}{\frac{1}{x}} = \\
			\lim\limits_{x \to \infty} \frac{\frac{x}{1 + x^2}}{\frac{1}{x}} = 
			\lim\limits_{x \to \infty} \frac{1}{(1 + x^2)\frac{1}{x^2}} = 
			\lim\limits_{x \to \infty} \frac{1}{\frac{1}{x^2} + 1} = 
			1
		\end{gather*}
		Б)\\
		\begin{comment}
		\begin{gather*}
			\lim\limits_{x \to 0} \frac{\cos\sin(x) - \cos(x)}{x^4} = 
			\lim\limits_{x \to 0} \frac{-\sin\sin(x) \cdot \cos(x) + \sin(x)}{4x^3} = 
			\lim\limits_{x \to 0} \frac{-\cos\sin(x) \cdot \cos^2(x) + \sin(x) \cdot \sin\sin(x) + \cos(x)}{12x^2} =\\
			\lim\limits_{x \to 0} \bigg( \frac{\cos(x) - \cos(x)^2 \cdot \cos\sin(x)}{12x^2} + \frac{\sin(x) \cdot \sin\sin(x)}{12x^2} \bigg) =\\
			\lim\limits_{x \to 0} \frac{\cos(x) - \cos^2(x) \cdot \cos\sin(x)}{12x^2} +
			\lim\limits_{x \to 0} \frac{\sin(x) \cdot \sin\sin(x)}{12x^2} = \\
			\lim\limits_{x \to 0} \frac{-\sin(x) + 2\cos(x) \cdot \sin(x) \cdot \cos\sin(x) + \sin\sin(x) \cdot \cos^3(x)}{24x} + 
			\lim\limits_{x \to 0} \frac{\cos(x) \cdot \sin\sin(x) + \sin(x) \cdot \cos\sin(x) \cdot \cos(x)}{24x} =\\
			\lim\limits_{x \to 0} -\frac{\sin(x)}{24x} +
			\lim\limits_{x \to 0} \frac{2\cos(x)}{24} \cdot
			\lim\limits_{x \to 0} \frac{\sin(x)}{x} \cdot
			\lim\limits_{x \to 0} \cos\sin(x) +
			\lim\limits_{x \to 0} \frac{\sin\sin(x) \cdot \cos^3(x)}{24x} +\\
			\lim\limits_{x \to 0} \frac{-\sin(x) \cdot \sin\sin(x) + \cos^2(x) \cdot \cos\sin(x) + \cos(x) \cdot \cos\sin(x) \cdot \cos(x)}{24} =\\
			-\frac{1}{24} + \frac{1}{12} + 
			\lim\limits_{x \to 0} \frac{\cos\sin(x) \cdot \cos^4(x) + \sin\sin(x) \cdot 3\sin^2(x)}{24} +\\
			\lim\limits_{x \to 0} \bigg( \frac{-\sin(x) \cdot \sin\sin(x) + \cos^2(x) \cdot \cos\sin(x)}{24} + \\ \frac{\cos(x) \cdot \cos\sin(x) \cdot \cos(x) - \sin(x) \cdot \sin\sin(x) \cdot \cos^2(x) - \cos\sin(x) \cdot \sin(x)}{24} \bigg)=\\
			-\frac{1}{24} + \frac{1}{12} + \frac{1}{24} + \frac{1+1}{24} = \frac{1}{12} + \frac{1}{12} = \frac{1}{6}
		\end{gather*}
		\end{comment}
		\begin{gather*}
			\lim\limits_{x \to 0} \frac{\cos\sin(x) - \cos(x)}{x^4} = 
			\lim\limits_{x \to 0} \frac{\cos(x - \frac{x^3}{3!} + \frac{x^5}{5!} - \ldots) - \cos(x)}{x^4} = \\
			\frac{1 - \frac{(x - \frac{x^3}{3!} + \ldots)^2}{2!} + \frac{(x - \ldots)^4}{4!} - \ldots - (1 - \frac{x^2}{2!} + \frac{x^4}{4!} - \ldots)}{x^4} = \\
			\lim\limits_{x \to 0} \frac{x^4 \cdot (\frac{1}{3!} + \frac{1}{4!} - \frac{1}{4!}) + x^2 \cdot (-\frac{1}{2!} + \frac{1}{2!}) + (\text{степени $>4$})}{x^4} = \\
			\lim\limits_{x \to 0} \frac{x^4 \cdot \frac{1}{3!} + (\text{степени $>4$})}{x^4} = 
			\frac{1}{6}
		\end{gather*}
		В)\\
		\begin{gather*}
			\lim\limits_{x \to 0} (\cot(x))^{\sin(x)} = 
			\lim\limits_{x \to 0} e^{\sin(x) \ln \cot(x)} = 
			e^{\lim\limits_{x \to 0} \sin(x) \ln \cot(x)}
		\end{gather*}
	 	рассмотрим $\lim\limits_{x \to 0} \bigg( \sin(x) \ln \cot(x) \bigg)$:
	 	\begin{gather*}
	 		\lim\limits_{x \to 0} \bigg( \sin(x) \ln \cot(x) \bigg) = 
	 		\lim\limits_{x \to 0} \frac{\ln \cot(x)}{\frac{1}{\sin(x)}} =
	 		\lim\limits_{x \to 0} \bigg( \frac{1}{\cot(x)} \cdot -\frac{1}{\sin^2(x)} \cdot \frac{1}{\frac{1}{\sin^2(x)} \cdot \cos(x)} \bigg) = 
	 		\lim\limits_{x \to 0} \frac{\sin(x)}{\cos^2(x)} = 
	 		\frac{0}{1} = 
	 		0
	 	\end{gather*}
		вернемся к изначальной задаче:
		\begin{gather*}
			e^{\lim\limits_{x \to 0} \sin(x) \ln \cot(x)} = e^{0} = 1
		\end{gather*}
		Ответ: а)$1\quad$ б)$\frac{1}{6}\quad$ в)$1$
		
		\subsection{2}
		
		\subsection{3}
		А)Б)\\
		Рассмотрим $\alpha = \frac{1}{2}$:\\
		Заметим, что 
		\begin{gather*}
			\lim\limits_{x \to 0} \frac{x^2 - \sin^2(x)}{x^2 \cdot \sin^2(x)} = \frac{1}{3} \quad \text{по правилу Лопиталя}
		\end{gather*}
		Откуда
		\begin{gather*}
			\lim\limits_{n \to \infty} \bigg( \frac{1}{a^2_{n+1}} - \frac{1}{a^2_{n}} \bigg) = \frac{1}{3}
		\end{gather*}
		Теперь докажем что:
		\begin{gather*}
			\lim\limits_{n \to \infty} na^2_n = 3
		\end{gather*}
		Доказательство:\\
		Докажем более общий факт:\\
		если
		\begin{gather*}
			\lim\limits_{n \to \infty} (a_{n+1} - a_{n}) = a
		\end{gather*}
		то
		\begin{gather*}
			\lim\limits_{n \to \infty} \frac{a_n}{n} = a
		\end{gather*}
		Это можно доказать с помощью теоремы Штольца, в которой $x_n$ заменим на $a_{n}$, а $y_n$ заменим на $n$:\\
		Формулировка:\\
		Пусть $x_n$ и $y_n$ -- две последовательности вещественных чисел, причём $y_n$ положительна, неограничена и строго возрастает (хотя бы начиная с некоторого члена).\\
		Тогда, если существует предел
		$\lim\limits_{n \to \infty} \frac{x_n - x_{n-1}}{y_n - y_{n-1}}$
		то существует и предел
		$\lim\limits_{n \to \infty} \frac{x_n}{y_n}$
		причём эти пределы равны.\\
		Доказательство:\\
		Допустим сначала, что предел равен конечному числу $L$, тогда для любого заданного $\varepsilon > 0$ существует такой номер $N > 0$, что при $n > N$ будет иметь место:
		\begin{gather*}
			L - \frac{\varepsilon}{2} < \frac{x_n - x_{n-1}}{y_n - y_{n-1}} < L + \frac{\varepsilon}{2}
		\end{gather*}		
		Значит, для любого $n > N$ все дроби:
		\begin{gather*}
			\frac{x_{N+1} - x_N}{y_{N+1} - y_N}, \frac{x_{N+2} - x_{N+1}}{y_{N+2} - y_{N+1}},...,\frac{x_n - x_{n-1}}{y_n - y_{n-1}}
		\end{gather*}
		лежат между этими же границами. Так как знаменатели этих дробей положительны (в силу строго возрастания последовательности $y_n$), то, по свойству медианты, между теми же границами содержится и дробь:
		\begin{gather*}
			\frac{x_n - x_N}{y_n - y_N}
		\end{gather*}
		числитель которой есть сумма числителей написанных выше дробей, а знаменатель — сумма всех знаменателей. Итак, при $n > N$:
		\begin{gather*}
			\left| \frac{x_n - x_N}{y_n - y_N} - L \right| < \frac{\varepsilon}{2}
		\end{gather*}
		Теперь рассмотрим следующее тождество (проверяемое непосредственно):
		\begin{gather*}
			\frac{x_n}{y_n} - L = \frac{x_N - L y_N}{y_n} + \left( 1 - \frac{y_N}{y_n} \right) \left( \frac{x_n - x_N}{y_n - y_N} - L \right)
		\end{gather*}
		откуда имеем
		\begin{gather*}
			\left| \frac{x_n}{y_n} - L \right| \le \left| \frac{x_N - L y_N}{y_n} \right| + \left| \frac{x_n - x_N}{y_n - y_N} - L \right| 
		\end{gather*}
		Второе слагаемое при $n > N$ становится меньше $\frac{\varepsilon}{2}$, первое слагаемое также станет меньше $\frac{\varepsilon}{2}$, при $n > M$, где $M$ — некоторый достаточно большой номер, в силу того, что $y_n \to +\infty$. Если взять $M > N$, то при $n > M$ будем иметь
		\begin{gather*}
			\left | \frac{x_n}{y_n} - L \right | < \varepsilon
		\end{gather*}
		что и доказывает наше утверждение.
		
		Случай бесконечного предела можно свести к конечному. Пусть, для определённости:
		\begin{gather*}
			\lim\limits_{n \to \infty} \frac{x_n - x_{n-1}}{y_n - y_{n-1}} = +\infty
		\end{gather*}
		из этого следует, что при достаточно больших $n$:
		$x_n - x_{n-1} > y_n - y_{n-1}$ и
		$\lim\limits_{n \to \infty} x_n = +\infty$,
		причём последовательность $x_n$ строго возрастает (начиная с определённого номера). В этом случае, доказанную часть теоремы можно применить к обратному отношению $y_n \over x_n$:
		\begin{gather*}
			\lim\limits_{n \to \infty} \frac{y_n}{x_n} = \lim\limits_{n \to \infty} \frac{y_n - y_{n-1}}{x_n - x_{n-1}} = 0
		\end{gather*}
		откуда и следует, что:
		\begin{gather*}
			\lim\limits_{n \to \infty} \frac{x_n}{y_n} = + \infty
		\end{gather*}
		Теорема доказана, откуда $\lim\limits_{x \to \infty} na^2_n = 3$, то есть $\lim\limits_{x \to \infty} \sqrt{n}a_n = \sqrt{3}$, что и требовалось
		
		\subsection{4}
		А)\\
		Заметим, что у уравнений вида $x^3 - kx - 1 = 0$ ровно $1$ корень при всех отрицательных $k$ (так как функция $x^3 - kx - 1$ будет монотонной и неограниченной), а также при всех $k < 1$, так как на интервале $[-1, 0)$ выражение $x^3 - kx$ будет меньше $1$, в силу того, что $kx \leqslant 1$ и $x^3 > 0$. На интервале $(-\infty, 1)$ функция монотонно возрастает, так как производная ($= 3x^2 - k$) больше $0$.\\
		Поэтому есть только одна функция, удовлетворяющая условию при $a = k$. Докажем, что она непрерывна.\\
		Заметим, что она монотонна, ведь при замене $k$ на $k+d$ выражение $\alpha^3 - (k+d)\alpha - 1$ уменьшается, откуда, единственный корень $\beta$ уравнения $x^3 - (k+d)x - 1 = 0$ больше, чем $\alpha$. При этом для любого значения $\gamma$, лежащего между $\alpha$ и $\beta$, существует $y$, такой что $\gamma$ является корнем соответствующего уравнения (потому что $\gamma^3 - y\gamma - 1$ непрерывно, и при $y = k$ оно меньше $0$, а при $y = k+d$ -- наоборот), откуда и следует непрерывность функции.
		\\
		Б)\\
		\\
		В)\\
		\\
		
		\subsection{5}
		Пусть $m = 27$, тогда задача имеет вид: $\ln\cos \frac{1}{m + 5}$.\\
		Посчитаем сперва $\cos \frac{1}{m + 5}$
		\begin{gather*}
			\cos x =  1 - \frac{x^2}{2!} + \frac{x^4}{4!} - \cdots = \sum_{n=0}^{\infty} {(-1)^n}\frac{x^{2n}}{(2n)!}, x\in\mathbb{C}\\
			\cos(\frac{1}{32}) = \sum_{n=0}^{\infty} {(-1)^n}\frac{\frac{1}{32}^{2n}}{(2n)!} = 1 - \frac{x^2}{2!} + \frac{x^4}{4!} =
			1 - \frac{\frac{1}{32}^2}{2!} + \frac{\frac{1}{32}^4}{4!} - \ldots \approx \\
			1 - \frac{\frac{1}{32}^2}{2!} + \frac{\frac{1}{32}^4}{4!} \approx 0.9995117
		\end{gather*}
		А теперь посчитаем логарифм от полученного значения
		\begin{gather*}
			\ln(1+x) = x - \frac{x^2}{2} + \frac{x^3}{3} - \ldots = \sum\limits^{\infty}_{n=0} \frac{(-1)^n x^{n+1}}{(n+1)} =  \sum\limits^{\infty}_{n=1} \frac{(- 1)^{n-1}x^n}{n} \quad \text{для $-1 < x < 1$}\\
			\ln(0.9995117) = \ln(1 - 0.0004883) = \sum\limits^{\infty}_{n=1} \frac{(- 1)^{n-1} 0.0004883^n}{n} = \\
			0.0004883 - \frac{0.0004883^2}{2} + \frac{0.0004883^3}{3} - \ldots \approx 0.0004883 - \frac{0.0004883^2}{2} + \frac{0.0004883^3}{3} \approx \\
			0.0004883
		\end{gather*}
		Откуда первые две значащие цифры это $4,\ 8$
		
		\subsection{6}
		Заметим, что $f'(x) = \alpha \ \Leftrightarrow \ \lim\limits_{t \to 0} \frac{1}{t} (f(t+x) - f(x)) = \alpha$, откуда в некоторой окресности нуля выражение $\frac{1}{t} (f(t+x) - f(x)) > 0$, поэтому при $t > 0:\ f(t+x) - f(x) > 0$ и наоборот. Но это и означает, что функция возрастает.
		
		\subsection{7}
		А)\\
		Возьмем минимальное $y = c$ и будем двигать вверх эту горизонтальную прямую. разность площадей верхнего и нижнего многоугольника меняется непрерывно, так как это разность непрерывных функций ( пусть $f_1$ -- площадь нижнего многоугольника, а $f_2$ -- площадь верхнего, тогда обе эти функции непрерывны, откуда $f = (f_1 - f_2)$ также непрерывна). Заметим, что если площадь всего многоугольника $S$ то значения $f$ лежат в $[S,\ -S]$. Тогда, так как $0 \in [S,\ -S]$ то $\exists c:\ f(c) = 0$\\
		\\
		Б)\\
		Пусть $S$ -- окружность с центром $(0,0) \in \mathbb{R}^2$, внутри которой лежат $M_1$ и $M_2$ ( $M_1$ и $M_2$ ограничены $\Rightarrow$ она существует ). Изменим масштаб так, чтобы диаметр $S$ стал равен $1$. Для $\forall x\in S$ рассмотрим диаметр $D_x$ проходящий через $x$. Пусть $L_t$ -- перпендикуляр к $D_x$, проходящий через точку на $D_x$, расположенную на расстоянии $t$ от $x$.\\
		Пусть $S_1(t)$ -- площадь части $M_1$, лежащей по одну сторону от $L_t$, что и $x$. Аналогично определим $S_2(t)$ для $M_2$. Заметим, что $S_1(0) = S_2(1) = 0$. Очевидно, что $S_1(t),\ S_2(t)$ -- непрерывные функции, отображающие $l$ в $\mathbb{R}$. Пусть $f:\ l \to \mathbb{R}$ это $f(t) = S_1(t) - S_2(t)$, это непрерывная функция и $f(0)f(1) \leqslant 0$. Откуда существует $t \in l:\ f(t)=0$ либо на отрезке $[a,b]$, либо в одной точке $c$. В первом случае определим $h_1(x) = \frac{a+b}{2}$, во втором случае $h_1(x) = c$.\\
		То есть перпендикуляр к $D_x$, проходящий через точку на $D_x$, расстояние от которой до $x$ равно $h_1(x)$, делит площадь $M_1$ пополам. Заметим, что $h_1(-x) = 1-h_1(x)$ и $h_1:\ S \to l$ -- непрерывная функция.\\
		Аналогично определим $h_2:\ S \to l$, где вместо $M_1$ действие происходит на $M_2$.\\
		Теперь определим $h(x)= h_1(x) - h_2(x)$. Так как $h_1(x)$ и $h_2(x)$ непрерывны, то и $h(x)$ непрерывна. Заметим, что $h(x) = -h(-x)\ \forall x\in S$. Но также есть и точка $y:\ h(y) = h(-y)$. Значит $h(y) = 0$ и $h_1(y) = h_2(y)$ и перпендикуляр к $D_y$, расстояние от которой до $y$ равно $h_1(y)$, делит пополам $M_1$ и $M_2$, что и требовалось доказать.
		\\
		
		\subsection{8}
		
		\subsection{9}
		А)\\
		\\
		Б)\\
		\\
		
		\subsection{10}
		$f:\ (p,q) \to \mathbb{R}$ выпуклая и дифференцируемая
		А)$f:\ (p,q) \to \mathbb{R}$ выпуклая\\
		Любая касательная не выше графика 
		$l(x) = f(a) + f'(a)(x - a)$, также $f(x)$ выпуклая $\Rightarrow \ f''(x) > 0$
		\begin{gather*}
			f(x) - l(x) = f(x) - f(a) - f'(a)(x - a)
		\end{gather*}
		По теорема Лагранжа на $(a,x)\ \exists c:\ \frac{f(x) - f(a)}{x - a} = f'(c)$
		\begin{gather*}
			f(x) - l(x) = (f'(c) - f'(a))(x - a)\\
			\\
			1)\ a = x\ f(x) = l(x)\\
			2)\ a < x\ f'(a) \leqslant f(c) \ \Rightarrow \ f(x) - l(x) \geqslant 0\\
			3)\ a > x\ \text{аналогично}\\
		\end{gather*}
		Что и требовалось
		\\
		Б) $f'(x) > 0\quad f'(x_1) \leqslant f'(x_2)$ если $x_1 \leqslant x_2$\\
		\begin{gather*}
			\alpha_1 x_1 + \alpha_2 x_2 = x \quad \alpha_1 + \alpha_2 = 1 \ \Rightarrow \ \alpha_1 = \frac{x_2 - x}{x_2 - x_1} \quad \alpha_2 = \frac{x - x_1}{x_2 - x_1}\\
			f(\alpha_1 x_1 + \alpha_2 x_2) = f(x) \leqslant \frac{x_2 - x}{x_2 + x_1} f(x_1) + \frac{x - x_1}{x_2 + x_1} f(x_2)\\
			(x_2 - x) + (x - x_1) = x_2 - x_1\\
			f(x)(x_2 - x_1) = f(x_1) (x_2 - x) + f(x_2)(x - x_1) \leqslant (x_2 - x) f(x_1) + (x - x_1) f(x_2)\\
		\end{gather*}
		Откуда
		\begin{gather*}	
			\frac{f(x) - f(x_1)}{x - x_1} \leqslant \frac{f(x_2) - f(x)}{x_2 - x} \ \Rightarrow \\
			x \to x_1: \quad f'(x_1) = \frac{f(x_1) - f(x_1)}{x_1 - x_1} \leqslant \frac{f(x_2) - f(x_1)}{x_2 - x_1}\\
			x \to x_2: \quad \frac{f(x_2) - f(x_1)}{x_2 - x_1} \leqslant \frac{f(x_2) - f(x_2)}{x_2 - x_2} = f'(x_2)\\
		\end{gather*}
		Следовательно
		\begin{gather*}
			f'(x_1) \leqslant \frac{f(x_2) - f(x_1)}{x_2 - x_1} \leqslant f'(x_2)
		\end{gather*}
		Что и требовалось\\
		\\
		В)\\
		Обратно
		\begin{gather*}
			a < x_1 < x_2 < b \ \text{откуда по Т.Лагранжа о среднем значении} \\ \frac{f(x) - f(x_1)}{x - x_1} = f'(c_1);\ \frac{f(x_2) - f(x)}{x_2 - x} = f'(c_2),\ \text{где}\ x_1<c_1<x<c_2<x_2\\
			f'(c_1) \leqslant f'(c_2) \ \Rightarrow \ \frac{f(x) - f(x_1)}{x - x_1} \leqslant \frac{f(x_2) - f(x)}{x_2 - x}\\
			\alpha x_1 + (1 - \alpha) x_2 = \alpha (x_1 - x_2) + x_2,\ \text{тогда если}\ x_1 < x_2,\ \text{то и}\ x_1<x<x_2\\
		\end{gather*}
		Что и требовалось
		\\
		
		\subsection{11}
		А)\\
		\begin{gather*}
			f\bigg( \frac{x_1}{n} + \frac{n-1}{n} \bigg( \frac{x_2 + \ldots + x_n}{n-1} \bigg) \bigg) \leqslant
			\frac{f(x_1)}{n} + \frac{n-1}{n} f \bigg( \frac{x_2 + \ldots + x_n}{n-1} \bigg) \leqslant \\
			\frac{f(x_1)}{n} + \frac{n-1}{n} f \bigg( \frac{x_2}{n-1} + \frac{n-2}{n-1}\frac{x_3 + \ldots + x_n}{n-2} \bigg) \leqslant
			\ldots \leqslant 
			\frac{f(x_1)}{n} + \frac{f(x_2)}{n} + \frac{n-2}{n} f \bigg( \frac{x_3 + \ldots + x_n}{n-2} \bigg) \leqslant \\
			\ldots \leqslant
			\frac{f(x_1)}{n} + \ldots + \frac{f(x_{n-2})}{n} + \frac{2}{n} f \bigg( \frac{x_{n-1} + x_{n}}{2} \bigg) \leqslant
			\frac{f(x_1)}{n} + \ldots + \frac{f(x_{n-2})}{n} + \frac{2}{n} \bigg( f  \frac{x_{n-1}}{2} + f \frac{x_{n}}{2} \bigg) = \\
			\frac{f(x_1) + \ldots + f(x_n)}{n}	
		\end{gather*}
		\\
		Б)\\
		Используем неравенство из (а) для выпуклой формы\\
		$f(x) = x^2$
		\begin{gather*}
			\bigg( \frac{x_1 + \ldots + x_n}{n} \bigg) \leqslant \frac{x^2_1 + \ldots + x^2_n}{n}\\
			\frac{x_1 + \ldots + x_n}{n} \leqslant \sqrt{\frac{x^2_1 + \ldots + x^2_n}{n}}
		\end{gather*}
		\\
		В)\\
		$f(x) = \log_a(x)$
		\begin{gather*}
			\log \bigg(\frac{x_1 + \ldots + x_n}{n} \bigg) \geqslant \frac{\log_a(x_1) + \ldots + \log_a(x_n)}{n}\\
			\text{так как $\log_a(x)$ - вогнутая функция}\\
			\frac{\log_a(x_1) + \ldots + \log_a(x_n)}{n} = \log_a(x_1 \cdot \ldots \cdot x_n)^{\frac{1}{n}} =
			\log_a{\sqrt{x_1 \cdot \ldots \cdot x_n}}
		\end{gather*}
		Откуда:
		\begin{gather*}
			\frac{x_1 + \ldots + x_n}{n} \geqslant \sqrt{x_1 \cdot \ldots \cdot x_n}
		\end{gather*}
		что и требовалось
		\\
		
		\subsection{12}
		По условию окружность должна лежать в $y \geqslant x^2$, то есть внутри параболы $y = x^2$. Также она доржна содержать точку $(0,0)$. Тогда в силу того, что $y = x^2$ симметрична относительно $x = 0$, и на $y = x^2$ также лежит и требуемая точка $(0,0)$, окружность касается параболы в точке $(0,0)$. Тогда уравнение окружности имеет вид $(x - a)^2 - (y - b)^2 = r^2$, где $a = 0,\ b = r$. То есть $x^2 + (y - r)^2 = r^2 \ \Leftrightarrow \ x^2 + y^2 - 2yr = 0$. Откуда $r = \frac{x^2 + y^2}{2y} \leqslant \frac{y + y^2}{2y} = \frac{y + 1}{2}\quad y\in [0; +\infty )$, следовательно $r \leqslant \frac{1}{2}$. Заметим, что при $r = \frac{1}{2}$ требования условия выполнены.
		
		
		\subsection{13}
		
\section{Problem List 4}

\begin{prob}
Construct an algorithm that convert a binary search tree into a linked list such that the linked list is ordered descending.
\end{prob}
\vskip 0.2in
\begin{proof}
\begin{itemize}
    \item[] 
    \item[1] Совершим RNL обход двоичного дерева поиска
    \item[2] Во время обхода, изменим указатель \texttt{LeftChild} на предыдущий узел в связанном списке, а \texttt{RightChild} на следующий
    \item[3] Будем отслеживать положение начала и конца списка по мере обхода
\end{itemize}

\begin{lstlisting}[language=Python]
def bst_to_list(root):
    head = None
    tail = None

    def reverse_inorder(node):
        nonlocal head, tail
        if node:
            reverse_inorder(node.right)
            if tail:
                tail.left = node
                node.right = tail
            else:
                head = node
            tail = node
            reverse_inorder(node.left)

    reverse_inorder(root)
    return head
\end{lstlisting}

Узлы бинарного дерева поиска имеют указатели \texttt{LeftChild} и \texttt{RightChild}, а узлы связанного списка также имеют указатели \texttt{LeftChild} и \texttt{RightChild}. Указатель \texttt{LeftChild} узла связанного списка указывает на предыдущий узел в списке, а указатель \texttt{RightChild} указывает на следующий узел в списке.
\end{proof}
\vskip 0.6in



\begin{prob}
A concatenate operation takes two sets $S_1$ and $S_2$, where every key in $S_1$ is smaller than any key in $S_2$, and merges them together. Give an algorithm to concatenate two binary search trees into one binary search tree. The worst-case running time should be $O(h)$, where $h$ is the minimal height of the two trees; heights of the trees are known.
\end{prob}
\vskip 0.2in
\begin{proof}
\begin{itemize}
    \item[]
    \item[1] Найдем самый большой ключ в первом дереве $S_1$, проходя по \texttt{RightChild} начиная от корня, пока это возможно. Назовем этот узел $x$.
    \item[2] Удалим $x$ из $S_1$.
    \item[3] Сделаем $x$ новым корнем объединенного дерева.
    \item[4] Сделаем корень $S_1$ левым ребенком $x$ и корень $S_2$ павым ребенком $x$.
\end{itemize}

Этот алгоритм занимает $O(h)$ времени, так как поиск и удаление самого большого ключа в $S_1$ занимает $O(h)$, где $h$ — высота $S_1$. Создание $x$ нового корня и прикрепление корней $S_1$ и $S_2$ в качестве его детей занимает $O(1)$.
\end{proof}
\vskip 0.6in



\begin{prob}
Describe how to modify any balanced tree data structure such that search, insert, delete, minimum, and maximum still take $O(\log n)$ time each, but successor and predecessor now take $O(1)$ time each. Which operations have to be modified to support this?
\end{prob}
\vskip 0.2in
\begin{proof}
Дополним вершины дерева указателями на прошлую и следующую вершины по порядку, и будем обновлять эти указатели при каждом изменении дерева (вставка, удаление)

\begin{itemize}
\item[]
\item[search] Без изменений
\item[insert] Обновим указатели на прошлую и следующую вершины (successor и predecessor) для всех узлов, затронутых операцией вставки. Для этого проведем обычную вставку, затем обновим указатели для всех вершин на пути от корня к новой вершине. $O(\log n)$ - время вставки и $O(\log n)$ - время обновления указателей
    \begin{lstlisting}
    function insert(node, value):
        // Perform a standard insert operation
        // ...
        // Update successor and predecessor pointers
        while node.parent is not null:
            if (node.parent.successor is null or
                    value < node.parent.successor.value):
                node.parent.successor = node
            if (node.parent.predecessor is null or
                    value > node.parent.predecessor.value):
                node.parent.predecessor = node
            node = node.parent
    \end{lstlisting}
\item[delete] Обновим указатели на прошлую и следующую вершины (successor и predecessor) для всех узлов, затронутых операцией удаления. Для этого проведем обычное удаление, затем обновим указатели для всех вершин на пути от корня к новой вершине. $O(\log n)$ - время удаления и $O(\log n)$ - время обновления указателей
    \begin{lstlisting}
    function delete(node):
        // Perform a standard delete operation
        // to remove the node from the tree
        // ...
        // Update successor and predecessor pointers
        if node.parent is not null:
            if node.parent.successor == node:
                node.parent.successor = node.successor
            if node.parent.predecessor == node:
                node.parent.predecessor = node.predecessor
            node = node.parent
            while node is not null:
                if (node.successor == null or
                        node.right.predecessor.value < node.successor.value):
                    node.successor = node.right.predecessor
                if (node.predecessor == null or
                        node.left.successor.value > node.predecessor.value):
                    node.predecessor = node.left.successor
                node = node.parent
    \end{lstlisting}

\item[minimum] Без изменений
\item[maximum] Без изменений
\item[successor] Следуем указателю, $O(1)$
    \begin{lstlisting}
    function successor(node):
        if node.successor is not null:
            return node.successor
        while node.parent is not null and node == node.parent.right:
            node = node.parent
        return node.parent
    \end{lstlisting}
\item[predecessor] Следуем указателю, $O(1)$
    \begin{lstlisting}
    function predecessor(node):
        if node.predecessor is not null:
            return node.predecessor
        while node.parent is not null and node == node.parent.left:
            node = node.parent
        return node.parent
    \end{lstlisting}
\end{itemize}
Дополняя вершины дерева указателями и обновляя эти указатели во время вставки и удаления, мы можем изменить любое дерево таким образом, что поиск, вставка, удаление, минимум и максимум работали $O(\log n)$
\end{proof}
\vskip 0.6in



\begin{prob}
The keys of the binary search tree had been written to the array according to the preorder traversal (NLR). So, for a tree from the picture, the array is $[4,2,1,3,5]$. The array for a tree is stored in RAM. Construct an $O(n)$ algorithm that gets on the input keys $u$ and $v$ and outputs whether $v$ is a descendent of $u$ in the original tree.
\end{prob}
\vskip 0.2in
\begin{proof}
\begin{itemize}
\item[]
\item[1] Создадим хэш-таблицу для хранения индексов всех ключей
\item[2] Найдем индексы ключей $u$ и $v$ в массиве с помощью хэш-таблицы
\item[3] Если индекс $u$ больше индекса $v$, то $v$ не может быть потомком $u$
\item[4] Найдем первый ключ в массиве, больший $u$ (назовем его $x$). Этот ключ - корень правого поддерева $u$
\item[5] Если индекс $v$ меньше индекса $x$, то $v$ является потомком $u$, в противном случае это не так
\end{itemize}

При обходе NLR все потомки вершины будут записаны в массиве после расмматриваемой вершины и до следующей вершины, не являющейся потомком. Найдя первый элемент, больший $u$, мы можем определить диапазон индексов, которые принадлежащих поддереву с корнем $u$. Если индекс $v$ попадает в этот диапазон, то $v$ является потомком $u$.
\vskip 0.1in
Сложность $\mathcal{O}(n)$, так как для создания хэш-таблицы требуется $\mathcal{O}(n)$ и $\mathcal{O}(1)$ для каждого поиска в ней.
\end{proof}
\vskip 0.6in



\begin{prob}
An array of size $n$ contains elements from the range from 1 to $n-1$ (may be not all of them). Construct an algorithm that finds a duplicate satisfying the conditions (subproblems are separate)
\begin{itemize}
\item[]
\item[1] Algorithm runs in $O(n)$
\item[2] Algorithm runs in $O(n)$ and uses $O(1)$ RAM; the array is available in read-only mode. It is known that the array contains all numbers from 1 to $n-1$
\item[3] Algorithm runs in $O(n)$ and uses $O(1)$ RAM; the array is available in read-only mode. You can solve only this subproblem
\end{itemize}
\end{prob}
\vskip 0.2in
\begin{proof}
Можно рассматривать массив как связанный список, где значение в каждом индексе представляет следующий узел в списке. Мы начинаем с первого элемента и перемещаемся по элементам, пока не найдем цикл. Точкой входа в цикл является дублирующийся элемент
\vskip 0.1in
Для обнаружения цикла мы можем использовать два указателя - медленный и быстрый. Медленный указатель делает 1 шаг за раз, быстрый 2. Если есть цикл, быстрый указатель догонит медленный и они встретятся внутри цикла
\vskip 0.1in
Чтобы применить эту концепцию к массиву, мы можем использовать значения в массиве в качестве указателей на следующий элемент. В частности, для каждого значения в массиве мы можем рассматривать его как указатель на элемент с соответсвующим индексом. Например, если $\text{arr}[i] = j$, то элемент $i$ указывает на элемент $j$

\begin{lstlisting}
function find_duplicate(arr):
    n = length(arr)
    slow = arr[0]
    fast = arr[arr[0]]

    # Phase 1: Find the meeting point
    while slow != fast:
        slow = arr[slow]
        fast = arr[arr[fast]]

    # Phase 2: Find the entry point to the cycle
    slow = 0
    while slow != fast:
        slow = arr[slow]
        fast = arr[fast]

    return slow
\end{lstlisting}

В фазе 1 мы используем медленный и быстрый указатели, чтобы найти точку встречи внутри цикла. Если цикла нет, быстрый указатель в конечном итоге достигнет конца массива и алгоритм завершится. В противном случае медленный и быстрый указатели в конечном итоге встретятся при некотором индексе k внутри цикла
\vskip 0.1in
Во 2 фазе мы сбрасываем медленный указатель в начало массива и перемещаем оба указателя в одном темпе, пока они не встретятся
\vskip 0.1in
Этот алгоритм работает $O(n)$ и использует $O(1)$ памяти, так как мы используем только 2 указателя
\end{proof}
\newpage
	{\large \hspace{3cm} \begin{center} Домашнее задание 18 $\bullet$ Мозговой Владислав \end{center} }
	\vspace{-1.5ex}
	\hrulefill
	
	\fontsize{12pt}{4.5mm}\selectfont
	\vspace{-3ex}
	\hrulefill
	\newline

	\section{}
		\subsection*{\textbf{Задача 1}}
		Пусть $G$ -- факторгруппа свободной абелевой группы $\mathbb{Z}^{4}$ по подгруппе порождённой строками матрицы
		\begin{gather*}
			\begin{bmatrix}
				{-130} & {245} & {0} & {120} \\
				{-10} & {5} & {0} & {0} \\
				{60} & {-80} & {20} & {-60} \\
				{-60} & {120} & {0} & {60}
			\end{bmatrix}
		\end{gather*}
		\subsubsection*{\textbf{А}}
		\textbf{Условие}\\
		Разложите G в прямую сумму циклических\\
		\\
		\textbf{Решение}\\
		$A = \langle a_1, a_2, a_3, a_4 \rangle$
		\begin{gather*}
			\begin{bmatrix}
				{-130} & {245} & {0} & {120} \\
				{-10} & {5} & {0} & {0} \\
				{60} & {-80} & {20} & {-60} \\
				{-60} & {120} & {0} & {60}
			\end{bmatrix}
			=
			\begin{bmatrix}
				{-130} & {245} & {0} & {120} \\
				{-10} & {5} & {0} & {0} \\
				{0} & {0} & {20} & {0} \\
				{-60} & {120} & {0} & {60}
			\end{bmatrix}
			=
			\begin{bmatrix}
				{-10} & {5} & {0} & {120} \\
				{-10} & {5} & {0} & {0} \\
				{0} & {0} & {20} & {0} \\
				{0} & {0} & {0} & {60}
			\end{bmatrix}
			=\\
			\begin{bmatrix}
				{-5} & {5} & {0} & {0} \\
				{-5} & {5} & {0} & {0} \\
				{0} & {0} & {20} & {0} \\
				{0} & {0} & {0} & {60}
			\end{bmatrix}
			=
			\begin{bmatrix}
				{-5} & {0} & {0} & {0} \\
				{-5} & {0} & {0} & {0} \\
				{0} & {0} & {20} & {0} \\
				{0} & {0} & {0} & {60}
			\end{bmatrix}
			=
			\begin{bmatrix}
				{5} & {0} & {0} & {0} \\
				{0} & {0} & {20} & {0} \\
				{0} & {0} & {0} & {60}
			\end{bmatrix}
		\end{gather*}
		$A \simeq \mathbb{Z}_5 \oplus \mathbb{Z}_20 \oplus \mathbb{Z}_60 \oplus \mathbb{Z}$\\
		
		\subsubsection*{\textbf{Б}}
		\textbf{Условие}\\
		Разложите G в прямую сумму примарных циклически\\
		\\
		\textbf{Решение}\\
		$A \simeq \mathbb{Z}_{2^4} \oplus \mathbb{Z}_{3} \oplus \mathbb{Z}_{5^3}$\\
		
		\subsubsection*{\textbf{В}}
		\textbf{Условие}\\
		Чему равен максимальный возможный порядок её элементов\\
		\\
		\textbf{Решение}\\
		Рассмотрим элементы вида $[\alpha, \beta, \gamma, 0]$\\
		Пусть $A^{\prime} = \mathbb{Z}_5 \oplus \mathbb{Z}_20 \oplus \mathbb{Z}_60$\\
		Тогда порядок элемента $[\alpha, \beta, \gamma, 0]$ в $A$ равен порядку элемента $[\alpha, \beta, \gamma]$ в $A^{\prime}$\\
		$[\alpha, \beta, \gamma]$ -- НОК порядков $\alpha$ в $\mathbb{Z}_{5}$, $\beta$ в $\mathbb{Z}_{20}$, $\gamma$ в $\mathbb{Z}_{60}$, откуда максимальный порядок элемента это 60, такой порядок достигается у $[1,1,1]$\\
		
		\subsubsection*{\textbf{Г}}
		\textbf{Условие}\\
		Вычислите порядок элемента $\begin{bmatrix} {-132} & {248} & {1} & {120} \end{bmatrix}$ в группе $G$\\
		\\
		\textbf{Решение}\\
		\begin{gather*}
			\begin{bmatrix}
				{-5} & {0} & {0} & {0} \\
				{0} & {0} & {20} & {0} \\
				{0} & {0} & {0} & {60} \\
				{-132} & {248} & {1} & {120}
			\end{bmatrix}
			=
			\begin{bmatrix}
				{-5} & {5} & {0} & {0} \\
				{0} & {0} & {20} & {0} \\
				{0} & {0} & {0} & {60} \\
				{2} & {248} & {1} & {0}
			\end{bmatrix}
		\end{gather*}
		Порядок $\begin{bmatrix} {-132} & {248} & {1} & {120} \end{bmatrix}$: $\#(2)$ в $\mathbb{Z}_5$, $\#(1)$ в $\mathbb{Z}_{20}$, $\#(0)$ в $\mathbb{Z}_{60}$, откуда: 
		
		
		\subsection*{\textbf{Задача 3}}
		\textbf{Теория}\\
		Напомним, что цепным комплексом абелевых групп называется набор абелевых групп $C_n$ и отображений $d_n:\ C_n \to C_{n-1}$, таких что композиция $d_n \circ d_{n+1} = 0$ для любого $n$. Для удобства комплекс обозначают $(C\bullet , d\bullet)$ или записывают в виде цепочки отображений:
		\begin{gather*}
			\ldots \rightarrow C_{n+1} \stackrel{d_{n}+1}{\rightarrow} C_{n} \stackrel{d_{n}}{\rightarrow} C_{n-1} \stackrel{d_{n-1}}{\rightarrow} \ldots
		\end{gather*}
		$n$-мерной группой гомологий $H_n$ называется фактор-группа $\text{ker}(d_n)/\text{Im}(d_{n+1})$. Гомологиями комплекса $(C\bullet , d\bullet)$ называется набор всех групп гомологий $(H\bullet )$.
		\\
		\textbf{Условие}\\
		Вычислите гомологии комплекса
		\begin{gather*}
			0 \rightarrow \mathbb{Z}^{3} \stackrel{d}{\rightarrow} \mathbb{Z}^{4} \rightarrow 0
		\end{gather*}
		где отображение $d$ задано матрицей
		\begin{gather*}
			\begin{bmatrix}
				{14} & {-14} & {0} \\
				{14} & {126} & {140} \\
				{28} & {-308} & {-266} \\
				{0} & {-140} & {-98}
			\end{bmatrix}
		\end{gather*}
		\\
		\textbf{Решение}\\
		\begin{gather*}
			d = 
			\begin{bmatrix}
				{14} & {-14} & {0} \\
				{14} & {126} & {140} \\
				{28} & {-308} & {-266} \\
				{0} & {-140} & {-98}
			\end{bmatrix}
			\to
			\begin{bmatrix}
				{14} & {-14} & {14} \\
				{14} & {126} & {14} \\
				{28} & {-308} & {42} \\
				{0} & {-140} & {42}
			\end{bmatrix}
			\to
			\begin{bmatrix}
				{14} & {-14} & {14} \\
				{0} & {140} & {0} \\
				{0} & {-336} & {14} \\
				{0} & {-140} & {42}
			\end{bmatrix}
			\to
			\begin{bmatrix}
				{14} & {0} & {0} \\
				{0} & {140} & {0} \\
				{0} & {-336} & {14} \\
				{0} & {-140} & {42}
			\end{bmatrix}
			\to\\
			\begin{bmatrix}
				{14} & {0} & {0} \\
				{0} & {140} & {0} \\
				{0} & {-56} & {14} \\
				{0} & {0} & {42}
			\end{bmatrix}
			\to
			\begin{bmatrix}
				{14} & {0} & {0} \\
				{0} & {140} & {0} \\
				{0} & {-56} & {14} \\
				{0} & {-168} & {0}
			\end{bmatrix}
			\to
			\begin{bmatrix}
				{14} & {0} & {0} \\
				{0} & {140} & {0} \\
				{0} & {-56} & {14} \\
				{0} & {-28} & {0}
			\end{bmatrix}
			\to
			\begin{bmatrix}
				{14} & {0} & {0} \\
				{0} & {28} & {0} \\
				{0} & {0} & {14}
			\end{bmatrix}\\
			\\
			\text{ker}\: d \cong 14\mathbb{Z} \oplus 28\mathbb{Z} \oplus 14\mathbb{Z}\\
			\text{im}\: d \cong \mathbb{Z}^3\slash_{\text{ker}\: d} = 14\mathbb{Z} \oplus 14\mathbb{Z} \oplus 28\mathbb{Z}\\
			\text{ker}\: d_0 = \mathbb{Z}^4,\quad \text{im}\: d_1 = \mathbb{Z}^3
		\end{gather*}
		Ответ:
		\begin{gather*}
			H_0 = \text{ker}\: d_0 \slash \text{im}\: d = \mathbb{Z}^4 \slash \Big( \mathbb{Z}_{14} \oplus \mathbb{Z}_{14} \oplus \mathbb{Z}_{28} \Big)\\
			H_1 = \text{ker}\: d \slash \text{im}\: d_1 = \Big(14\mathbb{Z} \oplus 14\mathbb{Z} \oplus 28\mathbb{Z}\Big)\slash \mathbb{Z}^3
		\end{gather*}
\newpage		
	\section{Линейные операторы на конечномерном пространстве}
		
		\subsection{ГЛ6 1}
		
		\subsection{ГЛ6 2}
		
		\subsection{ГЛ6 3}
		
		\subsection{ГЛ6 4}
		
		\subsection{ГЛ6 5}
		А)\\
		\\
		Б)\\
		\\
		
		\subsection{ГЛ6 6}
		
		\subsection{ГЛ6 7}
		
		\subsection{ГЛ6 8}
		
		\subsection{ГЛ6 9}
		
		\subsection{ГЛ6 10}
		
		\subsection{ГЛ6 11}
\section{HW 7}

\begin{prob}
Пусть $X$ схема, $f \in \mathcal{O}_X(X), X_f$ подмножество точек $X$, где $f$ не обращается в нуль (т е образ $f$ не лежит в максимальном идеале соответствующего локального кольца). Предположим, что $X$ нетерова, или же отделима и квазикомпактна. Покажите, что $X_f$ открыто и гомоморфизм ограничения индуцирует изоморфизм $\mathcal{O}_X(X)_f$ и $\mathcal{O}_X\left(X_f\right)$.
\end{prob}
\begin{proof}
$O_X$ - квазикомпактный пучок $\Rightarrow$ $\exists U_i = \operatorname{Spec} A_i$
\begin{gather*}
    O_X(U_i) = \tilde{A_i}\quad
    \bigcup_{i = 1}^{N} U_i = X \text{ $X$ - нетерово или квазикоспактное}\\
    V_{ii} = U_i \cap X_f = D(f_I) \text{, где $f_i$ ограничение $f$ на $U_i$}
    \text{так как} (f_i)_p = f_p \Rightarrow V_i \text{ - открытое}\\
    x_f = \bigcup V_i \Rightarrow X_f \text{ - открытое}\\
    O_X(V_i) \simeq O_X(U_i)_f = (A_i)_f\\
    O_X \text{ - пучок} \Rightarrow \exists \text{s.e.s}\\
    0 \to O_X(X) \to \oplus O_X(U_i) \to \oplus O_X(U_i \cap U_j)
\end{gather*}
% https://tikzcd.yichuanshen.de/#N4Igdg9gJgpgziAXAbVABwnAlgFyxMJZABgBpiBdUkANwEMAbAVxiRGJAF9T1Nd9CKMgEYqtRizYduvbHgJFh5MfWatEIAPIB9ABoAKXQEptAMy48QGOQMWlR1VZI06Dus0Yuz+ClACZlRwl1EAAdUIg0ZjgAAld9AFVtLBNzGSs+eUFkAIdxNTZwyOi4vX0ANWTPdOsfbIBmQPznMIioplj4pKwY8IBjOjQYpIArVK8Mm19kRrynEKL2zrLKnv7BmMqxrjEYKABzeCJQUwAnCABbJDIQHAgkP3Szy6QlW-vEeqfzq8QA96QABYggUNPsJs9fsCAYgAGwgloACwhPyQjRhAFYESF9gByFEvRBYmEAdmxbER+O+hP+dzR5I04UYaERdAJUOodKJDNaACMYDg2dQGFgwCE4BARVAQNRETA6NLEGAmAwGJy6FgGGxIGL2Uh4aSeeF9nQLhc2ZwKJwgA
\begin{tikzcd}
0 \arrow{r} & O_X(X)_f \arrow{r}{g} \arrow{d}{\alpha} & \oplus O_X(U_i)_f \arrow{r}{h} \arrow{d}{\beta} & \oplus O_X(U_i \cap U_j)_f \arrow{d}{\gamma} \\
0 \arrow{r} & O_X(X_f) \arrow{r}{g'} & \oplus O_X(V_i) \arrow{r}{h'} & \oplus O_X(V_i \cap V_j)
\end{tikzcd}
$\beta$ - ихоморфизм, $\alpha$ - инъекция, так как $g, \beta, g'$ - инъекции\\
\noindent
так как $U_i \cap U_j$ - афф, если $X$ - отделима или покрыта конечным числом афф., если $X$ - нетерова $\Rightarrow$ либо $\gamma$ - изоморфизм, либо $\gamma$ - инъекция по той же причине, что и $\alpha$ $\Rightarrow$ $\gamma$ как минимум инъекция $\Rightarrow$ по лемме о гомоморфизме $\alpha$ - сюръекция, а следовательно изомофизм
\end{proof}
\vskip 0.6in





\begin{prob}
Пусть $X$ схема и $f_1, \ldots, f_k$ порождают $\mathcal{O}_X(X)$. Предположим, что $X_{f_i}$ аффинны, докажите, что $X$ тоже аффинно.
\end{prob}
\begin{proof}
\begin{gather*}
    \varphi: X \to \operatorname{Spec}(\Gamma (X, O_X))\\
    \varphi_{f_i}: X_{f_i} \to \operatorname{Spec}(\Gamma(X, O_X)_{f_i}) \simeq \operatorname{Spec}(\Gamma(X_{f_i}, O_X))
    \quad\text{так как}
    \quad X_{f_i} - \operatorname{aff}
    \Rightarrow \varphi \text{ - изоморфизм}\\
    O_X = \langle f_1, \ldots, f_k \rangle
    \Rightarrow X = \bigcup X_{f_i}\\
    \operatorname{Spec}(\Gamma(X, O_X)) = \bigcup \operatorname{Spec}(\Gamma(X, O_X)_{f_i})
\end{gather*}
то есть $\varphi$ - изоморфизм на базе $\Rightarrow$ $\varphi$ - изоморфизм
\end{proof}
\vskip 0.6in





\begin{prob}
Выведите отсюда, что аффинность морфизма $f: X \rightarrow Y$ можно проверять на покрытии, то есть следующие условия равносильны:
\begin{itemize}
\item[(a)] $f$ аффинный, то есть прообраз любого аффинного открытого подмножества тоже аффинный
\item[(b)] существует открытое аффинное покрытие $U_i$ схемы $Y$, такое, что все $f^{-1}\left(U_i\right)$ аффинны.
\end{itemize}
\end{prob}
\begin{proof}
\begin{itemize}
\item[]
\item[$(a) \Rightarrow (b)$] -- очевидно 
\item[$(b) \Rightarrow (a)$]
    \begin{gather*}
        U \subset Y - \text{aff}\qquad
        U = \operatorname{Spec} A
        \quad Y = \bigcup U_i = \bigcup \operatorname{Spec} A_i\\
        U \cap U_i = \bigcup U_{i,j}\\
        \Rightarrow U_{i,j} = (\operatorname{Spec} A_i)_{g_j} = (\operatorname{Spec} A)_{h_{i,j}}\\
        f^{-1}(U_i) = V_i = \operatorname{Spec} B_i\\
        f^{-1}(U_{i,j}) = (\operatorname{Spec} B_i)_{f^{\#}(g_j)}\\
        (f^{-1}(U))_{f^{\#}(h_{i,j})} = (\operatorname{Spec} B_i)_{f^{\#}(g_j)}\\
        O_X(f^{-1}(U)) = \langle f^{\#}(h_{i,j}) \rangle\\
        \Rightarrow f^{-1}(U) \text{ -- aff (по 2)} 
    \end{gather*}
\end{itemize}
\end{proof}
\vskip 0.6in





\begin{prob}
Докажите, что конечность морфизма можно проверять на покрытии.
\end{prob}
\begin{proof}
Тут также $(a) \Rightarrow (b)$ -- очев, $(b) \Rightarrow (a)$ по прошлой задаче: $f^{-1}(U)$ -- aff. Факт из коммутативной алгебры, если $R = \langle f_1, \ldots, f_n \rangle$, (*) $g: R \to S$, $R_{f_i} \to S_{g(f_i)}$ - конечно $\Rightarrow$ $g$ - конечно; $O_y(U)_{h_{i,j}} \to O_X(f^{-1}(U))_{f^{\#}(h_{i,j})}$ - конечно $\Rightarrow$ $f^{-1}(U) \to U$ - конечно
\vskip 0.2in \noindent
(*): $g: R \to S$ - конечно $\Rightleftarrow$ $g$ - целый морфизм и $S$ - $R$-алгебра кон. типа\\
Зафиксируем $s \in S\quad I \subset R[x]\quad \forall p \in T\quad p(s) = 0$\\
$J \subset R$ - коэфф. при старших степенях у элем. $I$
\begin{gather*}
    s \in S
    \Rightarrow \frac{s}{1} \in S_{f_i}
    \Rightarrow \exists p_i \in R_{f_i}[x]:\ p_i(\frac{s}{1}) = 0\\
    \exists n_i: (f_i^{n_i} p_i) \in R[x]
    \Rightarrow f_i^{n_i} p_i \in I
\end{gather*}
так как $1 = \sum a_i f_i$ то $\exists N$ - достаточно большой что: $1 = 1^N = (\sum a_i f_i)^N \in J$ $\Rightarrow$ $g$-целый\\
\begin{gather*}
    S_{f_i} = \langle s_{i1}, \ldots, s_{in} \rangle \text{ - как } R_{f_i} \text{ - алг}\\
    \Rightarrow \frac{s}{1} = \sum_{i = 1}^{n} a_{ij} s_{ij}
    \Rightarrow \exists n_i: \frac{f_i^{n_i} s}{1} = \sum_{i=1}^{n} f_i^{n_i} a_{ij} s_{ij} \in S\\
    \Rightarrow s = 1 \cdot s
    = 1^{N} \cdot s
    = (\sum b_i f_i)^N s
    = (\sum b_i f_i)^N \sum a_{ij} s_{ij} \in S
\end{gather*}
\end{proof}
\vskip 0.6in





\begin{prob}
Пусть $X$ схема и $\mathcal{F}$ пучок $\mathcal{O}_X$-модулей. Докажите что $\mathcal{F}$ квазикогерентный тогда и только тогда, когда у любой точки есть окрестность $U$ и точная последовательность пучков $\mathcal{O}_X$-модулей
$$
\mathcal{O}_U^{\oplus I} \rightarrow \mathcal{O}_U^{\oplus J} \rightarrow \mathcal{F}\big|_U \rightarrow 0 .
$$
Здесь $I, J$ - некоторые множества индексов.
\end{prob}
\begin{proof}
\begin{itemize}
\item[]
\item[($\Rightarrow$)]
\begin{gather*}
    \mathcal{F}\big|_{U_i} = \tilde{M}_i\qquad
    O_x \big|_{U_i} = U_{U_i} = \tilde{A_i}\\
    X = \cup U_i = \cup \operatorname{Spec} A_i\\
    M_i \text{ - модуль над } A_i \Rightarrow
    \exists \text{точная последовательность}\\
    A_i^{\oplus J} \to A_i^{\oplus I} \to M \to 0
    \text{, где } |I| \text{ - количество порождающих у } M\\ \text{ и } |J| \text{ - количество соотношений на эти порождающие}\\
    \Rightarrow \forall q \in U_i = \operatorname{A_i}\\
    (A_i^{\oplus J})_q \simeq (A_i)^{\oplus J}_q \to (A_i^{\oplus I})_q \to M_q \to 0 \text{ - точная последовательность}\\
    \Rightarrow (\tilde{A}_i)^{\oplus J}_q \simeq  (O_{U_i})^{\oplus J}_q \to (\tilde{A}_i)^{\oplus I}_q \simeq (O_{U_i})^{\oplus I}_q \to \tilde{M}_q \simeq (\mathcal{F}\big|_{U_i})_q \to 0\\
    \text{точная последовательноть } \forall U_i\\
    \Rightarrow O_{U_i}^{\oplus J} \to O_{U_i}^{\oplus I} \to \mathcal{F}\big|_{U_i} \to 0 \text{ - точная последовательность}
\end{gather*}
\item[($\Leftarrow$)]
\begin{gather*}
    \text{зафиксируем} x \in X
    \quad \exists U^x \ni x
    \quad O^J_U \to O^I_U \to \mathcal{F}\big|_U \to 0\\
    \exists U^x_i:\ U^x = \cup U_i^x = \cup \operatorname{Spec} A_i\quad \text{без ограничения общности} x \in U^x_1\\
    O^J_{U^x_1} \to O^I_{U^x_1} \to \mathcal{F}\big|+{U^x_1} \to 0\quad \text{точная}\\
    M = \operatorname{coker} f_{U^x_1} = \frac{A^I_1}{f(A^J_1)}\quad \forall p \in U^x_1\\
    M_p = \left(\frac{A^I_1}{f(A^J_1)}\right)_p
    = \frac{(A^I_1)_p}{f(A^J_1)_p}
    = \frac{(A_1)^I_p}{f((A_1)^J_p)}
    = \operatorname{coker} f_p = \mathcal{F}_p\\
    \Rightarrow \mathcal{F}\big|_{U^x_1} \simeq \tilde{M}
\end{gather*}
этот процесс не зависит от выбора $x$ $\Rightarrow$ у них есть аффинное покрытие $X = \cup U^x_1$ и $\mathcal{F}$ огранич. на $\forall U^x_1 = \operatorname{Spec} A^x_1$ - это модуль над $A^x_1$
\end{itemize}
\end{proof}
\vskip 0.6in





\begin{prob}
Пусть $f: X \rightarrow Y$ аффинный морфизм, проверьте, что для квазикогерентного $\mathcal{F}$ на $X$ и $\mathcal{G}$ на $Y$ верно $f_*\left(\mathcal{F} \otimes f^* \mathcal{G}\right)=f_* \mathcal{F} \otimes \mathcal{G}$.
\end{prob}
\begin{proof}

\end{proof}


\end{document}
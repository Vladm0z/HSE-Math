\newpage
\section{Введение}
	$x^2 + y^2 + z^2 = 3xyz$\\
	Однородный многочлен от $x, y$ степени 2\\
	Заметим, что $f(tx,ty) = t^2 \cdot f(x,y)$\\
	\\
	\textbf{Пример:} $f(x,y) = ax^2 + bxy + cy^2 \quad a,b,c \in \mathbb{R}$ --- квадратичная форма\\
	\\
	Рассмотрим $\mathbb{Z}^2 \{ (m,n); \: m,n \in \mathbb{Z}\}$ и обозначим $\mathbb{Z}^{2 \star} = \mathbb{Z}^2 / (0,0)$\\
	Пусть $m(f) = \min{|f(x,y)|, \: (x,y) \in \mathbb{Z^{2\star}}}$, тогда $m(\lambda f) = m(f) \cdot \lambda$\\
	\textbf{Число Маркова}\\
	$C(f) = \frac{m(f)}{\sqrt{\Delta f}}$\\
	\\
	На курсе планируется рассмотреть несколько вопросов:\\
	1) Как устроен спектр Маркова(т.е. какие значения может принимать $C(f)$)\\
	2) Какую область $C(f)$ занимает на $\mathbb{R}$\\
	
\section{Появление группы}
	$x,y \in \mathbb{Z}^{2 \star}, \quad (x,y) \to (ax + by, \: cx + dy)$\\
	$A =
	\begin{pmatrix}
	a & b \\ c & d
	\end{pmatrix}
	\in GL(2, \mathbb{Z}) \ \text{-- группа обратимых матриц}$\\
	1) $a,b,c,d \in \mathbb{Z}$
	2) $\det \begin{pmatrix} a & b \\ c & d \end{pmatrix} = ad-bc = \pm 1$
	\\
	\textbf{Задача}\\
	Показать, что $\mathbb{Z}^2$ изоморфно $\mathbb{Z}^{2 \: \star}$\\
	Где $\mathbb{Z}^{2 \: \star} \ = \ \mathbb{Z}^2 / (0, 0)$\\
	\\
	Пусть $F(x,y) = f(\alpha x + \beta y,\: \gamma x + \delta y)$ \\ \\
	\textbf{Задача}\\
	Докажите, что: $|\Delta(F)| = |\Delta(f)|$\\ \\
	\begin{gather*}
	f_1 \sim f_2 \: \Leftrightarrow \\
	1) f_1(x,y) = f_2(\alpha x + \beta y,\: \gamma x + \delta y)\\
	2) 
		\begin{cases} 
			C(f_1) = C(f_2)\\
			C(\lambda f) = C(f)
		\end{cases}
	\end{gather*}
	
	\textbf{Примеры:}\\
	a)\\
	$f(x,y) = 0 \\
	\exists m,n \in \mathbb{Z}^{2 \star}$\\
	б)\\
	$(p_n, q_n): \: p_n,q_n \in \mathbb{Z}\\
	q_n > 0\\
	|f(p_n, q_n)| \to 0$
	\\

\section{Теория диофантовых приближений}
	Пусть $\omega \in \mathbb{R} / \mathbb{Q}$\\
	Существует бесконечно много пар $(p_n, q_n)$, $p_n, q_n \in \mathbb{Z}$, $q_n > 0$ таких что: \\
	\begin{equation*}
	|\omega - \frac{p_n}{q_n}| < \frac{1}{q^2_n}
	\end{equation*}
	\textbf{Задача}\\
	Проверить это утверждение, представив $\omega$ в виде цепной дроби, а также проверить, выполнено ли это утверждение для $\frac{1}{q^k_n}, \ k \in \mathbb{Z}$\\ \\
	$f = x^2 + \omega^2 y^2\\
	\Delta (f) = b^2 - 4ac = 4\omega^2 > 0$\\
	\begin{equation*}
	|f(p_n, q_n)| = |p^2_n - (\omega q_n)^2| = q^2_n |(\frac{p_n}{q-n} - \omega)(\frac{p_n}{q-n} + \omega)| < q^2_n \frac{C(\omega)}{q^3_n} = \frac{C(\omega)}{q_n} |\omega - \frac{p_n}{q_n}| < \frac{1}{q^3_n}
	\end{equation*}
	
	\textbf{Построим $\omega$}
	\begin{equation*}
	\omega = \sum_{n=1}^{\inf} 10^{-n!}
	\end{equation*}
	Заметим что в этом случае $\omega$ это десятичная непериодическая дробь из 0 и 1\\
	\\
	$\mathbf{\Delta (f) < 0}\\
	f_0(x,y) = x^2+xy+y^2\\
	|\Delta| = 3\\
	C(f_0) = \frac{1}{\sqrt{3}}$ \\ \\
	\textbf{Теорема}\\
	Если $\Delta(f) < 0$, то $C(f) \leqslant \frac{1}{\sqrt{3}}$ и $C(f) = \frac{1}{\sqrt{3}} \ \Leftrightarrow \text{\[ f_0 \]}$\\
	Спектр Маркова для $\Delta (f) < 0$ это отрезок $[ 0; \frac{1}{\sqrt{3}}] \ \Leftrightarrow \\
	\rho \in [ 0; \frac{1}{\sqrt{3}}] \quad \exists [f]: \: C(f)=\rho$
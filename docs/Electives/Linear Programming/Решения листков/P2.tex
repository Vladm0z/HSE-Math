\section{Лист 2}

\begin{prob}
\begin{gather*}
	-x_{1} + x_{2} - 2x_{3} + 3x_{4} + x_{5} \to \max \\
	x_{1} + 2x_{2} - x_{3} - 2x_{4} + x_{5} \leqslant 3 \\
	-x_{1} - x_{2} + x_{3} + 2x_{4} + x_{5} \leqslant 1 \\
	2x_{1} + x_{2} + x_{3} - x_{4} \leqslant 1 \\
	x_{i} \geqslant 0
\end{gather*}
\end{prob}
\begin{proof}
	\begin{gather*}
	\begin{array}{c|ccccc|c} 
		& x_{1} & x_{2} & x_{3} & x_{4} & x_{5} & \\
		\hline 
		y_{1} & 1 & 2 & -1 & -2 & 1 & -3\\
		y_{2} & -1 & -1 & 1 & 2 & 1 & -1 \\
		y_{3} & 2 & 1 & 1 & -1 & 0 & -1\\
		\hline 
		& -1 & 1 & -2 & 3 & 1 & 0
	\end{array}
	\end{gather*}
	Откуда
	\begin{gather*}
		x_{1} = y_{1} - y_{2} + 2y_{3} - 1\\
		x_{2} = 2y_{1} - y_{2} + y_{3} + 1\\
		x_{3} = -y_{1} + y_{2} + y_{3} - 2\\
		x_{4} = -2y_{1} + 2y_{2} - y_{3} + 3\\
		x_{5} = y_{1} + y_{2} + 1\\
	\end{gather*}
	В зависимости
	\begin{gather*}
		y_{1} = x_{1} + 2x_{2} - x_{3} - 2x_{4} + x_{5} - 3\\
		y_{2} = -x_{1} - x_{2} + x_{3} + 2x_{4} + x_{5} - 1\\
		y_{3} = 2x_{1} + x_{2} + x_{3} - x_{4} - 1
	\end{gather*}
	Выразим через $x_{1}$, $y_{1}$ остальные переменные
	\begin{gather*}
		x_{1} = y_{1} - 2x_{2} + x_{3} + 2x_{4} - x_{5} + 3
	\end{gather*}
	Получим
	\begin{gather*}
		y_{2}
		= -(y_{1} - 2x_{2} + x_{3} + 2x_{4} - x_{5} + 3) - x_{2} + x_{3} + 2x_{4} + x_{5} - 1
		= -y_{1} + x_{2} + 2x_{5} - 4\\
		y_{3}
		= 2(y_{1} - 2x_{2} + x_{3} + 2x_{4} - x_{5} + 3) + x_{2} + x_{3} - x_{4} - 1
		= 2y_{1} - 3x_{2} + 3x_{3} + 3x_{4} - 2x_{5} + 5
	\end{gather*}
	Выразим через $y_{1}, x_{2}, x_{3}, x_{4}, x_{5}$ функцию стоймости в двойственной задаче
	\begin{gather*}
		-x_{1} + x_{2} - 2x_{3} + 3x_{4} + x_{5}\\
		= -(y_{1} - 2x_{2} + x_{3} + 2x_{4} - x_{5} + 3) + x_{2} - 2x_{3} + 3x_{4} + x_{5}\\
		= -y_{1} + 3x_{2} - 3x_{3} + x_{4} + 2x_{5} - 3
	\end{gather*}
	Получим
	\begin{gather*}
	\begin{array}{c|ccccc|c} 
		& y_{1} & x_{2} & x_{3} & x_{4} & x_{5} & \\
		\hline 
		x_{1} & 1 & -2 & 1 & 2 & -1 & -3\\
		y_{2} & -1 & 1 & 0 & 0 & 2 & 4\\
		y_{3} & 2 & -3 & 3 & 3 & -2 & -5\\
		\hline 
		& -1 & 3 & -3 & 1 & 2 & 3
	\end{array}
	\end{gather*}
	Выразим через $x_{2}$, $y_{2}$ остальные переменные
	\begin{gather*}
		y_{2} = -y_{1} + x_{2} + 2x_{5} - 4\\
		x_{2} = y_{1} + y_{2} - 2x_{5} + 4\\
	\end{gather*}
	Получим
	\begin{gather*}
		x_{1} 
		= y_{1} - 2(y_{1} + y_{2} - 2x_{5} + 4) + x_{3} + 2x_{4} - x_{5} + 3
		= -y_{1} - 2y_{2} + x_{3} + 2x_{4} + 3x_{5} - 5\\
		y_{3}
		= 2y_{1} - 3(y_{1} + y_{2} - 2x_{5} + 4) + 3x_{3} + 3x_{4} - 2x_{5} + 5
		= -y_{1} - 3y_{2} + 3x_{3} + 3x_{4} + 4x_{5} - 7
	\end{gather*}
	Выразим через $y_{1}, y_{2}, x_{3}, x_{4}, x_{5}$ функцию стоймости в двойственной задаче
	\begin{gather*}
		-x_{1} + x_{2} - 2x_{3} + 3x_{4} + x_{5}\\
		= -y_{1} + 3(y_{1} + y_{2} - 2x_{5} + 4) - 3x_{3} + x_{4} + 2x_{5} - 3\\
		= 2y_{1} + 3y_{2} - 3x_{3} + x_{4} - 4x_{5} + 9
	\end{gather*}
	Получим
	\begin{gather*}
	\begin{array}{c|ccccc|c} 
		& y_{1} & y_{2} & x_{3} & x_{4} & x_{5} & \\
		\hline 
		x_{1} & -1 & 2 & 1 & 2 & 3 & 5\\
		x_{2} & 1 & -1 & 0 & 0 & -2 & -4\\
		y_{3} & 2 & -3 & 0 & 7 & 4 & 7\\
		\hline 
		& 2 & 3 & -3 & 1 & -4 & 9
	\end{array}
	\end{gather*}
	Откуда максимум равен $9$ и достигается при $x = (0, 3, 0, 2, 0)$
\end{proof}
\vskip 0.6in



\begin{prob}
\begin{gather*}
	3 x_{1}+4 x_{2}+5 x_{3} \to \max \\
	x_{1}+2 x_{2}+2 x_{3} \leqslant 1 \\
	-3 x_{1}+x_{3} \leqslant -1 \\
	-2 x_{1}-x_{2} \leqslant -1 \\
	x_{j} \geqslant 0
\end{gather*}
\end{prob}
\begin{proof}
	\begin{gather*}
	\begin{array}{c|ccc|c} 
		& x_{1} & x_{2} & x_{3} & \\
		\hline 
		y_{1} & 1 & 2 & 2 & -1\\
		y_{2} & -3 & 0 & 1 & 1 \\
		y_{3} & -2 & -1 & 0 & 1\\
		\hline 
		& 3 & 4 & 5 & 0
	\end{array}
	\end{gather*}
	Откуда
	\begin{gather*}
		x_{1} = y_{1} - 3y_{2} - 2y_{3} - 3\\
		x_{2} = 2y_{1} - y_{3} - 4\\
		x_{3} = 2y_{1} + y_{2} - 5
	\end{gather*}
	В зависимости
	\begin{gather*}
		y_{1} = x_{1} + 2x_{2} + 2x_{3} - 1\\
		y_{2} = -3x_{1} + x_{3} + 1\\
		y_{3} = -2x_{1} - x_{2} + 1
	\end{gather*}
	
	Выразим через $x_{1}$, $y_{2}$ остальные переменные
	\begin{gather*}
		y_{2} = -\frac{1}{3} x_{1} + \frac{1}{3} y_{1} - \frac{2}{3}y_{3} - 1
	\end{gather*}
	Получим
	\begin{gather*}
		x_{2}
		= 2y_{1} - y_{3} - 4\\
		x_{3}
		= 2y_{1} + y_{2} - 5
		= 2y_{1} -\frac{1}{3} x_{1} + \frac{1}{3} y_{1} - \frac{2}{3}y_{3} - 1 - 5
		= -\frac{1}{3} x_{1} + \frac{7}{3} y_{1} - \frac{2}{3} y_{3} - 6
	\end{gather*}
	Выразим через $y_{1}, x_{1}, y_{3}$ функцию стоймости в двойственной задаче
	\begin{gather*}
		y_{1} - y_{2} - y_{3}
		= y_{1} + \frac{1}{3} x_{1} - \frac{1}{3} y_{1} + \frac{2}{3}y_{3} + 1 - y_{3}
		= \frac{2}{3} y_{1} + \frac{1}{3} x_{1} - \frac{1}{3} y_{3} + 1
	\end{gather*}
	Получим
	\begin{gather*}
	\begin{array}{c|ccc|c} 
		& y_{2} & x_{2} & x_{3} & \\
		\hline 
		y_{1} & \frac{1}{3} & 2 & \frac{7}{3} & \frac{2}{3}\\
		x_{1} & -\frac{1}{3} & 0 & -\frac{1}{3} & \frac{1}{3}\\
		y_{3} & -\frac{2}{3} & -1 & -\frac{2}{3} & -\frac{1}{3}\\
		\hline 
		& -1 & -4 & -6 & 1
	\end{array}
	\end{gather*}

	Выразим через $x_{2}$, $y_{3}$ остальные переменные
	\begin{gather*}
		y_{3} = 2y_{1} - x_{2} - 4
	\end{gather*}
	Получим
	\begin{gather*}
		y_{2}
		= \frac{1}{3} y_{1} - \frac{1}{3} x_{1} - \frac{1}{3} (2 y_{1} - x_{2} - 4) - 1
		= -y_{1} - \frac{1}{3} x_{1} + \frac{2}{3} x_{2} + \frac{5}{3}\\
		x_{3}
		= \frac{7}{3} y_{1} - \frac{1}{3} x_{1} - \frac{2}{3} (2 y_{1} - x_{2} - 4) - 6
		= y_{1} - \frac{1}{3} x_{1} + \frac{2}{3} x_{2} - \frac{10}{3}
	\end{gather*}
	Выразим через $y_{1}, x_{1}, x_{2}$ функцию стоймости в двойственной задаче
	\begin{gather*}
		y_{1} - y_{2} - y_{3}
		= \frac{2}{3} y_{1} + \frac{1}{3} x_{1} - \frac{1}{3} (2y_{1} - x_{2} - 4) + 1
		= \frac{1}{3} x_{1} + \frac{1}{3} x_{2} + \frac{7}{3}
	\end{gather*}
	Получим
	\begin{gather*}
	\begin{array}{c|ccc|c} 
		& y_{2} & y_{3} & x_{3} & \\
		\hline 
		y_{1} & -1 & 2 & 1 & 0\\
		x_{1} & -\frac{1}{3} & 0 & -\frac{1}{3} & \frac{1}{3}\\
		x_{2} & \frac{2}{3} & -1 & \frac{2}{3} & \frac{1}{3}\\
		\hline 
		& \frac{5}{3} & -4 & -\frac{10}{3} & \frac{7}{3}
	\end{array}
	\end{gather*}

	Выразим через $y_{1}$, $y_{3}$ остальные переменные
	\begin{gather*}
		y_{3} = 2y_{1} - x_{2} - 4
	\end{gather*}
	Получим
	\begin{gather*}
		y_{2}
		= -\frac{1}{2} y_{3} + \frac{1}{6} x_{2} - \frac{1}{3} x_{1} - \frac{1}{3}\\
		x_{3}
		= \frac{1}{2} y_{3} + \frac{7}{6} x_{2} - \frac{1}{3} x_{1} - \frac{4}{3}
	\end{gather*}
	Получим
	\begin{gather*}
	\begin{array}{c|ccc|c} 
		& y_{2} & y_{3} & x_{3} & \\
		\hline 
		x_{1} & -\frac{1}{3} & 0 & -\frac{1}{3} & \frac{1}{3}\\
		x_{2} & \frac{2}{3} & -1 & \frac{2}{3} & \frac{1}{3}\\
		y_{1} & -1 & 2 & 1 & 0\\
		\hline 
		& \frac{5}{3} & -4 & -\frac{10}{3} & \frac{7}{3}
	\end{array}
	\end{gather*}

	Выразим через $x_{2}$, $y_{2}$ остальные переменные
	\begin{gather*}
		x_{2} = 6 y_{1} + 3 y_{3} + 2 x_{1} + 2
	\end{gather*}
	Получим
	\begin{gather*}
		y_{1}
		= \frac{1}{2} y_{3} + \frac{1}{2} (6 y_{2} + 3 y_{3} + 2 x_{1} + 2) + 2
		= 2 y_{3} + 3 y_{2} + x_{1} + 3\\
		x_{3}
		= 4 y_{3} + 2 x_{1} + 7 y_{2} + 1
	\end{gather*}
	Выразим функцию стоймости
	\begin{gather*}
		\frac{1}{3} x_{1} + \frac{1}{3} (6 y_{2} + 3 y_{3} + 2 x_{1} + 2) + \frac{7}{3}
		= \frac{1}{3} x_{1} + 2 y_{2} + y_{3} + \frac{2}{3} x_{1} + \frac{2}{3} + \frac{7}{3}
		= x_{1} + 2 y_{2} + y_{3} + 3
	\end{gather*}
	Получим
	\begin{gather*}
	\begin{array}{c|ccc|c} 
		& y_{1} & x_{2} & x_{3} & \\
		\hline 
		x_{1} & 1 & 2 & 2 & 1 \\
		y_{2} & 3 & 6 & 7 & 2\\
		y_{3} & 2 & 3 & 4 & 1\\
		\hline 
		& 3 & 2 & 1 & 3
	\end{array}
	\end{gather*}

	Откуда максимум равен $3$ и достигается при $x = (1, 0, 0)$	
\end{proof}
\vskip 0.6in



\begin{prob}
Найти все вершины многогранника в $\mathbb{R}^{4}$
\begin{gather*}
	x_{1} - 2x_{2} + 4x_{3} - x_{4} \leqslant 1 \\
	2x_{1} + 3x_{2} + x_{3} + 2x_{4} \leqslant 3 \\
	x_{i} \geqslant 0, i \in\{1,2,3,4\}
\end{gather*}
\end{prob}
\begin{proof}
	Вершина в $\mathbb{R}^4$ -- решение системы, в котором 4 неравенства обращены в равенства.
	\begin{itemize}
	\item
		В случае, когда зануляются неравенства $x_{i} \geqslant 0$, вершиной является $(0,0,0,0)$
	\item
		Если зануляются 3 неравенства $x_{i} \geqslant 0$ и $x_{1} - 2x_{2} + 4x_{3} - x_{4} = 1$, то вершинами являются $(0,0, \frac{1}{4}, 0)$ и $(1,0,0,0)$, а точки $(0,-\frac{1}{2},0,0), (0,0,0,1)$ вершинами не являются в силу $x_{2} < 0$ и $x_{4} < 0$
	\item
		Если зануляются 3 неравенства $x_{i} \geqslant 0$ и $2x_{1} + 3x_{2} + x_{3} + 2x_{4} = 3$, то вершины это $(0,0,0,\frac{3}{2}), (0,0,3,0), (0,1,0,0), (\frac{3}{2},0,0,0)$
	\item
		Если зануляются 2 неравенства $x_{i} \geqslant 0$, а также выполнено $x_{1} - 2x_{2} + 4x_{3} - x_{4} = 1$, $2x_{1} + 3x_{2} + x_{3} + 2x_{4} = 3$, то есть задачу можно представить в виде 6 систем:
	\begin{itemize}
	\item
		\begin{gather*}
		\begin{cases}
			x_1 - 2x_2 = 1\\
			2x_1 + 3x_2 = 3
		\end{cases}\\
		x_1 = \frac{9}{7}\quad
		x_2 = \frac{1}{7}
		\end{gather*}
	\item
		\begin{gather*}
		\begin{cases}
			x_1 + 4x_3 = 1\\
			2x_1 + x_3 = 3
		\end{cases}\\
		x_1 = \frac{11}{7}\quad
		x_3 = -\frac{1}{7}\quad \text{не вершина, так как не выполнено } x_i \geqslant 0
		\end{gather*}
	\item
		\begin{gather*}
		\begin{cases}
			x_1 - x_4 = 1\\
			2x_1 + 2x_4 = 3
		\end{cases}\\
		x_1 = \frac{5}{4}\quad
		x_4 = \frac{1}{4}
		\end{gather*}
	\item
		\begin{gather*}
		\begin{cases}
			-2x_2 + 4x_3 = 1\\
			3x_2 + x_3 = 3
		\end{cases}\\
		x_2 = \frac{11}{14}\quad
		x_3 = \frac{9}{14}
		\end{gather*}
	\item
		\begin{gather*}
		\begin{cases}
			-2x_2 - x_4 = 1\\
			3x_2 + 2x_4 = 3
		\end{cases}\\
		x_2 = -5\quad
		x_4 = 9\quad \text{не вершина, так как не выполнено } x_i \geqslant 0
		\end{gather*}
	\item
		\begin{gather*}
		\begin{cases}
			4x_3 - x_4 = 1\\
			x_3 + 2x_4 = 3
		\end{cases}\\
		x_3 = \frac{5}{9}\quad
		x_4 = \frac{11}{9}
		\end{gather*}
	\end{itemize}
	\end{itemize}
	То есть у данного многогранника 11 вершин:
	\begin{itemize}
		\item $(0,0,0,0)$
		\item $(0,0, \frac{1}{4}, 0)$
		\item $(1,0,0,0)$
		\item $(0,0,0,\frac{3}{2})$
		\item $(0,0,3,0)$
		\item $(0,1,0,0)$
		\item $(\frac{3}{2},0,0,0)$
		\item $(\frac{9}{7},\frac{1}{7},0,0)$
		\item $(\frac{5}{4},0,0,\frac{1}{4})$
		\item $(0,\frac{11}{14},\frac{9}{14},0)$
		\item $(0,0,\frac{5}{9},\frac{11}{9})$
	\end{itemize}
	Однако, так как по условию $x_{1} - 2x_{2} + 4x_{3} - x_{4} = 1$, $2x_{1} + 3x_{2} + x_{3} + 2x_{4} = 3$, то из этих вершин подходят только
	\begin{itemize}
		\item $(\frac{9}{7},\frac{1}{7},0,0)$
		\item $(\frac{5}{4},0,0,\frac{1}{4})$
		\item $(0,\frac{11}{14},\frac{9}{14},0)$
		\item $(0,0,\frac{5}{9},\frac{11}{9})$
	\end{itemize}
\end{proof}
\vskip 0.6in



\begin{prob}
Дано число $n$. Найти
\begin{gather*}
	\sum\limits_{i=1}^{n} u_i + \sum\limits_{j=1}^{n} v_j \to \max
\end{gather*}
при условии
\begin{gather*}
	u_i + v_j \leqslant 2^{i+j},\quad \forall 1 \leqslant i, j \leqslant n
\end{gather*}
\end{prob}
\begin{proof}
	Заметим что $u_i + v_j \leqslant 2^{i+j} \Rightarrow u_i \leqslant 2^{i+j},\ v_j \leqslant 2^{i+j}$, откуда $u_i + v_1 \leqslant 2^{i+1} \Rightarrow u_i \leqslant 2^{i+1}$, аналогично $v_j \leqslant 2^{j+1}$. Пусть $\{u_1', u_2', \ldots, u_n', v_1', v_2', \ldots, v_n'\}$ -- значения коэффициентов при максимуме суммы, заметим, что если $u_1$ или $v_1 > 0$, то сделав замену $u_1' = v_1' = 0, v_i' = v_i + v_1\ \forall i \geqslant 2$, сумма всех элементов не уменьшится. Так как $u_1 = v_1 = 0$, то $u_i = v_i = 2^{i+1}\ \forall i \geqslant 2$ является решением, и максимальная сумма элементов $2 \cdot (0 + 2^3 + \ldots + 2^{n+1}) = 2 \cdot (2^{n+2} - 2^{3})$
\end{proof}
\vskip 0.6in



\begin{prob}
Матрица размера $m \times n$ называется латинским прямоугольником, если элементы каждой строки этой матриџы образуют перестановку чисел от 1 до $n$, и в каждом столбще все числа разные.
\noindent
Докажите, что латинский прямоугольник $m \times n$ всегда можно дополнить до латинского квадрата.
\end{prob}
\begin{proof}
	Рассмотрим двудольный граф, где вершины одной доли соответствуют колонкам, назовем их $c_{1}, \ldots, c_{n}$, а другой доли числам $n_{1}, \ldots, n_{n}$. Пусть ребро соединяет 2 вершины $c_{i}, n_{j}$, если в колонке $i$ не стоит число $j$, заметим, что если рассмотреть латинский прямоугольник с $m$ колонками и $n$ строками, то вершины будут иметь степень $n-m$, тогда, убирая ребра, можно ставить числа в квадрат, дополняя его. (по факту решение задачи эквивалентно лемме Холла, где одна доля соответствет колонкам, а другая числам)
\end{proof}
\vskip 0.6in



\begin{prob}
Докажите теорему, "двойственную" к теореме Дилуорса. В конечном, частично упорядоченном множестве мощность длиннейшей цепи равна мощности наименьшего разбиения на антицепи
\end{prob}
\begin{proof}
	Пусть $L(a)$ -- длина длинной цепи с началом в $a$, заметим, что если $a_1 > a_2$, то $L(a_1) < L(a_2)$, и если $L(a_1) = L(a_2)$, то $a_1, a_2$ несравнимы и $A_k = \{a| L(a) = k\}$ -- антицепь. Пусть длина наибольшей цепи $b$, тогда в ней есть все значения $L$ от $1$ до $b$ (больше $b$ быть не может, так как в таком случае рассматриваемое множество не является самой длинной цепью), тогда заметим, что $A_{1}, \ldots, A_{b}$ -- является наименьшим разбиением на атицепи.
\end{proof}
\vskip 0.6in



\begin{prob}
В последоватетьности из $n m + 1$ различных действитетьных чисел П.Эрдёш ищет "длинную цепь" - $n + 1$ элемент, идущие слева направо в порядке возрастания. Д.Секерёш, напротив, ищет "длинную антицепь" - $m + 1$ элемент, идущие слева направо в порядке убывания. Докажите, что хотя бы один из них преуспеет.
\end{prob}
\begin{proof}
	Рассмотрим функцию $f(x),\ x \in [1, mn+1]$, значения которой равны длинам возрастающих последовательностей, начинающихся с числа на позиции $x$. Допустим цепи длины $n+1$ нет, то есть значения $f(x)$ лежат в $[0, n]$. Тогда по принципу Дирихле для какого-то значения существует хотя бы $m+1$ число $x_i$, такое что $f(x_1) = \ldots = f(x_{m+1})$, заметим, что эта последовательность является убывающей, так как иначе, если $\exists i,j: x_{i} \leqslant x_{j}$, то возрастающую последовательность можно продолжить на 1 элемент ($x_{j}$), а следовательно для какого-то $x_{k}$ равенство $f(x_1) = \ldots = f(x_{m+1})$ не будет выполнено.
\end{proof}



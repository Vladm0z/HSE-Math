\section{Problem List 1}


\begin{prob}
    Show that a code with minimal distance $d$ can fix up to $d-1$ erasures - unknown bits at fixed positions.
\end{prob}

\begin{proof}
    The decoder needs to fill in $d-1$ (known) coordinates in order to recover the transmitted codeword. Suppose that there are two different possible ways of filling in these $d-1$ coordinates and these give us two different codewords $c$ and $\hat{c}$. Hamming distance between $c$ and $\hat{c}$ can't be $d-1$ or smaller, because it contradicts the assumption that any two distinct codewords must differ in at least $d$ places. (this argument can be applied iff the locations of the errors are known)
\end{proof}
\vskip 0.4in





\begin{prob}
    Are there exists $(n, k, d)_q$-codes as follows (recall that $q$ is alphabet size, $n$ is block size, $k$ is number of information symbols, $d$ is the minimal distance):
    \begin{itemize}
    \item[(a)] $[n, n, 1]_q$
    \item[(b)] $[n, n-1,2]_q$
    \item[(c)] $[11,5,5]_2$
    \item[(d)] $[20,7,5]_2$
    \item[(e)] $[22,15,3]_2$
    \end{itemize}
\end{prob}

\begin{proof}
    \begin{itemize}
    \item[(a)] input = output
    \item[(b)] binary parity check code of length $n$
    \item[(c)] $d = 5$ means all patterns of up to two bit errors can be corrected, but $2^{11-5} = 64 < {{11} \choose {2}} + 11 + 1 = 67$
    \item[(d)] 
    \item[(e)] 
    \end{itemize}
\end{proof}
\vskip 0.4in





\begin{prob}
    Find $(n, k, d)_q$ satisfying Hamming condition but with $d>n-k+1$
\end{prob}

\begin{proof}
    Singleton bound
    If $C$ is a linear code with block length $n$, dimension $k$ and minimum distance $d$ over the finite field with $q$ elements, then the maximum number of codewords is $q^{k}$ and the Singleton bound implies:
    \begin{gather*}
        q^k \leq q^{n-d+1}\\
        k \leq n - d + 1\\
        d \leq n - k + 1
    \end{gather*}
\end{proof}
\vskip 0.4in





\begin{prob}
    Propose a $(n, n-1,2)_q$-code that can detect transposition of any pair of symbols (when symbols exchange their places).
\end{prob}

\begin{proof}
    Можно рассмотреть следующий код: $n-1$ символ тот же самый, а $n$ символ выбирается следующим образом:
    $x_n \equiv_{q+1} x_1 \cdot 2 + x_2 \cdot 3 + \ldots + x_{n-1} \cdot n$ (т.е. код, аналогичный ISBN-10)
\end{proof}
\vskip 0.4in





\begin{prob}
    Suppose $C \subset V=\mathbb{F}_q^n$ be a linear code. By dual code denote $C^*=\operatorname{Ann}(C) \subset V^*$. So if $C$ is $(n, k, d)_{q}$ - code, then $C^*$ is $\left(n, n-k, d^{\prime}\right)_q$ - code for some $d^{\prime}$.
    \begin{itemize}
        \item[(a)] Show that for natural dual bases we have generator and parity check matrices for $C^*$ coincide with parity check and generator matrices for $C$ respectively.
        \item[(b)] Show that group of repeating symbol corresponds to check sum of this group in the dual code.
        \item[(c)] Find a minimal distance for the code, dual to Hamming code. code generated by a polynomial $x^k h\left(x^{-1}\right)$, where $h(t)=\left(t^n-1\right) / g(x)$.
        \item[(d)] Suppose that $C$ is a cyclic polynomial $(n, k, d)_{q}$ - code generated by polynomial $g(x)$. Show that $C*$ is a cyclic code generated by a polynomial $x^k h(x^{-1})$, where $h(t) = (t^n - 1)/g(x)$
        \item[(e)] Is dual to polynomial code always polynomial?
    \end{itemize}
\end{prob}

\begin{proof}
    \begin{itemize}
        \item[(a)] $C\*$ is a linear code of length n over $F_{q}$. If $G$ is a generator matrix for $C$, with rows $r_1, \ldots, r_k$ and columns $c_1, \ldots, c_n$, then
            \begin{gather*}
                G = 
                \begin{bmatrix}
                    c_1 & \ldots & c_n
                \end{bmatrix}
                =
                \begin{bmatrix}
                    r_1 \\ \vdots \\ r_n
                \end{bmatrix}
            \end{gather*}
            Consider $\varphi: V(n,q) \mapsto V(k,q)$
            \begin{align*}
                x \mapsto x G^{T}
                & = x[r_1^{T}, \ldots, r_k^{T}]\\
                & = (x \cdot r_1, \ldots, x \cdot r_k)\\
                & = x_1 c_1^{T} + \ldots + x_n c_n^{T}
            \end{align*}
            Then $n = \dim(\operatorname{ker} \varphi) + \dim(\operatorname{im} \varphi)$\\
            As rank $G=k$, considering im $\varphi$ in terms of the columns of $G$, so $\operatorname{dim}(\operatorname{im} \varphi)=k$. Hence, $\operatorname{dim}(\operatorname{ker} \varphi)=n-k$.
            Aliter, let $G=\left[I_k A\right]$ be a generator matrix for $C$, then $x \in C^{\perp} \Leftrightarrow G x^T=0$ :
            \begin{gather*}
            \begin{bmatrix}
                1 & 0 & \cdots & 0 & a_{11} & \cdots & a_{1, n-k} \\
                0 & 1 & \cdots & 0 & a_{21} & \cdots & a_{2, n-k} \\
                \vdots & & & & \vdots & & \\
                0 & & \cdots & 1 & a_{k 1} & \cdots & a_{k, n-k}
            \end{bmatrix}
            \begin{bmatrix}
                x_1 \\ \vdots \\ x_k \\ \vdots \\ x_n
            \end{bmatrix}
            \end{gather*}
            \begin{align*}
                x_1& + a_{11} x_{k+1} + \cdots + &a_{1, n-k} x_n = 0\\
                x_2& + a_{21} x_{k+1} + \cdots + &a_{2, n-k} x_n = 0\\
                \vdots& \quad& \vdots\\
                x_k& + a_{k 1} x_{k+1} + \cdots + &a_{k, n-k} x_n=0
            \end{align*}
            So any choice can be made for $x_{k+1}, \ldots, x_n$; then $x_1, \ldots, x_k$ are determined. Hence $C^{\perp}=q^{n-k}$. Hence $\operatorname{dim} C^{\perp}=n-k$.
            \begin{gather*}
                G=\left[I_k A\right], \quad H=\left[-A^T I_{n-k}\right], \quad \operatorname{rank} H=n-k
            \end{gather*}
            Then
            \begin{gather*}
                G H^T=\left[\begin{array}{ll}
                I_k & A
                \end{array}\right]\left[\begin{array}{c}
                -A \\
                I_{n-k}
                \end{array}\right]=I_k(-A)+A I_{n-k}=-A+A=0 .
            \end{gather*}
            So $H G^T=0$; that is, the rows $s_1, \ldots, s_{n-k}$ of $H$ are in $C^{\perp}$. But rank $H=n-k$; so $H$ is a generator matrix for $C^{\perp}$.
        \item[(b)] 
        \item[(c)] Dual of the Hamming code is shortened Hadamard code $\left[2^{n}-1, n, d\right]$ and $d = 2^{n-1}$
        \item[(d)] The dual code of $C$ is generated by polynomial $\sum\limits_{j = 0}^{k} h_{k-j} x^{j}$. The pnly thing remaning is to note that $\sum\limits_{j = 0}^{k} h_{k-j} x^{j} = \sum\limits_{j = 0}^{k} h_j x^{k-j} = x^{k} h (x^{-1})$
        \item[(e)] 
    \end{itemize}
\end{proof}
\vskip 0.4in





\begin{prob}
    For a given graph with $m$ vertices and $n$ edges we associate $(n, n-m, d)_q$ code as follows. Bits are enumerated by edges, a vector is a code word iff sum of bits for all edges with a common vertex is zero.
    \begin{itemize}
        \item[(a)] Find a graph associated with repeating code.
        \item[(b)] Find minimal distance of the code associated with $n$-gone.
        \item[(c)] Find minimal distance of the code associated with a simplex on $m$ vertices.
        \item[(d)] Find minimal distance of the code associated with Petersen graph
    \end{itemize}
\end{prob}

\begin{proof}
    \begin{itemize}
        \item[(a)] 
        \item[(b)] 
        \item[(c)] 
        \item[(d)] 
    \end{itemize}
\end{proof}

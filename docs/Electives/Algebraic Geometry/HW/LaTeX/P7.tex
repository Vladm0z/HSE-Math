\section{HW 7}

\begin{prob}
Пусть $X$ схема, $f \in \mathcal{O}_X(X), X_f$ подмножество точек $X$, где $f$ не обращается в нуль (т е образ $f$ не лежит в максимальном идеале соответствующего локального кольца). Предположим, что $X$ нетерова, или же отделима и квазикомпактна. Покажите, что $X_f$ открыто и гомоморфизм ограничения индуцирует изоморфизм $\mathcal{O}_X(X)_f$ и $\mathcal{O}_X\left(X_f\right)$.
\end{prob}
\begin{proof}
$O_X$ - квазикомпактный пучок $\Rightarrow$ $\exists U_i = \operatorname{Spec} A_i$
\begin{gather*}
    O_X(U_i) = \tilde{A_i}\quad
    \bigcup_{i = 1}^{N} U_i = X \text{ $X$ - нетерово или квазикоспактное}\\
    V_{ii} = U_i \cap X_f = D(f_I) \text{, где $f_i$ ограничение $f$ на $U_i$}
    \text{так как} (f_i)_p = f_p \Rightarrow V_i \text{ - открытое}\\
    x_f = \bigcup V_i \Rightarrow X_f \text{ - открытое}\\
    O_X(V_i) \simeq O_X(U_i)_f = (A_i)_f\\
    O_X \text{ - пучок} \Rightarrow \exists \text{s.e.s}\\
    0 \to O_X(X) \to \oplus O_X(U_i) \to \oplus O_X(U_i \cap U_j)
\end{gather*}
% https://tikzcd.yichuanshen.de/#N4Igdg9gJgpgziAXAbVABwnAlgFyxMJZABgBpiBdUkANwEMAbAVxiRGJAF9T1Nd9CKMgEYqtRizYduvbHgJFh5MfWatEIAPIB9ABoAKXQEptAMy48QGOQMWlR1VZI06Dus0Yuz+ClACZlRwl1EAAdUIg0ZjgAAld9AFVtLBNzGSs+eUFkAIdxNTZwyOi4vX0ANWTPdOsfbIBmQPznMIioplj4pKwY8IBjOjQYpIArVK8Mm19kRrynEKL2zrLKnv7BmMqxrjEYKABzeCJQUwAnCABbJDIQHAgkP3Szy6QlW-vEeqfzq8QA96QABYggUNPsJs9fsCAYgAGwgloACwhPyQjRhAFYESF9gByFEvRBYmEAdmxbER+O+hP+dzR5I04UYaERdAJUOodKJDNaACMYDg2dQGFgwCE4BARVAQNRETA6NLEGAmAwGJy6FgGGxIGL2Uh4aSeeF9nQLhc2ZwKJwgA
\begin{tikzcd}
0 \arrow{r} & O_X(X)_f \arrow{r}{g} \arrow{d}{\alpha} & \oplus O_X(U_i)_f \arrow{r}{h} \arrow{d}{\beta} & \oplus O_X(U_i \cap U_j)_f \arrow{d}{\gamma} \\
0 \arrow{r} & O_X(X_f) \arrow{r}{g'} & \oplus O_X(V_i) \arrow{r}{h'} & \oplus O_X(V_i \cap V_j)
\end{tikzcd}
$\beta$ - ихоморфизм, $\alpha$ - инъекция, так как $g, \beta, g'$ - инъекции\\
\noindent
так как $U_i \cap U_j$ - афф, если $X$ - отделима или покрыта конечным числом афф., если $X$ - нетерова $\Rightarrow$ либо $\gamma$ - изоморфизм, либо $\gamma$ - инъекция по той же причине, что и $\alpha$ $\Rightarrow$ $\gamma$ как минимум инъекция $\Rightarrow$ по лемме о гомоморфизме $\alpha$ - сюръекция, а следовательно изомофизм
\end{proof}
\vskip 0.6in





\begin{prob}
Пусть $X$ схема и $f_1, \ldots, f_k$ порождают $\mathcal{O}_X(X)$. Предположим, что $X_{f_i}$ аффинны, докажите, что $X$ тоже аффинно.
\end{prob}
\begin{proof}
\begin{gather*}
    \varphi: X \to \operatorname{Spec}(\Gamma (X, O_X))\\
    \varphi_{f_i}: X_{f_i} \to \operatorname{Spec}(\Gamma(X, O_X)_{f_i}) \simeq \operatorname{Spec}(\Gamma(X_{f_i}, O_X))
    \quad\text{так как}
    \quad X_{f_i} - \operatorname{aff}
    \Rightarrow \varphi \text{ - изоморфизм}\\
    O_X = \langle f_1, \ldots, f_k \rangle
    \Rightarrow X = \bigcup X_{f_i}\\
    \operatorname{Spec}(\Gamma(X, O_X)) = \bigcup \operatorname{Spec}(\Gamma(X, O_X)_{f_i})
\end{gather*}
то есть $\varphi$ - изоморфизм на базе $\Rightarrow$ $\varphi$ - изоморфизм
\end{proof}
\vskip 0.6in





\begin{prob}
Выведите отсюда, что аффинность морфизма $f: X \rightarrow Y$ можно проверять на покрытии, то есть следующие условия равносильны:
\begin{itemize}
\item[(a)] $f$ аффинный, то есть прообраз любого аффинного открытого подмножества тоже аффинный
\item[(b)] существует открытое аффинное покрытие $U_i$ схемы $Y$, такое, что все $f^{-1}\left(U_i\right)$ аффинны.
\end{itemize}
\end{prob}
\begin{proof}
\begin{itemize}
\item[]
\item[$(a) \Rightarrow (b)$] -- очевидно 
\item[$(b) \Rightarrow (a)$]
    \begin{gather*}
        U \subset Y - \text{aff}\qquad
        U = \operatorname{Spec} A
        \quad Y = \bigcup U_i = \bigcup \operatorname{Spec} A_i\\
        U \cap U_i = \bigcup U_{i,j}\\
        \Rightarrow U_{i,j} = (\operatorname{Spec} A_i)_{g_j} = (\operatorname{Spec} A)_{h_{i,j}}\\
        f^{-1}(U_i) = V_i = \operatorname{Spec} B_i\\
        f^{-1}(U_{i,j}) = (\operatorname{Spec} B_i)_{f^{\#}(g_j)}\\
        (f^{-1}(U))_{f^{\#}(h_{i,j})} = (\operatorname{Spec} B_i)_{f^{\#}(g_j)}\\
        O_X(f^{-1}(U)) = \langle f^{\#}(h_{i,j}) \rangle\\
        \Rightarrow f^{-1}(U) \text{ -- aff (по 2)} 
    \end{gather*}
\end{itemize}
\end{proof}
\vskip 0.6in





\begin{prob}
Докажите, что конечность морфизма можно проверять на покрытии.
\end{prob}
\begin{proof}
Тут также $(a) \Rightarrow (b)$ -- очев, $(b) \Rightarrow (a)$ по прошлой задаче: $f^{-1}(U)$ -- aff. Факт из коммутативной алгебры, если $R = \langle f_1, \ldots, f_n \rangle$, (*) $g: R \to S$, $R_{f_i} \to S_{g(f_i)}$ - конечно $\Rightarrow$ $g$ - конечно; $O_y(U)_{h_{i,j}} \to O_X(f^{-1}(U))_{f^{\#}(h_{i,j})}$ - конечно $\Rightarrow$ $f^{-1}(U) \to U$ - конечно
\vskip 0.2in \noindent
(*): $g: R \to S$ - конечно $\Rightleftarrow$ $g$ - целый морфизм и $S$ - $R$-алгебра кон. типа\\
Зафиксируем $s \in S\quad I \subset R[x]\quad \forall p \in T\quad p(s) = 0$\\
$J \subset R$ - коэфф. при старших степенях у элем. $I$
\begin{gather*}
    s \in S
    \Rightarrow \frac{s}{1} \in S_{f_i}
    \Rightarrow \exists p_i \in R_{f_i}[x]:\ p_i(\frac{s}{1}) = 0\\
    \exists n_i: (f_i^{n_i} p_i) \in R[x]
    \Rightarrow f_i^{n_i} p_i \in I
\end{gather*}
так как $1 = \sum a_i f_i$ то $\exists N$ - достаточно большой что: $1 = 1^N = (\sum a_i f_i)^N \in J$ $\Rightarrow$ $g$-целый\\
\begin{gather*}
    S_{f_i} = \langle s_{i1}, \ldots, s_{in} \rangle \text{ - как } R_{f_i} \text{ - алг}\\
    \Rightarrow \frac{s}{1} = \sum_{i = 1}^{n} a_{ij} s_{ij}
    \Rightarrow \exists n_i: \frac{f_i^{n_i} s}{1} = \sum_{i=1}^{n} f_i^{n_i} a_{ij} s_{ij} \in S\\
    \Rightarrow s = 1 \cdot s
    = 1^{N} \cdot s
    = (\sum b_i f_i)^N s
    = (\sum b_i f_i)^N \sum a_{ij} s_{ij} \in S
\end{gather*}
\end{proof}
\vskip 0.6in





\begin{prob}
Пусть $X$ схема и $\mathcal{F}$ пучок $\mathcal{O}_X$-модулей. Докажите что $\mathcal{F}$ квазикогерентный тогда и только тогда, когда у любой точки есть окрестность $U$ и точная последовательность пучков $\mathcal{O}_X$-модулей
$$
\mathcal{O}_U^{\oplus I} \rightarrow \mathcal{O}_U^{\oplus J} \rightarrow \mathcal{F}\big|_U \rightarrow 0 .
$$
Здесь $I, J$ - некоторые множества индексов.
\end{prob}
\begin{proof}
\begin{itemize}
\item[]
\item[($\Rightarrow$)]
\begin{gather*}
    \mathcal{F}\big|_{U_i} = \tilde{M}_i\qquad
    O_x \big|_{U_i} = U_{U_i} = \tilde{A_i}\\
    X = \cup U_i = \cup \operatorname{Spec} A_i\\
    M_i \text{ - модуль над } A_i \Rightarrow
    \exists \text{точная последовательность}\\
    A_i^{\oplus J} \to A_i^{\oplus I} \to M \to 0
    \text{, где } |I| \text{ - количество порождающих у } M\\ \text{ и } |J| \text{ - количество соотношений на эти порождающие}\\
    \Rightarrow \forall q \in U_i = \operatorname{A_i}\\
    (A_i^{\oplus J})_q \simeq (A_i)^{\oplus J}_q \to (A_i^{\oplus I})_q \to M_q \to 0 \text{ - точная последовательность}\\
    \Rightarrow (\tilde{A}_i)^{\oplus J}_q \simeq  (O_{U_i})^{\oplus J}_q \to (\tilde{A}_i)^{\oplus I}_q \simeq (O_{U_i})^{\oplus I}_q \to \tilde{M}_q \simeq (\mathcal{F}\big|_{U_i})_q \to 0\\
    \text{точная последовательноть } \forall U_i\\
    \Rightarrow O_{U_i}^{\oplus J} \to O_{U_i}^{\oplus I} \to \mathcal{F}\big|_{U_i} \to 0 \text{ - точная последовательность}
\end{gather*}
\item[($\Leftarrow$)]
\begin{gather*}
    \text{зафиксируем} x \in X
    \quad \exists U^x \ni x
    \quad O^J_U \to O^I_U \to \mathcal{F}\big|_U \to 0\\
    \exists U^x_i:\ U^x = \cup U_i^x = \cup \operatorname{Spec} A_i\quad \text{без ограничения общности} x \in U^x_1\\
    O^J_{U^x_1} \to O^I_{U^x_1} \to \mathcal{F}\big|+{U^x_1} \to 0\quad \text{точная}\\
    M = \operatorname{coker} f_{U^x_1} = \frac{A^I_1}{f(A^J_1)}\quad \forall p \in U^x_1\\
    M_p = \left(\frac{A^I_1}{f(A^J_1)}\right)_p
    = \frac{(A^I_1)_p}{f(A^J_1)_p}
    = \frac{(A_1)^I_p}{f((A_1)^J_p)}
    = \operatorname{coker} f_p = \mathcal{F}_p\\
    \Rightarrow \mathcal{F}\big|_{U^x_1} \simeq \tilde{M}
\end{gather*}
этот процесс не зависит от выбора $x$ $\Rightarrow$ у них есть аффинное покрытие $X = \cup U^x_1$ и $\mathcal{F}$ огранич. на $\forall U^x_1 = \operatorname{Spec} A^x_1$ - это модуль над $A^x_1$
\end{itemize}
\end{proof}
\vskip 0.6in





\begin{prob}
Пусть $f: X \rightarrow Y$ аффинный морфизм, проверьте, что для квазикогерентного $\mathcal{F}$ на $X$ и $\mathcal{G}$ на $Y$ верно $f_*\left(\mathcal{F} \otimes f^* \mathcal{G}\right)=f_* \mathcal{F} \otimes \mathcal{G}$.
\end{prob}
\begin{proof}

\end{proof}

\section{HW 4}

\begin{prob}
Покажите, что гомоморфизм градуированных колец $f: R \rightarrow R^{\prime}$ индуцирует морфизм схем в $\operatorname{Proj}(R)$ из дополнения в $\operatorname{Proj}\left(R^{\prime}\right)$ к $V(I)$, где $I$ - идеал, порождаемый в $R^{\prime}$ образом $R_{+}$(вначале можно проверить почти очевидную склейку морфизмов: если схема $X$ покрыта открытыми $U_i$ и заданы морфизмы $g_i: U_i \rightarrow Y$, совпадающие на пересечениях, то существует и морфизм $g: X \rightarrow Y$, совпадающий с $g_i$ на $U_i$ )
\end{prob}
\begin{proof}
\begin{gather*}
	g: X \to Y\quad g^{\#}: O_Y \to g_{*} O_X\quad f_i: U_i \hookrightarrow X\\
	g_i: U_i \to Y\quad g_i^{\#} O_Y \to g_{i*} O_{U_i}\quad f_i^{\#}: O_x \to f_{i*} O_{U_i}\\
	\operatorname{Proj} R = \bigcup_{f \in R_+} D_+(f)
	\Rightarrow \operatorname{Proj} R' \backslash V(\varphi (R_+))
	= \Bigcup_{f \in R_+} D_+(\varphi f)\\
	\varphi_f \left(\frac{a}{f_n}\right) = \frac{\varphi (a)}{\varphi (f)^n}\\
	\text{т.к. $\varphi$ - гомоморфизм гр колец, то}\\
	\varphi_f(R_{(f)}) < R'_{(\varphi (f))}
	\Rightarrow \exists \varphi_{(f)}: R_{(f)} \to R'_{(\varphi (f))}
	\Rightarrow \varphi*_{(f)}: D_+ (\varphi (f)) \to D_+ (f)\quad \forall f \in R_+\\
	D_+(\varphi(f)) \cap D_+(\varphi(g)) = D_+ (\varphi(f) \varphi(g)) = D_+(\varphi(fg))\\
	\Rightarrow \varphi^*_{(f)}\bigg|_{D_+(\varphi(f)) \cap D_+(\varphi(g))}
	= \varphi^*(g)\bigg|_{D_+(\varphi(f)) \cap D_+(\varphi(g))}
	= \varphi^*(fg)\\
	\Rightarrow \text{ можно склеить } \varphi^*: \operatorname{Proj} R'\backslash V(\varphi(R_+)) \to \operatorname{Proj} R
\end{gather*}
\end{proof}
\begin{comment}

\end{comment}
\vskip 0.6in






\begin{prob}
\begin{itemize}
\item[(а)] Докажите, что схема $\mathbb{P}_k^1$ ( $k$ поле) изоморфна замкнутой подсхеме, определенной уравнением $y^2=x z$ в $\mathbb{P}_k^2$ (т.е. $\left.\operatorname{Proj}\left(k[x, y, z] /\left(y^2-x z\right)\right)\right)$.
\item[(б)] Докажите, что если $k$ алгебраически замкнуто, то все неприводимые коники над $k$ (то есть подсхемы $\mathbb{P}_k^2$, определенные обращением в нуль неприводимого однородного многочлена второй степени) изоморфны над $k$.
\item[(в)] Верно ли это в общем случае, т.е. если $k$ не является алгебраически замкнутым?
\end{itemize}
\end{prob}
\begin{proof}
\begin{itemize}
\item[(а)]
	\begin{gather*}
		\frac{k[x,y,z]}{y^2-xz} \simeq k[x_0^2, x_0 x_1, x_1^2]\\
		\operatorname{Proj} k[x_0^2, x_0 x_1, x_1^2] = {D_+}^{R'}(x_0^2) \cup {D_+}^{R'}(x_0 x_1) \cup {D_+}^{R'}(x_1^2)\\
		\operatorname{Proj} k[x_0, x_1] = {D_+}^{R}(x_0) \cup {D_+}^{R}(x_1)\\
		R'(x_0 x_1) = \langle \frac{x_1^2}{x_0 x_1} = \frac{x_1}{x_0}, \frac{x_0^2}{x_0 x_1} = \frac{x_0}{x_1} \rangle\\
		R'(x_0^2) = \langle \left(\frac{x_1}{x_0}\right)^2, \frac{x_1 x_0}{x_0^2} = \frac{x_1}{x_0} \rangle = \langle \frac{x_1}{x_0} \rangle\\
		R'(x_1^2) = \langle \frac{x_0}{x_1} \rangle\\
		\Rightarrow R'(x_0 x_1) = R'(x_0^2) x_k R'(x_1^2)
		\Rightarrow {D_+}^{R}'(x_0 x_1) = {D_+}^{R'}(x_0^2) \cup {D_+}^{R'}(x_1^2)
		\Rightarrow \operatorname{Proj} R' = {D_+}^{R'} (x_0^2) \cup {D_+}^{R'}(x_1^2)\\
		R(x_0) = \langle \frac{x_1}{x_0} \rangle\quad
		R(x_1) = \langle \frac{x_0}{x_1} \rangle
		\Rightarrow R(x_0) \simeq R'(x_1^2)\\
		R(x_1) \simeq R'(x_1^2)
		\Rightarrow {D_+}^{R'}(x_0^2) \simeq {D_+}^{R}(x_0)\\
		{D_+}^{R'}(x_1^2) \simeq {D_+}^{R}(x_1)\quad
		{D_+}^{R'}(x_0^2) \cap {D_+}^{R'}(x_1^2) \simeq
		{D_+}^{R'}(x_0^2 x_1^2) \simeq {D_+}^{R}(x_0 x_1) \simeq {D_+}^{R}(x_0) \cap {D_+}^{R}(x_1)\\
		\Rightarrow \operatorname{Proj} R \simeq \operatorname{Proj} R'
	\end{gather*}
\item[(б)]
	$R = k[x,y]\quad R'= \frac{k[x,y,z]}{f}$\\
	с точностью до линейной замены $\exists !$ неприводимая коника над алгебраически замкнутым полем, которая зажана $xz = y^2$. Если $x_0 = l_0(x,y,z)$ $x_1 = l_1(x,y,z)$ $x_2 = l_2(x,y,z)$, где $l_i$ - линейные уравнения, тогда $k[x_0, x_1, x_2] \simeq k[x,y,z]$, откуда $R' = \frac{k[x,y,z]}{y^2-xz} \Rightarrow \operatorname{Proj} R'\simeq \mathbb{P}'_{k}$
\item[(в)]
	\begin{gather*}
		X = \operatorname{Proj} \frac{\mathbb{R}[x,y,z]}{x^2+y^2+z^2}\\
		\{\mathbb{R} \text{ - точки } X\} \leftrightarrow \{\text{решения} x^2+y^2+z^2\}
		\Rightarrow \{\mathbb{R} \text{ - точки} X\} = \varnothing\\
		\text{но у } \operatorname{Proj} \frac{\mathbb{R}[x,y,z]}{xz-y^2}
		\text{ есть } \mathbb{R} \text{ - точки, так как } (1,1,1) \text{ - решение } xz-y^2
	\end{gather*}
\end{itemize}
\end{proof}
\vskip 0.6in






\begin{prob}
Пусть $k$ поле, рассмотрим градуированное кольцо $R\left(a_0, \ldots, a_n\right)$, которое представляет собой кольцо многочленов $k\left[X_0, \ldots, X_n\right]$ с нестандартной градуировкой $\operatorname{deg}\left(X_i\right)=a_i$ (так что обычное кольцо многочленов - это $R(1, \ldots, 1))$. Положим $\mathbb{P}_k\left(a_0, \ldots, a_n\right)=\operatorname{Proj}\left(R\left(a_0, \ldots, a_n\right)\right)$.
\begin{itemize}
\item[(а)] Покажите, что $\mathbb{P}_k\left(a_0, \ldots, a_n\right) \cong \mathbb{P}_k\left(d a_0, \ldots, d a_n\right)$ для всех $d \in \mathbb{Z}_{>0}$.
\item[(б)] Покажите, что $\mathbb{P}_k(a, b) \cong \mathbb{P}_k^1$ для всех $a, b \in \mathbb{Z}_{>0}$.
\item[(в)] Постройте замкнутое вложение $\mathbb{P}_k(1,1,2)$ в $\mathbb{P}_k^3$ и опишите $\mathbb{P}_k(1,1,2)$ геометрически.
\item[(г)] Изоморфны ли $\mathbb{P}_k(1,1,2)$ и $\mathbb{P}_k^2 ?$
\end{itemize}
\end{prob}
\begin{proof}
\begin{itemize}
\item[(а)]
	\begin{gather*}
		\operatorname{Proj}(R(a_0, \ldots, a_n)) = \bigcup_{i=0}^{n} D_+(X_i)\\
		A(x_i) = \left(\frac{x_0^{a_i}}{x_i^{a_0}}, \ldots, \frac{x_n^{a_i}}{x_i^{a_n}}\right) \text{ - как $k$ алгебра}\\
		\operatorname{Proj}(R(da_0, \ldots, da_n)) = \bigcup_{i=0}^{n} D'_+(X_i)\\
		A'(x_i) = \left(\left(\frac{x_0^{a_i}}{x_i^{a_0}}\right)^d, \ldots, \left(\frac{x_n^{a_i}}{x_i^{a_n}}\right)^d\right) \text{ - как $k$ алгебра}\\
		A(x_i) \simeq A'(x_i)\\
		\frac{f}{x_i^k} \to \frac{f'}{x_i^k}\quad f'=f\quad \deg f = k\quad \deg f'= dk\\
		\Rightarrow D_+(x_i) \simeq D'_+(x_i)\quad D_+(x_i) \cap D_+(x_j) \simeq D'_+(x_i) \cap D'_+(x_i)\\
		\Rightarrow \operatorname{Proj} R(a_0, \ldots, a_n) \simeq \operatorname{Proj}(da_0, \ldots, da_n)
	\end{gather*}
\item[(б)]
	\begin{gather*}
		\mathbb{R}'_k = \operatorname{Proj} k[x,y] = \operatorname{Proj} A\\
		\mathbb{R}'_k (a,b) = \operatorname{Proj} R(a,b) = \operatorname{Proj} A'\\
		\operatorname{Proj} A = D_+(x) \cup D_+(y)\quad \operatorname{Proj} A' = D'_+(x) \cap D'_x(y)\\
		A(x) = \langle \frac{y}{x} \rangle\quad A(y) = \langle \frac{x}{y} \rangle\\
		A'(x) = \langle \frac{y^a}{x^b} \rangle\quad A'(y) = \langle \frac{x^b}{y^a} \rangle\\
		A(x) \simeq A'(x)\quad A(y) \simeq A'(y) \Rightarrow D_+(x) \simeq D'_+(x)\\
		D_+(x) \cap D_+(y) = D_+(xy) \simeq \operatorname{Spec} A(xy)\\
		A(xy) = \langle \frac{x^2}{xy} = \frac{x}{y}, \frac{y^2}{xy} = \frac{y}{x} \rangle \simeq \frac{k[x_0, x_1]}{x_0 x_1 - 1}\\
		D'_+(x) \cap D'_+(y) = D'_+(xy) \simeq \operatorname{Spec} A'(xy)\\
		A'(xy) = \langle \frac{x^{a+b}}{(xy)^a} = \frac{x^b}{y^a}, \frac{y^{a+b}}{(xy)^b} = \frac{y^a}{x^b} \rangle \simeq \frac{k[x_0, x_1]}{x_0 x_1 - 1}\\
		\Rightarrow \operatorname{Proj} A \simeq \operatorname{Proj} A'
	\end{gather*}
\item[(в)]
	Из 2 задачи знаем что
	\begin{gather*}
		k[x_0^2, x_0 x_1, x_1^2] \simeq \frac{k[x,y,z]}{y^2-xz}
		\Rightarrow k[x_0^2, x_0 x_1, x_1^2][t]
		\simeq \frac{k[x,y,z]}{y^2-xz}[t]
		\simeq \frac{k[x,y,z,t]}{y^2-xz}\\
		\operatorname{deg}t = 2\quad
		R = k[x_0, x_1, t],\quad
		R'= k[x_0^2, x_0 x_1, x_1^2, t]\\
		\operatorname{Proj} R \simeq \operatorname{Proj} R'
		\Rightarrow \mathbb{P}_k(1,1,2) \simeq \operatorname{Proj} \frac{k[x,y,z,t]}{y^2-xz}
	\end{gather*}
	это проективный конус над коникой из 2 задачи
\item[(г)]
	$\mathbb{P}_k(1,1,2) \simeq \operatorname{Proj} \frac{k [x,y,z,t]}{y^2-xz}$\\
	аффинная окрестность вершины конуса это $\operatorname{Spec} \frac{k[x,y,z]}{y^2-xz}$, вершина $m = (x,y,z)$, $\frac{m}{m^2}$ - трехмерное векторное пространство над $k$, порожд. $[x],[y],[z]$\\
	$\frac{k[x,y,z]}{y^2-xz} \simeq k[x_0^2, x_0 x_1, x_1^2] \subset k[x_0, x_1] \Rightarrow htm \leq 2$ $(0) \subset (x,y) \subset (x,y,z) \Rightarrow htm \geq 2 \Rightarrow htm = 2$, то есть размерность лок. кольца в точке $(x,y,z) = 2$, $2 < 3$, то есть конус не регулярен в вершине, то есть он не может быть изоморфен регулярной схеме.
\end{itemize}
\end{proof}
\begin{comment}

\end{comment}
\vskip 0.6in






\begin{prob}
Если $\operatorname{Proj}(R)$ - конечное дискретное множество, докажите, что оно покрывается одной аффинной картой $D_{+}(f)$ (то есть на самом деле совпадает со $\operatorname{Spec}\left(R_{(f)}\right)$.
\end{prob}
\begin{proof}
$X = \operatorname{Proj} (R)$ Пусть $\not \exists f: X = D_+ (f) \Rightarrow \forall f \in R_+\quad \exists x \in X: f(x) = 0$. Пусть $Y < X$ - минимальное подмножество, такое что $\forall f \in R_+\ \exists y \in Y\quad f(y) = 0$, так как $\forall x \in X\quad x = I \cap R_0\quad R_+ \not\subset I$, тогда $|Y| \neq 1$ (только если $|X| = 1$, но тогда $X = \operatorname{Spec} O_{X,X}$)
\begin{gather*}
	\forall y \in Y\ \exists a_y \in R_+
	\text{т.ч. } a_y(y) = 0\quad \forall y'\in Y\quad a_y(y')\neq 0\\
	b_x = \Prod_{y \in Y - \{x\}}a_y\quad b_x(y) = 0\quad \forall y \neq X\\
	\Rightarrow \sum_{x \in Y} b_x (y) \neq 0\quad \forall y \in Y
\end{gather*}
\end{proof}

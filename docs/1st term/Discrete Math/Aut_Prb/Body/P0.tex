\newpage
	\section*{Листок 0. Индукция. Элементарная комбинаторика.}
		\subsection{1}
		а)\newline
		\begin{gather*}
		\sin(x) + \sin(2x) + ... + \sin(nx) = \frac{\sin(\frac{n+1}{2}) * \frac{\sin(nx)}{2}}{\sin(\frac{x}{2})}
		\end{gather*}
		\begin{gather*}
		\sum_{k = 0}^{n-1} \sin(a + kd) =\\ \sin(a) + \sin(a+k) + ... + \sin(a + (k-1)d) =\\ \frac{\sin(\frac{nd}{2})}{\sin(\frac{d}{2})} * \sin(\frac{2a + (n-1)d}{2})
		\end{gather*}
		пусть $a = x$ и $d = x$ тогда
		\begin{gather*}
		\frac{\sin(\frac{nd}{2})}{\sin(\frac{d}{2})} * \sin(\frac{2a + (n-1)d}{2}) =\\ \frac{\sin(\frac{nx}{2})}{\sin(\frac{x}{2})} * \sin(\frac{2x + (n-1)x}{2}) =\\
		\frac{\sin(\frac{nx}{2})}{\sin(\frac{x}{2})} * \sin(\frac{(n+1)x}{2})
		\end{gather*}
		б)\newline
		 докажем по индукции
		\begin{gather*}
		1^2+2^2+...+n^2 = \frac{n(n+1)(2n+1)}{6}
		\end{gather*}
		
		\begin{gather*}
		1^2+2^2+...+n^2 + (n+1)^2 = \frac{1}{6}n(n+1)(2n+1) + (n+1)^2  = \\
		\frac{1}{6}(n+1)(n(2n+1)+6(n+1)) = \frac{1}{6}(n+1)(2n^2+7n+6) = \\
		\frac{1}{6}(n+1)(n+2)(2n+3) = \frac{1}{6}(n+1)((n+1)+1)(2(n+1)+1) 
		\end{gather*}
		в)\newline
		 докажем по индукции
		\begin{gather*}
		1^3+2^3+...+n^3 = (\frac{n(n+1)}{2})^2
		\end{gather*}
		\begin{gather*}
		1^3+2^3+...+n^3 +(n+1)^3 = (\frac{n(n+1)}{2})^2 +(n+1)^3 =\\
		\frac{n^2(n+1)^2 + 4(n+1)^3}{4} = \frac{n^2(n+1)^2}{4} = (\frac{n(n+1)}{2})^2
		\end{gather*}
		\newpage
		г)\newline
		\begin{gather*}
		\frac{1}{1*2} + \frac{1}{2*3} + ... + \frac{1}{(n-1)n} = \frac{n}{n+1}
		\end{gather*}
		\begin{gather*}
		\frac{1}{1*2} + \frac{1}{2*3} + ... + \frac{1}{(n-1)n} = 
		(\frac{1}{1} - \frac{1}{2}) + (\frac{1}{2} - \frac{1}{3}) + ... = \frac{1}{1} - \frac{1}{n} = \frac{n-1}{n}
		\end{gather*}
		\subsection{2}
		\begin{gather*}
		\sqrt{1 +\sqrt{1 + \sqrt{1 + ...}}} < 2
		\end{gather*}
		пусть $a_{n+1} = \sqrt{1 + a_n}$, тогда
		\begin{gather*}
		\lim a_{n+1} = \sqrt{1 + \lim a_n} \Longrightarrow a = \sqrt{1 + a}
		\end{gather*}
		
		\subsection{3}
		Воспользуемся методом индукции для доказательства. Если n равно нулю, тогда:
		$n = 0 \Longrightarrow x^0 + (\frac{1}{x})^0 = 1 + 1 = 2$ — целое \\
		Если n равно единице, тогда:
			\begin{gather*}
			n = 1 \Longrightarrow x^1 + \frac{1}{x}^1 = x + \frac{1}{x} \text{	   целое по условию}
			\end{gather*}
		Пусть мы доказали, что 
			\begin{gather*} 
			x + \frac{1}{x}, x^2 + \frac{1}{x}^2, ..., x^{k-1} + \frac{1}{x}^{k-1}, x^k + \frac{1}{x}^k
			\end{gather*}
		это целые числа.\\
		Докажем, что $x^{k+1} + \frac{1}{x}^{k+1}$ также является целым числом. \\Для этого умножим $x^k + \frac{1}{x}^k$ на $x + \frac{1}{x}$: 
			\begin{gather*}
		 	(x^k + \frac{1}{x}^k)*(x + \frac{1}{x}) = x^{k+1} + x^{k-1} + \frac{1}{x}^{k-1} + \frac{1}{x}^{k+1} 
			 \end{gather*}
		 
		Сгруппируем слагаемые с одинаковыми степенями при x: 
		 	\begin{gather*}
			 (x^k + \frac{1}{x}^k)*(x + \frac{1}{x}) = (x^{k+1} + \frac{1}{x}^{k+1}) + (x^{k-1} + \frac{1}{x}^{k-1})
			 \end{gather*}
		Выразим, чему равно $x^{k+1} + \frac{1}{x}^{k+1}$:
			\begin{gather*}
		 	x^{k+1} + \frac{1}{x}^{k+1} = (x^k + \frac{1}{x}^k)*(x + \frac{1}{x}) - (x^{k-1} + \frac{1}{x}^{k-1})
			\end{gather*}
		 
		Поскольку множители $x^k + \frac{1}{x}^k$ и $x + \frac{1}{x}$ являются целыми числами (первое из доказанного, второе из условия), значит и их произведение также целое. Число $x^{k-1} + \frac{1}{x}^{k-1}$ является целым, значит и разность будет также целой. 		
		Следовательно, число $x^{k+1} + \frac{1}{x}^{k+1}$ является целым. При отрицательных n все выполняется аналогично.
		
		\subsection{4}
		Докажем это утверждение по индукции.
		\newline
		База: для $n = 2, 3$
		\newline
		$F_1 = 1, F_2 = 1, F_3 = 2, F_4 = 3$
		
		$n = 2: F_2*F_2 = 1, F_1*F_3 = 2$
		
		$n = 3: F_3*F_3 = 4, F_2*F_4 = 3$

		Переход:\newline
		пусть для н верно предположение индукции, тогда докажем его для н+1.
		Утверждение для $n+1: F_{n+1} * F_{n+1} = F_{n+2} * F_n \pm 1$.\newline
		Заметим, что 
		\begin{gather*}
		F_{n+2} * F_n = (F_n + F_{n+1}) * F_n = F_n*F_n + F_n*F_{n+1}
		\end{gather*}
		По предположению индукции $F_n*F_n = F_{n-1}*F_{n+1} \pm 1$, \newline
		следовательно
		\begin{gather*}
		F_n*F_n + F_n*F_{n+1} =\\ F_{n-1}*F_{n+1} + F_n*F_{n+1} \pm 1 =\\ F_{n+1}*(F_{n-1} + F_n) \pm 1 \\= F_{n+1}*F_{n+1} \pm 1
		\end{gather*}
		значит
		\begin{gather*}
		 F_{n+2} * F_n = F_{n+1}*F_{n+1} \pm 1 \Longleftrightarrow F_{n+1} * F_{n+1} = F_{n+2} * F_n \pm 1
		\end{gather*}
		
		
		\subsection{5}
		а)\newline
		из вершины можно провести диагональ в любую другую вершину, кроме 2 соседних.Также каждуюдиагональ мы учитываем 2 раза
		\begin{gather*}
		\frac{n(n-3)}{2}
		\end{gather*}
		б)\newline
		\begin{gather*}
		\frac{n(n-1)(n-2)}{3!}
		\end{gather*}
		\subsection{6}
		а)\newline
		Пока что будем считать, что буквы а различны, то есть существуют $a_1$, $a_2$, $a_3$. Аналогично существуют $\text{н}_1$, $\text{н}_2$.
		Тогда количество слов = 6! (на первое место можно поставить любую из 6 букв, на второе - любую из оставшихся и тд., из чего следует что кол-во слов = 6*5*4*3*2*1 = 6!)
		\newline
		Теперь объединим слова в группы так, что из каждого слова в группе можно получить другое перестановкой букв $a_1$, $a_2$ и $a_3$ между собой, а также $\text{н}_1$ и $\text{н}_2$ между собой
		Заметим, что в каждой группе будет 3! * 2! слов, так как есть 3! способов расставить $a_1$, $a_2$ и $a_3$ в каком либо порядке, аналогично 2! способов для "н".
		\newline
		Количество слов, которое можно составить из слова ананас равно количеству получившихся групп, потому что 2 слова с пронумерованными буквами будут одинаковыми словами без номеров, только если одно слово получается из второго перестановкой "а" между собой и/или "н" между собой, следовательно кол-во слов $= \frac{6!}{3! * 2!} = \frac{720}{6*2} = 60$.
		\newline
		б)\newline		
		Аналогично, как в пункте а, будет считать, что существуют $a_1$, $a_2$ ... $a_k$ и $b_1$, $b_2$ ... $b_m$, из чего получаем, что слов с пронумерованными буквами $= (k+m)!$, а слов в группе $= k!*m!$, 
		из чего следует, что всего слов $= \frac{(k+m)!}{k!*m!}$
		
		\subsection{7}
		а)\newline
		Каждую ручку можно поместить в любую из 5 коробок, из чего следует, что количество способов положить 10 разных ручек в 5 разных коробок - $5^{10}$.
		\newline		
		б)\newline
		Будем считать, что карандашей не 10, а 15, при этом в каждой коробке должен быть хотя бы 1 карандаш. Заметим, что кол-во способов разложить карандаши не изменилось.
		Также пронумеруем карандаши от 1 до 15, и будем считать, что если в коробках $x_1$, $x_2$, ... $x_5$ карандашей, то в первой коробке лежат первые $x_1$ карандашей, во второй $x_2$ и т.д. Заметим, что количество способов разложить карандаши всё ещё не изменилось.
		\newline
		Теперь заметим, что количество способов разложить карандаши с заданными условиями равно количеству способов поставить 4 перегородки между 15-тью шарами, пронумерованных от 1 до 15, при этом никакие 2 перегородки на могут стоять "вместе". Количество способов поставить перегородки - ${14\choose 4} = C_{14}^4 = \frac{14!}{4!(14-4)!} = 1001$ (есть 14 мест, куда можно поставить перегородки).
		
		
		\subsection{8}
		а)
		Есть $2^n$ способов освещения, так как каждая лампочка может быть вкл или выкл независимо от других. Количество освещений с чётным кол-вом лампочек равно кол-ву освещений с нечётным кол-вом лампочек, так как каждому освещению из первого множества можно сопоставить освещение с изменённой последней лампочкой из другого множества (и аналогично наоборот), из чего следует, что данное сопоставление - биекция.
		\\
		б)
		Пусть у числа k различных простых делителей, тогда для любого делителя свободного от квадратов числа можно сказать, делится оно на один из k простых делителей или нет, причём отдельно для каждого простого. Заметим, что по множеству "ответов" однозначно определяется делитель, при этом других делителей нет, так как иначе свободное от квадратов число делится на какой то ещё простой делитель или квадрат предыдущих, из чего кол-во делителей - $2^k$.
		
		\subsection{9}
		Очевидно в каждых столбце и строке ровно 1 ладья. 
		\newline 
		Заметим, что есть 8 способов поставить ладью в первый столбец, 7 - во второй, так, чтобы ладьи друг друга не били, ... 1 - в последний.
		Из чего следует, что кол-во способов расставить ладьи $= 8!$.
		\newpage
		\subsection{10}
		Рассмотрим 2 случая: когда Иванов сидит с краю, и когда - нет.\\
        Если Иванов сидит с краю, то кол-во способов посадить Петрова - $n - 2$, а остальных - $n - 2$, $n - 3$ ... $1$ способом $\Rightarrow$ кол-во способов = $2*(n - 2)*(n - 2)!$ (Иванов может сидеть с любого края).\\
        Второй случай - кол-во способов посадить Петрова - $n - 3$, остальных - $n - 2$, $n - 3$ ... $1$ способом $\Rightarrow$ кол-во способов $= (n - 2)*(n - 3)*(n - 2)!$
        В сумме получается 
        \begin{gather*}
        (n - 2)!*(2*(n - 2) + (n - 2)*(n - 3)) = (n - 2)!*(n - 2)*(n - 1) = (n - 1)!*(n - 2)
        \end{gather*}
        \\
        б)\\
        Кол-во способов расположить Иванова - $n$, Петрова (после Иванова) - $n - 3$, для оставшихся - $(n - 2)!$, итого - $n*(n - 3)*(n - 2)!$

		\subsection{11}
		а) Докажем по индукции, что количество частей равно $1 + \dfrac{1}{2} n (n + 1)$. \newline
		База $(n = 1)$ -- очевидно. Пусть $n > 1$. По предположению индукции перед проведением $n$-й прямой было $1 +  \dfrac{1}{2}n(n - 1)$ частей. Новая прямая делится точками пересечения со старыми прямыми на $n$ интервалов. Каждый из этих интервалов разбивает одну часть на две. Следовательно, добавится n частей. Поэтому всего частей станет  $1 + \dfrac{1}{2} (n - 1)n + n = 1 + \dfrac{1}{2} n(n + 1)$. \newline \newline
        б) Докажем по индукции, количество частей равно $2 + n(n - 1)$. \newline
        База $(n = 1)$ -- очевидно. Пусть $n > 1$. По предположению индуцкии перед проведением $n$-й окружности было $2 +  (n - 1)(n - 2)$ частей. Рассмотрим $n$-ю окружность. Она пересекает предыдущие $(n - 1)$ окружностей не более чем в $2(n - 1)$ точках (каждую окружность -- не более, чем в двух точках). Следовательно, $n$-я окружность разбивается первыми $(n - 1)$ окружностями не более чем на $2(n - 1)$ дуг. Каждая дуга делит одну из частей, на которые плоскость была разделена $(n - 1)$ окружностями, еще на две части. Тем самым, каждая дуга прибавляет одну часть плоскости, и $n$-я окружность прибавляет не более $2(n - 1)$ частей плоскости. Более того, $n$-я окружность прибавляет ровно $2(n - 1)$ частей плоскости тогда и только тогда, когда она пересекает каждую из предыдущих окружностей в двух точках и все эти точки различны. Таким образом, $n$ окружностей делят плоскость не более чем на $2 + (2 + 4 + 6 + ... + 2(n - 1)) = n(n - 1) + 2$ части, причём равенство достигается, если каждая пара окружностей пересекается в двух точках и все эти точки пересечения различны (то есть никакие три окружности не проходят через одну точку).
		
		\subsection{12}
        Заметим, что все кратчайшие пути имеют длину $n$, так как за каждое ребро соединяет 2 вершины, отличающиеся в 1 координате $\Rightarrow$ если следить за "концом" пути, то после каждого перемещения сумма координат меняется не больше чем на 1, при этом изначально сумма 0, а в конце - $n$. Это доказывает, что пути имеют длину >= $n$, при этом существует путь длины $n$
        $(0, 0, 0, .., 0) -> (1, 0, 0, .., 0) -> (1, 1, 0, .., 0) -> ... -> (1, 1, 1, .., 1)$. Заметим, что если следить за концом пути, то в координатах будет заменяться один "0" на "1", кол-во способов выбрать "0" равно $n$ для первого шага, $n - 1$ для второго шага ... 1 для последнего $\Longrightarrow$  всего таких путей $n!$.
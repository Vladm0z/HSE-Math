\newpage	
	\section{Листок 2}
		
		\subsection{1}
		\begin{gather*}
		(x_1 + x_2 + ... + x_m)^n = \sum_{\substack{k_1, k_2, ... k_m \geqslant 0 \\ k_1 + k_2 + ... + k_m = n}}^{} \frac{n!}{k_1! ... k_m!} * x_1^{k_1} * ... * x_m^{k_m}
		\end{gather*}
		
		\subsection{2}
		1)\\
		\begin{gather*}
		(a + b)^{n} = (a + b)^{n-1} * (a + b) 
		\end{gather*}
		Заметим, что:
		\begin{gather*}
		{n \choose k} = \\
		\frac{n!}{k!(n-k)!} = \frac{n!}{k!(n-k)!} * \biggl( \frac{n-k}{n} + \frac{k}{n} \biggl)\\
		\frac{(n-1)!}{k!(n-k-1)!} + \frac{(n-1)!}{(k-1)!(n-k)!} = \\
		\frac{(n-1)!}{k!(n-1-k)!} + \frac{(n-1)!}{(k-1)!(n-1-k+1)!} = \\
		 = {{n-1} \choose {k}} + {{n-1} \choose {k-1}}
		\end{gather*}
		Тогда:
		\begin{gather*}
		(a + b)^{n} = \\
		{n \choose 0}a^{n} + {n \choose 1}a^{n-1}b + ... + {n \choose n}b^{n} =	\\ \\
		{n-1 \choose 0}a^{n} + \biggl({n-1 \choose 1} + {n-1 \choose 0}\biggl)a^{n-1}b + 
		\biggl({n-1 \choose 2} + {n-1 \choose 1}\biggl)a^{n-2}b^2 + ... + {n-1 \choose n-1}b^{n} = \\ \\
		\biggl({n-1 \choose 0}a^{n} + {n-1 \choose 1}a^{n-1}b + ... + {n-1 \choose n-1}ab^{n-1}\biggl) + 
		\biggl({n-1 \choose 0}a^{n-1}b + {n-1 \choose 1}a^{n-2}b^{2} + ... + {n-1 \choose n-1}b^{n}\biggl) = \\ \\
		\biggl({n-1 \choose 0}a^{n-1} + {n-1 \choose 1}a^{n-2}b + ... + {n-1 \choose n-1}b^{n-1}\biggl) * 
		(a + b) = \\ \\
		\biggl({n-1 \choose 0}a^{n-1} + {n-1 \choose 1}a^{n-2}b + ... + {n-1 \choose n-1}b^{n-1}\biggl) * 
		\biggl({1 \choose 0}a + {1 \choose 1}b\biggl) = \\ \\
		(a + b)^{n-1} * (a + b) 
		\end{gather*}
		2)\\
		\begin{gather*}
		(a + b)^{n+m} = (a + b)^{n} * (a + b) ^{m}
		\end{gather*}
		Заметим, что $m = \underbrace{1 + ... + 1}_m$, тогда мыможем прибавлять по $1$ как в (1) пункте
		
		\subsection{3}
		A)\\
		Заметим, что количество способов выбрать $n$ упорядоченных пар (то есть существуют первая, вторая ... n-тая пары, при этом люди в самих парах неупорядочены)\\
		= $\frac{(2n)!}{2^n}$, так как для первой пары есть $\frac{n*(n-1)}{2}$ способов выбрать 2х людей в любом порядке, из-за чего кол-во способов в $2!$ раза меньше, чем $n*(n-1)$, аналогично для второй пары $\frac{(n-2)*(n-3)}{2}$ и так далее. При этом все выборы независимы, из чего получаем, что всего $\frac{(2n)!}{2^n}$ способов. Тогда кол-во способов выбрать $n$ неупорядоченных пар в $n!$ меньше $\Longrightarrow$ ответ - $\frac{(2n)!}{2^n * n!}$\\
		
		B)\\
		Аналогично кол-во упорядоченных троек =  $\frac{(3n)!}{3!^n}$, из чего кол-во неупорядоченных троек $= \frac{(3n)!}{3!^n * n!}$
		
		\subsection{4}
		A)\\
		Пусть переменные это $x_1$, $x_2$, ... $x_n$.\\
		Расположим подряд $d + n - 1$ шарик, выкинем из этого ряда произвольным образом $n-1$ шарик. Тогда расстояние от начала ряда до первого выброшенного шарика - степень $x_1$, расстояние от первого выброшенного до второго выброшенного - степень $x_2$, и т.д.\\
		Тогда количество различных одночленов степени d от n переменных совпадает с количеством способов выкинуть $n-1$ шарик из $d + n - 1$, т.е. ${{d + n - 1} \choose {n-1}}$
		\\
		B)\\
		Пусть есть d шариков и требуется поставить $n-1$ перегородку(причем $d > n-1$, иначе у какого-то $x_i$ степень нулевая), тогда расстояние от 1 шарика до перегородки - степень $x_1$, от первой перегородки до 2 перегородки - степень $x_2$, и т.д.\\
		Всего есть ${{d-1}\choose{n-1}}$ способов расставить $n-1$ перегородку средиd шариков. 
		
		\subsection{5}
		Представим n в виде n подряд идущих единиц, тогда мы можем расставлятьмежду ними "+", объединяя числа(т.е. $1+1, 1$ = $2, 1$ ). Тогда, так как всего n единиц и последовательность не играет роль, то мы можем ставить "+" в $n-1$ промежуток, так мы получим $2^{n-1}$ вариант.
		
		\subsection{6}
		Заметим, что 2 ожерелья одинаковые - если из одного можно получить комбинацией поворотов и переворотов. Будем называть конкретное расположение бусин картинкой. Тогда всего есть $n!$ картинок и у каждого ожерелья "есть" $2k$ разных картинок, при этом очевидно, что каждая картинка принадлежит ровно одному ожерелью, из чего следует, что ожерелий = $\frac{n!}{2k}$
		
		\newpage
		\subsection{7}
		Заметим, что k-мерная грань - гмт из точек вида ($x_1$, $x_2$, $x_3$, ... $x_n$), где некоторые из x - константы, равные "0" или "1", при этом кол-во констант $= n - k$. (Сюдя по всему это и есть определение k-мерной плоскости) %почему??
		Тогда нетрудно видеть, что кол-во граней равно ${k \choose {n-k}} *2^{n-k}$, так как есть $k \choose {n-k}$ способов выбрать положение констант и $2^{n-k}$ способов выбрать сами константы.
		
		\subsection{8}
		Заметим, что каждый из таких кратчайших путей длины $n + m + p$, причём среди рёбер этого пути $n$ параллельны оси $x$, $m$ параллельны оси $y$, $p$ параллельны оси $z$. Тогда каждому пути можно сопоставить последовательность длины $n + m + p$, состоящей из $n$ "1", $m$ "2" и $p$ "3", где каждая цифра соответствует выбору ребра, || одной из осей. Кол-во последовательностей такой длины 
		\begin{gather*}
		C_{n + m + p}^n + C_{m + p}^m = \frac{(n+m+p)! * (m+p)!}{n!*(m+p)!*m!*p!} = \frac{(n+m+p)!}{n!*m!*p!}
		\end{gather*}
		
		\subsection{9}
		A)\\
		Рассмотрим все способы выбрать $n$ объектов из $2n$. Заметим, что в каждом способе есть такие $x$ и $y$, что среди первых $n$ выбрано $x$ объектов, среди вторых $n$ выбрано $y$ объектов. Таким образом каждому выбору $n$ объектов из $2n$ можно сопоставить 2 выбора $x$ и $y$ объектов из множеств размером с $n$, где $x+y = n$, при этом 2м разным выборам из $2n$ сопоставляются разные выборы из 2х $n$, и по 2м выборам из 2 $n$ можно восстановить выбор из $2n$, таким образом это сопоставление - биекция $\Rightarrow$ в обоих множествах одинаковое кол-во элементов.\\ \\
		\\
		B)\\
		Заметим, что $C_{n-r-1}^{k-r}$ равно количеству последовательностей длины $n$, в котором $k$ "1" и $n-k$ "0" и в котором первый "0" встречается на $r+1$ позиции (отсчёт начинается с 1), так как первые $r+1$ цифр заданы однозначно, а среди последующих $n-r-1$ - $k-r$ "1". Заметим, что при $n = k$ равенство так-же выполнено.
		
		\subsection{10}
		A)\\
		$m*n$;\\
		0 при $m > n$, $\frac{n!}{(n-m)!}$ при $m \leqslant n$;\\
		0 при $m!=n$, $m!$ при $m = n$;\\
		0 при $m > n$, ${n \choose m}$ при $m \leqslant n$; (${n \choose m}$ способов выбрать m элементов из N, далее отображение однозначно задаётся) \\ 
		\\
		B)\\
		Заметим, что число неубывающих отображений равно количеству способов распределить $m$ одинаковых шаров по $n$ разным коробкам, причём в коробке может быть 0 шаров, что равно кол-ву способов распределить $m+n$ шаров по $n$ коробкам, где в каждой коробке хотя бы 1 шар, что равно $C_{m+n-1}^{n-1}$.
		\\
		C)\\
		
		
		\subsection{11}
		Формулировка:
		\begin{gather*}
		\biggl | \bigcup_{i=1}^{n}A_i \biggl | = \sum_{i} | A_i | - \sum_{i<j} | A_i \cap A_j | + \sum_{i<j<k} | A_i \cap A_j \cap A_k | - \ldots + (-1)^{n-1} | A_1 \cap A_2 \cap \ldots \cap A_n |
		\end{gather*}
		Доказательство:\\
		Возьмём произвольный элемент $x \in A_1 \cup A_2 \cup ... \cup A_n$. Покажем, что $x$ учитывается правой частью формулы ровно один раз. Пусть $x$ принадлежит пересечению ровно $k$ множеств. Без ограничения общности, можно считать, что $x$ принадлежит множествам $A_1$, $A_2$, ..., $A_k$ и не принадлежит множествам $A_{k+1}$, $A_{k+2}$, ... $A_n$. \\
		Тогда посчитаем сколько разучитывается $x$ в различных суммах:\\
		1)\\
		В первой сумме $\sum_{i}^{}|A_i|$ элемент $x$ посчитан ${k \choose 1}$ раз
		\\
		2)\\
		Во второй сумме $\sum_{i<j} | A_i \cap A_j |$ элемент $x$ посчитан ${k \choose 2}$ раз
		\\
		...
		\\
		$k$)\\
		В k сумме $\sum_{i<j< ... <k} | A_i \cap A_j \cap ... \cap A_k|$ элемент $x$ посчитан ${k \choose k}$ раз
		\\
		$>k$)\\
		Cуммы, состоящие из $k + 1$ и более пересечений, не учитывают элемент $x$, так как $x$ не
		входит в пересечение более чем $k$ множеств
		\\
		Таким образом, элемент $x$ посчитан ${k \choose 1} - {k \choose 2} + ... (-1)^{k+1}{k \choose k}$ раз.\\		
		Заметим, что\\
		\begin{gather*}
		0 = (1-1)^n = \sum_{k = 0}^{n}\biggl({n \choose k} * 1^{n-k} * (-1)^k\biggl) = \\ =
		1 + \sum_{k = 1}^{n}\biggl((-1)^k * {n \choose k}\biggl) = 1 - \sum_{k = 1}^{n}\biggl((-1)^{k+1} * {n \choose k}\biggl)
		\end{gather*}
		$\Longrightarrow$
		\begin{gather*}
		1 = \sum_{k = 1}^{n}\biggl((-1)^{k+1} * {n \choose k}\biggl) = {n \choose 1} - {n \choose 2} + ... + (-1)^{n+1}{n \choose n}
		\end{gather*}
		Т.е. каждый элемент посчитан ровно 1 раз
		
		\subsection{12}
		Заметим, что любое число, которое является четвёртой степенью целого числа $x$ является и квадратом целого числа $x^2$. Заметим, что полных квадратов от 1 до 1 000 000 - 1 000, при этом среди них 10 кубов (каждый квадрат - число вида $a*a$, где $1 \leqslant a \leqslant 1000$, при этом тогда среди этих квадратов 10 кубов, так как среди чисел (от 1 до 1 000) 10 кубов). Полных кубов от 1 до 1 000 000 - 100. Итого - $1 000 + 100 - 10 = 1 090$. Это количество чисел, которые квадрат или куб $\Rightarrow$ чисел, которые ни квадрат, ни куб - $1 000 000 - 1 090 = 998 910$.
		
		
		\subsection{13}
		\begin{gather*}
		\frac{(p_1^{\alpha_1 + 1} - 1) * ... * (p_k^{\alpha_k + 1} - 1)}{(p_1 - 1) * ... * (p_k - 1)}
		\end{gather*}
		, докажем индукцией по k.\\
		База: $k = 1$\\
		$n = p^{\alpha_k}$, следовательно сумма делителей 
		\begin{gather*}
		= 1 + p + p^2 + ... + p^{\alpha_k} = \frac{p^{\alpha_k + 1} - 1}{p - 1}
		\end{gather*}
		Переход:-\\
		Рассмотрим число $n_1 = p_1^{\alpha_1} * $ ... $ * p_{k-1}^{\alpha_{k-1}}$ и числом $S_1$ назовём сумму делителей $n_1$. Тогда $S$ (сумма делителей $n$) = $S_1*(\frac{p_k^{\alpha_k + 1} - 1}{p_k - 1})$, так как каждый делитель из $n$ представим в виде $p_k^{\alpha _k}*A$, где $A$ - делитель $n_1$.
			
		
		\subsection{14} 
		A)\\
		1)\\
		\begin{Young}
			& & & & & & \cr
		\end{Young}\\ 
		\\
		2)\\
		\begin{Young}
			& & & & & \cr
			\cr
		\end{Young}\\ 
		\\
		3)\\
		\begin{Young}
			& & & & \cr
			& \cr
		\end{Young}\\
		\\
		4)\\
		\begin{Young}
			& & & & \cr
			\cr
			\cr
		\end{Young}\\ 
		\\
		5)\\
		\begin{Young}
			& & & \cr
			& &\cr
		\end{Young}\\ 
		\\
		6)\\
		\begin{Young}
			& & & \cr
			& \cr
			\cr
		\end{Young}\\ 
		\\
		7)\\
		\begin{Young}
			& & \cr
			& & \cr
			\cr
		\end{Young}\\ 
		\\
		8)\\
		\begin{Young}
			& & \cr
			& \cr
			& \cr
		\end{Young}\\ 
		\\
		итого - ответ = 8.\\ \\
		B)\\
		Заметим, что кол-во диаграмм Юнга размером не более $p * q$ равно кол-ву диаграмм $p+1 * q+1$, так как к каждой диаграмме можно добавить фигуру в форме "Г" слева-сверху длиной $q+1$ и высотой $p+1$, таким образом сопоставив каждой диаграмме из первой группы диаграмму из второй группы (причём очевидно, что это инъекция). Также заметим, что любая диаграмма $p+1 * q+1$ в первой строке имеет $q+1$ квадрат, и в левом столбце - $p+1$ квадрат, из чего следует, что можно сопоставить каждой диаграмме их второй группы диаграмму из первой (то есть отображение $(1) \longrightarrow (2)$  - сюръекция) из чего следует что выше была показана биекция. 
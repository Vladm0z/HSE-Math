\newpage	
	\section{Листок 5}
		
		\subsection{1}
		A)\\
		Нет, так как в $\mathbb{Q}$ нет минимального элемента, в то время как в $\mathbb{N}$ -- есть $\Rightarrow$ биекции нет $\Rightarrow$ множества не изоморфны.
		\\ \\
		B)\\
		Заметим, что в $\mathbb{Q}$ нет минимума, а в $\mathbb{Q} \cap [0, 1]$ он есть $\Rightarrow$ биекции нет $\Rightarrow$ множества не изоморфны.
		\\ \\
		C)\\
		Разобьем отрезок $(0, 1)$ на отрезки следующим образом: пусть числу $n \in \mathbb{Z}, \ n \ne -1,0,1,$ соответствует 
		\begin{gather*}
			a_n:
			\begin{cases}
				[ 1 - \frac{1}{2^{n}} ,\: 1 - \frac{1}{2^{n+1}}) \quad \text{при $n > 0$}\\
				(\frac{1}{2^{|n|+1}} ,\: \frac{1}{2^{|n|}}] \quad \text{при $n < 0$}
			\end{cases}
		\end{gather*}
		А в качестве $a_0$ возьмем точку $0.5$, $a_1 = (0.5,\: 0.75)$, $a_{-1} = (0.25,\: 0.5)$
		Переведем отрезки в $\mathbb{Q}$ -- $a_n \to [n-1, n)$, если $n > 1$ и $a_n \to (n, n+1]$, если $n < -1$, $a_1 = (0, 1)$, $a_{-1} = (-1, 0)$, $a_0 = 0$.\\
		Так мы показали явную биекцию между $\mathbb{Q}$ и $\mathbb{Q} \cap (0, 1)$
		
		\subsection{2}
		A)\\
		Рассмотрим всё множество без одного элемента $\alpha$. В нём есть минимум $\beta$. Заметим, что $\min{(\alpha,\: \beta)}$ -- минимум всего множества.\\ \\
		B)\\
		Рассмотрим все множества, в которых $x$ -- минимально (хотя бы одно нетривиальное такое есть, иначе $x$ -- максимум). Рассмотрим множество "вторых" минимумов в этих множествах. Среди них есть минимум $z$. Заметим, что это и есть непосредственно следующий, так как если есть $y: \: x < y < z$, то в множестве $[x,y,z]$ второе минимальное - $y$, откуда $z$ -- не минимум из множества "вторых".\\ \\
		C)\\
		Рассмотрим множество всех элементов(далее $A$), превосходящих все элементы из данного множества. Так как множество ограничено, то множество превосходящих не пустое. Выберем минимум множества $A$. Заметим, что это и есть точная верхняя грань.\\ \\
		\\ \\
		\subsection{3}
		A)\\
		\\ \\
		B)\\
		Заметим, что для всех $x \in (1,2)$ верно, что $\forall y: \: (x,y) \in P$. При $x = 1,\: y \geqslant 2$ при $x = 2,\: y \leqslant 1$. Остальные $x$ не подходят, откуда следует, какое именно это множество точек.
		\\ \\
		\subsection{4}
		Пронумеруем элементы от 1 до $m$ (элементы вида $a_i$). Сопоставим каждому множеству $U$ такую последовательность 0 и 1 - на $i$ позиции 1, если $a_i \in U$, иначе 0. Заметим, что если одно множество внутри другого, то образ первого больше образа второго, так как "напротив" $\ $"0" $\ $из первого множества стоят "0" $\ $во втором. Откуда следует, что выше указанное сопоставление - изоморфизм.
		\\ \\
		\subsection{5}
		Пронумеруем простые числа (каждое простое число вида $p_i$). Тогда сопоставим финитной последовательности следующее число: $(p_1^{\alpha_1} \cdot p_2^{\alpha_2} \cdot ... \cdot p_k^{\alpha_k} \cdot ...)$, где $\alpha_k$ - $k$ элемент последовательности. Заметим, что все $\alpha_k$, кроме конечного числа, $= 0$, поэтому вышеуказанное произведение определено, и это биекция. Теперь проверим, что отношение перенеслось: $b \ | \ a$ $\Leftrightarrow$ в разложении на простые множители $b$ каждый простой множитель имеет степень $\leq$ степени вхождения в $a$.	Откуда следует, что вышеуказанное сопоставление - изоморфизм.
		\\ \\
		\subsection{6}
		A)\\
		Заметим, что пустое множество может перейти только в себя, так как любое множество содержит $\varnothing$, а $\varnothing$ содержит только $\varnothing$.\\
		Заметим, что множество из одного элемента может перейти только в множество из одного элемента, так как множество больше чем из одного элемента содержит хотя бы 2 подмножества, кроме себя и пустого, а множество из одного элемента - только пустое(помимо себя).\\ \\
		Заметим, что перестановка одноэлементных множеств однозначно задаёт автоморфизм. \\
		Докажем по индукции.\\
		База: $n = 2$\\ 
		Заметим, что любое множество из двух элементов содержит два одноэлементных подмножества и пустое, что однозначно задаётся перестановкой множеств из одного элемента, так как по двум одноэлементным множествам однозначно восстанавливается множество из двух элементов.\\
		Переход\\
		Существует лишь одно множество, которое больше $k$ одноэлементных множеств (и с другими не сравнимо), при этом это же множество будет больше любого объединения этих одноэлементных множеств, которые в свою очередь однозанчно задаются по предположению индукции.\\
		Тогда автоморфизмов - $m!$.
		\\ \\
		B)\\
		Заметим, что автоморфизмов столько же, сколько перестановок простых чисел, что равномощно количеству перестановок натуральных чисел, так как простых чисел счётно. Заметим, что количество перестановок натуральных чисел не меньше количества последовательностей из 0 и 1, не заканчивающихся бесконечной последовательностью 0. А именно: сопоставим перестановке $A$  ($A = [a_1,\: a_2,\: ... \:]$) последовательность $B$ ($B = [b_1,\: b_2,\: ... \:]$) следующим образом - $b_i = 1$ если $a_i < a_{i+1}$, иначе $b_i = 0$. Заметим, что у каждой последовательности $B$ прообраз: мнимый "нулевой" $\ $элемент равен 0, на позициях где предыдущий равный 1 (и первый элемент), будет равен (предыдущему + 1 + количество нулей до следующей 1), на остальных позициях - (предыдущий - 1). Нетрудно видеть, что такая последовательность - биекция (покажем, что любое натуральное число встретится: разобьём натуральный ряд на группы подряд идущих чисел, длины групп равные максимальным группам из ряда "0" $\ $и "1" $\ $вида "1000...000". Заметим, что записанная прообразная перестановка - записанные в обратном порядке (внутри самой группы) группы, откуда и следует, что любое число встретится). При этом количество последовательностей из 0 и 1, не заканчивающихся бесконечной последовательностью 0 - континуум, а количество перестановок $\geqslant$ чем континуум, откуда автоморфизмов континуум, что и требовалось доказать.
		\\ \\
		\subsection{7}
		Выберем счётное упорядоченное подмножество, пронумеруем их целыми числами. Применим следующий алгоритм для двух соседних (0 и 1), для остальных будем делать аналогично:\\
		2 операция: \\
		0 и 1 не соседи, откуда следует, что есть $z$ между ними, назовём его $\frac{1}{2}$.\\
		$n$  операция:\\
		Рассмотрим все рациональные числа вида $\frac{i}{n}$, $(i,\: n) = 1$, и для каждого из них найдём 2 неприводимых дроби $\frac{p_1}{q_1},\: \frac{p_2}{q_2}$ со знаменателями меньшими $n$, такие что число $\frac{i}{n}$ лежит между ними. Также $\frac{p_1}{q_1}$ (меньшее) максимальное среди возможных (такое есть, т.к. всего рациональных числ удовлетворяющих уловию конечно), а $\frac{p_2}{q_2}$ - минимально. Тогда между прообразами $\frac{p_1}{q_1}$, $\frac{p_2}{q_2}$ есть элемент, назовём его $\frac{i}{n}$. Таким образом любой элемент между $0$ и $1$ получил какой-то образ.\\
		Аналогично можно сделать для любых двух пар. Так мы получили изоморфизм из двух указанных множеств.
		\\ \\		
		\subsection{8}
		1) $\textbf{Лемма Цорна}$\\
		Если в частично упорядоченном множестве $M$ для всякого линейного упорядоченного подмножества существует верхняя грань, то в $M$ существует максимальный элемент.\\ \\
		2) $\textbf{Аксиома выбора}$\\
		Для каждого семейства $A$ непустых непересекающихся множеств существует множество $B$, имеющее один и только один общий элемент с каждым из множеств $X \subset A$\\ \\
		3) $\textbf{Теорема Цермело}$\\
		На всяком множестве можно ввести такое отоношение порядка, что это множество будет вполне упорядочено.\\
		\\
		$\textbf{Частично упорядоченое множество}$ -- множество на котором введен частичный порядок\\
		$\textbf{Линейно упорядоченное множество}$ -- частично упорядоченное множество, в котором для любых двух элементов $a$ и $b$ имеет место $a \leqslant b$ или $b\leqslant a$.\\
		$\textbf{Вполне упорядоченное множество}$ -- линейно упорядоченное множество $M$ такое, что в любом его непустом подмножестве есть минимальный элемент.\\
		$\textbf{Семейство множеств}$ -- множество, элементами которого являются другие множества.\\
		\\ \\
		\subsection{9}
		A*)\\
		Рассмотрим частично упорядоченное множество $Z$, элементами которого будут частичные порядки на $X$ (то есть подмножества множества $X \times X$, обладающие свойствами рефлексивности, транзитивности и антисимметричности), упорядоченные по включению: $\leqslant_1$ считается меньшим или равным $\leqslant_2$, если $\leqslant_2$ продолжает $\leqslant_1$ ($x \leqslant_1 y \ \Rightarrow \ x \leqslant_2 y$).\\
		Условие леммы Цорна выполнено: если у нас есть семейство частичных порядков, линейно упорядоченное по включению, то объединение этих порядков является частичным порядком, и этот порядок будет верхней границей семейства.\\
		(Проверим что объединение обладает свойством транзитивности: пусть $x \leqslant_1 y$ в одном из порядков семейства ($\leqslant_1$), а $y \leqslant_2 z$ в другом; один из порядков (например, $\leqslant_1$) продолжает другой, тогда $x \leqslant_1 y \leqslant_1 z$ и потому $x \leqslant z$ в объединении. Рефлексивность и антисимметричность проверяются аналогично)\\
		Тогда, по лемме Цорна на множестве $X$ существует максимальный частичный порядок, продолжающий исходный. Обозначим его как $\leqslant$ (путаницы с исходным порядком не возникнет, так как исходный нам больше не нужен). Нам надо показать, что он будет линейным. Пусть $x, y \in X$ — два несравнимых элемента. Расширим порядок до нового порядка $\leqslant'$, при котором $x \leqslant' y$. Этот новый порядок определяется так: $a \leqslant' b$, если (1) $a \leqslant b$ или (2) $a \leqslant x$ и $y \leqslant b$.\\
		Несложно проверить, что $\leqslant'$ будет частичным порядком.\\
		Рефлексивность очевидна.\\
		Транзитивность: если $a \leqslant' b$ и $b \leqslant' c$, то есть четыре возможности. Если в обоих случаях имеет место случай (1), то $a \leqslant b \leqslant c$ и всё очевидно. Если $a \leqslant' b$ в силу (1), а $b \leqslant c$ в силу (2), то $a \leqslant b \leqslant x$ и $y \leqslant c$, так что $a \leqslant' c$ в силу (2). Аналогично рассматри вается и симметричный случай. Наконец, двукратная ссылка на (2) невозможна, так как тогда ($a \leqslant x$), ($y \leqslant b$), ($b \leqslant x$) и ($y \leqslant c$), и получается, что $y \leqslant b \leqslant x$, а мы предполагали, что $x$ и $y$ не сравнимы.
		Антисимметричность доказывается аналогично. Таким образом, отношение $\leqslant'$ будет частичным порядком, строго содержащим $\leqslant$, что противоречит максимальности.
		\\ \\
		B*)\\
		\\ \\
		C*)\\
		\\ \\
		D*)\\
		\\ \\
		E*)\\
		Пусть даны два непустых множества $A$ и $B$. Рассмотрим множество $M$ всех функций $f$ таких, что область определения $D_f\subset A$, $f\colon D_f\to B$ и $f$ -- иньекция.\\
		Для $f,g\in M$ полагаем $f\leqslant g$, если $D_f\subset D_g$ и $f=g|_{D_f}$, т.е. $g$ является продолжением $f$. Это частичный порядок.
		Проверим выполнение условия леммы Цорна. Если $\{f_p: p\in I\}$ -- линейно упорядоченное подмножество, то его верхней гранью будет функция, у которой ${\mathrm {graph}} \  f = \bigcup_{p\in I} {\mathrm {graph}} \ f_p$.
		Пусть $f$ -- максимальный элемент. Тогда либо $D_f=A$, либо $f(D_f)=B$, так как в противном случае найдется точка $a\in A\setminus D_f$ и точка $b\in B\setminus f(D_f)$, и можно продолжить $f$, полагая $f(a)=b$.
		Теорема доказана.
		\\ \\
		\subsection{10}
		A*)\\
		Аксиома выбора из теоремы Цермело (предполагается, что человек сперва сдал $10(\text{b})$)\\
		Пусть $S$ -- данное семейство непустых множеств. По теореме Цермело множество $U = \bigcup S$ может быть вполне упорядочено. Для каждого $x \in S$ имеем $x \subset U$. Пусть $\min(x)$ означает наименьший элемент $x$ в смысле порядка на $U$. Поскольку $\varnothing \in S$, соответствие $x \mapsto \min(x)$ является функцией выбора на $S$.
		\\ \\
		B*)\\
		Вполне упорядоченное множество $(S,\: <S)$ назовём вполне упорядоченным подмножеством $X$, если $S \subset X$. Для данного множества $X$ рассмотрим совокупность $W(X)$ всех его вполне упорядоченных подмножеств. На $W(X)$ определим отношение строгого частичного порядка $\prec$ следующим образом:\\
		$(S,\: <S) \prec (T,\: <T )$, если и только если $S \subset T$ есть собственный начальный отрезок $(T,\: <T )$, и $<S$ совпадает с ограничением $<T4$ на $S$.\\
		Докажем, что $(W(X),\: \prec)$ удовлетворяет условию леммы Цорна. Рассмотрим любую цепь $C \subset W(X)$. Цепи $C$ соответствует возрастающая по включению цепь подмножеств $X$ и возрастающая по включению цепь бинарных отношений на этих множествах. Обозначим через $U$ объединение этой цепи подмножеств $X$, а через $<U$ — объединение соответствующей цепи отношений. Ясно, что $<U$ есть отношение линейного порядка на $U$ и каждое $(S,\: <S) \in C$ есть начальный отрезок $(U,\: <U)$. Отсюда получаем, что $(U,\: <U )$ — вполне упорядоченное подмножество $X$. Таким образом, $(U,\: <U )$ есть элемент $W(X)$ и верхняя грань цепи $C$.\\
		Применяя лемму Цорна получаем, что в $(W(X),\: \prec)$ найдётся некоторый максимальный элемент $(M,\: <M)$. Тогда $M$ обязано совпадать со всем $X$: в противном случае мы можем взять $a \in X \ M$ и продолжить порядок $<M$ на большее множество $N = M \cup {a}$ полагая $x <N a$ для всех $x \in M$. Тогда $(N,\: <N)$ будет вполне упорядоченным подмножеством $X$ и $(M,\: <M) \prec (N,\: <N )$, что противоречит максимальности $(M,\: <M)$.
		\\ \\
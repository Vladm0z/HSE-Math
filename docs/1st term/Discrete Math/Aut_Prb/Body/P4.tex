\newpage
	\section{Листок 4}
		\subsection{1}
		Сопоставим тройке чисел $(a;b;c)$ числу $\frac{(\frac{(a+b)(a+b-1)}{2} + b + c - 1)(\frac{(a+b)(a+b-1)}{2} + b + c)}{2} + c$.\\
		Заметим, что отображение $(a;b) \: \to \: \frac{(a+b-1)(a+b)}{2} + b$ --  биекция, так как для фиксированного $a+b$ выражение принимает значения от $(1 + 2 + ... + (b - 1)) + 1$ до $(1 + 2 + ... + (b - 1)) + b$. Если $a_2+b_2 > a_1+b_1$, то очевидно, что образ $(a_2;b_2)$ > образа $(a_1;b_1)$.\\
		Так мы сопоставляем каждой тройке $(a;b;c)$ пару $(\frac{(a+b-1)(a+b)}{2} + b;c)$, а каждой такой паре - число $\frac{(\frac{(a+b)(a+b-1)}{2} + b + c - 1)(\frac{(a+b)(a+b-1)}{2} + b + c)}{2} + c$.
		
		\subsection{2}
		Заметим, что отрезок $[0,1]$ равномощен прямой с дополнительной точкой $I$(это можно доказать через проецированиае отрезка на прямую, тогда $0 \to 0$, $1 \to \infty \ \text{(I) = $\infty$}$). Заметим, что каждая прямая пересекает каждую ось (если нет, считаем, что проходит через $I$) в какой то точке. Таким образом, прямая задаётся парой расширенных вещественных чисел, кроме $(I;I)$ (что равномощно $R^2$), при этом любой паре можно сопоставить прямую (проходящие через соотв. 2 точки). Покажем, что $R^2$ равномощно $R$: считаем $R$ -- последовательностью счётной длины из 0 и 1. Заметим, что каждой паре $(r_1;r_2)$, где $r_1 = [a_1,\: a_2,\: a_3,\: ...]$, $r_2 = [b_1,\: b_2,\: b_3,\: ...]$ можем сопоставить $r_3 = [a_1,\: b_1,\: a_2,\: b_2,\: a_3,\: b_3,\: ...]$. При этом аналогично можно сопоставить каждому $r_3$ пару $(r_1;r_2)$.
		Откуда множество прямых равномощно $R$.
		
		\subsection{3}
		A)\\
		Рассмотрим одноэлементные подмножества, заметим что если рассмотреть два одноэлементных множества, то можно однозначно определить отношение элементов из этих множеств(т.е. что единственный элемент множества $A$ больше единственного элемента множества $B$). Тогда сопоставим одноэлементному множеству с наименьшим элементом первое простое число(т.е. 2), следующему по величине элемента множеству сопоставим следующее простое число(т.е. 3) и т.д. Так мы сопоставили одноэлементные множества простым числам. Далее рассмотрим множество из $n$ элементов(пусть это $\{ a_1,\: a_2,\: ... ,\: a_n \}$) Рассмотрим одноэлементные множества с этими элементами, пусть они сопоставлены числам $p_1,\: p_2,\:... ,\: p_n $, тогда сопоставим множеству $\{ a_1,\: a_2,\: ... ,\: a_n \}$ число $p_1 \cdot p_2 \cdot ... \cdot p_n$. Очевидно, что разным множества сопоставлены разные числа(так как если двум множества сопоставлено одно и то же число, то элементы этих множест совпали). Так мы показали, что множество всех конечных подмножеств равномощно $\mathbb{N}$, т.е. множество всех конечных подмножеств счетно, что и требовалось доказать.
		\\
		B)\\
		Сопоставим каждой последовательности из задачи (множество $A$) паре 2 конечных последовательностей: предпериод и сам период (множество $B$). Заметим, что $A \leqslant B$. При этом $A \geqslant C$, где $C$ м ножество из пункта $a$. (а именно можем сопоставить каждому множеству из $C$ последовательность упорядоченно записанных элементов, после которой идёт период из 1). При этом $B = D^2$, где $D$ -- множество конечных последовательностей натуральных чисел. Покажем, что $D$ -- счётно. Заметим, что $D$ -- объединение счётного количества счётного количества конечных множеств, а именно: множеств последовательностей фиксированной длины с фиксированной суммой. Откуда $D$ -- счётно $\Rightarrow$ $B$ счётно, при этои $C$ тоже счётно, откуда $A$ -- счётно
		
		\subsection{4}
		Заметим, что если корней(алгебраических чисел) конечное количество, то трансцендентных континуум, так как всего чисел континуум.\\
		Заметим, что алгебраические числа --  подмножество описываемых чисел, то есть тех, которым можно сопоставить хотя бы одну конечную строку символов, которая бы "означала" это число.\\
		Докажем, что множество описываемых чисел счётно:\\
		Заметим, что их "меньше" чем возможных конечных последовательностей из конечного набора символов, что очевидно счётно, но любое рациональное число можно обозначить/описать(записав '$p/q$'). При этом алгебраических хотя бы счётно, так как все рациональные числа - алгебраические (являются корнями уравнений вида $x - \alpha = 0$). Откуда алгебраических чисел счётно.
		
		\subsection{5}
		A)\\
		\\
		B)\\
		Заметим, что последовательностей из действительных чисел столько же, сколько последовательностей из наборов 0 и 1. Докажем, что последовательностей из наборов 0 и 1 столько же, сколько и наборов 0 и 1: Пусть $a_{i \ j}$ --  $j$-тый символ $i$-той последовательности. Тогда сопоставим последовательности из бесконечных наборов следующую последовательность: $\{ a_{1 \ 1},\: a_{1 \ 2},\: a_{2 \ 1},\: a_{1 \ 3},\: a_{2 \ 2},\: a_{3 \ 1},\: ... \}$, то есть будем последовательно записывать символы с фиксированной суммой индексов. Заметим, что если взять 2 различные последовательности наборов, то у них будут различные образы, при этом любая последовательность из 0 и 1 имеет прообраз. Так мы доказали, что последовательностей действительных континуум.
		\\
		C)\\
		Заметим, что отображений столько же, сколько и последовательностей из натуральных чисел. (сопоставляем отображению последовательность такую, что на $i$-той позиции - образ $i$).
		Заметим, что таких последовательностей столько же, сколько счётных последовательностей из 0 и 1, а именно запишем число $n$ в виде $n - 1$ подряд идущих единиц, и разделим числа между собой нулями($1,\; 2,\; 3,\; 4,\; 5,\; ... \: \to \: 0010110111011110...$). 
		\\
		D)\\
		\subsection{6}
		A)\\
		Предположим есть биекция из $\mathbb{R}$ в $\mathbb{R} \to \mathbb{R}$ (сопоставляющаяя $i$ функцию $f_i(x)$ ). Тогда рассмотрим следующий элемент из $\mathbb{R}^{\mathbb{R}}$: каждому $i$ сопоставим $1 + f_i(i)$. Заметим, что этот элемент из $\mathbb{R}^{\mathbb{R}}$ не имеет прообраза, откуда следует, что биекции нет. 
		\\
		B*)\\
		\\
		C*)\\
		
		\subsection{7*}
		Если одна из частей квадрата содержит отрезок, то можно воспользоваться теоремой Кантора-Бернштейна.
		Допустим первая часть не содержит отрезков, тогда в каждом горизонтальном сечении квадрата есть точка второй части, тогда с помощью аксиомы выбора во второй части можно найти подмножество, равномощное
		отрезку -- после чего снова можно сослаться на теорему Кантора -- Бернштейна.
		
		\subsection{8*}
		Заметим, что если $\mathbb{R}^{\mathbb{R}}$ равномощно $2^{\mathbb{R}}$ , то $\mathbb{N}^{\mathbb{R}}$ равномощно $\mathbb{R}^{\mathbb{R}}$ и $2^{\mathbb{R}}$.\\
		Заметим, что множество $\mathbb{R}^{\mathbb{R}}$ можно разбить на континуальное количество счётных множеств, а именно: фактор-множество отношения эквивалентности: $a \ \blacklozenge \ b \: \Leftrightarrow \:  (a - b) \in \mathbb{Z}$. Теперь покажем, что $\mathbb{R}^{\mathbb{R}} \: \sim \: 2^{\mathbb{R}}$. Это эквивалентно ${[0,\: 1)}^{[0,\: 1)} \: \sim \: 2^{\mathbb{R}}$. Рассмотрим $X$ принадлежащий первому множеству, оно - множество пар $(i,\: a_i)$, где $i$ принадлежит отрезку $[0,\: 1)$. Теперь скажем, что если есть пара $(x,\:a_x)$, то представим $a_x$ в виде последовательности 0 и 1, и классу эквивалентности $x$ сопоставим нули и единицы соответсвенно их позиции. Таким образом показана биекция ${[0,\: 1)}^{[0,\: 1)} \: \to \: 2^{\mathbb{R}}$, что и требовалось.
		
		\subsection{9}
		A)\\
		Заметим, что выполнены рефлексивность (то есть существует тождественная биекция), симметричность (если есть биекция, то есть и обратная биекция) и транзитивность (если есть биекция из $A$ в $B$: $a_i \to b_i$ и есть биекция из $B$ в $C$: $b_i \to c_i$, то есть биекция из $A$ в $C$: $a_i \to c_i$.
		\\
		B)\\
		Рассмотрим множество из двух элементов: $[a,b]$. Тогда $[a] \prec [b]$ и $[b] \prec [a]$ но $[a] \ne [b]$.
		\\
		C)\\
		Заметим, что если между $a$ и $b$ есть биекция, то есть и инъекция, откуда любые 2 элемента из одного класса эквивалентности "равны" с точки зрения $\prec$. Если же биекции нет, но есть инъекция (без ограничения общности инъекция $a \to b$), то для любых элементов $a_i$ из $A$ (класс эквивалентности $a \in A$) и $b_j$ из $B$ верно, что есть инъекция $a_i \to b_j$ ($a_i\: \in \:a; \ b\: \in \:b_j$).\\		
		Проверим, что это отношение частичного порядка: \\
		Рефлексивность (наличие тождественной биекции доказывает рефлексивность)\\
		Антисимметричность (если есть инъекция из $c \in C$ в $d \in D$ и наоборот, то из Т. Кантора-Бернштейна следует, что есть биекция, то есть $d \in C$)\\
		Транзитивность (переносится из основного отношения, то есть, если $\overline{\prec}$ обладает транзитивностью, то и $\prec$ обладает транзитивностью).
		
		\subsection{10}
		Теорема Кантора-Бернштейна утверждает, что если существуют инъективные отображения $f: \: A \to B \: \ \text{и} \ g: \: B \to A$, то $|A| = |B|$\\ 
		Докажем это.\\
		Пусть $f: \: A \to A_2$ биекция и $A_3 = f(A_1) \subset A_2$, $A_4 = f(A_2) \subset A_3$ и т.д. Тогда мы получили систему из множеств: $A_0 \supset A_1 \supset A_2 \supset ... $\\
		В которой $A_{2n}$ есть результат n-кратного применения отображения f к множеству $A_0$, а
		$A_{2n+1}$ есть результат n-кратного применения отображения f к множеству $A_1$.\\
		Представим множество $A_0$ в виде объединения непересекающихся слоев $C_k = A_k \backslash A_{k+1}$ с центром в $C = \bigcap\limits_{k} A_k$\\
		Так как $f(C) = C_{2}$, $f(C_2) = C_{4}$ и т.д. то $C,\: C_2,\: C_4,\: ...$ равномощны(так как $f$ биекция)\\
		Поэтому мы можем построить биекцию между множествами $A_0$ и $A_1$:\\
		\begin{gather*}
		A_0 = C_0 \cup C_1 \cup C_2 \cup C_3 \cup C_4 \cup ... \cup C\\
		A_1 = \: \qquad C_1 \cup C_2 \cup C_3 \cup C_4 \cup ... \cup C\\
		A_1 = \: \qquad C_1 \cup C_0 \cup C_3 \cup C_2 \cup ... \cup C
		\end{gather*}
		Если элемент $a$ множества $A$ принадлежит слою с четным номером($a \in C_{2k}$), сопоставим ему элемент $f(a)$, если же элемент $a$ в слое с нечетным номером или в сердцевине($a \in C_{2k+1}$), то оставим его на месте (поставив ему в соответствие его же, но как элемент множества $A_1$). Тогда множества $A = A_0$ и $A_1$ равномощны.
		
		\subsection{11}
		
		Заметим, что не меньше, так как есть биекция из наборов принадлежащих $2^M$($2^M$  мы считаем набором из $0$ и $1$, где мы ставим $1$, если рассматриваемый элемент принадлежит подмножеству $M$, и $0$, если не принадлежит), содержащих ровно одну $1$ в $M$.\\
		Покажем, что нет биекции из $M$ в $2^M$. Предположим, что есть(назовем ее $f$). Тогда рассмотрим следующий набор(назовем его $m$) из $2^M$: $i$-тый элемент противоположен $i$-тому элементу образа $i$ (где $i \in M$). Заметим, что у этого набора нет прообраза (по построению для каждого i-элемента верно, что $m \ne f(i)$).
		Откуда следует, что мощность $2^M$ больше $M$.
		\subsection{12**}
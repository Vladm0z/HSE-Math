 \documentclass[]{book} 
%These tell TeX which packages to use.
\usepackage[utf8]{inputenc}
\usepackage[T2B]{fontenc}
\usepackage{array,epsfig}
\usepackage{amsmath}
\usepackage{amsfonts}
\usepackage{amssymb}
\usepackage{amsxtra}
\usepackage{amsthm}
\usepackage{mathrsfs}
\usepackage{mathtools}
\usepackage{color}
\usepackage{natbib}
\usepackage{graphicx}
\usepackage{mathtext}
\usepackage[english, russian] {babel}
\usepackage{mathtools}
\usepackage{rotating}
\usepackage{tikz}
\usepackage{wrapfig}
\usepackage{multicol}
\usepackage{verbatim}
\usepackage{eufrak}
\usepackage{subfiles}
\usepackage{subfigure}
\usepackage{stackrel}
\usepackage[arrow]{xy}
\usepackage{units}
\usetikzlibrary{babel,calc,arrows,shapes.geometric,intersections,through,backgrounds}

%Here I define some theorem styles and shortcut commands for symbols I use often

\renewcommand{\thesection}{\arabic{chapter}.\arabic{section}}

\renewcommand{\thesection}{\arabic{section}}

\usepackage[framemethod=TikZ]{mdframed}
\usepackage{amsthm}
%%%%%%%%%%%%%%%%%%%%%%%%%%%%%%
%Theorem
\newcounter{theo}[section] \setcounter{theo}{1}
\renewcommand{\thetheo}{\arabic{section}.\arabic{theo}}
\newenvironment{theo}[2][]{%
\refstepcounter{theo}%
\ifstrempty{#1}%
{\mdfsetup{%
frametitle={%
\tikz[baseline=(current bounding box.east),outer sep=0pt]
\node[anchor=east,rectangle,fill=blue!20]
{\strut Теорема};}}
}%
{\mdfsetup{%
frametitle={%
\tikz[baseline=(current bounding box.east),outer sep=0pt]
\node[anchor=east,rectangle,fill=blue!20]
{\strut Теорема:~#1};}}%
}%
\mdfsetup{innertopmargin=10pt,linecolor=blue!20,%
linewidth=2pt,topline=true,%
frametitleaboveskip=\dimexpr-\ht\strutbox\relax
}
\begin{mdframed}[]\relax%
\label{#2}}{\end{mdframed}}

\newcommand{\lra}{\Longleftrightarrow}
\newcommand{\Lra}{\Longrightarrow}
\newcommand{\Ra}{\Rightarrow}
\newcommand{\surj}{\twoheadrightarrow}
\newcommand{\graph}{\mathrm{graph}}
\newcommand{\bb}[1]{\mathbb{#1}}
\newcommand{\Z}{\bb{Z}}
\newcommand{\Q}{\bb{Q}}
\newcommand{\R}{\bb{R}}
\renewcommand{\C}{\mathbb{C}}
\newcommand{\N}{\bb{N}}
\newcommand{\M}{\mathbf{M}}
\newcommand{\K}{\mathbb{K}}
\newcommand{\mB}{\mathfrak{B}}
\newcommand{\B}{\mathbb{B}}
\newcommand{\m}{\mathbf{m}}
\newcommand{\MM}{\mathscr{M}}
\newcommand{\HH}{\mathscr{H}}
\newcommand{\Om}{\Omega}
\newcommand{\Ho}{\in\HH(\Om)}
\newcommand{\bd}{\partial}
\newcommand{\del}{\partial}
\newcommand{\bardel}{\overline\partial}
\newcommand{\textdf}[1]{\textbf{\textsf{#1}}\index{#1}}
\newcommand{\img}{\mathrm{img}}
\newcommand{\ip}[2]{\left\langle{#1},{#2}\right\rangle}
\newcommand{\inter}[1]{\mathrm{int}{#1}}
\newcommand{\exter}[1]{\mathrm{ext}{#1}}
\newcommand{\cl}[1]{\mathrm{cl}{#1}}
\newcommand{\ds}{\displaystyle}
\newcommand{\vol}{\mathrm{vol}}
\newcommand{\cnt}{\mathrm{ct}}
\newcommand{\osc}{\mathrm{osc}}
\newcommand{\LL}{\mathbf{L}}
\newcommand{\UU}{\mathbf{U}}
\newcommand{\support}{\mathrm{support}}
\newcommand{\AND}{\;\wedge\;}
\newcommand{\OR}{\;\vee\;}
\newcommand{\Oset}{\varnothing}
\newcommand{\wh}{\widehat}
\renewcommand{\leq}{\leqslant}
\renewcommand{\geq}{\geqslant}
\newcommand{\tgn}{\mathrm{tg}}
\newcommand{\ro}{\rho}
\newcommand{\ex}{\exists\:}
\newcommand{\fo}{\forall\:}
\newcommand{\la}{\langle}
\newcommand{\ra}{\rangle}
\newcommand{\defn}{\textbf{Определение. }}
\newcommand{\eps}{\varepsilon}

\DeclareMathOperator{\id}{id}
\DeclareMathOperator{\pt}{pt}
\DeclareMathOperator{\Int}{Int}
\DeclareMathOperator{\fin}{fin}
%\DeclareMathOperator{\i}{i}

%Pagination stuff.
%\setlength{\topmargin}{-.3 in}
%\setlength{\oddsidemargin}{0in}
%\setlength{\textheight}{9.in}
%\setlength{\textwidth}{7in}
\pagestyle{plain}

\usepackage[a4paper,%
            left=1cm,right=1cm,top=1in,bottom=1cm]{geometry}

\everymath{\displaystyle}

\linespread{1.5}

\usepackage[labelformat=empty]{caption}

\begin{document}

\begin{titlepage}
	\centering
	{\scshape\LARGE НИУ «Высшая школа экономики» \par}
	\vspace{1cm}
	{\scshape\Large Факультет математики\par}
	\vspace{1.5cm}
	{\huge\bfseries Введение в топологию\par}
	\vspace{2cm}
	{\Large\itshape лектор: Пирковский Алексей Юльевич\par}
	\vfill

% Bottom of the page
	{\large 2019/2020 г.}
\end{titlepage}

\tableofcontents

\newpage

\graphicspath{{lectures/lecture1/}{pictures/}}

\subfile{lectures/lecture1/lecture1.tex}

\vspace{30pt}

\graphicspath{{lectures/lecture2/}{pictures/}}

\subfile{lectures/lecture2/lecture2.tex}

\vspace{30pt}

\graphicspath{{lectures/lecture3/}{pictures/}}

\subfile{lectures/lecture3/lecture3.tex}

\vspace{30pt}

\graphicspath{{lectures/lecture4/}{pictures/}}

\subfile{lectures/lecture4/lecture4.tex}

\vspace{30pt}

\graphicspath{{lectures/lecture5/}{pictures/}}

\subfile{lectures/lecture5/lecture5.tex}

\vspace{30pt}

\graphicspath{{lectures/lecture6/}{pictures/}}

\subfile{lectures/lecture6/lecture6.tex}

\graphicspath{{lectures/lecture7/}{pictures/}}

\subfile{lectures/lecture7/lecture7.tex}

\graphicspath{{lectures/lecture8/}{pictures/}}

\subfile{lectures/lecture8/lecture8.tex}

\graphicspath{{lectures/lecture9/}{pictures/}}

\subfile{lectures/lecture9/lecture9.tex}

\graphicspath{{lectures/lecture_10/}{pictures/}}

\subfile{lectures/lecture_10/lecture_10.tex}

\graphicspath{{lectures/lecture_11}{pictures/}}

\subfile{lectures/lecture_11/lecture_11.tex}

\graphicspath{{lectures/lecture_12}{pictures/}}

\subfile{lectures/lecture_12/lecture_12.tex}

\graphicspath{{lectures/lecture_13}{pictures/}}

\subfile{lectures/lecture_13/lecture_13.tex}

\vspace{30pt}


\end{document}



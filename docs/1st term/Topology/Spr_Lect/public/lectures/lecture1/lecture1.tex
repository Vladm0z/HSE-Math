\documentclass[../../main.tex]{subfiles}
 
\begin{document}

\section{Лекция 1. 30 октября 2019 г.}
\vspace{10pt}

{\large Введение: какая бывает топология и чем она занимается. Метрические пространства, нормированные пространства, евклидовы пространства. Примеры: $\R_n$, $C[a,b]$, дискретная метрика, $l^{\infty}(S), l^1, l^2,$ \\ $p $-адическая метрика на $\Q$, метрика Хаусдорфа.}

\vspace{10pt}

\subsection{Метрики, метрические пространства}

$X$ - множество

\textbf{Определение.} \textit{Метрика} на $X$ - функция $\ro \colon X \times X \longrightarrow [0;\: +\infty)$, удовлетворяющая следующим условиям:

(1) $\ro(x,y) = \ro(y,x) \quad \fo x, y \in X$

(2) $\ro(x,x) = 0 \quad \fo x \in X$

(3) $\ro(x,z) \leq \ro(x,y) + \ro(y,z) \quad \fo x, y, z \in X$ (неравенство треугольника)

(4) $\ro(x,y) > 0 \quad \fo x \neq y$


\textbf{Определение.} \textit{Метрическим пространством} называется пара $\left(X, \ro \right)$, то есть некоторое множество $X$ с заданной на нем метрикой.

\textbf{Определение.} Функция $\ro$ называется \textit{полуметрикой}, если выполняются (1)-(3) условия, а \textbf{$\left(X, \ro \right)$} тогда - полуметрическое пространство. 

\textbf{Пример 0.} $X$ - мн-во $\ro(x,y) = \begin{cases} 1, \text{ если } x \neq y \\ 0 \text{, если } x = y \end{cases}$

\textbf{Пример 1.} $X = \R \quad \ro(x, y) = |x - y|$

\textbf{Пример 2.} Три метрики на $\R^n:$

$\ro_1(x, y) = \sum^n_{i = 1} |x_i - y_i|$

$\ro_2(x, y) = \sqrt{\sum^n_{i = 1} \left(x_i - y_i \right)^2}$ - евклидова метрика

$\ro_{\infty}(x, y) = \max_{1 \leq i \leq n}|x_i - y_i|$

\textbf{Пример 3.} $X = C[a, b]$ - мн-во всех непрерывных функций из $[a, b]$ в $\R$.

Равномерная метрика: $\ro(f, g) = \sup_{t \in [a, b]} \left| f(t) - g(t) \right|$

\textbf{Упражнение.} Доказать, что $\ro$ из примера 3, действительно, является метрикой.

\textbf{Наблюдение:} В примере 3 $X$ - векторное пространство над $\R$, и $\ro(x,y) = \ro(x-y, 0)$

\textbf{Определение} Пусть $X$ - векторное пространство над $\mathbb{K}$ (где $\K = \R$, либо $\K = \C$)

Функция $X \longrightarrow [0, +\infty), \;\; x \in X \longrightarrow ||x||$, называется \textbf{\textit{нормой}} на векторном пространстве $X$, если она удовлетворяет следующим условиям:

(1) $|| \lambda x|| = |\lambda| \:||x|| \quad (\lambda \in \R, x \in X)$

(2) $||x + y|| \leq ||x|| + ||y|| \quad (x, y \in X)$

(3) $||x|| > 0 \quad \fo x \neq 0$

$\left(X, ||\cdot|| \right)$ - \textit{нормированное пространство}.

\textbf{Наблюдение:} Пусть $\left( X, ||\cdot|| \right)$ - нормированное пространство. Тогда $\ro(x,y) = ||x - y|| \quad \left( x, y \in X \right)$ - метрика на $X$ (метрика, порожденная нормой)

\textbf{Упражнение.} Доказать, что $\ro$ из наблюдения, действительно, является метрикой.

\textbf{Пример 4.} Три нормы на $\K^n$:

$||x||_1 = \sum^n_{i = 1} |x_i|$

$||x||_2 = \sqrt{\sum^n_{i=1}|x_i|^2}$ (евклидова норма)

$||x||_{\infty} = \max_{1 \leq i \leq n} |x_i|$

Они порождают метрики из примера 2.

\textbf{Пример 5.} Равномерная норма на $C[a, b]$: $||f|| = \sup_{t \in [a, b]} |f(t)| \quad \text{Она порождает метрику из примера 3}$

\textbf{Обозначение} $X,Y$ - множества.

$Y^X$ - множество всех отображений из $X$ в $Y$

В частности: $\K^{\N}$ - множество числовых последовательностей в $\K$

\textbf{Пример 6.} $S$ - множество
$$l^{\infty}\left(S \right) = \Big\{ f \in \K^S \colon f \text{ ограничена} \Big\} \quad \quad ||f|| = \sup_{s \in S} |f(s)|  \text{ - равномерная норма}$$
Частный случай: $l^{\infty} = \l^{\infty}\left(\N \right)$ - пространство ограниченных последовательностей

\textbf{Пример 7.} $l^1 = \Big\{ x = (x_i) \in \K^{\N} \colon \text{ ряд} \sum^{\infty}_{i = 1} |x_i| \text{ сходится} \Big\}$

Норма на $l^1 \colon ||x||_1 = \sum^{\infty}_{i=1} |x_i|$

\textbf{Пример 8.} $l^2 = \Big\{ x = (x_i) \in \K^{\N} \colon \text{ ряд} \sum^{\infty}_{i = 1} |x_i|^2 \text{ сходится} \Big\}$

$l^2$ - векторное подпространство в $\K^{\N}$ - след. из неравенства $(a + b)^2 \leq 2(a^2 + b^2) \quad (a, b \geq 0)$
\vspace{10pt}

\begin{minipage}{0.3\textwidth}
Норма на $l^2$: $||x||_2 = \sqrt{\sum^{\infty}_{i=1}|x_i|^2}$
\end{minipage}
\begin{minipage}{0.7\textwidth}
Неравенство треугольника в $l^2$ следует из неравенства треугольника для $||\cdot||_2$ на $\K^{\N}$ при $n \to \infty$
\end{minipage}

\vspace{20pt}

\textbf{Определение} $E$ - векторное пространство над $\R$

\textit{Скалярное произведение} на $E$ - функция $E \times E \longrightarrow \R$
$(x,y) \in E \times E \longrightarrow \la  x, y\ra \in \R$, удовлетворяющая условиям:

(1) $\la \alpha x + \beta y, z\ra = \alpha\la x, z\ra + \beta\la y, z\ra \quad (\alpha, \beta \in \R, \;\; x, y, z \in E)$

(2) $\la x, y\ra = \la y, x\ra \;\; (\fo x, y \in E )$

(3) $\la x, x\ra > 0 \;\; \fo x \neq 0$

\textit{Евклидово пространство} - векторное пространство $E$ над $\R$, снабженное скалярным произведением.

\textit{Факты:} 

(1) $\left| \la x, y\ra \right| \leq \sqrt{\la x, x\ra}\sqrt{\la y, y\ra}$ - неравенство Коши-Буняковского

(2) $||x|| = \sqrt{\la x, x\ra}$ - норма на $E$

\textbf{Пример 9.} Норма $||\cdot||_2$ на $\R^n$ порождено скалярным произведением $\la x, y\ra = \sum^{n}_{i=1} x_i y_i$. Норма $||\cdot||_2$ на $l^2$ порождена скалярным произведением $\la x, y\ra = \sum^n_{i = 1} x_i y_i$ (\textit{упражнение}: доказать сходимость этого ряда)

\textbf{Упражнение.} Доказать сходимость этого ряда.

\textbf{Упражнение*.} Доказать, что остальные нормы из примеров не порождаются скалярным произведением.

\textbf{Пример 10.} \textit{$p$ - адическая метрика} на $\Q$

Пусть $p \in \N$ - простое

\textit{Наблюдение:} Каждый $x \in \Q \setminus \{0\}$ имеет вид $x = p^r \frac a b$, где $a, r \in \Z, b \in \N,$ причем $p \nmid a, p \nmid b$

\textbf{Определение} \textit{$p$ - адическая норма} $x \in Q \setminus \{0 \}$ - это $|x|_p = p^{-r};\:\: |0|_p = 0$

\textit{Упражнение}

(1) $|-x|_p = |x|_p$

(2) $|xy|_p = |x|_p |y|_p $

(3) $|x|_p > 0 \quad  \fo x \neq 0$

(4) $|x + y|_p \leq \max\Big\{ |x|_p, |y|_p \Big\} \leq |x|_p + |y|_p$

(5) $\ro_p(x, y) = |x- y|_p$ - метрика на $\Q$

\textbf{Пример 11.} \textit{метрика хаусдорфа}

\textbf{Определение} $X$ - метрическое пространство, $x \in X, A \subset X$

$$\ro(x, A) = \inf \Big\{ \ro(x,a): a \in A \Big\} \text{ - расстояние от } x \text{ до } A$$

$X$ - метрическое пространство

\textbf{Определение.} Подмножество $A \subset X$ \textit{ограничено}, если $\ex C > 0 : \ro(x,y) \leq C \:\; \fo x, y \in A$

\textit{Обозначение} $\mB(x) = \Big\{ A \subset X \colon A \text{ ограничено}  \Big\}$

\textbf{Определение} \textit{Расстояние Хаусдорфа} между $A, B \in \mB(x)$ - это 
$$\ro_{H}(A, B) = \max \Big\{ \sup_{a\in A} \ro(a, B),\;\; \sup_{b \in B}\ro(b, A) \Big\}$$

\textit{Упражнение} $\ro_{H}$ - полуметрика на $\mB(x)$


\end{document}
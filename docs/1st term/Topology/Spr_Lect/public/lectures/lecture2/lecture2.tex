\documentclass[../../main.tex]{subfiles}
 
\begin{document}

\section{Лекция 2. 6 ноября 2019 г.}
\vspace{10pt}

{\large Открытые множества в метрическом пространстве. Топологические пространства. Метризуемость. Хаусдорфовость. Сравнение топологий. Замкнутые мнжества. Примеры: дискретная и антидискретная топологии, топология Зарисского.}

\vspace{10pt}

\subsection{Открытые множества в метрическом пространстве}

$(X, \ro)$ - метрическое пространство, $x \in X, r \geq 0$.

\defn Открытый шар с центром в $x$ радиуса $r$ - это 
$$B_r(x) = \left\{ y \in X \colon \ro(y, x) < r \right\}$$

\textit{Замкнутый шар} с центром в $x$ радиуса $r$ - это
$$\overline{B}_r(x) = \left\{ y\in X \colon \ro(y, x) \leq r \right\}$$

\textbf{Пример. } \quad $x = \R \rightarrow B_r(x) = (x - r, x + r); \quad \overline{B}_r(x) = [x - r, x+r] $

\textbf{Упражнение.} Нарисовать $B_1(o)$ на $(\R^2, \ro_p)$ для $p = 1, p = 2, p = \infty$.

\textbf{Пример. } $X = C[a,b]$ с равномерной метрикой


\begin{minipage}{0.4\linewidth}
(картинка графика)
\end{minipage}
\begin{minipage}{0.6\linewidth}
$\overline{B}_r(f)$ состоит из тех непрерывных функций, графики которых содержатся в заштрихованном множестве
\end{minipage}
\vspace{10pt}

\defn $(X, \ro)$ - метрическое пространство, $A \subset X, x \in A$

$x$ - \underline{внутренняя точка} $A \lra \ex \eps > 0 \colon B_{\eps}(x) \subset A$ 

$A$ называется \underline{открытым} $\lra$ все его точки - внутренние

\textbf{Предложение 1}. Открытый шар $B_r(x)$ открыт.

\textbf{Доказательство}

\begin{minipage}{0.4\linewidth}
(картинка док-ва)
\end{minipage}
\begin{minipage}{0.6\linewidth}
Пусть $y \in B_r(x)$, т.е $\ro(y,x) < r$.

Положим $\eps = r - \ro(y,x)$

Покажем: $B_{\eps}(y) \subset B_r(x)$ (*)

Пусть $z \in B_{\eps}(y)$

$\ro(z, x) \leq \ro(z,y) + \ro(y,x)< \eps + \ro(y,x) = r \Rightarrow z \in B_r(x) \Rightarrow $ (*) доказано $\Rightarrow B_r(x)$ открыто $\square$
\end{minipage}

\textbf{Предложение 2}.

(1) $\varnothing$ открыто

(2) $X$ открыто

(3) $\left\{ U_i \right\}_{i \in I}$ - семейство открытых множеств в $X \rightarrow \bigcup_{i \in I} U_i$ открыто

(4) $U_1, U_2, \ldots, U_n \subset X$ - открыты $\Rightarrow \bigcap^{n}_{i = 1} U_i$ открыто

\textbf{Доказательство} (1), (2) очевидны

(3) $x \in \bigcup_{i \in I} U_i \Rightarrow \ex i_0 \in X \colon x \in U_{i_o} \Rightarrow \ex \eps > 0 \colon B_{\eps}(x) \subset U_{i_0} \Rightarrow B_{\eps}(x) \subset \bigcup_{i \in I} U_i$

(4) достаточно для $n = 2$

$x \in U_1 \cap U_2$

$\ex \eps_1 \eps_2 > 0 \colon B_{\eps_1}(x) \subset U_1, B_{\eps_2} \subset U_2$

Обозначим $\eps = \min\left\{\eps_1, \eps_2 \right\} \Rightarrow B_{\eps}(x) \subset  U_1 \cap U_2 $

\subsection{Топологические пространства}

\defn Пусть $X$ - множество, $\tau \subset 2^{X} $

$\tau$ называется \underline{топологией} на $X$, если

(1) $\varnothing \in \tau$

(2) $X \in \tau$

(3) $\left\{U_i \right\}_{i \in I}$ - семейство множеств из $\tau \Rightarrow \bigcup_{i \in I} U_i \in \tau$

(4) $U_1, \ldots U_n \in \tau \Rightarrow \bigcap^n_{i = 1} U_i \in \tau$

$\left(X, \tau \right)$ называется \underline{топологическим пространством}

Множества из $\tau$ называются \underline{открытыми}

\textbf{Наблюдение.} Из предложения 2: каждая метрика $\ro$ на множестве $X$ порождает топологию $\tau_{\ro}$ на $X$

\defn 

Топологическое пространство $\left(X, \tau \right)$ называется \underline{метризуемым} $\lra \ex$ метрика $\ro\colon X\times X \rightarrow [0;\; +\infty) \colon \tau_{\ro} = \tau$

\textit{Замечение} Если $\tau = \tau_{\ro}$, то такая $\ro$ не единственная!

Например: $\tau_{\ro} = \tau_{2\ro}$

\textbf{Пример-упражнение} Метрики $\ro_1, \ro_2, \ro_{\infty}$ на $\K^n$ (где $\K = \R$ либо $\mathbb{C}$) порождают одну и ту же топологию на $\K^n$

\textbf{Пример 1} (дискретная топология)

$X$ - $\fo$ множество, $\tau = 2^X$

Рассмотрим $\ro\colon X\times X \rightarrow [0;\:+\infty), \quad \ro(x,y) = \begin{cases} 1, \text{ если } x\neq y, \\ 0 \text{ иначе}  \end{cases}$

Заметим: $\tau = \tau_{\ro}$

Действительно: $B_1(x) = {x} \Rightarrow {x}$ открыто в $\tau_{\ro} \: \fo x \in X \Rightarrow$ каждое $A \subset X$ открыто в $\tau_{\ro}$, т.к. $A = \bigcup_{x\in A} {x} \Rightarrow \tau_{\ro} = \tau$ - дискретная топология

\textbf{Пример 2}. (антидискретная топология)

$X$ - $\fo$ множество, $\tau = \left\{ \varnothing, X \right\}$

\defn Пусть $\tau_1, \tau_2$ - топологии на множестве $X$

Говорят, что $\tau_1$ грубее $\tau_2$ ($\tau_2$ \textit{тоньше} $\tau_1$), если $\tau_1 \subset \tau_2$

Синонимы: грубее = слабее, тоньше = сильнее

Дискретная - самая тонкая, антидискретная - самая грубая.

\defn \underline{Окрестность} точки $x$ в топологическом пространстве $X$ - любое открытое множество $U \subset X$, содержащее $x$

\subsection{Хаусдорфово топологическое пространство}

\defn 

Топологическое пространство $X$ называется \underline{хаусдорфовым} $\lra \fo x, y \in X, x \neq y, \ex$ окрестности $U \ni x, V \ni y \colon U \cap V = \varnothing$

\textbf{Предложение}. Метризуемое топологическое пространство хаусдорфово.

\textit{Доказательство:} Пусть $\left(X, \ro \right)$ - метр. пространство, $x, y \in X, x \neq y$. Обозначим $a = \ro(x, y),\;\; a > 0$

\begin{minipage}{0.8\linewidth}
Из неравенства треугольника: $B_{\frac a 2}(x) \cap B_{\frac a 2}(y) = \varnothing$
\end{minipage}
\begin{minipage}{0.2\linewidth}
(картинка с шарами)
\end{minipage}

\textit{Следствие:} Антидискретная тополония на множестве, содержащем более 1 элемента, неметризуема (т.к. неухаусдорфова)

\defn Пусть $X$ - топологическое пространство.

Множество $F \subset X$ называется \underline{замкнутым} $\lra$ $X \backslash F$ открыто.

\begin{minipage}{0.8\linewidth}
\textit{Предложение.} Пусть $X$ - топологическое пространство, $\tau' = \left\{ F \subset X: F \text{ замкнуто} \right\}$. Тогда:

(1) $\varnothing \in \tau'$

(2) $x \in \tau'$

(3) $\left\{ F_i \right\}$ - семейство множество из $\tau' \Rightarrow \bigcap_{i \in I} F_i \in \tau'$

(4) $F_1, F_2, \ldots, F_n \in \tau' \Rightarrow \bigcup^n_{i = 1} F_i$ замкнуто
\end{minipage}45
\begin{minipage}{0.2\linewidth}
Напоминание:

$X \backslash \bigcap_{i \in I} F_i = \bigcup_{i \in I}\left( X \backslash F_i \right)$

$X \backslash \bigcup_{i \in I} F_i = \bigcap_{i \in I}\left( X \backslash F_i \right)$

\end{minipage}

\textit{Наблюдение:} Если $X$ - множество, $\tau' \subset 2^X$ удовлетворяет (1) - (4) из предл. $\Rightarrow \left\{X \backslash F \colon F \in \tau' \right\}$ - топология на $X$

\subsection{Топология Зарисского}

\textbf{Пример} (топология Зарисского)

$X$ - множество, $\K = \R$ или $\mathbb{C}$.

\defn $A \subset \K^X$ - \underline{подалгебра} в $K^X$, если

(1) $A$ - векторное подпространство в $K^X$

(2) $1 \in A$ (где 1 - функция, тождественно равная единице)

(3) $f, g \in A \Rightarrow fg \in A$ ($fg$ - поточечное произведение $f$ и $g$)

Зафиксируем какую-либо подалгебру $A \subset \K^X$

$\fo S \subset A$ обозначим $V(S) = \left\{ x \in x \colon\;\; \fo f \in S \;\; f(x) = 0 \right\}$ 

\textit{Упражнение.} На $X$ существует топология, в которой $F \subset X$ замкнуто $\lra  F = V(S)$ для некоторого $S \subset A$. \\Она называется \underline{топологией Зарисского}

Важный частный случай: $X = \K^n,\; A = \K[t_1,\ldots, t_n]$

\textit{Упражнение:} Описать топологию Зарисского в явном виде для следующих случаев:

(1) $X$ - любое множество, $A = \K^X$

(2) $X = \K, \; A = \K[t]$

(3) $X = [a, b] \subset \R, \; A = C[a,b]$




\end{document}
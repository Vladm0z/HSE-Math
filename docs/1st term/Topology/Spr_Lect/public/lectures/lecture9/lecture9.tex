\documentclass[../../main.tex]{subfiles}
 
\begin{document}

\section{Лекция 9. 5 декабря 2019 г.}

\subsection{Связные компоненты.}
\defn Пусть $(X,\;\tau)$ - топологическое просранство. Связное множество $A\subset X$ называется \textbf{\textit{связной компонентой}} (или \textbf{\textit{компонентой связности}}), если оно является максимальным по вложению среди связных множеств. Иначе говоря, $\forall\; B\supsetneq A\; \Rightarrow\; B$ - несвязное. 

\begin{theo}[(Свойства связных компонент)]{thm:sv_comp}
(1) Связные компоненты образуют разбиение $X$ (то есть $\forall\; X_1,\; X_2$ - связные компоненты $\Rightarrow\; X_1 = X_2$ или $X_1\cap X_2 = \varnothing$ и $X = \bigcup \{A\subset X\; |\;A\; \text{- компонента свзности в }X \} $).\\
(2) $\forall\; x\in X\; \Rightarrow\; C(x):=\bigcup\{A\subset X\; |\; A\ni x,\; A\text{ - связно}\}$ является связной компонентой, содержащей $x$.\\
(3) Всякое непустое связное множество $A\subset X$ содержится ровно в одной связной компоненте.\\
(4) Связные компоненты замкнуты в $X$.
\end{theo}
\textit{Доказательство:}

(1), (2): пусть $X_1,\; X_2$ - копмоненты связности, $X_1\cap X_2 \neq \varnothing$. Тогда (по свойствам свяных множеств) $X_1\cup X_2$ тоже связное, $X_1\cup X_2 \supset X_1,\; X_2$, а так как $X_1,\; X_2$ - максимальные, то $X_1 = X_1\cup X_2 = X_2$.\\
Аналогично,  $C(x)$ - связное и содержит в себе любое связное множество, содержащее точку $x$, значит оно максимальное, то есть является компонентой связности.\\
Теперь множество $ \{C(x)\; |\; x\in X\}$ есть множество всех связных компонент и очевидно, что обьединение элементов этого множества является все множество $X$. 

(3): пусть В - такая компонента связности, что $A\cap B \neq \varnothing$ (такая существует, так как $\fo\; x\in A\; \exists\; B = C(x):\; x\in B\cap A\neq \varnothing$), тогда $A\cup B\supset B$ - связное множество, а значит $A\cup B = B$. $B$ единственно в силу (1).

(4): пусть $B$ - связная компонента, тогда $\overline B \supset B$ - связное множество $\Rightarrow \overline B = B$. $\square$

\textit{Следствие:} если $X$ - пространство, состоящее из конечного числа связных компонент, то компоненты связности открыты.\\
\textit{Доказательство:}
$X = C_1\cup...\cup C_n \; \Rightarrow C_i = X\backslash (C_1\cup...\cup C_{i-1}\cup C_{i+1}\cup...\cup C_n) = X\backslash [\text{замкнутое множество}]\; \Rightarrow C_i$ - открытое множество. $\square$

\textbf{Пример 1.} Связные компоненты $\R \backslash \{0\} $ - это $ (-\infty,\; 0)$ и $(0,\; +\infty)$.

\textbf{Пример 2.} $X_1,\; X_2$ - непустые связные пространства, $X = X_1\sqcup X_2 \; \Rightarrow \; X_1, \; X_2$ - связные компоненты $X$. 

\textbf{Пример 3.} $X$ - дискретное топологическое пространство $\Rightarrow$ связные компоненты - все одноэлементые множества.

\textbf{Пример 4.} То же самое верно для $X = \Q$ со стандартной топологией.

\textit{Упражнение:} описать компоненты канторова множества.\\
\defn Линейно связное $A\subset X$ называется \textbf{\textit{линейно связной компонентой}}, если $A$ - максимальное линейно связное множество.

\begin{theo}{thm:lin_sv_comp}

(1) Линейно связные компоненты образуют разбиение $X$.\\
(2) $\fo x\in X$ множество $PC(x):=\bigcup\{A\subset X\;-\; \text{линейно связное } |\; A\ni x \} = \{y\in X \; | \; \exists \text{ путь из }x \text{ в }y\}$ является линейно связной компонентой, содержащей $x$.\\
(3) Всякое непустое линейно связное подмножество $A\subset X$ содержится ровно в одной линейно связной компоненте.
\end{theo}
\textit{Доказательство:} в качестве упражнения. $\square$

\textit{Следствие:} разбиение пространства на линейно свзные компоненты является измельчением разбиения на связные компоненты.

\textit{Упражнение:} описать линейно связные компоненты $X = \{(x,\sin\frac1x)\}\cup [(0,\; -1),\; (0,\; 1)]$. Замкнуты ли они?\\
\defn Топологическое пространство $X$ называется \textbf{\textit{локально линейно связным}}, если $\fo\; x\in X\; \fo \; U$ - окрестности точки $ x \; \exists \; V\subset U$ - линейно связная окрестность точки$x$.

\textbf{Пример 5.} Открытое множество нормированного пространства локально линейно связно.

\textbf{Пример 6.} Любое топологическое многообразие локально линейно связно.

\textbf{Предложение.} Пусть $X$ - локально линейно связное пространство, тогда его линейно связные компоненты открыты в $X$ и совпадают со связными компонентами.\\
\textit{Доказательство:}
Пусть $A$ - линейно связная компонента. $\Rightarrow \fo\; x\in A\; \exists\; U_x \ni x\;$ - линейно связная окрестность $A\cup U_x \supset A$ - линейно связно $\Rightarrow\; A = A\cup U_x $, то есть $U_x\subset A\; \Rightarrow\; A = \bigcup\limits_{x\in A} U_x$ - объединение открытых множеств, значит $A$ - открыто.\\
Пусть $B$ - связная компонента, такая что $A\subset B$. $B\backslash A = \bigcup \{$линейно связные компоненты, содержащиеся в $B$ и отличные от  $A\}$ $\Rightarrow\; B\backslash A$ - открыто $\Rightarrow\; B = A\cup B\backslash A$ - объединение открытых $\Rightarrow\; B\backslash A = \varnothing$. $\square$

\subsection{Компактные топпологические пространства.}

Пусть $X$ - множество, $\Gamma\subset 2^X$.\\
\defn $\Gamma$ - \textbf{\textit{покрытие}} подмножества $Y\subset X$ ($\Gamma$ \textbf{\textit{покрывает}} $Y$ или $Y$ \textbf{\textit{покрывается}} семейством $\Gamma$), если $\bigcup\Gamma\supset Y$. Если на $X$ есть топология, то открытое покрытие означает покрытие открытыми множествами, замкнутое покрытие - замкнутыми.\\
\defn Топологическое пространство $X$ - \textbf{\textit{компактно}}, если из всякого открытого покрытия $\Gamma$ можно извлечь конечное подпокрытие (то есть конечное подсемейство $\Omega\subset\Gamma$, являющееся покрытием).\\
\defn $F\subset 2^X$. $F$ - центрированное семейство, если всякое конечное семейство $F_0\subset F$ имеет непустое пересечение: $\bigcap F_0 \neq \varnothing$.

\textbf{Предложение.} Топологическое пространство $X$ компактно $\Longleftrightarrow\; \fo \; \Gamma\subset 2^X $ - центрированное семейство его замкнутых подмножеств $\Rightarrow \bigcap \Gamma \neq \varnothing$.\\
\textit{Доказательство:}
Пусть $X$ - множество, $\mathcal F\subset 2^X$, $\Gamma = \{X\backslash F\; |\; F\in\mathcal F\}$. Заметим:\\
(1) $\bigcap \mathcal{F} = \varnothing\; \Leftrightarrow\; \Gamma$ - покрытие $X$;\\
(2) $\bigcap \mathcal{F}$ - центрированное $\Leftrightarrow$ никакое конечное подсемейство в $\Gamma$ не покрывает $X$.\\
Скомбинировав эти два замечания, получим требуемое. $\square$

\textbf{Предложение.} $X$ - топологическое пространство, $Y\subset X$.\\
$Y$ - компактно в индуцированной топологии $\Longleftrightarrow$ каждое покрытие $Y$ множествами, открытыми в $X$, имеет конечное подпокрытие.\\
\textit{Доказательство:}
$\Rightarrow$: Пусть $\{U_i\; |\; i\in I\}$ - покрытие $Y$ множестами, открытыми в $X$. Обозначим $V_i = U_i\cap Y$ - открытые в индуцированной топологии множества $\Rightarrow\; \{V_i\}$ - открытое покрытие $Y\; \Rightarrow\; \exists\; i_1,\;...\;,\;i_n:\; Y=V_{i_1}\cup\;...\;\cup V_{i_n}\; \Rightarrow \; Y\subset U_{i_1}\cup\;...\;\cup U_{i_n}$.\\
$\Leftarrow$: Пусть$\{V_i\; |\; i\in I\}$ - покрытие $Y$ множестами, открытыми в $Y$, тогда $V_i = U_i\cap Y$, где $U_i$ - некоторые открытые в $X$ множества $\Rightarrow\; \{U_i\}$ - открытое покрытие $Y\; \Rightarrow\; \exists\; i_1,\;...\;,\;i_n:\; Y=U_{i_1}\cup\;...\;\cup U_{i_n}\; \Rightarrow \; Y\subset V_{i_1}\cup\;...\;\cup V_{i_n}$. $\square$

\textbf{Пример 7.} Конечное топологическое пространство компактно.

\textbf{Пример 8.} Дискретное пространство компактно $\Longleftrightarrow$ оно конечно.

\textbf{Пример 9.} Антидискретное пространство компактно

\begin{theo}[]{thm:otr}

Отрезок $[a,\;b]\subset\R$ компактен
\end{theo}
\textbf{Доказательство:} смотреть курс анализа

\end{document}
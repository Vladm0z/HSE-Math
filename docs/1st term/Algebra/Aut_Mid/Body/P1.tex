	\section{Задачи}
		\subsection{1 Диофантовы уравнения}
		1)\\
		Рассмотрим $ax+by = k$, разделим все, если возможно(иначе корней нет), на $\gcd(a,b)$,\\
		Получим $\frac{a}{\gcd(a,b)} x+ \frac{b}{\gcd(a,b)} y = \frac{k}{\gcd(a,b)} \ = \ a_0 x + b_0 y = k_0$, теперь $\gcd(a_0,b_0) = 1$\\ 
		По алгоритму евклида найдем 1 пару $a_1,\ b_1$ при которой равенство выполнено, тогда все решения можно записать как:\\
		\begin{gather*}
			\begin{cases}
				x_n \ = \ a_1 + n \cdot \frac{a}{\gcd(a,b)} \ = \ a_1 + n \cdot a_0\\
				y_n \ = \ b_1 - n \cdot \frac{b}{\gcd(a,b)} \ = \ b_1 - n \cdot b_0
			\end{cases}
			\quad n \in \mathbb{Z}\\
			\text{gcd(a,b) \ = \ НОД(a,b) \ = \ (a,b)}
		\end{gather*}
		2)\\
		Рассмотрим $ax+by+cz = k$. Если $k \mod \gcd(a, b, c) = 0$, то у уравнения есть решения, иначе их нет.\\
		Пусть $p = \gcd(a, b)$, и $a^{\star} = \frac{a}{p} \quad b^{\star} = \frac{b}{p}$\\
		Тогда решим уравнение $a^{\star} u + b^{\star} v = c$ -- его решения $u_0$ и $v_0$ (по (1) пункту)\\
		$z_0$ и $t_0$ -- решения $cz + pt = d$ (по (1) пункту)\\
		$x_0$ и $y_0$ -- решения $a^{\star} x + b^{\star} y = t_0$ (по (1) пункту)\\
		Тогда решения системы это:
		\begin{gather*}
			\begin{cases}
				x \ = \ x_0 + b^{\star} k - u_0 m \\
				y \ = \ y_0 - a^{\star} k - v_0 m \\
				z \ = \ z_0 + p m
			\end{cases}
			\\
			k, m \in \mathbb{Z}
		\end{gather*}
		
		
		\subsection{2}
		Докажем что все конечные поля одинакового порядка изоморфны\\
		Рассмотрим поля $A$ и $B$ порядка $p^n$. Пусть $a \in A$ и $b \in B$ -- примитивные элементы полей. Ненулевых элементов в $A$ и $B$ ровно $p^n - 1$. \\
		У многочлена $x^{p^n - 1} - 1$ ровно $p^n - 1$ ненулевых корней. Все эти корни различны и лежат как в $A$, так и в $B$. Тогда, так как порядки полей совпадают, то некий $\alpha \in A$ перешел в $\beta \in B$. И тогда $\alpha^k = \beta$, а это отоношение задает изоморфизм полей.
		\\
		2:
		\begin{center}
			\begin{tabular}{|c|c|c|}
				\hline
				$+ $&$ 0 $&$ 1 $\\
				\hline
				$0 $&$ 0 $&$ 1 $\\
				\hline
				$1 $&$ 1 $&$ 0 $\\
				\hline
			\end{tabular}
		\end{center}
		\begin{center}
			\begin{tabular}{|c|c|c|}
				\hline
				$\times $&$ 0 $&$ 1 $\\
				\hline
				$0 $&$ 0 $&$ 0 $\\
				\hline
				$1 $&$ 0 $&$ 1 $\\
				\hline
			\end{tabular}
		\end{center}		
		4:
		\begin{center}
			\begin{tabular}{|c|c|c|c|c|}
				\hline
				$+ $&$ 0 $&$ 1 $&$ x $&$ x+1 $\\
				\hline
				$0 $&$ 0 $&$ 1 $&$ x $&$ x+1$\\
				\hline
				$1 $&$ 1 $&$ 0 $&$ x+1 $&$ x$\\
				\hline
				$x $&$ x $&$ x+1 $&$ 0 $&$ 1$\\
				\hline
				$x+1 $&$ x+1 $&$ x $&$ 1 $&$ 0$\\
				\hline
			\end{tabular}
		\end{center}
		\begin{center}
			\begin{tabular}{|c|c|c|c|c|}
				\hline
				$\times $&$ 0 $&$ 1 $&$ x $&$ x+1 $\\
				\hline
				$0 $&$ 0 $&$ 0 $&$ 0 $&$ 0$\\
				\hline
				$1 $&$ 0 $&$ 1 $&$ x $&$ x+1$\\
				\hline
				$x $&$ 0 $&$ x $&$ 0 $&$ 1$\\
				\hline
				$x+1 $&$ 0 $&$ x+1 $&$ 1 $&$ 0$\\
				\hline
			\end{tabular}
		\end{center}	
		8:
		\begin{center}
			\begin{tabular}{|c|c|c|c|c|c|c|c|c|}
				\hline
				$+ $&$ 0 $&$ 1 $&$ x $&$ x+1 $&$ x^2 $&$ x^2+1 $&$ x^2+x $&$ x^2+x+1$\\
				\hline
				$0 $&$ 0 $&$ 1 $&$ x $&$ x+1 $&$ x^2 $&$ x^2+1 $&$ x^2+x $&$ x^2+x+1$\\
				\hline
				$1 $&$ 1 $&$ 0 $&$ x+1 $&$ x $&$ x^2+1 $&$ x^2 $&$ x^2+x+1 $&$ x^2+x$\\
				\hline
				$x $&$ x $&$ x+1 $&$ 0 $&$ 1 $&$ x^2+x $&$ x^2+x+1 $&$ x^2 $&$ x^2+1$\\
				\hline
				$x+1 $&$ x+1 $&$ x $&$ 1 $&$ 0 $&$ x^2+x+1 $&$ x^2+x $&$ x^2+1 $&$ x^2$\\
				\hline
				$x^2 $&$ x^2 $&$ x^2+1 $&$ x^2+x $&$ x^2+x+1 $&$ 0 $&$ 1 $&$ x $&$ x+1$\\
				\hline
				$x^2+1 $&$ x^2+1 $&$ x^2 $&$ x^2+x+1 $&$ x^2+x $&$ 1 $&$ 0 $&$ x+1$&$ x$\\
				\hline
				$x^2+x $&$ x^2+x $&$ x^2+x+1 $&$ x^2 $&$ x^2+1 $&$ x $&$ x+1 $&$ 0 $&$ 1$\\
				\hline
				$x^2+x+1 $&$ x^2+x+1 $&$ x^2+x $&$ x^2+1 $&$ x^2 $&$ x+1 $&$ x $&$ 1 $&$ 0$\\
				\hline
			\end{tabular}
		\end{center}
		\begin{center}
			\begin{tabular}{|c|c|c|c|c|c|c|c|c|}
				\hline
				$\times $&$ 0 $&$ 1 $&$ x $&$ x+1 $&$ x^2 $&$ x^2+1 $&$ x^2+x $&$ x^2+x+1$\\
				\hline
				$0 $&$ 0 $&$ 0 $&$ 0 $&$ 0 $&$ 0 $&$ 0 $&$ 0 $&$ 0$\\
				\hline
				$1 $&$ 0 $&$ 1 $&$ x $&$ x+1 $&$ x^2 $&$ x^2+1 $&$ x^2+x $&$ x^2+x+1$\\
				\hline
				$x $&$ 0 $&$ x $&$ x^2 $&$ x^2+x $&$ x+1 $&$ 1 $&$ x^2+x+1 $&$ x^2+1$\\
				\hline
				$x+1 $&$ 0 $&$ x+1 $&$ x^2+x $&$ x^2+1 $&$ x^2+x+1 $&$ x^2 $&$ 1 $&$ x$\\
				\hline
				$x^2 $&$ 0 $&$ x^2 $&$ x+1 $&$ x^2+x+1 $&$ x^2+x $&$ x $&$ x^2+1 $&$ 1$\\
				\hline
				$x^2+1 $&$ 0 $&$ x^2+1 $&$ 1 $&$ x^2 $&$ x $&$ x^2+x+1 $&$ x+1 $&$ x^2+x$\\
				\hline
				$x^2+x $&$ 0 $&$ x^2+x $&$ x^2+x+1 $&$ 1 $&$ x^2+1 $&$ x+1 $&$ x $&$ x^2$\\
				\hline
				$x^2+x+1 $&$ 0 $&$ x^2+x+1 $&$ x^2+1 $&$ x $&$ 1 $&$ x^2+x $&$ x^2 $&$ x+1$\\
				\hline
			\end{tabular}
		\end{center}	
		3:
		\begin{center}
			\begin{tabular}{|c|c|c|c|}
				\hline
				$+ $&$ 0 $&$ 1 $&$ 2$\\
				\hline
				$0 $&$ 0 $&$ 1 $&$ 2$\\
				\hline
				$1 $&$ 1 $&$ 2 $&$ 0$\\
				\hline
				$2 $&$ 2 $&$ 0 $&$ 1$\\
				\hline
			\end{tabular}
		\end{center}
		\begin{center}
			\begin{tabular}{|c|c|c|c|}
				\hline
				$\times $&$ 0 $&$ 1 $&$ 2$\\
				\hline
				$0 $&$ 0 $&$ 0 $&$ 0$\\
				\hline
				$1 $&$ 0 $&$ 1 $&$ 2$\\
				\hline
				$2 $&$ 0 $&$ 2 $&$ 1$\\
				\hline
			\end{tabular}
		\end{center}
		9:
		\begin{center}
			\begin{tabular}{|c|c|c|c|c|c|c|c|c|c|}
				\hline
				$+ $&$ 0 $&$ 1 $&$ 2 $&$ x $&$ x+1 $&$ x+2 $&$ 2x $&$ 2x+1 $&$ 2x+2$\\
				\hline
				$0 $&$ 0 $&$ 1 $&$ 2 $&$ x $&$ x+1 $&$ x+2 $&$ 2x $&$ 2x+1 $&$ 2x+2$\\
				\hline
				$1 $&$ 1 $&$ 2 $&$ 0 $&$ x+1 $&$ x+2 $&$ x $&$ 2x+1 $&$ 2x+2 $&$ 2x$\\
				\hline
				$2 $&$ 2 $&$ 0 $&$ 1 $&$ x+2 $&$ x $&$ x+1 $&$ 2x+2 $&$ 2x+1 $&$ 2x$\\
				\hline
				$x $&$ x $&$ x+1 $&$ x+2 $&$ 2x $&$ 2x+1 $&$ 2x+2 $&$ 0 $&$ 1 $&$ 2$\\
				\hline
				$x+1 $&$ x+1 $&$ x+2 $&$ x $&$ 2x+1 $&$ 2x+2 $&$ 2x $&$ 1 $&$ 2 $&$ 0$\\
				\hline
				$x+2 $&$ x+2 $&$ x $&$ x+1 $&$ 2x+2 $&$ 2x $&$ 2x+1 $&$ 2 $&$ 0 $&$ 1$\\
				\hline
				$2x $&$ 2x $&$ 2x+1 $&$ 2x+2 $&$ 0 $&$ 1 $&$ 2 $&$ x $&$ x+1 $&$ x+2$\\
				\hline
				$2x+1 $&$ 2x+1 $&$ 2x+2 $&$ 2x $&$ 1 $&$ 2 $&$ 0 $&$ x+1 $&$ x+2 $&$ x$\\
				\hline
				$2x+2 $&$ 2x+2 $&$ 2x $&$ 2x+1 $&$ 2 $&$ 0 $&$ 1 $&$ x+2 $&$ x $&$ x+1$\\
				\hline
			\end{tabular}
		\end{center}
		\begin{center}
			\begin{tabular}{|c|c|c|c|c|c|c|c|c|c|}
				\hline
				$\times $&$ 0 $&$ 1 $&$ 2 $&$ x $&$ x+1 $&$ x+2 $&$ 2x $&$ 2x+1 $&$ 2x+2$\\
				\hline
				$0 $&$ 0 $&$ 0 $&$ 0 $&$ 0 $&$ 0 $&$ 0 $&$ 0 $&$ 0 $&$ 0$\\
				\hline
				$1 $&$ 0 $&$ 1 $&$ 2 $&$ x $&$ x+1 $&$ x+2 $&$ 2x $&$ 2x+1 $&$ 2x+2$\\
				\hline
				$2 $&$ 0 $&$ 2 $&$ 1 $&$ 2x $&$ 2x+2 $&$ 2x+1 $&$ x $&$ x+2 $&$ x+1$\\
				\hline
				$x $&$ 0 $&$ x $&$ 2x $&$ 2 $&$ x+2 $&$ 2x+2 $&$ 1 $&$ x+1 $&$ 2x+1$\\
				\hline
				$x+1 $&$ 0 $&$ x+1 $&$ 2x+2 $&$ x+2 $&$ 2x $&$ 1 $&$ 2x+1 $&$ 2 $&$ x$\\
				\hline
				$x+2 $&$ 0 $&$ x+2 $&$ 2x+1 $&$ 2x+2 $&$ 1 $&$ x $&$ x+1 $&$ 2x $&$ 2$\\
				\hline
				$2x $&$ 0 $&$ 2x $&$ x $&$ 1 $&$ 2x+1 $&$ x+1 $&$ 2 $&$ 2x+2 $&$ x+2$\\
				\hline
				$2x+1 $&$ 0 $&$ 2x+1 $&$ x+2 $&$ x+1 $&$ 2 $&$ 2x $&$ 2x+2 $&$ x $&$ 1$\\
				\hline
				$2x+2 $&$ 0 $&$ 2x+2 $&$ x+1 $&$ 2x+1 $&$ x $&$ 2 $&$ x+2 $&$ 1 $&$ 2x$\\
				\hline
			\end{tabular}
		\end{center}
		5:
		\begin{center}
			\begin{tabular}{|c|c|c|c|c|c|}
				\hline
				$+ $&$ 0 $&$ 1 $&$ 2 $&$ 3 $&$ 4 $\\
				\hline
				$0 $&$ 0 $&$ 1 $&$ 2 $&$ 3 $&$ 4$\\
				\hline
				$1 $&$ 1 $&$ 2 $&$ 3 $&$ 4 $&$ 0$\\
				\hline
				$2 $&$ 2 $&$ 3 $&$ 4 $&$ 0 $&$ 1$\\
				\hline
				$3 $&$ 3 $&$ 4 $&$ 0 $&$ 1 $&$ 2$\\
				\hline
				$4 $&$ 4 $&$ 0 $&$ 1 $&$ 2 $&$ 3$\\
				\hline
			\end{tabular}
		\end{center}
		\begin{center}
			\begin{tabular}{|c|c|c|c|c|c|}
				\hline
				$\times $&$ 0 $&$ 1 $&$ 2 $&$ 3 $&$ 4 $\\
				\hline
				$0 $&$ 0 $&$ 0 $&$ 0 $&$ 0 $&$ 0$\\
				\hline
				$1 $&$ 0 $&$ 1 $&$ 2 $&$ 3 $&$ 4$\\
				\hline
				$2 $&$ 0 $&$ 2 $&$ 4 $&$ 1 $&$ 3$\\
				\hline
				$3 $&$ 0 $&$ 3 $&$ 1 $&$ 4 $&$ 2$\\
				\hline
				$4 $&$ 0 $&$ 4 $&$ 3 $&$ 2 $&$ 1$\\
				\hline
			\end{tabular}
		\end{center}
		7:
		\begin{center}
			\begin{tabular}{|c|c|c|c|c|c|c|c|}
				\hline
				$+ $&$ 0 $&$ 1 $&$ 2 $&$ 3 $&$ 4 $&$ 5 $&$ 6$\\
				\hline
				$0 $&$ 0 $&$ 1 $&$ 2 $&$ 3 $&$ 4 $&$ 5 $&$ 6$\\
				\hline
				$1 $&$ 1 $&$ 2 $&$ 3 $&$ 4 $&$ 5 $&$ 6 $&$ 0$\\
				\hline
				$2 $&$ 2 $&$ 3 $&$ 4 $&$ 5 $&$ 6 $&$ 0 $&$ 1$\\
				\hline
				$3 $&$ 3 $&$ 4 $&$ 5 $&$ 6 $&$ 0 $&$ 1 $&$ 2$\\
				\hline
				$4 $&$ 4 $&$ 5 $&$ 6 $&$ 0 $&$ 1 $&$ 2 $&$ 3$\\
				\hline
				$5 $&$ 5 $&$ 6 $&$ 0 $&$ 1 $&$ 2 $&$ 3 $&$ 4$\\
				\hline
				$6 $&$ 6 $&$ 0 $&$ 1 $&$ 2 $&$ 3 $&$ 4 $&$ 5$\\
				\hline
			\end{tabular}
		\end{center}
		\begin{center}
			\begin{tabular}{|c|c|c|c|c|c|c|c|}
				\hline
				$\times $&$ 0 $&$ 1 $&$ 2 $&$ 3 $&$ 4 $&$ 5 $&$ 6$\\
				\hline
				$0 $&$ 0 $&$ 0 $&$ 0 $&$ 0 $&$ 0 $&$ 0 $&$ 0$\\
				\hline
				$1 $&$ 0 $&$ 1 $&$ 2 $&$ 3 $&$ 4 $&$ 5 $&$ 6$\\
				\hline
				$2 $&$ 0 $&$ 2 $&$ 4 $&$ 6 $&$ 1 $&$ 3 $&$ 5$\\
				\hline
				$3 $&$ 0 $&$ 3 $&$ 6 $&$ 2 $&$ 5 $&$ 1 $&$ 4$\\
				\hline
				$4 $&$ 0 $&$ 4 $&$ 1 $&$ 5 $&$ 2 $&$ 6 $&$ 3$\\
				\hline
				$5 $&$ 0 $&$ 5 $&$ 3 $&$ 1 $&$ 6 $&$ 4 $&$ 2$\\
				\hline
				$6 $&$ 0 $&$ 6 $&$ 5 $&$ 4 $&$ 3 $&$ 2 $&$ 1$\\
				\hline
			\end{tabular}
		\end{center}
\newpage
		\subsection{3}
		Приводимые в $\mathbb{F}_3$ приводимы и в $\mathbb{F}_9$.\\
		В $\mathbb{F}_3$ неприводимые это $x^2 + 1$; $x^2 + x + 2$; $x^2 + 2x + 2$\\
		В $\mathbb{F}_9$ $y^2 + 1 = 0$; $y^2 + y + 2$ имеет корень $(x + 1)$; $y^2 + 2y + 2$ имеет корень $(x + 2)$.
		\\
		\subsection{4}
		Покажем, что в $F_{16} = F[y]/(y^2 + y\overline{(x+1)} + 1)$ есть корни.\\
		Заметим, что если любое уравнение можно свести к $x^2 + x + c = 0$ или $x^2 + c = 0$. Докажем это: пусть уравнение вида $ax^2 + bx + c = 0$ (считаем $a$ не нулевым), тогда оно эквивалентно уравнению $x^2 + \frac{b}{a}x + \frac{c}{a}$. Сделаем замену переменных: $z \cdot \frac{b}{a} = x$. Тогда уравнение эквивалентно $z^2 + z + \frac{c \cdot a^2}{b^2} = 0$ с заменой корней (при $b \ne 0$, иначе эквивалентно уравнению вида $x^2 + c = 0$). Заметим, что $x^2 + x + c = 0$ имеют корни в $F16$: \\
		\begin{gather*}
			x^2 + x = 0 \qquad \ \ x = 0; \quad x = 1 \\
			x^2 + x + 1 = 0 \quad x = \overline{x}; \quad x = \overline{x+1}
		\end{gather*}
		Примечание: \\$(y \cdot \overline{x} + \alpha)^2 = y^2 \cdot \overline{x+1} + \alpha^2 = y \cdot \overline{x} + \overline{x+1} + \alpha^2$, поэтому решение двух оставшихся уравнений сводится к двум первым: $(y \cdot \overline{x} + \alpha)^2 + (y \cdot \overline{x} + \alpha) + c = \alpha^2 + \alpha + \overline{x+1} + c$\\
		\begin{gather*}
		x^2 + x + \overline{x} = 0 \qquad \ \ x = y \cdot \overline{x} + \overline{x}; \quad x = y \cdot \overline{x} + \overline{x+1} \\
		x^2 + x + \overline{x+1} = 0 \quad x = y \cdot \overline{x}; \qquad \ \ x = y \cdot \overline{x} + 1 
		\end{gather*}
		Покажем, что у уравнений вида $x^2 + c = 0$ есть решения: \\
		$0 \cdot 0 = 0$, откуда $x^2 + 0 = 0$ имеет корень. \\
		$1 \cdot 1 = 1$, откуда $x^2 + 1 = 0$ имеет корень. \\
		$\overline{x} \cdot \overline{x} = \overline{x+1}$, откуда $x^2 + \overline{x+1} = 0$ имеет корень. \\
		$\overline{x+1} \cdot \overline{x+1} = \overline{x}$, откуда $x^2 + \overline{x} = 0$ имеет корень. 
		
		
		\subsection{5}
		Рассмотрим группу обратимых для $n = 12 \quad \mathbb{Z}/ 12 \mathbb{Z}$, это $\{ 1,\ 5,\ 7,\ 11 \}$.\\
		Заметим, что $5 \cdot 5 = 25 = 1,\ 7 \cdot 7 = 49 = 1,\ 11 \cdot 11 = -1 \cdot -1 = 1$, откуда следует, что эта группа не циклична.
		
		\subsection{6}
		Докажем, что существует первообразный корень в $Z/pZ$.\\
		(1) \textbf{Теорема Ферма}: $\alpha^{p-1} = 1$ при $\alpha \ne 0$. Доказательство:\\
		Рассмотрим всевозможные произведения $\alpha$ на другие элемента поля. Так как это поле, значит в нём нет делителей $0$, откуда не может быть такого, что $\alpha \cdot x = \alpha \cdot y$ при $x \ne y$, поэтому всевозможные произведения различны. Откуда следует, что $\alpha \cdot 2\alpha \cdot \: ... \: \cdot (p-1)\alpha = (p-1)!  \ \Leftrightarrow \  a^{p-1}(p-1)! = (p-1)!  \ \Leftrightarrow \  a^{p-1} = 1 \ ((p-1)! \ne 0)$.\\
		\\
		(2) \textbf{Лемма}: $n = \sum \phi (i)$, где $i$ пробегает по всем делителям $n$ (Здесь мы работаем в натуральных числах).\\
		Доказательство:\\
		Будем говорить, что $\alpha \in [1,n]$ принадлежит множеству $M_i$ (где $i$ -- делитель $n$), если $\frac{\alpha}{\frac{n}{i}}$ целое, меньше $i$ и взаимнопросто с $i$. Нетрудно видеть, что каждое $\alpha$ может принадлежать не более $1$ множеству, так как то, что $\alpha \in M_{i_1} \ \Leftrightarrow \ (\alpha,\ n) = \frac{n}{i_1}$. Также видно, что любое $\alpha$ принадлежит хоть какому то множеству.\\ 
		Теперь заметим, что в каждом множестве $M_i$ ровно $\phi(i)$ элементов, так как таких $\alpha:\ \frac{\alpha}{n/i} \in Z$ - $i$ $(\frac{n}{i}, \frac{2n}{i},\ ... \: , \frac{in}{i})$, при этом среди чисел в промежутке $[1,i]$ взаимнопростых $\phi(i)$. Откуда следует то, что и требовалось доказать.\\
		\\
		(3) \textbf{Замечание}: элементов порядка $k$ либо $0$, либо $\phi(k)$. Доказательство:\\
		Предположим есть хотя бы $1$. (элемент $g$) Тогда элементы вида $g,\ g^2,\ g^3,\ ... ,\ g^k$ различны и являются корнями уравнения $x^k - 1 = 0$, откуда следует, что других корней нет, при этом если $\alpha \in [1,k]$ не взаимнопросто с $k$ (пусть $(\alpha,k) = y$), то порядок у $g^{\alpha} = \frac{k}{y}$, что не равно k. Поэтому элементов порядка $k$ ровно $\phi(k)$.\\
		\\
		Следствие \textbf{(1)}, \textbf{(2)} и \textbf{(3)}:\\
		Заметим, что если $k$ - не делитель $p-1$, то чисел порядка $k$ - ноль, так как порядок не может быть больше чем $p-1$ (иначе среди чисел $g,\ g^2,\ ...,\ g^i$ найдутся $2$ одинаковых, тогда разделим одно на другое и получим, что порядок меньше, чем предполагался -- противоречие), при этом любое число в степени $p-1$ равно $1$.\\
		Теперь рассмотрим все $k$, которые делят $p-1$. Заметим, что для всякого $k$ количество чисел порядка $k$ не больше чем $\phi(k)$, при этом сумма всех $\phi(i)$, где $i$ делит $p-1$, равна $p-1$, то есть ($\sum_{i \: | \: p-1} \phi(i) \: = \: p-1$), и всякое ненулевой элемент принадлежит хоть какому то порядку, откуда следует, что для всякого k количество чисел порядка k ровно $\phi(k)$. \\
		\\
		Откуда есть элементы порядка $p-1$, что и требовалось доказать.\\
		\\ \\
		Пусть первообразный корень это $g$, и $g^{p-1} = 1 + pk$. Рассмотрим числа вида $(g + pt)^{p-1} \quad \forall t \in Z$. Тогда $(g + pt)^{p-1} = 1 + p \cdot (k + (p-1)g^{p-2} \cdot t + p \cdot X)$. Заметим, что существует такое $t_1$, что $k + (p-1)g^{p-2} \cdot t_1 + p \cdot X = 1$ ($\mod \quad \mathbb{Z}/p^{n-1} \mathbb{Z}$), так как существует обратное у $(p-1) \cdot g^{p-2}$ (назовем его $t_0$). Рассмотрим $t_1 = t_0 \cdot (1 - k - p \cdot X)$, заметим, что $1+p$ принадлежит показателю вида $p^\alpha$, так как $g + pt_1$ принадлежит показателю вида $p^\beta \cdot (p-1)$, так как все возможные непустые показатели являются делителями $p^{n-1} \cdot (p-1)$, при этом $(g + pt_1)^{p-1} \ne 1$. \\ \\
		Рассмотрим $(1+p) ^{p^{\alpha}} = 1 + p^{\alpha + 1} \cdot (1 + p \cdot Y) = 1 + p^{\alpha + 1} \cdot u_{\alpha}$, где $u_{\alpha}$ взаимнопросто с $p$. Предположим, что $(1+p) ^{p^{\alpha}} = 1$, тогда $p^{\alpha + 1} = 0$, откуда $\alpha + 1 = n$, поэтому $1 + p$ принадлежит показателю $p^{n-1}$, следовательно $g + pt_1$ принадлежит показателю $p^{n-1} \cdot (p-1)$, и тогда $g + pt_1$ -- первообразный корень.\\
		\\
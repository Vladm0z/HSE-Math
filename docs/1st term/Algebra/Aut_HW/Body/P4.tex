\newpage
	\section{Алгебра ДЗ 4}
		\subsection{1}
		$\mathbb{F}_2 / x^3+x^2+1$ \\
		\begin{table}[h]
			\begin{center}
				\begin{tabular}{|c|c|c|c|c|c|c|c|}
					\hline
					$$ & $1$ & $x$ & $x+1$ & $x^2$ & $x^2+1$ & $x^2+x$ & $x^2+x+1$ \\
					\hline
					$1$ & $1$ & $$ & $$ & $$ & $$ & $$ & $$ \\
					\hline
					$x$ & $$ & $$ & $$ & $$ & $$ & $1$ & $$ \\
					\hline
					$x+1$ & $$ & $$ & $$ & $1$ & $$ & $$ & $$ \\
					\hline
					$x^2$ & $$ & $$ & $1$ & $$ & $$ & $$ & $$ \\
					\hline
					$x^2+1$ & $$ & $$ & $$ & $$ & $$ & $$ & $1$ \\
					\hline
					$x^2+x$ & $$ & $1$ & $$ & $$ & $$ & $$ & $$ \\
					\hline
					$x^2+x+1$ & $$ & $$ & $$ & $$ & $1$ & $$ & $$ $$\\
					\hline
				\end{tabular}
			\end{center}
		\end{table}
		\\
		$(x^2+1)(x^2+x+1) = x^4+x^3+x+1 = x(x^3+x^2+1) + 1$\\
		Т.к. каждый элемент обратим, то $\mathbb{F}_2 / x^3+x^2+1$ - поле
	
		\subsection{2}
		$(\overline{x+1})^{k} = 1$\\
		$(x+1)^{2} = x^{2} + 1$\\
		$(x+1)^{3} = (x^{2} + 1) + (x^{3} + x) = x$\\
		$(x+1)^{4} = x^{2} + x$\\
		$(x+1)^{5} = x^{2} + x + 1$\\
		$(x+1)^{6} = x^{6} + x^{4} + x^{2} + 1 = (x^{3} + x^{2} + 1)^{2} + x^{2} = x^{2}$\\
		$(x+1)^{7} = x^{2}(x + 1) = 1 \ \Rightarrow \ k = 7$
		
		\subsection{3}$F_4$ состоит из 4 элементов - $\{ 0,\: 1,\: x,\: x+1 \}$ \\
		Назовём $F_{16}$ кольцо из условия. \\
		Заметим, что если $k = y^2 + y(\overline{x+1}) + 1$ неприводим над $F_4[y]$, то все $ak$, где $a \in F_4$, неприводимы над $F_4[y]$, т.к. $F_4$ - поле, откуда если $ak = m \cdot n$, то $k = \frac{m}{a} \cdot n$, при этом $\frac{m}{a}$ определён при ненулевых $a$, откуда вытекает противоречие.\\
		Покажем, что $k$ неприводим. Он степени $2$, поэтому $2$ множителя (если они есть) степени $1$. Если есть множитель вида $x-a$, то значит, что $a$ - корень $k$. Покажем, что $k$ не имеет корней:\\
		\begin{gather*}
		k(0) = 1  \\
		k(1) = 1 + (\overline{x+1}) + 1 = \overline{x+1} \\
		k(\overline{x}) = (\overline{x+1}) + (\overline{x})(\overline{x+1}) + 1 = (\overline{x+1}) + 1 + 1 = \overline{x+1} \\
		k(\overline{x+1}) = (\overline{x}) + (\overline{x+1})(\overline{x+1}) + 1 = (\overline{x}) + (\overline{x}) + 1 = 1 
		\end{gather*}
		Заметим, что мы доказали, что в кольце $F_4[y]/k$ нет делителей $0$.\\
		Теперь покажем, что любой ненулевой обратим --- рассмотрим ненулевой элемент $\alpha \in F_4[y]/k$. Пусть $\alpha \cdot \beta = \alpha \cdot \gamma$ для некоторых $\beta,\: \gamma \in F_4[y]/k$. Тогда $\alpha \cdot (\beta - \gamma) = 0$ , т.к. $F_4$ - поле. При этом $0$ не имеет делителей, поэтому $(\beta - \gamma) = 0$, что значит, что $\beta = \gamma$. Поэтому все возможные произведения $\alpha$ с разными элементами из $F_4[y]/k$ будут давать разные результаты, поэтому один из результатов будет $1$.
		
		\subsection{4}
		Рассмотрим какую-то функцию $f_1: \ f_1 \ne f(x) = 0$, в которой $0$ принадлежит образу. Так как произведение $f_1$ и функции, равной $0$ в дополнении к образу $0$ в $f_1$, и равной $1$ во всех остальных точках, равно $0$, то $f_1$ является делителем $0$. Заметим, что любая другая функция $f$ обратима. Рассмотрим следующую функцию $g: \ g(x) = \frac{1}{f(x)}$, заметим, что она обратная для $f: \ f(x) \ne 0$, поэтому функция определена. Заметим, что в поле нет делителей $0$, поэтому если множество всех функций - поле, то любая функция $k$, в которой $0$ принадлежит образу, вида $k(x) = 0$, откуда следует, что мощность $|M| = 1$. Замтим, что если $|M| = 1$, то множество функций изоморфно множеству $R$, которое является полем.
		
		\subsection{5}
		$(a + b)^p = a^p + b^p; \quad a,b \in \mathbb{F}_p$\\
		т.к. $(a + b)^p = \sum_{b = 0}^{p} {{p \choose k} a^{p-k} b^{k}}$, то все коэффициенты кроме первого и последнего $\vdots p$\\ 
		$\Rightarrow \: (a + b)^p = a^p + b^p$

	\section{Алгебра ДЗ 5}
		\subsection{1}
			\begin{gather*}
				1: y \cdot (\overline{x+1}) + (\overline{x})
				\\ \\2: (y \cdot (\overline{x+1}) + (\overline{x}))  \cdot  (y \cdot (\overline{x+1}) + (\overline{x})) = \\ 
				y \cdot ((\overline{x+1})  \cdot  (\overline{x}) + (\overline{x})  \cdot  (\overline{x+1}) + (\overline{x+1})  \cdot  (\overline{x+1})  \cdot  (\overline{x+1}) ) + ((\overline{x})  \cdot  (\overline{x}) + (\overline{x+1})  \cdot  (\overline{x+1}) ) = \\
				y \cdot ((1) + (1) + (1) ) + ((\overline{x+1}) + (\overline{x}) = \\ y \cdot (1) + (1)
				\\ \\3: (y \cdot (1) + (1))  \cdot  (y \cdot (\overline{x+1}) + (\overline{x})) = \\ 
				y \cdot ((1)  \cdot  (\overline{x}) + (1)  \cdot  (\overline{x+1}) + (1)  \cdot  (\overline{x+1})  \cdot  (\overline{x+1}) ) + ((1)  \cdot  (\overline{x}) + (1)  \cdot  (\overline{x+1}) ) = \\
				y \cdot ((\overline{x}) + (\overline{x+1}) + (\overline{x}) ) + ((\overline{x}) + (\overline{x+1}) = \\ y \cdot (\overline{x+1}) + (1)
				\\ \\4: (y \cdot (\overline{x+1}) + (1))  \cdot  (y \cdot (\overline{x+1}) + (\overline{x})) = \\ 
				y \cdot ((\overline{x+1})  \cdot  (\overline{x}) + (1)  \cdot  (\overline{x+1}) + (\overline{x+1})  \cdot  (\overline{x+1})  \cdot  (\overline{x+1}) ) + ((1)  \cdot  (\overline{x}) + (\overline{x+1})  \cdot  (\overline{x+1}) ) = \\
				y \cdot ((1) + (\overline{x+1}) + (1) ) + ((\overline{x}) + (\overline{x}) = \\ y \cdot (\overline{x+1}) + (0)
				\\ \\5: (y \cdot (\overline{x+1}) + (0))  \cdot  (y \cdot (\overline{x+1}) + (\overline{x})) = \\ 
				y \cdot ((\overline{x+1})  \cdot  (\overline{x}) + (0)  \cdot  (\overline{x+1}) + (\overline{x+1})  \cdot  (\overline{x+1})  \cdot  (\overline{x+1}) ) + ((0)  \cdot  (\overline{x}) + (\overline{x+1})  \cdot  (\overline{x+1}) ) = \\
				y \cdot ((1) + (0) + (1) ) + ((0) + (\overline{x}) = \\ y \cdot (0) + (\overline{x})
				\\ \\6: (y \cdot (0) + (\overline{x}))  \cdot  (y \cdot (\overline{x+1}) + (\overline{x})) = \\ 
				y \cdot ((0)  \cdot  (\overline{x}) + (\overline{x})  \cdot  (\overline{x+1}) + (0)  \cdot  (\overline{x+1})  \cdot  (\overline{x+1}) ) + ((\overline{x})  \cdot  (\overline{x}) + (0)  \cdot  (\overline{x+1}) ) = \\
				y \cdot ((0) + (1) + (0) ) + ((\overline{x+1}) + (0) = \\ y \cdot (1) + (\overline{x+1})
				\\ \\7: (y \cdot (1) + (\overline{x+1}))  \cdot  (y \cdot (\overline{x+1}) + (\overline{x})) = \\ 
				y \cdot ((1)  \cdot  (\overline{x}) + (\overline{x+1})  \cdot  (\overline{x+1}) + (1)  \cdot  (\overline{x+1})  \cdot  (\overline{x+1}) ) + ((\overline{x+1})  \cdot  (\overline{x}) + (1)  \cdot  (\overline{x+1}) ) = \\
				y \cdot ((\overline{x}) + (\overline{x}) + (\overline{x}) ) + ((1) + (\overline{x+1}) = \\ y \cdot (\overline{x}) + (\overline{x})
				\\ \\8: (y \cdot (\overline{x}) + (\overline{x}))  \cdot  (y \cdot (\overline{x+1}) + (\overline{x})) = \\ 
				y \cdot ((\overline{x})  \cdot  (\overline{x}) + (\overline{x})  \cdot  (\overline{x+1}) + (\overline{x})  \cdot  (\overline{x+1})  \cdot  (\overline{x+1}) ) + ((\overline{x})  \cdot  (\overline{x}) + (\overline{x})  \cdot  (\overline{x+1}) ) = \\
				y \cdot ((\overline{x+1}) + (1) + (\overline{x+1}) ) + ((\overline{x+1}) + (1) = \\ y \cdot (1) + (\overline{x})
				\\ \\9: (y \cdot (1) + (x))  \cdot  (y \cdot (\overline{x+1}) + (x)) = \\ 
				y \cdot ((1)  \cdot  (x) + (x)  \cdot  (\overline{x+1}) + (1)  \cdot  (\overline{x+1})  \cdot  (\overline{x+1}) ) + ((x)  \cdot  (x) + (1)  \cdot  (\overline{x+1}) ) = \\
				y \cdot ((x) + (1) + (x) ) + ((\overline{x+1}) + (\overline{x+1}) = \\ y \cdot (1) + (0)
				\\ \\10: (y \cdot (1) + (0))  \cdot  (y \cdot (\overline{x+1}) + (x)) = \\ 
				y \cdot ((1)  \cdot  (x) + (0)  \cdot  (\overline{x+1}) + (1)  \cdot  (\overline{x+1})  \cdot  (\overline{x+1}) ) + ((0)  \cdot  (x) + (1)  \cdot  (\overline{x+1}) ) = \\
				y \cdot ((x) + (0) + (x) ) + ((0) + (\overline{x+1}) = \\ y \cdot (0) + (\overline{x+1})
		\end{gather*}
		\begin{gather*}
				11: (y \cdot (0) + (\overline{x+1}))  \cdot  (y \cdot (\overline{x+1}) + (x)) = \\ 
				y \cdot ((0)  \cdot  (x) + (\overline{x+1})  \cdot  (\overline{x+1}) + (0)  \cdot  (\overline{x+1})  \cdot  (\overline{x+1}) ) + ((\overline{x+1})  \cdot  (x) + (0)  \cdot  (\overline{x+1}) ) = \\
				y \cdot ((0) + (x) + (0) ) + ((1) + (0) = \\ y \cdot (x) + (1)
				\\ \\12: (y \cdot (x) + (1))  \cdot  (y \cdot (\overline{x+1}) + (x)) = \\ 
				y \cdot ((x)  \cdot  (x) + (1)  \cdot  (\overline{x+1}) + (x)  \cdot  (\overline{x+1})  \cdot  (\overline{x+1}) ) + ((1)  \cdot  (x) + (x)  \cdot  (\overline{x+1}) ) = \\
				y \cdot ((\overline{x+1}) + (\overline{x+1}) + (\overline{x+1}) ) + ((x) + (1) = \\ y \cdot (\overline{x+1}) + (\overline{x+1})
				\\ \\13: (y \cdot (\overline{x+1}) + (\overline{x+1}))  \cdot  (y \cdot (\overline{x+1}) + (x)) = \\ 
				y \cdot ((\overline{x+1})  \cdot  (x) + (\overline{x+1})  \cdot  (\overline{x+1}) + (\overline{x+1})  \cdot  (\overline{x+1})  \cdot  (\overline{x+1}) ) + ((\overline{x+1})  \cdot  (x) + (\overline{x+1})  \cdot  (\overline{x+1}) ) = \\
				y \cdot ((1) + (x) + (1) ) + ((1) + (x) = \\ y \cdot (x) + (\overline{x+1})
				\\ \\14: (y \cdot (x) + (\overline{x+1}))  \cdot  (y \cdot (\overline{x+1}) + (x)) = \\ 
				y \cdot ((x)  \cdot  (x) + (\overline{x+1})  \cdot  (\overline{x+1}) + (x)  \cdot  (\overline{x+1})  \cdot  (\overline{x+1}) ) + ((\overline{x+1})  \cdot  (x) + (x)  \cdot  (\overline{x+1}) ) = \\
				y \cdot ((\overline{x+1}) + (x) + (\overline{x+1}) ) + ((1) + (1) = \\ y \cdot (x) + (0)
				\\ \\15: (y \cdot (x) + (0))  \cdot  (y \cdot (\overline{x+1}) + (x)) = \\ 
				y \cdot ((x)  \cdot  (x) + (0)  \cdot  (\overline{x+1}) + (x)  \cdot  (\overline{x+1})  \cdot  (\overline{x+1}) ) + ((0)  \cdot  (x) + (x)  \cdot  (\overline{x+1}) ) = \\
				y \cdot ((\overline{x+1}) + (0) + (\overline{x+1}) ) + ((0) + (1) = \\ y \cdot (0) + (1)
			\end{gather*}
		\subsection{2}
		
		Сначала найдём все неприводимые многочлены степени $\leqslant 4$.
		Степени 1: 
		\begin{gather*}
			x\\
			x+1
		\end{gather*}
		Степени 2,3:\\
		(это многочлены, не имеющих корней, т.к. иначе они раскладываются на множители, хотя бы 1 из который степени 1):
		\begin{gather*}
			x^2 + x + 1\\
			x^3 + x^2 + 1\\
			x^3 + x + 1
		\end{gather*}
		Степени 4:\\
		Заметим, что в разложении на простые (у приводимых) либо есть многочлен первой степени, либо это $(x^2 + x + 1)^2 = 
		x^4 + x^2 + 1$, откуда неприводимые :
		\begin{gather*}
			x^4 + x^3 + x^2 + x + 1\\
			x^4 + x^3 + 1\\
			x^4 + x + 1
		\end{gather*}
		Заметим, что любые приводимые многочлены 6 степени раскладываются в произведение неприводимых, при этом среди множителей либо есть многочлен первой степени, либо это произведение многочленов степени 2 и 4, либо степеней 2, и 2, и 2, либо степеней 3 и 3. Всего многочленов 6той степени, не делящихся на многочлены первой степени - $\frac{2^6}{2 \cdot 2} = 2^4 = 16$. Возможных произведений многочленов степени 2, 4: $1 \cdot 3 = 3$, степеней 2, 2, 2: $1$, степеней 3, 3: $2 \cdot 2 - 1 = 3$, откуда неприводимых многочленов: $16 - 3 - 1 - 3 = 9$\\
		\\
		Ответ:$9$.
		
		\subsection{3}
		Рассмотрим $F_9 = F_3[y]/(y^2 + 1)$. Заметим, что любой автоморфизм переводит 0 в 0, 1 в 1. Тогда $(-1) \to (-1)$. Пусть $ay + b \to y$, тогда заметим, что автоморфизм однозначно задался, т.к. любой элемент лежит в линейной оболочке $[1;y]$. Но тогда $(ay + b)^2 = -1$, откуда $(ay + b)$ - корень уравнения $\alpha^2 + 1 = 0$. Корни : $y$,$-y$. Откуда есть только 1 нетривиальный автоморфизм со следующим отображением : $ay + b \to -ay + b$
		
		\subsection{4 теорема Вильсона}
		Заметим, что у каждого элемента есть обратный, при этом только для $2х$ элементов обратный равен элементу (только 2 корня уравнения $x^2 = 1$), а именно $1$ и $-1$. Поэтому разобьём все элементы, кроме $1$ и $-1$, на пары обратных. Тогда нетрудно видеть, что произведение всех, кроме $1$ и $-1$, равны $1$. Откуда произведение всех равно $-1$.
		
		\subsection{5}
		Пусть элементы поля --- $a_1, a_2 ... a_n$\\
		Пусть $f_{a_i}(x) = (x-a_1)(x-a_2)...(x-a_{i-1})(x-a_{i+1})...(x-a_n)$.\\
		Заметим, что $f_{a_i}(x)$ --- многочлен, принимающий во всех точках, кроме $a_i$ и $0$.\\
		Рассмотрим следующий многочлен:\\
		\\
		$g(x) = b_1 \cdot \frac{f_{a_1}(x)}{f_{a_1}(a_1)} + b_2 \cdot \frac{f_{a_2}(x)}{f_{a_2}(a_2)} + ... + b_n \cdot \frac{f_{a_n}(x)}{f_{a_n}(a_n)}$\\
		Заметим, что $g(a_i) = b_i$, т.к. все выражения, кроме $b_i \cdot \frac{f_{a_i}(x)}{f_{a_i}(a_i)}$, обнуляются, а оставшееся равно $b_i$, откуда следует, что такими многочленами можно задать любую функцию, поэтому любая функция - многочлен.
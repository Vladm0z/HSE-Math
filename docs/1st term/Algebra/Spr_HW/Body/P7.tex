\newpage
	{\large \hspace{3cm} \begin{center} Домашнее задание 20 и 21 $\bullet$ Мозговой Владислав \end{center} }
	\vspace{-1.5ex}
	\hrulefill
	
	\fontsize{12pt}{4.5mm}\selectfont
	\vspace{-3ex}
	\hrulefill
	\newline

	\section{}
		\subsection*{\textbf{Задача 1}}
		\subsubsection*{\textbf{А}}
		\textbf{Условие}\\
		 Найдите жорданову форму и жорданов базис матрицы
		 \begin{gather*}
			 A:=
			 \left[\begin{array}{ccc}
				 -4 & 1 & 2 \\
				 -3 & 0 & 8 \\
				 1 & -1 & -6
			 \end{array}\right]
		\end{gather*}
		\\
		\textbf{Решение}\\
		\begin{gather*}
			|A- \lambda E| = 
			\begin{pmatrix}
				-4-\lambda & 1 & 2\\
				-3 & -\lambda & 8\\
				1 & -1 & -6-\lambda
			\end{pmatrix}
			= -\lambda^3 - 10\lambda^2 - 33\lambda - 36 = -(\lambda + 3)^2(\lambda + 4) = 0\\
			\lambda_1 = -3\quad \lambda_2 = -4
		\end{gather*}
		Тогда жорданова форма это
		\begin{gather*}
			\begin{pmatrix}
				-4 & 0 & 0\\
				0 & -3 & 1\\
				0 & 0 & -3
			\end{pmatrix}
		\end{gather*}
		Теперь найдем жорданов базис
		\begin{gather*}
			\text{ker}(A + 4E) = \text{ker}
			\begin{pmatrix}
				0 & 1 & 2\\
				-3 & 4 & 8\\
				1 & -1 & -2
			\end{pmatrix}
			\\
			e_1 = 
			\begin{pmatrix}
				0 \\ -2 \\ 1
			\end{pmatrix}\\
			\\
			\text{ker}(A + 3E) = \text{ker}
			\begin{pmatrix}
				-1 & 1 & 2\\
				-3 & 3 & 8\\
				1 & -1 & -3
			\end{pmatrix}
			\\
			e_2 = 
			\begin{pmatrix}
				1 \\ 1 \\ 0
			\end{pmatrix}
			\\
			\text{ker}(A + 3E)^2\\
			(A + 3E) 
			\begin{pmatrix}
				x_1 \\ x_2 \\ x_3
			\end{pmatrix}
			=
			\begin{pmatrix}
				1 \\ 1 \\ 0
			\end{pmatrix}
			\\
			\begin{pmatrix}
				-1 & 1 & 2\\
				-3 & 3 & 8\\
				1 & -1 & -3
			\end{pmatrix}
			\begin{pmatrix}
				x_1 \\ x_2 \\ x_3
			\end{pmatrix}
			= 
			\begin{pmatrix}
				-x_1 + x_2 + 2x_3\\
				-3x_1 + 3x_2 + 8x_3\\
				x_1 - x_2 - 3x_3
			\end{pmatrix}
			=\\ 
			\begin{pmatrix}
				-x_1 + x_2 + 2x_3\\
				-3x_1 + 3x_2 + 8x_3\\
				-x_3
			\end{pmatrix}
			= 
			\begin{pmatrix}
				-x_1 + x_2\\
				-3x_1 + 3x_2\\
				-x_3
			\end{pmatrix}
			= 
			\begin{pmatrix}
				-x_1 + x_2\\
				0\\
				-x_3
			\end{pmatrix}
			\\
			e_3 = 
			\begin{pmatrix}
				3 \\ 0 \\ 1
			\end{pmatrix}
		\end{gather*}
		\\
		Тогда жорданов базис это
		\begin{gather*}
			e_1 = 
			\begin{pmatrix}
				0 \\ -2 \\ 1
			\end{pmatrix}
			\quad
			e_2 = 
			\begin{pmatrix}
				1 \\ 1 \\ 0
			\end{pmatrix}
			\quad
			e_3 = 
			\begin{pmatrix}
				3 \\ 0 \\ 1
			\end{pmatrix}
		\end{gather*}
		
		\subsubsection*{\textbf{Б}}
		\textbf{Условие}\\
		 Опишите все $A$-инвариантные подпространства в $C^3$
		\\
		\textbf{Решение}\\
		Так как имеется 2 линейно независимых собственныз вектора, то $A$-инвариантные подпространства состоят из линейной комбинации этих векторов и подпространства $A$:\\
		$\{c_1 V_1\}, \{c_2 V_2\}, \{c_1 V_1 + c_2 V_2\}, \{A\}, \{0\}\qquad c_1,c_2 \in \mathbb{C}$
		\\
		
		\subsection*{\textbf{Задача 2}}
		Рассмотрим блочную матрицу
		\begin{gather*}
			\left[\begin{array}{ccccc}
				7 & 9 & 0 & 0 & 0 \\
				-4 & -5 & 0 & 0 & 0 \\
				0 & 0 & -1 & -2 & -1 \\
				0 & 0 & 1 & 1 & 0 \\
				0 & 0 & -1 & -1 & 0
			\end{array}\right]
		\end{gather*}
		\subsubsection*{\textbf{А}}
		\textbf{Условие}\\
		 Найдите все инвариантные подпространства размерности 3 соответствующего оператора\\
		\\
		\textbf{Решение}\\
		Рассмотрим матрицу:
		\begin{gather*}
			F = A \oplus B\\
			F = 
			\begin{pmatrix}
				A & 0\\
				0 & B
			\end{pmatrix}
			\qquad 
			A = 
			\begin{pmatrix}
				7 & 9\\
				-4 & -5
			\end{pmatrix}
			\qquad
			B = 
			\begin{pmatrix}
				-1 & -2 & -1\\
				1 & 1 & 0\\
				-1 & -1 & 0
			\end{pmatrix}
		\end{gather*}
		Пространство $V_1$ называется инвариантным, если $F(V_1) \subset V_1$, $v$ -- собственный вектор, если $Fv = \lambda_1 v\ \Rightarrow FV \in V_1$\\
		Инвариантные подпространства -- $\{\sum x_i v_i\}$, где $v_i$ -- собственный, $x_i \in k$\\
		Найдем собственные векторы
		\begin{enumerate}
		\item 
			\begin{gather*}
				A:\
				\begin{pmatrix}
					7-\lambda & 9\\
					-4 & -5-\lambda	
				\end{pmatrix}
				= (\lambda -1)^2\\
				\lambda_1 = 1\\
				\begin{pmatrix}
					7-\lambda & 9\\
					-4 & -5-\lambda	
				\end{pmatrix}
				=
				\begin{pmatrix}
					6 & 9\\
					-4 & -6	
				\end{pmatrix}
				=
				\begin{pmatrix}
					2 & 3
				\end{pmatrix}\\
				V_1 = 
				\begin{pmatrix}
					-3 & 2
				\end{pmatrix}
			\end{gather*}
		\item 
			\begin{gather*}
				B:\
				\begin{pmatrix}
					-1-\lambda & -2 & -1\\
					1 & 1 - \lambda & 0\\
					-1 & -1 & -\lambda
				\end{pmatrix}
				=
				-\lambda^3\\
				\lambda_2 = 0\\
				\begin{pmatrix}
					-1 & -2 & -1\\
					1 & 1 & 0\\
					-1 & -1 & 0
				\end{pmatrix}
				=
				\begin{pmatrix}
					0 & -1 & -1\\
					1 & 1 & 0\\
					0 & 0 & 0
				\end{pmatrix}\\
				V_2 = 
				\begin{pmatrix}
					1 \\ -1 \\ 1
				\end{pmatrix}
			\end{gather*}
		\end{enumerate}
		Так как имеется всего 2 собственных вектора, то не существует инвариантных пространств размерности $3$ помимо $B$\\
		
		\subsubsection*{\textbf{Б}}
		\textbf{Условие}\\
		Докажите, что множество операторов, коммутирующих с данным, образует векторное пространство и найдите его размерность\\
		\\
		\textbf{Решение}\\
		Так как $F = A \oplus B$, то рассмотрим $A$ и $B$ отдельно
		\begin{gather*}
			AC = CA\\
			\begin{pmatrix}
				7 & 9\\
				-4 & -5
			\end{pmatrix}
			\begin{pmatrix}
				a & b\\
				c & d
			\end{pmatrix}
			=
			\begin{pmatrix}
				a & b\\
				c & d
			\end{pmatrix}
			\begin{pmatrix}
				7 & 9\\
				-4 & -5
			\end{pmatrix}\\
			\begin{pmatrix}
				7a + 9c & 7b + 9d\\
				-4a - 5c & -4b -5d
			\end{pmatrix}
			=
			\begin{pmatrix}
			7a - 4b & 9a - 5b\\
			7c - 4d & 9c - 5d
			\end{pmatrix}\\
			\\
			\begin{cases}
				7a + 9c = 7a - 4b\\
				7b + 9d = 9a - 5b\\
				-4a - 5c = 7c - 4d\\
				-4b - 5d = 9c - 5d
			\end{cases}
			\qquad
			\begin{cases}
				9c = - 4b\\
				13b + 9d = 9a\\
				-4a = 13c - 4d\\
				-4b = 9c
			\end{cases}
			\\	
			\begin{cases}
				9c = -4b\\
				13b + 9d = 9a\\
				-4a = -4b + 4c - 4d
			\end{cases}
			\qquad
			\begin{cases}
				9c = -4b\\
				b + d = a + c
			\end{cases}
			\\
			C =
			\begin{pmatrix}
				-\frac{1}{4}(13c - 4d) & -\frac{1}{4}9c\\
				c & d
			\end{pmatrix}
			=
			\frac{1}{4}
			\begin{pmatrix}
				4d - 13c & -9c\\
				c & d
			\end{pmatrix}
		\end{gather*}
		Рассмотрим векторное пространство $X_1$, где $D \in X_1$ имеет вид $C$. Проверим, что $\alpha D + \beta E \in X_1$
		\begin{gather*}
			\alpha
			\frac{1}{4}
			\begin{pmatrix}
				4d_1 - 13c_1 & -9c_1\\
				c_1 & d_1
			\end{pmatrix}
			+ \beta
			\frac{1}{4}
			\begin{pmatrix}
				4d_1 - 13c_1 & -9c_1\\
				c_1 & d_1
			\end{pmatrix}
			= \\ =
			\frac{1}{4}
			\begin{pmatrix}
				\alpha(4d_1 - 13c_1) + \beta(4d_2 - 13c_2) & -9(\alpha c_1 + \beta c_2)\\
				\alpha c_1 + \beta c_2 & \alpha d_1 + \beta d_2
			\end{pmatrix}
		\end{gather*}
		Пусть
		\begin{gather*}
			\alpha c_1 + \beta c_2 = c^{\prime}\\
			\alpha d_1 + \beta d_2 = d^{\prime}
		\end{gather*}
		Тогда
		\begin{gather*}
			\alpha D + \beta E = \\ =
			\frac{1}{4}
			\begin{pmatrix}
			\alpha(4d_1 - 13c_1) + \beta(4d_2 - 13c_2) & -9(\alpha c_1 + \beta c_2)\\
			\alpha c_1 + \beta c_2 & \alpha d_1 + \beta d_2
			\end{pmatrix}
			= \\ =
			\frac{1}{4}
			\begin{pmatrix}
			4d^{\prime} - 13c^{\prime} & -9c^{\prime}\\
			c^{\prime} & d^{\prime}
			\end{pmatrix}
			\in X_1
		\end{gather*}
		Следовательно множество операторов, комутирующих с $A$ является векторным пространством\\
		\\
		Если матрицы в $B$ и $A^{\prime}$ имеют общую систему собстенных векторов, то $A^{\prime}B = B A^{\prime}$\\
		Так как $A^{\prime} = H D_1 H^{-1},\ B = H D_2 H^{-1}$ то 
		\begin{gather*}
			A^{\prime}B = HD_1H^{-1} HD_2H^{-1} = HD_1D_2H^{-1}\\
			BA^{\prime} = HD_2H^{-2} HD_1H^{-1} = HD_2D_1H^{-1}
		\end{gather*}
		Найдем собственные векторы $B$
		\begin{gather*}
			e_1 =
			\begin{pmatrix}
				1 \\ -1 \\ 1
			\end{pmatrix}\\
			(B - \lambda E)
			\begin{pmatrix}
				x_1 \\ x_2 \\ x_3
			\end{pmatrix}
			=
			\begin{pmatrix}
				1 \\ -1 \\ 1
			\end{pmatrix}
			\\
			\begin{pmatrix}
			-1 & -2 & -1 & | & 1\\
			1 & 1 & 0 & | & -1\\
			-1 & -1 & 0 & | & 1
			\end{pmatrix}
			\to
			\begin{pmatrix}
			0 & -1 & -1 & | & 0\\
			1 & 1 & 0 & | & -1
			\end{pmatrix}
			\\
		\end{gather*}
		 у $B$ только 1 собственный вектор
		
		\subsection*{\textbf{Задача 3}}
		\textbf{Условие}\\
		Вычислите след оператора $A^4 - 3A^3 - 2A^2 + A$, где $A$ -- оператор, заданный матрицей
		\begin{gather*}
			\left[\begin{array}{ccc}
				3 & 1 & -1 \\
				-5 & -1 & 3 \\
				2 & 1 & 0
			\end{array}\right]
		\end{gather*}
		\textbf{Решение}\\
		\begin{gather*}
			A^2 = 
			\begin{pmatrix}
				3 & 1 & -1\\
				-5 & -1 & 3\\
				2 & 1 & 0
			\end{pmatrix}
			\begin{pmatrix}
				3 & 1 & -1\\
				-5 & -1 & 3\\
				2 & 1 & 0
			\end{pmatrix}
			=
			\begin{pmatrix}
				2 & 3 & 0\\
				-4 & -7 & 2\\
				11 & 3 & -5
			\end{pmatrix}\\
			A^3 =
			\begin{pmatrix}
				2 & 3 & 0\\
				-4 & -7 & 2\\
				11 & 3 & -5
			\end{pmatrix}
			\begin{pmatrix}
				3 & 1 & -1\\
				-5 & -1 & 3\\
				2 & 1 & 0
			\end{pmatrix}
			= 
			\begin{pmatrix}
				-9 & -1 & 7\\
				27 & 1 & 17\\
				8 & 13 & -2
			\end{pmatrix}\\
			A^4 = 
			\begin{pmatrix}
				-9 & -1 & 7\\
				27 & 1 & 17\\
				8 & 13 & -2
			\end{pmatrix}
			\begin{pmatrix}
				3 & 1 & -1\\
				-5 & -1 & 3\\
				2 & 1 & 0
			\end{pmatrix}
			=
			\begin{pmatrix}
				-8 & -15 & 6\\
				42 & 43 & -24\\
				-45 & -3 & 31
			\end{pmatrix}\\
			A^4 - 3A^3 - 2A^2 + A=\\ 
			\begin{pmatrix}
				-8 & -15 & 6\\
				42 & 43 & -24\\
				-45 & -3 & 31
			\end{pmatrix}
			- 3
			\begin{pmatrix}
				-9 & -1 & 7\\
				27 & 1 & 17\\
				8 & 13 & -2
			\end{pmatrix}
			- 2
			\begin{pmatrix}
				2 & 3 & 0\\
				-4 & -7 & 2\\
				11 & 3 & -5
			\end{pmatrix}
			+
			\begin{pmatrix}
				3 & 1 & -1\\
				-5 & -1 & 3\\
				2 & 1 & 0
			\end{pmatrix}
			= \\ =
			\begin{pmatrix}
				18 & -17 & -16\\
				-36 & 53 & 26\\
				-89 & -47 & 47
			\end{pmatrix}
			\\
			\text{tr}
			\begin{pmatrix}
			18 & -17 & -16\\
			-36 & 53 & 26\\
			-89 & -47 & 47
			\end{pmatrix}
			= 
			18 + 53 + 47
			=
			118
		\end{gather*}
		\\
		
		\subsection*{\textbf{Задача 4}}
		\subsubsection*{\textbf{А}}
		\textbf{Условие}\\
		Приведите матрицу 
		\begin{gather*}
			B:=
			\left[\begin{array}{lll}
				1 & 2 & 2 \\
				4 & 3 & 4 \\
				4 & 4 & 3
			\end{array}\right]
		\end{gather*}
		над полем $F_5$ к нормальной фробениусовой форме
		\\
		\textbf{Решение}\\
		\begin{gather*}
			\begin{pmatrix}
				1 - \lambda & 2 & 2 \\
				4 & 3 - \lambda & 4 \\
				4 & 4 & 3 - \lambda 
			\end{pmatrix}
			=
			-\lambda^3 + 7\lambda^2 + 17\lambda + 9 \underset{5}{\equiv} 4\lambda^3 + 2\lambda^2 + 2\lambda + 4 = 2(\lambda + 1)(2\lambda^2 - \lambda + 2)
		\end{gather*}
		\\
		
		\subsubsection*{\textbf{Б}}
		\textbf{Условие}\\
		 Вычислите количество $B$-инвариантных подпространств в $F_5$
		\\
		\textbf{Решение}\\
		Так как различным собственным числам $\lambda_1 \ne \lambda_3$ соответствует $v_1 \ne v_3$, то количество инвариантных подпространств:\\
		$\{c_1V_1\}, \{c_3V_3\}, \{c_1V_1 + c_3V_3\}, \{B\}, \{0\}$\\
		То есть 5 
		\\
		
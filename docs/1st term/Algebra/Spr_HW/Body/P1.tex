
\newpage
	{\large \hspace{3cm} \begin{center} Домашнее задание 14 $\bullet$ Мозговой Владислав \end{center} }
	\vspace{-1.5ex}
	\hrulefill
	
	\fontsize{12pt}{4.5mm}\selectfont
	\vspace{-3ex}
	\hrulefill
	\newline

	\section{}
		\subsection*{Пример}
		\textbf{Условие}\\
		Рассмотрим циклические группы $C_2$ и $C_3$ поряков $2$ и $3$, состоящие из элементов $\{1, a\}$ ($a^2 = 1$) и $\{1, b, b^2\}$ ($b^3 = 1$) соответственно. Зададим их полупрямое произведения $3 \circ 2$, построенное по автоморфизму $\phi$ порядка $2$ группы $C_3$, которое переводит $b \mapsto b^{-1}$, как множество произведений $\{b^j a^i\: |\: 0 \leqslant j < 3,\: 0 \leqslant i < 1\}$ со следующим правилом умножения: 
		\begin{gather*}
			(b^{j} a^{i}) \cdot(b^{k} a^{l}):=b^{j}(a^{i} b^{k} a^{-i}) a^{i} a^{l}=b^{j} \varphi^{i}(b^{k}) a^{i+l}=b^{j+(-1)^{i} k} a^{i+l}
		\end{gather*}
		Покажите, что умножение ассоциативно и задает группу, изоморфную группе перестановок $S_3$\\
		\\
		\textbf{Решение}\\
		\begin{gather*}
			((b^{j} a^{i})(b^{k} a^{l}))(b^{p} a^{q}) = (b^{j + (-1)^{i}k} a^{i+l})(b^{p} a^{q}) = (b^{j + (-1)^{i}k + (-1)^{i+l}p} a^{i+l+q})\\
			(b^{j} a^{i})((b^{k} a^{l})(b^{p} a^{q})) = (b^{j} a^{i})(b^{k + (-1)^{l}p} a^{l + q}) = (b^{j + (-1)^{i}k + (-1)^{i}(-1)^{l}p} a^{i+l+q})
		\end{gather*}
		Рассмотрим 
		\begin{gather*}
			C_2 = \{1, a\} \simeq \{e, (12)\}\quad \text{вместо $(12)$ можно взять $(23)$ или $(31)$}\\
			C_3 = \{1, b, b^2\} \simeq \{e, (123), (132)\}\\
			S_3 = \{e, (12), (23), (31), (123), (132)\}			
		\end{gather*}
		Если $i = 0$ то $(b^{j} a^{0})(b^{k} a^{l}) = (b^{j + k} a^{l})$ что соответствует $(123)^{j} (123)^{k} (12)^{l} = (123)^{j+k} (12)^{l}$\\
		Если $i = 1$ то $(123)^{j} (12) (123)^{k} (12)^{l} = (123)^{j-k} (12)^{1 + l}$
		
		
		
		\subsection*{Задача}
			\subsubsection*{А}
			\textbf{Условие}\\
			Вычислите порядок числа 2 в группе обратимых элементов кольца $\mathbb{Z} \slash (631 \mathbb{Z})^{\times}$\\
			\\
			\textbf{Решение}\\
			Заметим, что $\text{ord}(2) = n,\ \phi(631) = 630,\ \frac{\phi(631)}{n} \in \mathbb{N}$\\
			Тогда $\text{ord}(2) = 2^{x_1} 3^{x_2} 5^{x_3} 7^{x_4},\ \text{ord}(2) \leqslant 630,\ x_1, x_2, x_3, x_4 \in \mathbb{N}_0$\\
			Тогда можно рассматривать $2$ в степенях вида $2^{x_1} 3^{x_2} 5^{x_3} 7^{x_4}$, и тогда $\text{ord}(2) = 45$, так как это наименьшая степень подобного вида, при которой $2^n = 1\ \text{mod}\: 631$
			
			
			\subsubsection*{Б}
			\textbf{Условие}\\
			Найдите автоморфизм порядка 7, у циклической группы порядка 631\\
			\\
			\textbf{Решение}\\
			\begin{gather*}
				F:\ \mathbb{Z}^{\times}_7 \hookrightarrow \text{Aut}(\mathbb{Z}^{\times}_{631})\\
				F^7: b \to b^{\alpha} \to b^{\alpha^2} \to \ldots \to b^{\alpha^7} = b\ \text{mod}(631)\ \Leftrightarrow\ \alpha^7 = 1\ \text{mod}(631)\\
				\alpha = 1,\: 21,\: 133,\: 269,\: 427,\: 441,\: 601
			\end{gather*}
			Автоморфизмы порядка 7: $b \to b^{\alpha},\ \alpha = 1,\: 21,\: 133,\: 269,\: 427,\: 441,\: 601$
			
			
			\subsubsection*{В}
			\textbf{Условие}\\
			Опишите какую-нибудь неабелеву группу $G$, построенную, как полупрямое произведение двух циклических групп порядков 7 и 631. Точнее опишите какое-нибудь правило умножения на множестве пар
			\begin{gather*}
				\{b^{j} a^{i}\: |\: 0 \leqslant j<631,\: 0 \leqslant i<7\}
			\end{gather*}
			которое задаст структуру группы\\
			\\
			\textbf{Решение}\\
			Зададим полупрямое произведение, построенное по автоморфизму из прошлого пункта задачи, тогда:
			\begin{gather*}
				(b^{j} a^{i})(b^{k} a^{l}) = b^{j}(a^{i}b^{k}a^{-i})a^{i}a^{l} = b^{j + (21)^{i}k} a^{i+l}
			\end{gather*} 
			Тогда заметим, что:
			\begin{enumerate}
				\item 
					\begin{gather*}
						((b^{j} a^{i})(b^{k} a^{l}))(b^{m} a^{n}) = (b^{j + (21)^{i}k} a^{i+l})(b^{m} a^{n}) = b^{j + (21)^{i}k + (21)^{i+l}m} a^{i+l+m}\\
						(b^{j} a^{i})((b^{k} a^{l})(b^{m} a^{n})) = (b^{j} a^{i})(b^{k + (21)^{l}m} a^{l+n}) = b^{j + (21)^{i}k + (21)^{i+l}m} a^{i+l+m}
					\end{gather*}
				\item 
					\begin{gather*}
						b^{0} a^{0} = 1
					\end{gather*}
				\item 
					\begin{gather*}
						(b^{j} a^{i})^{-1} = (b^{-j(21)^{7-i}} a^{7-i})\\
						(b^{j} a^{i})^{-1}(b^{-j(21)^{7-i}} a^{7-i}) = b^{j +(21)^{i}(-j)(21)^{7-i}} a^{7} = b^{j - j(21)^{7}} = b^{j-j} = 1
					\end{gather*}
			\end{enumerate}
			Откуда группа неабелева
			
			
			\subsubsection*{Г}
			\textbf{Условие}\\
			Вычислите порядок коммутанта группы $G$\\
			\\
			\textbf{Решение}\\
			$G = Z_{631} \rtimes Z_{7}$\\
			$a^{-1} b a = b^{21}$\\
			Коммутатор $[b,a] = b^{-1} a^{-1} ba = b^{20}$, то есть циклическая подгруппа $\langle b^{20} \rangle \in [G,G]$\\
			\begin{gather*}
				(b^{j} a^{i})((b^{20})^{n})(b^{-j(21)^{7-i}} a^{7-i}) = (b^{j + 20n} a^{i})(b^{-j(21)^{7-i}} a^{7-i}) = b^{j + 20n + (21)^{7}(-j)} a^{7} = b^{20n} \in \langle b^{20} \rangle
			\end{gather*}
			в $G\slash \langle b^{20} \rangle$: $[b,a]\langle b^{20} \rangle = b^{20}\langle b^{20} \rangle = \langle b^{20} \rangle$, откуда $[G,G] \subset \langle b^{20} \rangle \Leftrightarrow [G,G] = \langle b^{20} \rangle$\\
			Тогда порядок $[G,G] = \frac{631}{\text{gcd}(631, 20)} = 631$\\
			\\
			Коммутатор имеет вид:
			\begin{gather*}
				[b^{j} a^{i}, b^{k} a^{l}] = 
				(b^{j} a^{i})(b^{k} a^{l})((b^{k} a^{l})(b^{j} a^{i}))^{-1} =\\
				(b^{j + (21)^{i} k} a^{l+i})(b^{k+(21)^{l}j} a^{i+l})^{-1} = 
				(b^{j + (21)^{i}k} a^{i+l})(b^{(-k-(21)^{l}j)(21)^{7-i-j}} a^{7-i-l}) = \\
				b^{j + (21)^{i}k + (21)^{i+l}(-k-(21)^{l}j)(21)^{7-i-j}} a^{7} = 
				b^{j + (21)^{i}k + (21)^{7}(-k-(21)^{l}j)} = b^{j + (21)^{i}k - k - (21)^{l}j - j} a^{0}
			\end{gather*} 
			
			
			\subsubsection*{Д}
			\textbf{Условие}\\
			Сколько классов сопряженности в группе $G$ и каковы их порядки?\\
			\\
			\textbf{Решение}\\
			Рассмотрим класс сопряженности элеммента $b^{k} a^{l}$:
			\begin{gather*}
				(b^{i} a^{j})(b^{k} a^{l})(b^{-j(21)^{7-i}} a^{7-i}) = b^{j + (21)^{i}k - j(21)^{l} a^{l}} = b^{k} (b^{j + (21)^{i}k - (21)^{l}j - k} a^{0}) a^{l}
			\end{gather*}
			Тогда все классы сопряженности равномощны $[G,G]$, так как $|H| = |gH|\quad \forall g \in G$\\
			Откуда порядок всех классов сопряженности равен $631$ 
			
			
			\subsubsection*{Е}
			\textbf{Условие}\\
			Для каждого возможного порядка класса сопряженности опишите представителей в каком-нибудь одном классе с данным порядком\\
			\\
			\textbf{Решение}\\
			
			
\newpage
	{\large \hspace{3cm} \begin{center} Домашнее задание 22 $\bullet$ Мозговой Владислав \end{center} }
	\vspace{-1.5ex}
	\hrulefill
	
	\fontsize{12pt}{4.5mm}\selectfont
	\vspace{-3ex}
	\hrulefill
	\newline

	\section*{}
		\subsection*{\textbf{Задача 1}}
		\subsubsection*{\textbf{А}}
		\textbf{Условие}\\
		Проверьте,что 3 – является собственным значением матрицы
		\begin{gather*}
			A := 
			\begin{pmatrix}
				-2 & -2 & -3 \\
				-2 & -2 & 3 \\
				-3 & 3 & -3
			\end{pmatrix}
		\end{gather*}
		и найдите оставшиеся собственные значения этой матрицы.\\
		\\
		\textbf{Решение}\\
		\begin{gather*}
			\text{det}(A-\lambda E) = 
			\text{det}
			\begin{pmatrix}
				-2-\lambda & -2 & -3 \\
				-2 & -2-\lambda & 3 \\
				-3 & 3 & -3-\lambda
			\end{pmatrix}
			=\\=
			-\lambda^3 - 7\lambda^2 + 6\lambda + 72
			=
			-(\lambda - 3)(\lambda + 4)(\lambda + 6)\\
			\lambda_1 = 3,\ \lambda_2 = -4,\ \lambda_3 = -6
		\end{gather*}
		собственные значения $A$ это $3, -4, -6$
		
		\subsubsection*{\textbf{Б}}
		\textbf{Условие}\\
		Найдите собственные векторы матрицы A\\
		\\
		\textbf{Решение}\\
		\begin{enumerate}
		\item $\lambda = 3$
			\begin{gather*}
				(A-\lambda E)X = 
				(A-3E)X = 
				\begin{pmatrix}
					-2-3 & -2 & -3 \\
					-2 & -2-3 & 3 \\
					-3 & 3 & -3-3
				\end{pmatrix}
				\begin{pmatrix}
					x_1 \\ x_2 \\ x_3
				\end{pmatrix}
				=\\
				\begin{pmatrix}
					-5 & -2 & -3 \\
					-2 & -5 & 3 \\
					-3 & 3 & -6
				\end{pmatrix}
				\begin{pmatrix}
					x_1 \\ x_2 \\ x_3
				\end{pmatrix}
				=
				\begin{pmatrix}
					-5x_1 -2x_2 -3x_3 \\
					-2x_1 -5x_2 +3x_3 \\
					-3x_1 + 3x_2 -6x_3
				\end{pmatrix}
				\\
				\text{det}
				\begin{pmatrix}
					-5x_1 -2x_2 -3x_3 \\
					-2x_1 -5x_2 +3x_3 \\
					-3x_1 + 3x_2 -6x_3
				\end{pmatrix}
				= 0
				\\
				\alpha
				\begin{pmatrix}
					1\\ -1 \\ -1
				\end{pmatrix}
			\end{gather*}
		\item $\lambda = -4$
			\begin{gather*}
				(A-\lambda E)X = 
				(A+4E)X = 
				\begin{pmatrix}
					-2+4 & -2 & -3 \\
					-2 & -2+4 & 3 \\
					-3 & 3 & -3+4
				\end{pmatrix}
				\begin{pmatrix}
					x_1 \\ x_2 \\ x_3
				\end{pmatrix}
				=\\
				\begin{pmatrix}
					2 & -2 & -3 \\
					-2 & 2 & 3 \\
					-3 & 3 & 1
				\end{pmatrix}
				\begin{pmatrix}
					x_1 \\ x_2 \\ x_3
				\end{pmatrix}
				=
				\begin{pmatrix}
					2x_1 -2x_2 -3x_3 \\
					-2x_1 +2x_2 +3x_3 \\
					-3x_1 + 3x_2 +x_3
				\end{pmatrix}
				\\
				\text{det}
				\begin{pmatrix}
					2x_1 -2x_2 -3x_3 \\
					-2x_1 +2x_2 +3x_3 \\
					-3x_1 + 3x_2 +x_3
				\end{pmatrix}
				= 0
				\\
				\alpha
				\begin{pmatrix}
					1\\1\\0
				\end{pmatrix}
			\end{gather*}
		\item $\lambda = -6$
			\begin{gather*}
				(A-\lambda E)X = 
				(A+6E)X = 
				\begin{pmatrix}
					-2+6 & -2 & -3 \\
					-2 & -2+6 & 3 \\
					-3 & 3 & -3+6
				\end{pmatrix}
				\begin{pmatrix}
					x_1 \\ x_2 \\ x_3
				\end{pmatrix}
				=\\
				\begin{pmatrix}
					4 & -2 & -3 \\
					-2 & 4 & 3 \\
					-3 & 3 & 3
				\end{pmatrix}
				\begin{pmatrix}
					x_1 \\ x_2 \\ x_3
				\end{pmatrix}
				=
				\begin{pmatrix}
					4x_1 -2x_2 -3x_3 \\
					-2x_1 +4x_2 +3x_3 \\
					-3x_1 + 3x_2 +3x_3
				\end{pmatrix}
				\\
				\text{det}
				\begin{pmatrix}
					4x_1 -2x_2 -3x_3 \\
					-2x_1 +4x_2 +3x_3 \\
					-3x_1 + 3x_2 +3x_3
				\end{pmatrix}
				= 0
				\\
				\alpha
				\begin{pmatrix}
					1\\-1\\2
				\end{pmatrix}
			\end{gather*}
		\end{enumerate}
		
		
		\subsubsection*{\textbf{В}}
		\textbf{Условие}\\
		Выпишите ортогональную замену переменных, приводящую квадратичную форму
		\begin{gather*}
			-2 x^{2}-4 x y-6 x z-2 y^{2}+6 y z-3 z^{2}
		\end{gather*}
		к главным осям (т.е. к виду $\sum a_{i} x_{i}^{2}$ )\\
		\\
		\textbf{Решение}\\
		\begin{gather*}
			\begin{pmatrix}
				-2 & -2 & -3\\
				-2 & -2 & 3\\
				-3 & 3 & -3
			\end{pmatrix}\\
			e_1 = \frac{(1,-1,-1)}{|(1,-1,-1)|} = \frac{(1,-1,-1)}{\sqrt{3}} = (\frac{1}{\sqrt{3}},-\frac{1}{\sqrt{3}},-\frac{1}{\sqrt{3}})\\
			e_2 = \frac{(1,1,0)}{|(1,1,0)|} = \frac{(1,1,0)}{\sqrt{2}} = (\frac{1}{\sqrt{2}},-\frac{1}{\sqrt{3}},0)\\
			e_3 = \frac{(1,-1,2)}{|(1,-1,2)|} = \frac{(1,-1,2)}{\sqrt{6}} = (\frac{1}{\sqrt{6}},-\frac{1}{\sqrt{6}},\frac{2}{\sqrt{6}})\\
			\\
			\begin{pmatrix}
				\frac{1}{\sqrt{3}} & \frac{1}{\sqrt{2}} & \frac{1}{\sqrt{6}}\\
				\\
				\frac{-1}{\sqrt{3}} & \frac{1}{\sqrt{2}} & \frac{-1}{\sqrt{6}}\\
				\\
				\frac{-1}{\sqrt{3}} & \frac{0}{\sqrt{2}} & \frac{2}{\sqrt{6}}
			\end{pmatrix}
			=
			\begin{pmatrix}
				\frac{1}{\sqrt{3}} & \frac{1}{\sqrt{2}} & \frac{1}{\sqrt{6}}\\
				\\
				-\frac{1}{\sqrt{3}} & \frac{1}{\sqrt{2}} & -\frac{1}{\sqrt{6}}\\
				\\
				-\frac{1}{\sqrt{3}} & 0 & \frac{2}{\sqrt{6}}
			\end{pmatrix}
		\end{gather*}
		
		
		\subsection*{\textbf{Задача 2}}
		\subsubsection*{\textbf{А}}
		\textbf{Условие}\\
		Предьявите базис в $\mathbb{R}^3$, в котором пара квадратичных форм
		\begin{gather*}
			Q_{1}(x, y, z):=x(x+2 z)+y(y-z)+z(2 x-y+6 z)\\
			Q_{2}(x, y, z):=-7 x^{2}-10 x y-28 x z+17 y^{2}-92 y z+44 z^{2}
		\end{gather*}
		одновременно приводятся к сумме квадратов $a_ix^2 + b_iy^2 + c_iz^2$ ($i = 1; 2$). Выпишите получившиеся формы в новом базисе.\\
		\\
		\textbf{Решение}\\
		\begin{gather*}
			Q_1:\\
			\begin{pmatrix}
				1 & 0 & 2\\
				0 & 1 & -1\\
				2 & -1 & 6
			\end{pmatrix}
			\\
			Q_2:\\
			\begin{pmatrix}
				-7 & -5 & -14\\
				-5 & 17 & -46\\
				-14 & -46 & 44
			\end{pmatrix}\\
			\\
			\text{det}(Q_2 - \lambda Q_1) = 0\\
			\text{det}
			\Bigg(
			\begin{pmatrix}
				-7 & -5 & -14\\
				-5 & 17 & -46\\
				-14 & -46 & 44
			\end{pmatrix}
			-
			\lambda
			\begin{pmatrix}
				1 & 0 & 2\\
				0 & 1 & -1\\
				2 & -1 & 6
			\end{pmatrix}
			\Bigg)
			=
			\text{det}
			\begin{pmatrix}
				-7-\lambda & -5 & -14-2\lambda\\
				-5 & 17-\lambda & -46+\lambda\\
				-14-2\lambda & -46+\lambda & 44-6\lambda
			\end{pmatrix}
			=\\
			= -\lambda^3 + 27\lambda^2 + 360\lambda - 1296 = -(\lambda-3)(\lambda+12)(\lambda-36)\\
			\lambda_1 = 3\\
			\lambda_2 = -12\\
			\lambda_3 = 36\\		
		\end{gather*}
		\begin{enumerate}
		\item $\lambda_1 = 3$
			\begin{gather*}
				\begin{pmatrix}
					-7-\lambda & -5 & -14-2\lambda\\
					-5 & 17-\lambda & -46+\lambda\\
					-14-2\lambda & -46+\lambda & 44-6\lambda
				\end{pmatrix}	
				\begin{pmatrix}
					x_1 \\ x_2 \\ x_3
				\end{pmatrix}
				=
				\begin{pmatrix}
					-7-3 & -5 & -14-6\\
					-5 & 17-3 & -46+3\\
					-14-6 & -46+3 & 44-18
				\end{pmatrix}	
				\begin{pmatrix}
					x_1 \\ x_2 \\ x_3
				\end{pmatrix}
				=\\
				\begin{pmatrix}
					-10 & -5 & -20\\
					-5 & 14 & -43\\
					-20 & -43 & 26
				\end{pmatrix}	
				\begin{pmatrix}
					x_1 \\ x_2 \\ x_3
				\end{pmatrix}
				=
				\begin{pmatrix}
					-10x_1 -5x_2  -20x_3\\
					-5x_1 14x_2  -43x_3\\
					-20x_1  -43x_2 +26x_3
				\end{pmatrix}
				=
				\begin{pmatrix}
					0 \\ 0 \\ 0
				\end{pmatrix}\\
				\\
				\begin{pmatrix}
					x_1 \\ x_2 \\ x_3
				\end{pmatrix}
				=
				\begin{pmatrix}
					3 \\ -2 \\ -1
				\end{pmatrix}
			\end{gather*}
		\item $\lambda_2 = -12$
			\begin{gather*}
				\begin{pmatrix}
					-7-\lambda & -5 & -14-2\lambda\\
					-5 & 17-\lambda & -46+\lambda\\
					-14-2\lambda & -46+\lambda & 44-6\lambda
				\end{pmatrix}	
				\begin{pmatrix}
					x_1 \\ x_2 \\ x_3
				\end{pmatrix}
				=
				\begin{pmatrix}
					-7+12 & -5 & -14+24\\
					-5 & 17+12 & -46-12\\
					-14+24 & -46-12 & 44+72
				\end{pmatrix}	
				\begin{pmatrix}
					x_1 \\ x_2 \\ x_3
				\end{pmatrix}
				=\\
				\begin{pmatrix}
					5 & -5 & 10\\
					-5 & 29 & -58\\
					10 & -58 & 116
				\end{pmatrix}	
				\begin{pmatrix}
					x_1 \\ x_2 \\ x_3
				\end{pmatrix}
				=
				\begin{pmatrix}
					5x_1 -5x_2  +10x_3\\
					-5x_1 +29x_2  -58x_3\\
					10x_1  -58x_2 +116x_3
				\end{pmatrix}
				=
				\begin{pmatrix}
					0 \\ 0 \\ 0
				\end{pmatrix}\\
				\\
				\begin{pmatrix}
					x_1 \\ x_2 \\ x_3
				\end{pmatrix}
				=
				\begin{pmatrix}
					0 \\ 2 \\ 1
				\end{pmatrix}
			\end{gather*}
		\item $\lambda_3 = 36$
			\begin{gather*}
				\begin{pmatrix}
					-7-\lambda & -5 & -14-2\lambda\\
					-5 & 17-\lambda & -46+\lambda\\
					-14-2\lambda & -46+\lambda & 44-6\lambda
				\end{pmatrix}	
				\begin{pmatrix}
					x_1 \\ x_2 \\ x_3
				\end{pmatrix}
				=
				\begin{pmatrix}
					-7-36 & -5 & -14-72\\
					-5 & 17-36 & -46+36\\
					-14-72 & -46+36 & 44-216
				\end{pmatrix}	
				\begin{pmatrix}
					x_1 \\ x_2 \\ x_3
				\end{pmatrix}
				=\\
				\begin{pmatrix}
					-43 & -5 & -86\\
					-5 & -19 & -10\\
					-86 & -10 & -172
				\end{pmatrix}	
				\begin{pmatrix}
					x_1 \\ x_2 \\ x_3
				\end{pmatrix}
				=
				\begin{pmatrix}
					0 \\ 0 \\ 0
				\end{pmatrix}
				\\
				\begin{pmatrix}
				x_1 \\ x_2 \\ x_3
				\end{pmatrix}
				=
				\begin{pmatrix}
				2 \\ 0 \\ -1
				\end{pmatrix}
			\end{gather*}
		\end{enumerate}
		\begin{gather*}
			\begin{pmatrix}
				\frac{3}{\sqrt{3^2 + 2^2 + 1^2}} & \frac{0}{\sqrt{2^2 + 1^2}} & \frac{2}{\sqrt{2^2 + 1^2}}\\
				\frac{-2}{\sqrt{3^2 + 2^2 + 1^2}} & \frac{2}{\sqrt{2^2 + 1^2}} & \frac{0}{\sqrt{2^2 + 1^2}}\\
				\frac{-1}{\sqrt{3^2 + 2^2 + 1^2}} & \frac{1}{\sqrt{2^2 + 1^2}} & \frac{-1}{\sqrt{2^2 + 1^2}}
			\end{pmatrix}
			=
			\begin{pmatrix}
				\frac{3}{\sqrt{14}} & 0 & \frac{2}{\sqrt{5}}\\
				-\frac{2}{\sqrt{14}} & \frac{2}{\sqrt{5}} & 0\\
				-\frac{1}{\sqrt{14}} & \frac{1}{\sqrt{5}} & -\frac{1}{\sqrt{5}}
			\end{pmatrix}
		\end{gather*}
	
	
		\begin{comment}
			-43a -5b -86c, -5a -19b -10c, -86a -10b -172c
		\begin{gather*}
			Q_{1}(x, y, z):=x(x+2 z)+y(y-z)+z(2 x-y+6 z) = \\
			(x+2z)^2 + (y-z)^2 + z^2\\
			x' = x + 2z\\
			y' = y-z\\
			z' = z
			\\
			Q_{2}(x', y', z'):=-7 x'^{2}-10 x' y'-28 x' z'+17 y'^{2}-92 y' z'+44 z'^{2} = \\
			-7 (x + 2z)^{2}-10 (x + 2z) (y-z)-28 (x + 2z) z+17 (y-z)^{2}-92 (y-z) z+44 z^{2} =\\
			-7x^2 - 10xy - 46 xz + 17y^2 - 146 yz + 89z^2\\
			\begin{pmatrix}
				-7 & -5 & -23\\
				-5 & 17 & -73\\
				-23 & -73 & 89
			\end{pmatrix}\\
			\text{det}
			\begin{pmatrix}
			-7-\lambda & -5 & -23\\
			-5 & 17-\lambda & -73\\
			-23 & -73 & 89-\lambda
			\end{pmatrix}
			=
			-\lambda^3 + 99\lambda^2 + 5112\lambda - 1296 =\\
			-(\lambda-33)^3 + 8379(\lambda - 33) + 239274 = 0\\
			(\lambda-33)^3 - 8379(\lambda - 33) - 239274 = 0\\
			\lambda -33 = x + \frac{c}{x}\\
			(x + \frac{c}{x})^3 - 8379(x + \frac{c}{x}) - 239274 = 0\\
			x^3 = 21(5697 + 8i\sqrt{264837})\\
			x_1 = 21^{\frac{1}{3}}(5697 + 8i\sqrt{264837})^{\frac{1}{3}} \approx -70.659\\
			x_2 = -1 \cdot -21^{\frac{1}{3}}(5697 + 8i\sqrt{264837})^{\frac{1}{3}} \approx -32.748\\
 			x_3 = (-1)^{\frac{2}{3}}21^{\frac{2}{3}}(5697 + 8i\sqrt{264837})^{\frac{2}{3}} \approx 103.41\\
 			\\
 			\lambda_1 \approx -37.659\\
 			\lambda_2 \approx 0.252\\
 			\lambda_3 \approx 136.41\\
		\end{gather*}
		Собственные вектора
		\begin{enumerate}
		\item $\lambda_1 = -37.659$
			\begin{gather*}
				\begin{pmatrix}
					-7-\lambda_1 & -5 & -23\\
					-5 & 17-\lambda_1 & -73\\
					-23 & -73 & 89-\lambda_1
				\end{pmatrix}
				=
				\begin{pmatrix}
					1 & 0 & -0.983\\
					0 & 1 & -1.425\\
					0 & 0 & 0
				\end{pmatrix}
				\\
				\begin{pmatrix}
					0.983 \\ 1.425 \\ 1
				\end{pmatrix}
			\end{gather*}
		\item $\lambda_2 = 0.252$
			\begin{gather*}
				\begin{pmatrix}
					-7-\lambda_2 & -5 & -23\\
					-5 & 17-\lambda_2 & -73\\
					-23 & -73 & 89-\lambda_2
				\end{pmatrix}
				=
				\begin{pmatrix}
					1 & 0 & 5.122\\
					0 & 1 & -2.83\\
					0 & 0 & 0
				\end{pmatrix}
				\\
				\begin{pmatrix}
					-5.122 \\ 2.83 \\ 1
				\end{pmatrix}
			\end{gather*}
		\item $\lambda_3 = 136.41$
			\begin{gather*}
				\begin{pmatrix}
					-7-\lambda_3 & -5 & -23\\
					-5 & 17-\lambda_3 & -73\\
					-23 & -73 & 89-\lambda_3
				\end{pmatrix}
				=
				\begin{pmatrix}
					1 & 0 & 0.139\\
					0 & 1 & 0.606\\
					0 & 0 & 0 
				\end{pmatrix}
				\\
				\begin{pmatrix}
					-0.139 \\ -0.606 \\ 1
				\end{pmatrix}
			\end{gather*}
		\end{enumerate}	
		Тогда матрица перехода имеет вид
		\begin{gather*}
			P = 
			\begin{pmatrix}
				\frac{0.983}{\sqrt{0.983^{2} + 1.425^{2} + 1}} & -\frac{5.122}{\sqrt{5.122^{2} + 2.83^{2} + 1}} & -\frac{0.139}{\sqrt{0.139^{2} + 0.606^{2} + 1}}\\
				\frac{1.425}{\sqrt{0.983^{2} + 1.425^{2} + 1}} & \frac{2.83}{\sqrt{5.122^{2} + 2.83^{2} + 1}} & -\frac{0.606}{\sqrt{0.139^{2} + 0.606^{2} + 1}}\\
				\frac{1}{\sqrt{0.983^{2} + 1.425^{2} + 1}} & \frac{1}{\sqrt{5.122^{2} + 2.83^{2} + 1}} & \frac{1}{\sqrt{0.139^{2} + 0.606^{2} + 1}}
			\end{pmatrix}
		\end{gather*}
		И формы имеют вид:
		\begin{gather*}
			Q'_1 = P Q_1 P^{T}\\
			Q'_2 = P Q_2 P^{T}
		\end{gather*}
		\end{comment}
	
		
		\subsubsection*{\textbf{Б}}
		\textbf{Условие}\\
		Вычислите корни кубического многочлена
		\begin{gather*}
			f(t):=
			\text{det}
			\left(t
			\begin{bmatrix}
				1 & 0 & 2 \\
				0 & 1 & -1 \\
				2 & -1 & 6
			\end{bmatrix}
			-
			\begin{bmatrix}
				-7 & -5 & -14 \\
				-5 & 17 & -46 \\
				-14 & -46 & 44
			\end{bmatrix}
			\right)
		\end{gather*}
		\\
		\textbf{Решение}\\
		\begin{gather*}
			f(t):=
			\text{det}
			\left(t
			\begin{bmatrix}
				1 & 0 & 2 \\
				0 & 1 & -1 \\
				2 & -1 & 6
			\end{bmatrix}
			-
			\begin{bmatrix}
				-7 & -5 & -14 \\
				-5 & 17 & -46 \\
				-14 & -46 & 44
			\end{bmatrix}
			\right)
			=\\
			\text{det}
			\begin{pmatrix}
				t-7 & -5 & 2t-14 \\
				-5 & t+17 & -t-46 \\
				2t-14 & -t-46 & 6t+44
			\end{pmatrix}
			= \\
			t^3 + 27t^2 - 360t - 1296
			=
			(t + 3)(t - 12)(t + 36)\\
			t_1 = -3,\ t_2 = 12,\ t_3 = -36			
		\end{gather*}
		
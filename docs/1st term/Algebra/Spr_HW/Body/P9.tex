\newpage
{\large \hspace{3cm} \begin{center} Домашнее задание 23 $\bullet$ Мозговой Владислав \end{center} }
\vspace{-1.5ex}
\hrulefill
	
\fontsize{12pt}{4.5mm}\selectfont
\vspace{-3ex}
\hrulefill
\newline

\section*{}
	\subsection*{\textbf{Задача 1}}
	\textbf{Условие}\\
	Дана операция
	$
	A:=
	\begin{bmatrix}
		0 & -1 & 0 & 0 & 0 \\
		1 & -1 & 0 & 0 & 0 \\
		0 & 0 & 0 & 0 & -1 \\
		0 & 0 & 1 & 0 & 2 \\
		0 & 0 & 0 & 1 & 0
	\end{bmatrix}
	$
	Найдите $\text{tr}(A^{-2})$\\
	\\
	\textbf{Решение}
	\begin{gather*}
		\begin{bmatrix}
			0 & -1 & 0 & 0 & 0 \\
			1 & -1 & 0 & 0 & 0 \\
			0 & 0 & 0 & 0 & -1 \\
			0 & 0 & 1 & 0 & 2 \\
			0 & 0 & 0 & 1 & 0
		\end{bmatrix}
		^{-1}
		=
		\frac{1}{\text{det}(A)} A^{T}_{+}
		=
		-1 \cdot A^{T}_{+}
		=
		-A^{T}_{+}
		=
		\begin{bmatrix}
			-1 & 1 & 0 & 0 & 0 \\
			-1 & 0 & 0 & 0 & 0 \\
			0 & 0 & 2 & 1 & 0 \\
			0 & 0 & 0 & 0 & 1 \\
			0 & 0 & -1 & 0 & 0
		\end{bmatrix}
		\\
		\\
		\begin{bmatrix}
			0 & -1 & 0 & 0 & 0 \\
			1 & -1 & 0 & 0 & 0 \\
			0 & 0 & 0 & 0 & -1 \\
			0 & 0 & 1 & 0 & 2 \\
			0 & 0 & 0 & 1 & 0
		\end{bmatrix}
		^{-2}
		=
		\begin{bmatrix}
			-1 & 1 & 0 & 0 & 0 \\
			-1 & 0 & 0 & 0 & 0 \\
			0 & 0 & 2 & 1 & 0 \\
			0 & 0 & 0 & 0 & 1 \\
			0 & 0 & -1 & 0 & 0
		\end{bmatrix}
		^{2}
		=
		\begin{bmatrix}
			0 & -1 & 0 & 0 & 0 \\
			1 & -1 & 0 & 0 & 0 \\
			0 & 0 & 4 & 2 & 1 \\
			0 & 0 & -1 & 0 & 0 \\
			0 & 0 & -2 & -1 & 0
		\end{bmatrix}
		\\
		\\
		\text{tr}(A^{-2}) = (-1)^{2} + 4^{2} = 1 + 16 = 17
	\end{gather*}
	
	\newpage
	\subsection*{\textbf{Задача 2}}
	Известно, что суммы $i$-тых степеней корней многочлена $f(x)$ третьей степени $3,3$ и $3$ для $i = 1,2,4$ соответственно
	\subsubsection*{\textbf{А}}
	\textbf{Условие}\\
	Найдите $f(x)$\\
	\\
	\textbf{Решение}\\
	\begin{gather*}
		f(x) = (x - x_1)(x - x_2)(x - x_3)\\
		x_1 + x_2 + x_3 = 3\\
		x_1^{2} + x_2^{2} + x_3^{2} = (x_1 + x_2 + x_3)^2 - 2(x_1 x_2 + x_1 x_3 + x_2 x_3) = 3\\
		x_1^{4} + x_2^{4} + x_3^{4} = (x_1 + x_2 + x_3)^4 - 6(x_1 x_2 + x_1 x_3 + x_2 x_3)^{2}\\
		- 4(x_1 x_2 + x_1 x_3 + x_2 x_3)(x_1^2 + x_2^2 + x_2^3) + 4(x_1 + x_2 + x_3)x_1x_2x_3 = 3\\
		\\
		3^2 - 2(x_1 x_2 + x_1 x_3 + x_2 x_3) = 3\\
		x_1 x_2 + x_1 x_3 + x_2 x_3 = 3\\
		\\
		(x_1 + x_2 + x_3)^4 - 6(x_1 x_2 + x_1 x_3 + x_2 x_3)^{2}\\
		- 4(x_1 x_2 + x_1 x_3 + x_2 x_3)(x_1^2 + x_2^2 + x_2^3) + 4(x_1 + x_2 + x_3)x_1x_2x_3 = 3\\
		(x_1 + x_2 + x_3)^4 - 6(x_1 x_2 + x_1 x_3 + x_2 x_3)^{2}\\
		- 4(x_1 x_2 + x_1 x_3 + x_2 x_3)((x_1 + x_2 + x_3)^2 - 2(x_1 x_2 + x_1 x_3 + x_2 x_3)) + 4(x_1 + x_2 + x_3)x_1x_2x_3 = 3\\
		3^4 - 6 \cdot 3^{2} - 4 \cdot 3 \cdot 3 + 4 \cdot 3  \cdot x_1x_2x_3 = 3\\
		81 - 54 - 36 + 12 x_1x_2x_3 = 3\\
		12 x_1x_2x_3 = 12\\
		x_1x_2x_3 = 1\\
		\\
		f(x) = x^3 - 3x^2 + 3x - 1 = (x-1)^{3}
	\end{gather*}
	
	\subsubsection*{\textbf{Б}}
	\textbf{Условие}\\
	Найдите суммы 5-ых степеней корней многочлена $f(x)$\\
	\\
	\textbf{Решение}\\
	\begin{gather*}
		f(x) = x^3 - 3x^2 + 3x - 1 = (x-1)^{3}\\
		x_1^5 + x_2^5 + x_3^5 = (x_1 + x_2 + x_3)^5 - 5(x_1^3 + x_2^3 + x_3^3)(x_1x_2 + x_1x_3 + x_2x_3)\\
		- 20x_1x_2x_3(x_1x_2 + x_1x_3 + x_2x_3) - 15(x_1^2 + x_2^2 + x_3^2)x_1x_2x_3 -10(x_1 + x_2 + x_3)(x_1^2 x_2^2 + x_1^2 x_3^2 + x_2^2 x_3^2)\\
		\\
		x_1^5 + x_2^5 + x_3^5 =\\
		(x_1 + x_2 + x_3)^5 - 5((x_1 + x_2 + x_3)^3 - 3(x_1+x_2+x_3)(x_1x_2+x_1x_3+x_2x_3) + 3x_1x_2x_3) \cdot \\
		(x_1x_2 + x_1x_3 + x_2x_3) - 20x_1x_2x_3(x_1x_2 + x_1x_3 + x_2x_3)\\
		- 15((x_1 + x_2 + x_3)^2 - 2(x_1 x_2 + x_1 x_3 + x_2 x_3))x_1x_2x_3\\
		- 10(x_1 + x_2 + x_3)((x_1x_2 + x_1x_3 + x_2x_3)^2 - 2x_1x_2x_3(x_1+x_2+x_3))\\
		\\
		1^5 + 1^5 + 1^5 = 3
	\end{gather*}
	
	\newpage
	\subsection*{\textbf{Задача 3}}
	\subsubsection*{\textbf{А}}
	\textbf{Условие}\\
	Пусть многочлены $A(x), B(x)$ -- многочлены с коэффициентами в поле, со страшим коэффициентом 1, и пусть $Q(x)$ -- остаток от деления $B(x)$ на $A(X)$. Покажите что результанты многочленов $A(x)$ и $B(x)$ и многочленов $A(x)$ и $Q(x)$ совпадают.\\
	\\
	\textbf{Решение}\\
	\begin{gather*}
		A(x) = x^{n} + a_{n-1}x^{n-1} + \ldots + a_1\\
		B(x) = x^{m} + b_{m-1}x^{m-1} + \ldots + b_1\\
		Q(x) = x^{k} + q_{k-1}x^{k-1} + \ldots + q_1\\
		B = AS + Q\quad S \text{ -- какой-то многочлен}\\
		S(x) = x^{l} + s_{l-1}x^{l-1} + \ldots + a_1\\
		\\
		AS = S_0A + S_1xA + \ldots + x^{l} A
	\end{gather*}
	При умножении $A$ на $x_i$ коэффициенты будут смещаться на $i$ так как
	\begin{gather*}
		x^{i} A = x^{n+i} + a_{n-1+i} x^{n-1+i} + \ldots + a_{0 + i} x^{i}
	\end{gather*}
	Следовательно домножение $A$ на $x^{i}$ происходит элементарное преобразование строк, а следовательно дискриминант не изменяется.\\
	Откуда следует, что так как $-SA + B$ -- результат элементарных преобразований, то
	\begin{gather*}
		R(A,B) = R(-SA+B, B) = R(Q,B)
	\end{gather*}
	
	\newpage
	\subsubsection*{\textbf{Б}}
	\textbf{Условие}\\
	Вычислите дискриминант многочлена $x^8 + ax + b$ и выясните при каких $a,b \in \mathbb{F}_{7}$ многочлен $x^{8} + ax + b$ делится на квадрат неприводимого многочлена над $\mathbb{F}_{7}$?\\
	\\
	\textbf{Решение}
	\begin{gather*}
		f'(x) = 8x^7 + a\\
		D = a_{0}^{2n-2} \prod_{i<j}(a_i - a_j)^{2}\\
		R(f,f')= (-1)^{\frac{n(n-1)}{2}}a_{0}D\\
		\\
		R(f,f') = 
		\left|
		\begin{array}{ccccccccccccccc}
			1 & 0 & 0 & 0 & 0 & 0 & 0 & a & b & 0 & 0 & 0 & 0 & 0 & 0 \\
			0 & 1 & 0 & 0 & 0 & 0 & 0 & 0 & a & b & 0 & 0 & 0 & 0 & 0 \\
			0 & 0 & 1 & 0 & 0 & 0 & 0 & 0 & 0 & a & b & 0 & 0 & 0 & 0 \\
			0 & 0 & 0 & 1 & 0 & 0 & 0 & 0 & 0 & 0 & a & b & 0 & 0 & 0 \\
			0 & 0 & 0 & 0 & 1 & 0 & 0 & 0 & 0 & 0 & 0 & a & b & 0 & 0 \\
			0 & 0 & 0 & 0 & 0 & 1 & 0 & 0 & 0 & 0 & 0 & 0 & a & b & 0 \\
			0 & 0 & 0 & 0 & 0 & 0 & 1 & 0 & 0 & 0 & 0 & 0 & 0 & a & b \\
			8 & 0 & 0 & 0 & 0 & 0 & 0 & a & 0 & 0 & 0 & 0 & 0 & 0 & 0 \\
			0 & 8 & 0 & 0 & 0 & 0 & 0 & 0 & a & 0 & 0 & 0 & 0 & 0 & 0 \\
			0 & 0 & 8 & 0 & 0 & 0 & 0 & 0 & 0 & a & 0 & 0 & 0 & 0 & 0 \\
			0 & 0 & 0 & 8 & 0 & 0 & 0 & 0 & 0 & 0 & a & 0 & 0 & 0 & 0 \\
			0 & 0 & 0 & 0 & 8 & 0 & 0 & 0 & 0 & 0 & 0 & a & 0 & 0 & 0 \\
			0 & 0 & 0 & 0 & 0 & 8 & 0 & 0 & 0 & 0 & 0 & 0 & a & 0 & 0 \\
			0 & 0 & 0 & 0 & 0 & 0 & 8 & 0 & 0 & 0 & 0 & 0 & 0 & a & 0 \\
			0 & 0 & 0 & 0 & 0 & 0 & 0 & 8 & 0 & 0 & 0 & 0 & 0 & 0 & a
		\end{array}
		\right|
		=
		8^{8}b^7 - 7^{7} a^8\\
		\\
		D = \frac{R(f,f')}{(-1)^{\frac{n(n-1)}{2}} a_0} = \frac{8^{8}b^7 - 7^{7} a^8}{(-1)^{\frac{8(8-1)}{2}} \cdot 1} = 8^{8}b^7 - 7^{7} a^8
	\end{gather*}
	Рассмотрим неприводимый многочлен $A = x^n + a_{n-1} x^n + \ldots + a_{0},\ \forall i:\ a_i \in \mathbb{F}_7$. По основной теореме алгебры он имеет минимум один корень над $\mathbb{C}$, тогда $A = \prod_{i = 1} (x - x_i)^{k_i},\ A^2 = \prod_{i = 1} (x - x_i)^{2k_i}$, то есть имеет кратные корни. Следовательно если $x^8 + ax + b$ делится на $A^2$, то $D(A) = 0$.
	$D(A) = 8^8b^7 - 7^7a^8 \equiv 6b^7 = 0$, а следовательно $b = 0$, $a$ -- любое.\\

	
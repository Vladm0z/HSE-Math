\newpage		
	\section{Задачи для подготовки к экзамену}
		
		\subsection{}
		$\sigma = \sigma_{\rho_{1}} \circ \ldots \circ \sigma_{\rho_{l}}$
 		\begin{enumerate}
			\item Знак перестановки\\
				\\
				Заметим, что любой цикл длины $a$ можо разложить в $a-1$ цикл длины 2. Так как знак перестановки определяется количеством транспозиций(то есть циклов длины 2), то он равен $(-1)^{(\rho_{1} - 1) \cdot \ldots \cdot (\rho_{l} - 1)}$
				
			\item Порядок перестановки\\
				\\
				Заметим, что порядок перестановки это НОК длин независимых циклов, тогда он равен $\text{gcd}(\rho_{1}, \ldots , \rho_{l})$\\
				Докажем это утверждение:\\
				Заметим, что для того, чтобы при перестановка при возведении в степень перешла сама в себя, необходимо и достаточно, чтобы любой ее элемент перешел сам в себя, что равносильно тому, что цикл, в который он входит, пройден целое число раз (если пройден не целое, то элемент не перейдет сам в себя).\\
				Так как при умножении на себя каждый цикл сдвигается на 1, то для того, чтобы перестановка перешла сама в себя, необходимо и достаточно, чтобы степень перестановки делилась на все длины циклов.\\ Минимальным таким числом является НОК длин циклов.
			
			\item Порядок класса сопряженности\\
				\\
			
			
			\item Количесвто перестановок, коммутирующих с $\sigma$\\
				\\
			
		\end{enumerate}
		
		\subsection{}
		Докажем по индукции\\
		\\
		База:\\
		2 вершины, очевидна\\
		Переход:\\
		Выполнен для всех деревьев на $\leqslant n-1$ вершине, доказать что выполнено и на дереве из $n$ вершин.\\
		Рассмотрим дерево на $n$ вершинах и выкинем любую из висящих вершин, для полученного дерева на $n-1$ вершине $(a_1 b_1)\ldots(a_{n-2} b_{n-2})$ -- цикл длины $n$. Вернем выброшенную вершину, он соединена ребром $(a_{n-1} b_{n-1})$ с одной из вершин (пусть она имеет номер $i$), тогда заметим, что ессли рассмотреть перестановку $(a_1 b_1)\ldots(a_{n-1} b_{n-1})$, то вершина, попавшая в $i$ после $(a_1 b_1)\ldots(a_{n-2} b_{n-2})$, попадает в вершину $n$ (так как $(a_{n-1} b_{n-1}) = (i\ n)$), а выходит оттуда только после следующей перестановки (так как $n$ висячая вершина). Тогда в цикл добавилась еще одна вершины и он теперь проходит по всем $n$ вершинам $arrow$ его длина также увеличилась на 1 и стала равна $n$.
			
		\subsection{}
		\begin{enumerate}
			\item Это группы $\mathbb{Z} \slash 4 \mathbb{Z}$ и $\mathbb{Z} \slash 2 \mathbb{Z} \times \mathbb{Z} \slash 2 \mathbb{Z}$\\
			Пусть есть группа $A$ порядка 4, каждый её элемент имеет порядок 1, 2 или 4. Если в $A$ есть элемент порядка 4, то она циклическая, то есть $A \simeq \mathbb{Z} \slash 4 \mathbb{Z}$, тогда все группы порядка 4 изоморфны.\\
			Пусть в группе $A$ нет элемента порядка, тогда все элементы $A$ имеют порядок 2, то есть $\forall a \in A:\ a^2 = e$. Рассмотрим любые два элемента: $a_1^{2} = a_2^{2} = e$ и $(a_1 a_2)^{2} = e$. Тогда $a_1 a_2 = (a_1 a_2)^{-1} = a_2^{-1} a_2^{-1} = a_2 a_1$, то есть $a_1$ и $a_2$ коммутируют. Тогда любая группа, где все элементы(кроме $e$) имеют порядок 2, является абелевой.\\
			
			\item Это группы $\mathbb{Z} \slash 6 \mathbb{Z}$ и $S_3$\\
			Если существует одна подгруппа порядка $2$, то она нормальна, и любой элемент порядка $3$ коммутирует с одним элементом порядка $2$. Так мы получаем циклическую группу.\\
			Так как существует пять неидентичных элементов и пара элементов порядка $3$, то всего есть два элемента порядка $3$ и три элемента порядка $2$, каждый из которых генерирует подгруппу порядка $2$.\\		
			Группа действует транзитивно и сопрягая эти три подгруппы порядка $2$, что дает изоморфизм с $S_3$.\\
			
			\item Это группы $\mathbb{Z} \slash 8 \mathbb{Z}$, $\mathbb{Z} \slash 2 \mathbb{Z} \times \mathbb{Z} \slash 4 \mathbb{Z}$, $D_8$, $Q_8 = \{ \pm 1, \pm i, \pm j, \pm k\ |\ i^2 = j^2 = k^2 = ijk \}$, $E_8$\\
			\\
			Если в группе есть элемент порядка $8$, то она циклическая. Если все неидентичные элементы имеют порядок $2$, то группа абелева ($1 = (ab)^2 = abab$, так что $ab = a(abab)b = a^2 ba b^2 = ba$) -- и существует только один вариант.\\ 
			\\
			Поэтому любая другая группа должна иметь хотя бы один элемент порядка $4$. Заметим, что подгруппа $<a>$, сгенерированная элементом $a$, имеет индекс $2$ и, следовательно, является нормальной. Рассмотрим элементы $b \in <a>$.\\ 
			\\
			Если существует такой $b$ порядка $2$, который коммутирует с $a$, то группа абелева $\mathbb{Z} \slash 2 \mathbb{Z} \times \mathbb{Z} \slash 4 \mathbb{Z}$.\\ 
			\\
			Если существует элемент $b$ порядка $2$, который не коммутирует с $a$, то $b^{-1}ab$ должен быть элементом $<a>$ (нормальной подгруппы) порядка $4$, который не равен $a$, поэтому $a^{-1} = a^3$.\\
			\\
			Иначе все элементы вне $<a>$ имеют порядок $4$, образуя шесть элементов порядка $4$, один порядка $2$ и идентичность. Квадрат любого элемента порядка $4$ должен быть элементом порядка $2$. Поэтому мы берем $b \in <a>$ порядка $4$ -- который не может коммутировать с $a$, так как это сделает всю группу коммутативной, а это не работает со всеми эти элементы порядка $4$. Как и раньше, $<a>$ нормально, поэтому мы должны иметь $b^{-1}ab = a^{-1}$ и вместе с $a^4 = b^4 = 1$ и $a^2 = b^2$\\
			\\
			Таким образом существует ровно $5$ групп порядка $8$.
		\end{enumerate}
		
		\subsection{}
		\begin{enumerate}
			\item Заметим, что если два разных простых числа делят порядок группы, то у нее будут подгруппы каждого из этих порядков, поэтому только одно простое число может быть делителем порядка группы. Кроме того, $p$-группа имеет подгруппы любого порядка, делящие порядок группы, поэтому порядок должен быть $p^2$. Наконец, группа циклическая, если она имеет ровно одну подгруппу любого порядка, делящего порядок группы, поэтому наша группа должна быть циклической. В заключение, $\mathbb{Z} \slash p^2 \mathbb{Z}$ является единственной возможной группой, причем для всех простых $p$.
			\\
			\item Заметим, что если более трех различных простых чисел делят порядок группы, то у нее будет больше 4 подгрупп, поэтому не более двух простых чисел могут быть делителями порядка группы. Кроме того, группа имеет подгруппы любого порядка, делящие порядок группы, поэтому порядок должен быть не больше $p^3$ (иначе групп уже больше 4). Наконец, группа циклическая, если она имеет ровно одну подгруппу любого порядка, делящего порядок группы, поэтому наша группа должна быть циклической. Тогда есть несколько вариантов: либо наша группа это $\mathbb{Z} \slash p^3 \mathbb{Z}$, либо $\mathbb{Z} \slash p_1 p_2 \mathbb{Z}$.
		\end{enumerate}
		
		\subsection{}
		
		\subsection{}
		\begin{enumerate}
			\item
		\end{enumerate}
		
		\subsection{}
		\begin{enumerate}
			\item классы сопряженности $A_3$\\
				\\
				Классы сопряженности: $\overline{e}$, $\overline{(123)}$, $\overline{(132)}$\\
				
			\item классы сопряженности $A_4$\\
				\\
				Классы сопряженности: $\overline{e}$, $\overline{(123)}$, $\overline{(234)}$, $\overline{(12)(34)}$\\
				
			\item классы сопряженности $A_5$\\
				\\
				Классы сопряженности: $\overline{e}$, $\overline{(abc)}$, $\overline{(ab)(cd)}$, $\overline{(abcde)}$, $\overline{(acbde)}$\\
				
			\item классы сопряженности $A_6$\\
				\\
				Классы сопряженности: $\overline{e}$\\
				
		\end{enumerate}
		
		\subsection{}
		
		\subsection{}
		Пусть $G$ -- групп, $\text{Aut}(G)$ -- множество её автоморфизмов $\phi:\ G \simeq G$.
		\begin{enumerate}
			\item Докажем, что $\text{Aut}(G)$ -- группа\\
				\\
				Применим $\phi^{-1}$ к $\phi(gh) = \phi(g) \phi(h)$ получим $gh = \phi^{-1} (\phi(g) \phi(h))$, но $g = \phi^{-1} (\phi(g))$ и $h = \phi^{-1} (\phi(h))$. Обозначим $g^{\prime} = \phi(g)$ и $h^{\prime} = \phi(h)$, тогда $\phi^{-1} (g^{\prime} h^{\prime})=\phi^{-1}(g^{\prime}) \phi^{-1}(h^{\prime})$. Так как $\phi$ биекция, то $(g^{\prime}, h^{\prime})$ покрывает $G \times G$. Откуда $\phi^{-1} \in \text{Aut}(G)$ и $\phi \circ \phi^{-1} = \phi^{-1} \circ \phi = \text{id}_G$. Тогда $\phi^{-1} \psi^{-1}$ обратное к $\psi \circ \phi$, для $\psi, \phi \in \text{Aut}(G)$. Так как композиция двух групп гомоморфизмов является группой
				гомоморфизмов, мы заключаем, что $\text{Aut}(G)$ является группой.
				
			\item Для $g \in G$ определим $c_g:\ G \simeq G$ как левое сопряжение: $c_g(g^{\slash}) = g g^{\slash} g^{-1}$. И докажем, что $g \to c_g$ является группой гомоморфизмов $G \to \text{Aut}(G)$\\
				\\
				Так как $c_{g} \circ c_{g^{\prime}} = c_{g g^{\prime}}$ и $c_1 = \text{id}_G$, то $c_{g^{-1}}$ -- обратный для $c_g$ и $g \to c_g$ -- гомоморфизм $G \to \text{Aut}(G)$ (поскольку свойство группового гомоморфизма требует только проверки совместимости с групповым законом). Тогда так как $c_g = \text{id}_G$ только в том случае когда $g g^{\prime} g^{-1} = g^{\prime}\quad \forall g^{\prime} \in G$, так же заметим что $c_g$ однозначно определено когда все $g$ коммутируют со всеми $g^{\prime} \in G$.					
			
			\item Докажем, что $\text{Inn}(G)$ -- нормальная подгруппа $\text{Aut}(G)$\\
				\\
				Для $\phi \in \text{Aut}(G)$ заметим $\phi \circ c_{g} \circ \phi^{-1}: g^{\prime} \mapsto \phi(g \phi(g^{\prime}) g^{-1})=\phi(g) g^{\prime} \phi(g)^{-1}=c_{\phi(g)}(g^{\prime})$. Тогда $\phi \circ c_g \circ \phi^{-1} = c_{\phi(g)}$. Откуда $\text{Inn}(G)$ -- нормальная подгруппа $\text{Aut}(G)$.
		\end{enumerate}
		%https://www.math.arizona.edu/~cais/594Page/soln/solnexam1.pdf
		
		\subsection{}
			
			
		
		\subsection{}
		Пусть $U(n) = \{ k\ |\ 1 \leqslant k < n\ \text{and}\ (k,n) = 1 \}$ 
		\begin{enumerate}
			\item Докажем что $\text{Aut}(Z_n) \cong U(n)$\\
			\\
			Пусть $n$ это целое число, тогда распишем разложение $n$ на простые в виде: $n = 2^{n_0} p_1^{n_1} \ldots p_k^{n_k}$, где $n_0 \geqslant 0$ и $n_i \leqslant 1$, $p_i$ - простое число.\\
			Тогда: $\text{Aut}(\mathbb{Z}_{n})=\mathbb{Z}_{2} \oplus \mathbb{Z}_{2^{n_{0}-2}} \times \mathbb{Z}_{(p_{1}-1) p_{1}^{n_{1}-1}} \times \cdots \times \mathbb{Z}_{(p_{k}-1) p_{k}^{n_{k}-1}}$. \\
			Или же $\text{Aut}(\mathbb{Z}_{n})=\mathbb{Z}_{(p_{1}-1) p_{1}^{n_{1}-1}} \times \cdots \times \mathbb{Z}_{(p_{k}-1) p_{k}^{n_{k}-1}}$. \\
			Заметим, что $U(p^{n}) \cong \mathbb{Z}_{p^{n-1}}(p-1)$ и $U(2^{n}) \cong \mathbb{Z}_{2} \times \mathbb{Z}_{2^{n-2}}$,\\ Откуда $\text{Aut}(Z_n) \cong U(2^{n}) \times U(p_1^{n_1}) \times \ldots \times U(p_k^{n_k}) \cong U(n)$
			
			\item Докажем что $\text{Aut}(D_n) \cong Z_n \times U(n)$, ведь $\text{Aut}(D_3)$ и $\text{Aut}(D_4)$ являются частными случаями\\
			\\
			Пусть $a,b \in D_n$ будут гененраторами, такие что $|a| = 2$, $|b| = n$ и $aba = b^{-1}$. Элементы порядка $n$ из $D_n$ переходят в $b^i$ для $(i,n) = 1$. Элементы порядка $2$ это $b^j a$ при $j = 0, \ldots, n-1$ и $b^{\frac{n}{2}}$ для чётных $n$.\\
			Определим автоморфизм $\phi_{i, j}$ для $D_n$ для $i \in D_n$ и $j \in \mathbb{Z}_n$ задающий  $\phi_{i,j}(a) = ab^j$ и $\phi_{i,j}(b) = b^i$.\\
			Тогда $\text{Aut}(D_{n}) = \{\phi_{i, j} | i \in U(n), j \in \mathbb{Z}_{n}\}$.\\
			Композиция удовлетворяет уравнению (1): $\phi_{i_{1}, j_{1}} \circ \phi_{i_{2}, j_{2}}=\phi_{i_{1} i_{2}, j_{1}+i_{1} j_{2}}$.\\
			Определим подгруппы $N = \{ \phi_{1,j}\ |\ j \in \mathbb{Z}_n \}$ и $U = \{ \phi_{i,0}\ |\ i \in U(n) \}$. Тогда уравнение дает изоморфизм $N \cong \mathbb{Z}_n$ и $U \cong U(n)$. Заметим что $\phi_{i,j}^{-1} = \phi_{i^{-1}, -ji^{-1}}$ (где $i^{-1}$ это обратный по умножению для $i$ в $U(n)$) заметим что $N$ -- нормальная подгруппа $\text{Aut}(D_n)$. Тогда $\text{Aut}(D_n) \cong N \times_{\Phi} U$ для некоторого $\Phi:\ U \to \text{Aut}(N)$. Чтобы определить $\Phi$ мы используем (1): $\Phi_{\phi_{i, 0}}(\phi_{1, j})=\phi_{i, 0} \circ \phi_{1, j} \circ \phi_{i, 0}^{-1}=\phi_{i, 0} \circ \phi_{1, j} \circ \phi_{i^{-1}, 0}=\phi_{1, i j}$. Так мы доказали, что $\text{Aut}(D_n) \cong Z_n \times U(n)$. 
			
		\end{enumerate}
		
		\subsection{}
		\begin{enumerate}
			\item
		\end{enumerate}
		
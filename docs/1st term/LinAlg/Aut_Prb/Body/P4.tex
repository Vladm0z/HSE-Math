\newpage		
	\section{Линейные отображения и матрицы}
		
		\subsection{ГЛ4 1}
		Пусть верхняя левая клетка,граничащая по точке с прямоугольником имеет координаты $(0,0)$ и в ней записано число $a_{(0\ 0)}$. Тогда в каждой клетке записано число $a_{(i\ j)}$. Причем если мы рассматриваем прямоугольник $n \times m$ то для точек $a_{(i\ j)} \quad i \in [1,\ n];\ j \in [1, m];$ выполнено: $a_{(i\ j)} = \frac{a_{(i+1\ j+1)} + a_{(i+1\ j-1)} + a_{(i-1\ j+1)} + a_{(i-1\ j-1)}}{4}$. Тогда 
		
		\subsection{ГЛ4 2}
		Заметим, что по числам в четырех вершинах однозначно восстанавливаются оставшиеся (в предположении, что на грани также можно написать числа так, чтобы сумма сходящихся в вершине граней была равна значению в вершине). Пронумеруем вершины следующим образом: в верхнем квадрате по кругу от $1$ до $4$, каждая пара противоположных вершин имеет сумму $9$. Тогда по значениям в вершинах $1,\ 2,\ 3$ и $8$. Заметим, что сумма во всех противоположных вершинах одинакова, так как их сумма равна сумме всех граней. Также четвертая вершина в верхнем квадрате однозначно задаётся, так как $b_1 - b_2 - b_3 + b_4 = 0$ (грань этого квадрата посчитана $4$ раза: $2$ со знаком $+$, $2$ с $-$. Противоположная сторона не учитывается, а каждая "боковая" грань посчитана $2$ раза с разными знаками). Откуда следует, что по значениям $b_1,b_2,b_3,b_8$ однозначно задаются оставшиеся значения, откуда вытекают условия на $b_i$. Покажем, что при любых значениях в четырех вершинах есть решение, и покажем, какие они. Рассмотрим матрицу:
		\begin{gather*}
			\begin{pmatrix} 
				1 & 1 & 1 & 0 & 0 & 0 & | & b_1\\ 
				1 & 1 & 0 & 1 & 0 & 0 & | & b_2\\
				1 & 0 & 1 & 0 & 1 & 0 & | & b_3\\
				0 & 0 & 0 & 1 & 1 & 1 & | & b_8 
			\end{pmatrix}
		\end{gather*} 
		Приведём её к ступенчатому виду:
		\begin{gather*}
			\begin{pmatrix}
				1 & 1  & 1  & 0 & 0 & 0 & | & b_1\\ 
				0 & 0  & -1 & 1 & 0 & 0 & | & b_2 - b_1\\
				0 & -1 & 0  & 0 & 1 & 0 & | & b_3 - b_1\\
				0 & 0  & 0  & 1 & 1 & 1 & | & b_8 
			\end{pmatrix}
		\end{gather*} 
		
		\begin{gather*}
			\begin{pmatrix}
				1 & 1 & 1 & 0  & 0  & 0 & | & b_1\\ 
				0 & 1 & 0 & 0  & -1 & 0 & | & b_1 - b_3\\
				0 & 0 & 1 & -1 & 0  & 0 & | & b_1 - b_2\\
				0 & 0 & 0 & 1  & 1  & 1 & | & b_8
			\end{pmatrix}
		\end{gather*}
		
		\begin{gather*}
			\begin{pmatrix}
				1 & 0 & 0 & 0 & 0  & -1 & | & b_2 + b_3 - b_1 - b_8\\ 
				0 & 1 & 0 & 0 & -1 &  0 & | & b_1 - b_3\\
				0 & 0 & 1 & 0 & 1  &  1 & | & b_1 - b_2 + b_8\\
				0 & 0 & 0 & 1 & 1  &  1 & | & b_8
			\end{pmatrix}
		\end{gather*}
		Заметим, что значения на гранях не задаются однозначно, а зависят от значений $x_5,\ x_6$ (значения на гранях -- $x_i$), откуда следуют решения в зависимости от $b_i,\ x_5,\ x_6$.
	
		\subsection{ГЛ4 3}
		Ядро имеет нулевое пересечение с образом, откуда следует, что $V = \ker F + \text{im} F$ (так как $\dim \ker F + \dim\text{im} F = \dim V$). Докажем, что $F$ проецирует на $\text{im} F$: рассмотрев базисы $\ker F$ и $\text{im} F$ можно заметить, что при разложении $x$ и $F(x)$ в виде линейной комбинации, вектора из $\ker F$ занулятся в $F(x)$, а вектора из $\text{im} F$ останутся теми же.
		
		\subsection{ГЛ4 4}
		
		\subsection{ГЛ4 5}
		Пусть $X = x_1 + x_2 + ... + x_r$ (где $\text{rank}\ X = r$, $\text{rank}\ x_i = 1$ и $x_i$ пока что "неизвестны"), тогда приведём левую часть к ступенчатому виду (сначала горизонтальному, потом вертикальному), копируя действия в правую часть. Теперь такие $x_i$ точно существуют, откуда следует, что применяя оперции в обратном порядке мы получим $r$ матриц ранга $1$, сумма которых равна $X$. Покажем что меньше $r$ быть не может. Рассмотрим пространство $A$, в котором вектора из $X$ порождающие. В нем $\dim A = r$ и если $X = x_1 + x_2 + ... + x_{r-1}$, то $A$ лежит в линейной оболочке пространств, образованных из $x_1,\ x_2,\ ... ,\ x_{r-1}$, но размерность этих пространств равна $1$, откуда их линейная оболочка имеет размерность не более чем $r-1$.
		
		\subsection{ГЛ4 6}
		(б) и (в) независимы. Пример:
		$\begin{pmatrix}
			1 & 0 & 0\\ 0 & 1 & 0 
		\end{pmatrix}$
		и
		$\begin{pmatrix}
			0 & 0 & 1\\
			0 & 0 & 1
		\end{pmatrix}$\\
		$U_1 \cap U_2 = 0, W_1 \cap W_2 = W_2$
		аналогичный пример в обратную сторону для транспонированных матриц.
		
		\subsection{ГЛ4 7}
		A)\\
		$\dim\text{im} A + \dim \ker A = \dim\text{im} (B\ A) + \dim \ker (B\ A)$\\
		Покажем, что $\dim (\text{im} A \cap \ker B) = \dim \ker (B\ A) - \dim \ker (A)$\\
		(1): Заметим, что $ker (A) \in \ker (B\ A)$, поэтому $\dim ( \ker (B\ A) / \ker (A) ) = \dim \ker (B\ A) - \dim \ker (A)$, при этом нетрудно видеть, что $\text{im} A \cap \ker B = \ker (B\ A) / \ker (A)$ откуда следует (1) (откуда очевидно утверждение задачи).\\
		\\
		B)\\
		из предыдущего пункта:\\
		\begin{gather*}
			\dim\text{im} A = \dim\text{im} (B\ A) + \dim (\text{im} A \cap \ker B)\\
			\dim\text{im} AC = \dim\text{im} (B\ AC) + \dim (\text{im} AC \cap \ker B)
		\end{gather*}
		Откуда требуется доказать, что\\
		$ - \dim (\text{im} A \cap \ker B) \leq - \dim (\text{im} AC \cap \ker B)$, то есть $ \dim (\text{im} A \cap \ker B) \geq  \dim (\text{im} AC \cap \ker B)$, что верно, так как $\text{im} AC \in\text{im} A$.\\
		
		
		\subsection{ГЛ4 8}
		A)\\
		Рассмотрим матрицы $A = \begin{pmatrix} 1 & -1 \\ 1 & -1 \end{pmatrix}$ и $B = \begin{pmatrix} 1 & 1 \\ -1 & -1 \end{pmatrix}$. Заметим, что $A^2 = B^2 = 0$, при этом $A + B = \begin{pmatrix} 2 & 0 \\ 0 & -2 \end{pmatrix} \ \Rightarrow \ \det(A + B) = -4 \ \Rightarrow \ \det(A+B)^k \ne 0 \ \Rightarrow \ (A+B)^K \ne 0$
		\\
		B)\\
		Ответ: да.\\
		Пусть $A^N = 0,\ B^M = 0$.\\
		Тогда 
		\begin{gather*}
			(A + B)^{N+M} = 
			A^{N+M} + {N+M \choose 1} 
			\cdot A^{N+M-1} \cdot B + ... + {N+M \choose M} 
			A^N \cdot B^M + ... + B^{M+N} 
			=\\
			A^N \cdot A^M + A^N \cdot {N+M \choose 1}
			\cdot A^{M-1} \cdot B + ... + A^N \cdot B^M + ... + B^M \cdot B^N
			=
			0 + 0 + ... + 0 + ... + 0 = 0
		\end{gather*}
		(заметим, что мы явно пользуемся тем, что $AB = BA$ при разложении по биному Ньютона).
		\\
		C)\\
		\\
		\subsection{ГЛ4 9}
		A)\\
		Пусть $A = \begin{pmatrix} a & b\\ c & d \end{pmatrix}$. Тогда $\beta = 0\ \Leftrightarrow \ \det A = 0$, $\alpha = 0\ \Leftrightarrow \ a + d = 0$. То есть $c$ и $d$ однозначно выражаются через $a$ и $b$ : $d = -a, c = - \frac{a^2}{b}$, при этом для любых $a$ и $b$ с такими $c$ и $d$: $A^2 = 0$
		\\
		B)\\
		Заметим, что $\det (A\ B) = \det A \times \det B$, откуда если $A^3 = 0$, то $\det A = 0$, откуда если рассмотреть некоторое $B$, для которого $B^2 \ne 0$: $B^2 + \alpha B = 0$, то $B^3 + \alpha B^2 = 0$, откуда он может быть корнем $X^3 = 0$ при $\alpha = 0$, но тогда $B^2 = 0$. Поэтому нет корней из $X^3 = 0$, не являющимися корнями $X^2 = 0$, при этом очевидно, что все корни из $X^2 = 0$ являются корнями из $X^3 = 0$.
		\\
		C)\\
		Пусть $A = \begin{pmatrix} a & b\\ c & d \end{pmatrix}$. Тогда $\beta = 1\ \Leftrightarrow \ \det A = 1$, $\alpha = 0\ \Leftrightarrow \ a + d = 0$. То есть $c$ и $d$ однозначно выражаются через $a$ и $b$ : $d = -a, c = - \frac{1 + a^2}{b}$, при этом для любых $a$ и $b$ с такими $c$ и $d$: $A^2 = -E$
		
		\subsection{ГЛ4 10}
		
		\subsection{ГЛ4 11*}
		
		\subsection{ГЛ4 12*}
		Если $A \ne E \cdot v$, где $v$ -- матрица 1 на $n$ (что равносильно $\exists i,j:\ i \ne j,\ a_{i\ j} \ne 0$), то рассмотрим 2 матрицы $X,Y:\ \text{Tr}(X) = \text{Tr}(Y) = 0,\ (X - Y) = B$, где в $B$ все элементы нулевые кроме $a_{j\ i}$. Тогда $\text{Tr}(AX) \ne \text{Tr}(AY)$, так как $AX - AY = AB$, а $\text{Tr}(B) \ne 0$. Тогда $A = E \cdot v$. И если $\text{Tr}(AX) = 0$, то $a_{1\ 1} \cdot x_{1\ 1} + ... + a_{i\ i}  \cdot  x_{i\ i} + ... + a_{n\ n} \cdot x_{n\ n} = 0$ при $\sum x{i\ i} = 0$. Рассмотрим $\exists l,m: a_{l\ l} \ne a_{m\ m}$, при замене $x_{l\ l}$ на $x_{l\ l} + 1$ и $x_{m\ m}$ на $x_{m\ m} - 1$ -- $\sum x{i\ i}$ не изменится, но $\sum a_{i\ i} \cdot x_{i\ i}$ увеличится на $a_{l\ l} - a_{m\ m}$. Тогда если $A$ удовлетворяет условию, то такой пары $l,\ m$ нет. Откуда $A = E \cdot c$, где $с$ -- некое число, подходит.
		
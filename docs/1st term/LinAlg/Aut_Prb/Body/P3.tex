\newpage		
	\section{Многомерие}
		
		\subsection{ГЛ3 1}		
		сперва построим 1-мерные пространства(прямые через 0), их $\frac{q^n-1}{q-1}$ потом для каждого из них выбираем вторые вектора которых может быть $q^n-q$ (так как он не может заканчиваться в точке на первой прямой). Так мы получили двумерные пространства натянутые на эти пары векторов, их $\frac{(q^n-1)(q^n-q)}{q-1}$ каждое из них содержит $q^2-1$ ненулевых точек, поэтому разных двумерных пространств получаем $\frac{(q^n-1)(q^n-q)}{(q-1)(q^2-1)}$. Далее, по индукции:
		\begin{gather*}
		\frac{
		\prod\limits_{i = 1}^k (q^n - q^{i-1})
		}{
		\prod\limits_{i = 1}^k (q^k - q^{i-1})
		}
		\end{gather*}
		
		\subsection{ГЛ3 2}		
		Заметим, что в любом $k$ мерном подпространстве можно выбрать репер. Этот репер задаётся $k+1$ точкой, при этом они должны быть общего положения, так как вектора, образуемые парами из первой и остальных точек, должны быть линейно независимы. Тогда количество подпространств размерности $k$: $\frac{q^n(q^n-1)(q^n-q)(q^n-q^2)\ldots(q^n-q^{k-1}) }{q^k(q^k-1)(q^k-q)(q^k-q^2)\ldots(q^k-q^{k-1}) }$, так как числитель -- количество способов выбрать упорядоченное множество $k+1$ точек, знаменатель -- количество таких упорядоченных множеств в одном пространстве.
		
		\subsection{ГЛ3 3}		
		А)\\
		Рассмотрим следующие 6 плоскости проходящие через $(0,\ 0,\ 0,\ 0)$: $(a_1,\ a_2,\ 0,\ 0)$, $(a_1,\ 0,\ a_3,\ 0)$, $(a_1,\ 0,\ 0,\ a_4)$, $(0,\ a_2,\ a_3,\ 0)$, $(0,\ a_2,\ 0,\ a_4)$, $(0,\ 0,\ a_3,\ a_4)$, где $a_i$ -- параметры. Покажем, что любая плоскость содержит хотя бы 1 точку с хотя бы двумя нулевыми координатами. Рассмотрим базисные вектора: они линейно независимы. Заметим, что тогда есть пара линейно независимых координат у этих векторов, так как иначе можем считать что первый вектор $(1,\ 1,\ 1,\ 1)$ (можем считать, что он один из следующих: $(1,\ 1,\ 1,\ 1)$, $(0,\ 1,\ 1,\ 1)$, $(0,\ 0,\ 1,\ 1)$, $(0,\ 0,\ 0,\ 1)$). 
		\begin{gather*}
			\begin{matrix}
				\text{Для второго рассмотрим базис:} & \text{Для третьего:} & \text{Для четвёртого:} \\
				(1,\ -1,\ 0,\ 0) & ( 1,\ -1,\ 0,\ 0) & ( 1 ,\ -\frac{1}{2},\ -\frac{1}{2},\ 0) \\
				(0,\ 2,\ 0,\ 0) & (-1,\ 1,\ \frac{1}{2},\ 0) & ( -\frac{1}{2},\ 1 ,\ -\frac{1}{2},\ 0) \\
				(0,\ 0,\ 1,\ 0) & ( 0,\ 0,\ \frac{1}{2},\ 0) & ( -\frac{1}{2},\ -\frac{1}{2},\ 1,\ 0) \\
				(0,\ 0,\ 0,\ 1) & ( 0,\ 0,\ 0,\ 1) & ( 0,\ 0,\ 0,\ 1) 
			\end{matrix}
		\end{gather*}
		Тогда второй вектор $(\alpha,\ \alpha,\ \alpha,\ \alpha)$, откуда следует, что они линейно зависимы.
		Рассмотрим эту пару координат. Тогда заметим, что существует точка в этой плоскости с нулями в этих координтах, а все такие точки принадлежат шести плоскостям, проходящим через 0, откуда следует, что любая плоскость пересекает хотя бы одну из 6 плоскостей.\\
		\\
		Б)\\
		Заметим, что все плоскости, пересекающие плоскость $(a_1,\ a_2,\ 0,\ 0)$, вида $(x_1,\ x_2,\ a_3,\ a_4)$, где $x_i$ - константы, $a_i$ - параметры. Тогда для пары плоскостей $(a_1,\ a_2,\ 0,\ 0)$ и $(0,\ 0,\ a_3,\ a_4)$ верно, что нет плоскости, пересекающие обе в одной точке.
		\\
		
		\subsection{ГЛ3 4}		
		А) $(U + V) \cap (W + U) \cap (V + W) = (U + W) \cap V + (U + V) \cap W$\\
		Представим $V$ как $[v_1,\ v_2,\ \ldots]$, $U$ как $[u_1,\ u_2,\ \ldots]$, $W$ как $[w_1,\ w_2,\ \ldots]$\\
		Тогда заметим, что если $v_i \notin U+W$, то мы можем выкинуть $v_i$ из $V$ и тогда уравнение останется верным.
		Тогда проведем такую операцию "выкидывания" для $U, V, W$ и получим $U_1, V_1, W_1$.\\
		Теперь рассмотрим уравнение, заметим что в правой части есть как $w$ (так как $w \subset (u+v) \cdot w$), так и $v$ (так как $v \subset (u+w) \cdot v$), а следовательно там есть и $u$ ($u \subset v+w$). \\
		\\
		Б) $(U + V) \cap (W + U) \cap (V + W) \subseteq (U \cap V) + (W \cap U) + (V \cap W)$\\
		Пусть $U = [1, 1]$, $V = [0, 1]$ ,$W = [1, 0]$ - заметим что для таких множест уравнение имеет вид: $U + V + W \subseteq 0$, что очевидно неверно\\
		\\
		В*)$(U + V) \cap (W + U) \cap (V + W) \supseteq (U \cap V) + (W \cap U) + (V \cap W)$\\
		Представим $V$ как $[v_1,\ v_2,\ \ldots]$, $U$ как $[u_1,\ u_2,\ \ldots]$, $W$ как $[w_1,\ w_2,\ \ldots]$\\
		Тогда заметим, что если $v_i \notin U+W$, то мы можем выкинуть $v_i$ из $V$ и тогда уравнение останется верным.
		Тогда проведем такую операцию "выкидывания" для $U, V, W$ и получим $U_1, V_1, W_1$. Тогда правая часть не меньше левой и все что входит в $(U \cap V) + (W \cap U) + (V \cap W)$, входити в $(U + V) \cap (W + U) \cap (V + W)$\\
		\\
		Г)\\
		\\
		
		\subsection{ГЛ3 5}		
		
		
		\subsection{ГЛ3 6}		
		Заметим, что базис объединения подпространств не больше, чем объединение базисов этих подпространств. Заметим, что если нельзя построить отображение из объединения базисов в базис всего пространства, то пространство "больше" объединения подпространств. Заметим, что $2*X ~ X$ если X бесконечно, откуда следует, что $k*X ~ X$ для всякого целого k, поэтому нельзя построить отображение из объединения базисов в базис пространства.\\
		А)\\
		\\
		Б)\\
		\\
		
		\subsection{ГЛ3 7}		
		Рассмотрим куб($\mathbf{R}^3$), для куба существует прямая, непересекающая его ребра($\mathbf{R}^1$), один изпримеров этой прмой - прямая соединяющая середины противоположных сторон. Увеличим размерность на 1, теперь куб в размерности $\mathbf{R}^4$ и прямая стала плоскостью $\mathbf{R}^2$, причем эта прямая не пересекает ни одну плоскость гиперкуба по построению.
		
		\subsection{ГЛ3 8}		
		Куб из задачи (7) пересекается с плоскостью $\sum_i x_i = c$ при $c = [-2,\ 2]$ так как $\sum_i x_i = c \ \Leftrightarrow \ x_1 + x_2 + x_3 + x_4 = c$. Заметим что куб имеет координаты $\forall i:\ |x_i| < \frac{1}{2}$, а 4-мерная гиперплоскость $x_1 + x_2 + x_3 + x_4 = c$ перпендикулярна прямой соединяющей точки $\bigm( \frac{1}{2},\frac{1}{2},\frac{1}{2},\frac{1}{2} \bigm)$ и $\bigm( -\frac{1}{2},-\frac{1}{2},-\frac{1}{2},-\frac{1}{2} \bigm)$ -- тогда, так как эти две точки являются противоположными вершинами куба, то плоскость пересекает куб в том случае, если она пересекает этот (соединяющий точки $\bigm( \frac{1}{2},\frac{1}{2},\frac{1}{2},\frac{1}{2} \bigm)$ и $\bigm( -\frac{1}{2},-\frac{1}{2},-\frac{1}{2},-\frac{1}{2} \bigm)$ ) отрезок. То есть $x_1 + x_2 + x_3 + x_4 \in [-2, 2]$.
		\subsection{ГЛ3 9}		 
		Набор векторов $\nu_0 ,\ \nu_1 ,\ \ldots,\ \nu_n \in \mathbb{R}^n$, $n$ из $n+1$ образуют базис\\
		А)\\
		$w \in \mathbb{R}^n$ представим как $w = \sum x_i e_i;\ \sum x_i = 0$	
		\\
		Б)\\
		Рассмотрим набор из $n$ векторов, образующий базис, пусть это $\nu_1 ,\ \nu_2 ,\ \ldots,\ \nu_n$. Тогда $\nu_0$ выражается через все остальные вектора -- заметим, что тогда $\nu_0 = a_{1} \nu_{1} + \ldots + a_{n} \nu_{n}$, тогда заметим, что если $\forall i:\ a_{i} \ne 0$, то любой вектор можно заменить на $\nu_0$ имеет ненулевую ортогональную проекцию, если же $\exists i:\ a_{i} = 0$, без ограничения общности $a_{1} = 0$, то набор векторов $\nu_{2},\ \ldots ,\ \nu_{n}$ является базисом в $\mathbb{R}^{n-1}$, а $\nu_{1},\ \nu_{2},\ \ldots ,\ \nu_{n}$ базисом не является.\\
		\\
		В)\\
		\\
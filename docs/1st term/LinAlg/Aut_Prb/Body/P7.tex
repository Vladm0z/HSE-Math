\newpage		
	\section{Евклидова геометрия}
		
		\subsection{ГЛ7 1}
		Да, всегда найдется
		
		\subsection{ГЛ7 2}
		Нет, неверно\\
		\begin{gather*}
			\cos(a,b) = \frac{a \cdot b}{|a| \cdot |b|}
		\end{gather*}
		Пусть $a, b$ лежит в $\mathbb{R}^3$, будем проектировать гиперплоскость $(xy)$\\
		\begin{gather*}
			\begin{matrix}
				\vec{a} = (1,1,1) \to \vec{a}_1 = (1,1,0)\\
				\vec{b} = (1,2,-5) \to \vec{b}_1 = (1,2,0)
			\end{matrix}
		\end{gather*}\\
		Тогда $\vec{a} \vec{b} = 1 + 2 - 5 = -2 < 0$ тогда угол тупой\\
		Но $\vec{a}_1 \vec{b}_1 = 1 + 2 = 3 > 0$ тогда угол острый\\

		\subsection{ГЛ7 3}
		Очевидно, что в каждой размерности $\mathbb{R}^{n}$ есть хотя бы $n$ гиперплоскостей, относительно которых симметричен куб. Далее заметим, что плоскость относительно которой куб симметричен также может проходить через пары противоположных ребер (в размерности $\mathbb{R}^{n}$ это гиперплоскостей размерности $\mathbb{R}^{n-2}$ ровно $n \cdot (n-1) \cdot 2$).\\
		Заметим, что для $\mathbb{R}^n$ есть ровно $\dim(\mathbb{R}^n) + \frac{n \cdot (n-1) \cdot 2}{2} = n + n \cdot (n-1) = n \cdot n$.\\
		Тогда для 4: $4 \cdot 4 = 16$ 
		
		\begin{comment}
			Решим задачу для $n$: Заметим, что если рассмотреть полный граф, состоящий из вершин, являющихся серединами граней размерности $n-1$ ($K_{2^{n}}$), то любая плоскость симметрии будет пересекать часть ребер в середине, с частью не будет пересекаться, а оставшиеся будут лежать в ней. 
		\end{comment}
		
		\subsection{ГЛ7 4}
		$\Delta_{n}=\left\{x \in \mathbb{R}^{n+1} | \sum x_{i}=1, x_{i} \geqslant 0\right\}$\\
		А)\\
		\\
		Б)\\
		\\
		
		\subsection{ГЛ7 5*}
		Рассмотрим октаплекс $\{3,4,3\}$\\
		А)\\
		\begin{gather*}
			\begin{matrix}
			0 & 1 & 2\ \{3\} & 3\ \{3,4\} \\
			24 & 96 & 96 & 24 
			\end{matrix}
		\end{gather*}
		\\
		Б)\\
		%непонятно что хотят в ответе
		\\
		В)\\
		Правильные октаэдры $\{3,4\}$
		\\
		Г)\\
		Правильные треугольники $\{3\}$
		\\
		
		\subsection{ГЛ7 6}
		
		\subsection{ГЛ7 7}
		А) $\sigma_{\pi_{1}} \circ \sigma_{\pi_{2}}=\rho_{v, \varphi}$\\
		\\
		Б) $\sigma_{\pi_{1}} \circ \sigma_{\pi_{2}}=\tau_{v}$\\
		\\
		В) $\sigma_{\pi} \circ \varrho_{u, \varphi} \circ \sigma_{\pi}=\varrho_{v, \psi}$\\
		\\
		Г) $\varrho_{u, \varphi} \circ \varrho_{w, \psi}=\tau_{v} \circ \varrho_{v, \vartheta}$\\
		\\
		Д) $\varrho u, \varphi \circ \sigma_{\pi} \circ \varrho_{u,-\varphi}=\sigma_{\pi_{2}}$\\
		\\
		Е) $\varrho_{u, \varphi} \circ \sigma_{\pi_{1}}=\sigma_{\pi_{2}}$\\
		\\
		Ж) $\tau_{u_{2}} \circ \sigma_{\pi_{2}} \circ \tau_{u_{1}} \circ \sigma_{\pi_{1}}=\tau_{v} \circ \varrho_{v, \varphi}$\\
		\\
		
		\subsection{ГЛ7 8}
		А)\\
		Рассмотрим \\
		Тогда $\varrho_{A D} \circ \varrho_{A C} \circ \varrho_{A B}$ является поворотом вокруг оси $\perp AC$ и проходящей через $A$ на $\frac{\pi}{4}$
		\\
		Б)\\
		\\
		
\newpage
	\section{Задачи для подготовки к коллоквиуму 1}
		\subsection{1}
		Пусть аффинное преобразование из условия - $M(x) = \begin{pmatrix} M_{1\ 1} & M_{2\ 1} \\ M_{1\ 2} & M_{2\ 2} \end{pmatrix} \begin{pmatrix} x_1 \\ x_2 \end{pmatrix} + \begin{pmatrix} b_1 \\ b_2 \end{pmatrix} $. Заметим, что для всех $c_1, c_2, x_1, x_2$ : 
		$\begin{pmatrix} M_{1\ 1} & M_{2\ 1} \\ M_{1\ 2} & M_{2\ 2} \end{pmatrix} \begin{pmatrix} x_1 + c_1 \\ x_2 + c_2 \end{pmatrix} + \begin{pmatrix} b_1 \\ b_2 \end{pmatrix} = 
		\begin{pmatrix} M_{1\ 1} & M_{2\ 1} \\ M_{1\ 2} & M_{2\ 2} \end{pmatrix} \begin{pmatrix} x_1 \\ x_2 \end{pmatrix} + \begin{pmatrix} b_1 + c_1 \\ b_2 + c_2 \end{pmatrix} $\\ то есть \\
		$\begin{pmatrix} M_{1\ 1}*x_1 + M_{1\ 1}*c_1 + M_{2\ 1}*x_2 + M_{2\ 1}*c_2 + b_1 \\ M_{1\ 2}*x_1 + M_{1\ 2}*c_1 + M_{2\ 2}*x_2 + M_{2\ 2}*c_2 + b_2 \end{pmatrix} =
		\begin{pmatrix} M_{1\ 1}*x_1 + M_{2\ 1}*x_2 + b_1 + c_1 \\ M_{1\ 2}*x_1 + M_{2\ 2}*x_2 + b_2 + c_2 \end{pmatrix}$\\ откуда \\ 
		1.$M_{1\ 1}*c_1 + M_{2\ 1}*c_2 = c_1$ \\
		2.$M_{1\ 2}*c_1 + M_{2\ 2}*c_2 = c_2$ \\
		\\
		Заметим, что (1.) выполнено для всех пар $(c_1;c_2)$, то есть и для пары $(1;0)$, откуда $M_{1\ 1} = 1$. Поэтому $M_{2\ 1} = 0$.\\
		Аналогично $M_{1\ 2} = 0;M_{2\ 2} = 1$. Заметим, что тогда $M(x) = \begin{pmatrix} x_1 + b_1 \\ x_2 + b_2 \end{pmatrix}$, то есть $M(x)$ - сдвиг, что очевидно коммутирует со всеми сдвигами.
		
		\subsection{2}
		Заметим, что можно выбрать такой базис, что одна из 3х прямых - $x_1 = 0$, вторая - $x_1 = 1$, тогда третья - $x_1 = \alpha$. Далее будем работать в этом базисе. Покажем, что нет преобразования переводящее $x_1 = 0 -> x_1 = 0$, $x_1 = 1 -> x_1 = 1$, $x_1 = \alpha -> x_1 = \beta$ 
		Пусть есть такое аффинное преобразование $M(x) = \begin{pmatrix} M_{1\ 1} & M_{2\ 1} \\ M_{1\ 2} & M_{2\ 2} \end{pmatrix} \begin{pmatrix} x_1 \\ x_2 \end{pmatrix} + \begin{pmatrix} b_1 \\ b_2 \end{pmatrix}$. При этом точка вида $(0;y) -> (M_{2\ 1}*y + b_1; M_{2\ 2}*y + b_2)$, откуда $M_{2\ 1}*y + b_1 = 0$ для всех $y$, откуда $b_1 = 0, M_{2\ 1} = 0$. Аналогично $(1;y) -> (M_{1\ 1} + M_{2\ 1}*y + b_1; M_{1\ 2} + M_{2\ 2}*y + b_2)$, откуда $M_{1\ 1} = 1$. Поэтому точка вида $(q;w) -> (q;e)$, откуда следует, что прямая $x_1 = \alpha -> x_1 = \alpha$, откуда следует, что $M(x)$ не является "подхоящим" аффинным преобразованием, если $\beta \ne \alpha, \beta \ne 0, \beta \ne 1$.
		
		\subsection{3}
		$2x - y = 5 \quad x + 3y = 2$\\ 
		
		Пусть $n_1 = (1;2), n_2 = (3;-1)$. Тогда прямые задаются уравнениями:
		\begin{gather*}
			l_1 : (n_1,x) = (n_1,(0;-5)) = -10
			l_2 : (n_2,x) = (n_2,(2;0)) = 6
		\end{gather*}
		Заметим, что множество точек, равноудалённых от прямых $l_1$ и $l_2$ :
		\begin{gather*}
			|\frac{-10 - (n_1, a)}{|n_1|}| = |\frac{6 - (n_2, a)}{|n_2|}|\\
			1. \ (-10 - (n_1,a)) \cdot \sqrt{10} = +(6 - (n_2,a)) \cdot \sqrt{5}\\
			-(n_1,a) \cdot \sqrt{2} + (n_2,a) = 10\sqrt{2} + 6\\
			((-\sqrt{2} + 3;-2\sqrt{2} - 1),a) = 10\sqrt{2} + 6\\
			2. \ (-10 - (n_1,a)) \cdot \sqrt{10} = -(6 - (n_2,a)) \cdot \sqrt{5}\\
			-(n_1,a) \cdot \sqrt{2} - (n_2,a) = 10\sqrt{2} - 6\\
			((-\sqrt{2} - 3;-2\sqrt{2} + 1),a) = 10\sqrt{2} - 6\\
		\end{gather*}
		Откуда нормальные векторы биссектрисс: $(-\sqrt{2} + 3;-2\sqrt{2} - 1)$ и $(-\sqrt{2} - 3;-2\sqrt{2} + 1)$,\\
		поэтому напрявляющие - $(2\sqrt{2} + 1;-\sqrt{2} + 3)$ и $(2\sqrt{2} - 1;-\sqrt{2} - 3)$
		
		\subsection{4}
		Заметим, что размерность любого подпространства меньше или равна размерности пространства, так как базис пространства является порождающим для подпространства. При этом все возможные мощности, меньшие или равные счётной - конечные или счётные (так как любое подмножество натуральных чисел либо конечно, либо можно в явном виде посчитать).\\
		При этом несчётное множество линейно зависимо, так как иначе базис не может быть счётным.
		
		\subsection{5}
		Считаем, что вершины гиперкуба имеют координаты "0" или "1" в каждой из осей (не обязательно одновременно).\\
		Заметим, что каждый куб можно задать уравнением вида $(x,y,z,t)$, где ровно одно из чисел - константа, равная 0 или 1. (к примеру $(x,y,z,0)$). Аналогично можно задать стену в кубе, зафиксировав одну из переменных, которые остались (к примеру $(x,y,0,0)$). 
		\\
		A)\\
		Заметим, что противоложная дверь стены (x,y,0,0) для куба (x,y,z,0) : (x,y,1,0), то есть имеющую противоположную координату в соотв. оси. Аналогично это верно для всех стен во всех кубах.
		Теперь начнём путь: без огр общ считаем, что изначально мы в кубе (x,y,z,0) и вошли из стены (x,y,0,0), далее:
		\begin{gather*}
				\begin{matrix}
					\text{Стена}& \text{Куб}& \\
					(x,y,0,0)& (x,y,z,0)& \text{(начало)}\\
					(x,y,1,0)& (x,y,1,t)& \text{(x,y,1,0) в 2х кубах: (x,y,z,0) и (x,y,1,t). Из 1 куба Алиса пришла$ \Rightarrow $ она вошла во 2}\\
					(x,y,1,1)& (x,y,z,1)& \text{(x,y,1,1) в 2х кубах: (x,y,z,1) и (x,y,1,t). Из 2 куба Алиса пришла$ \Rightarrow $ она вошла в 1}\\
					(x,y,0,1)& (x,y,0,t)& \text{(x,y,0,1) в 2х кубах: (x,y,z,1) и (x,y,0,t). Из 1 куба Алиса пришла$ \Rightarrow $ она вошла во 2}\\
					(x,y,0,0)& (x,y,z,0)& \text{(x,y,0,0) в 2х кубах: (x,y,z,0) и (x,y,0,t). Из 2 куба Алиса пришла$ \Rightarrow $ она вошла в 1}\\
				\end{matrix}
		\end{gather*}
		откуда видно, что всего комнат -- $4$.
		\\
		B)\\
		
		\subsection{6}
		Заметим, что если $l$ параллельна П (это русская букова), то нет, не заметают, так как любая точка $a$ из плоскости $\alpha$, параллельной П и проходещей через $l$ не может быть полученна, так как любая прямая, проходящая через $a$ и точку на $l$ лежит в плоскости $\alpha$, что не лежит в плоскости П.\\
		Заметим, что если же $l$ не параллельна П, то есть существует пересечение $l$ и П, то любая точка $b$ может быть получена, так как рассмотрим плоскость, натянутую на $b$ и $l$. Заметим, что она пересекает П хотя бы по прямой, откуда следует, что если на плоскости есть 2 прямые и все прямые, проходящие через пары точек на этих 2х прямых, замощают плоскость, то в трёхмерной задаче прямые через $l$ и П замощают пространство. Заметим, что утверждение на плоскости выполнено, так как для любой точки можно выбрать прямую, не параллельную ни одной из 2х данных, тогда эта прямая пересекает 2 данные в каких то точках, откуда и следует, что для каждой точки можно указать прямую, проходящую через неё.\\
		
		\subsection{7}
		Центры треугольников имеют координаты $(\frac{1}{3};\frac{1}{3};\frac{1}{3};0;0)$ и $(0;0;\frac{1}{3};\frac{1}{3};\frac{1}{3})$ соотв.\\
		Тогда середина между этими центрами имеет координаты $X:(\frac{1}{6};\frac{1}{6};\frac{1}{3};\frac{1}{6};\frac{1}{6})$\\
		Заметим, что $Y:(\alpha;0;0;0;1-\alpha)$, $Z:(0;b;c;d;0)$ (при b + c + d = 1). Откуда прямая XY не может проходить через Z, если $\alpha \ne 1-\alpha$, поэтому $\alpha = \frac{1}{2}$, при этом $Z = \beta*Y + (1-\beta)*X$, откуда $\beta*\frac{1}{2} = -(1 - \beta)*\frac{1}{6} \Leftrightarrow \beta*\frac{1}{3} = -\frac{1}{6} \Leftrightarrow \beta = -\frac{1}{2}$ $\Rightarrow $\\
		$Z : (0;\frac{1}{4};\frac{1}{2};\frac{1}{4};0)$. Из того, что $\beta = -\frac{1}{2}$ следует, что $\overrightarrow{XY} : \overrightarrow{YZ} = 2:1$\\
		
		\subsection{8}
		Рассмотрим пространство $\alpha$ над $ABCD$, рассмотрим точку пересечения прямой $EP$ с $\alpha$. (пусть это точка $E_1$). Назовём такую операцию $P$-проекцией (которая была проведена с точкой $E$). Заметим, что $P$-проекция точки, лежащей на отрезке $EA,$ лежит на прямой $E_1 A$ . Заметим, что в плоскости $\alpha$ можно выбрать такую прямую $l$, которая не будет пересекаться ни с одной из прямых пар точек $A$,$B$,$C$,$D$,$E_1$. Проведём через эту прямую $l$ и $P$ плоскость. Заметим, что она не пересекает ни одно "ребро", так как это означает, что через
		
		\subsection{9}
		
		\subsection{10}
		Докажем, что любой прямой ровно $q$ точек: рассмотрим $2$ различные точки $a,b$, проведём черех них прямую. Рассмотрим всевозможные произведения $v = \overrightarrow{ab}$ со всеми элементами из поля. Заметим, что получится $q$ различных векторов, так как иначе $\alpha \cdot v = \beta \cdot v \ \Leftrightarrow \ v \cdot (\alpha - \beta) = 0 \ \Leftrightarrow \ \alpha - \beta = 0 \ \Leftrightarrow \ \alpha = \beta$. Тогда все точки вида $\gamma \cdot v + a$ лежат на прямой $ab$, при этом все они различны и других нет, откуда и следует, что точек на прямой -- $q$. \\
		Тогда заметим, что всего прямых -- $\frac{A(A-1)}{q(q-1)}$, где $A = q^n = |M|$ ($M$ - аффинное пространство из условия), так как каждая прямая однозначно задаётся парой точек, то каждая прямая была посчитана $q(q-1)$ раз.\\
		Заметим, что треугольников -- $\frac{A(A-1)(A-q)}{3!}$, так как каждый треугольник задаётся тремя точками общего положения, и каждая тройка точек была посчитана $3!$ раз.\\
		Каждая плоскость содержит $q^2$ точек, так как каждая плоскость изоморфна $2$-мерному пространству: каждая плоскость задаётся тремя неколлиниарными точками $o,a,b$, остальные задаются $o + \overrightarrow{oa} \cdot \alpha + \overrightarrow{ob} \cdot \beta$ , при этом все точки различны (это нетрудно видеть, так как иначе точки лежат на одной прямой). Тогда сопоставим точку $o$ точке $(0;0)$, точку $a,b$ -- точкам $(1;0),\ (0;1)$. Остальные точки задаются однозначно.\\
		Тогда плоскостей -- $\frac{A(A-1)(A-q)}{q^2 (q^2 - 1) (q^2 - A)}$.
		
		\subsection{11}
		Если поле из $27$ элементов -- расширение поля из $9$ размерности $n$, то в нем $9^n = 27$ элементов. Но в этом случае $n$ -- не целое число, а размерность расширения нецелой быть не может.\\
		Ответ: нет\\
		
		\subsection{12}
		a,b) Нет, так как пусть есть $N$ прямых. Заметим, что тогда точек с целочисленными координатами от $\alpha$ до $\alpha + N+1$: $(N+1)^2$ (выберем такое $\alpha$, что нет прямых вида $x_2 = c$, где $\in [\alpha,\alpha + N + 1]$), при этом каждая прямая (не вида $x_2 = c$), содержит не более $N+1$ точку (то есть каждой $x_2$ сопоставляется не более одной $x_1$), при этом прямых - N, откуда точек, принадлежащим прямым не более $N*(N+1)$, что меньше $(N+1)^2$.\\
		с) Пусть $N$ подпространств размерности $k$. (если есть пространства меньшей размерности - разширим их до $k$) (Тогда пространство размерности хотя бы $k+1$). Рассмотрим точки с целочисленными координатами от $0$ до $N+1$. Таких точек хотя бы $(N+1)^{k+1}$, при этом каждое подпростанство размерности $k$ имеет не более $(N+1)^k$ точек (заметим, что подпространство размерности k либо вырождено, то есть вида $(a_1,a_2,...,a_k,0)$, где $a_i$ - параметры, $0$ - константа, при этом константа не обязательно в последней координате, либо по первым k координатам однозначно восстанавливается последняя), откуда всего точек в объединении не более $(N+1)^k*N < (N+1)^{k+1}$.
		
		\subsection{13}
		Да. Докажем по индукции:\\
		База: $V$ содержит все многочлены нулевой степени -- это верно, так как есть хотя бы 1 многочлен нулевой степени, остальные выражаются линейно комбинацией.
		Переход: $V$ содержит все многочлены степени $< k$, тогда содержит и все многочлены степени $k$ (при $k \leq m$). Заметим, что есть хотя бы один многочлен степени $k$, тогда $ax^k$ лежит в этом векторном пространстве (так как можно вычесть многочлен с коэффициэнтами меньших степеней). Любой многочлен степени $k$ можно представить в виде $\alpha*x^k + P(x)$, где степень $P(x) < k$. Заметим, что из этого следует, что любой многочлен степени $k$ лежит в этом векторном пространстве.
		
		\subsection{14}
		
		\subsection{15}
		
		\subsection{16}
		1) Покажем, что сложение в $F_2$ такое же, что и в нашем векторном пространстве $V$ (которое образовано следующим образом: это $|M|$-мерное пространство, то есть сопоставим каждой координате элемент из $M$. Каждому множеству $C$ сопоставим следующий вектор $v = [v_1,\ v_2,\ ...]$: пусть $m_i \in M$ ($m_i$ соответсвует $i$	-той координате), тогда $m_i \in C\ \Leftrightarrow \ v_i = 1$.):\\
		\\
		Складываем множества $A$ и $B$ (и соответствующие вектора $a = [a_1,\ a_2,\ ...],\ b = [b_1,\ b_2,\ ...]$) Проверим для каждого элемента, что соответствующая ему координата соответствует нахождению (или отсутствию) элемента в сумме. Если $m_i \in A$ и $m_i \notin B$, то $m_i \in A+B$, как и $1 + 0 = 1$. Аналогично для случая наоборот. Если $m_i \notin A,B$, то $m_i \notin A,B$, как и $0 + 0 = 0$. Если $m_i \in A,B$, то $m_i \notin A,B$, как и $1 + 1 = 0$.\\
		Так же заметим, что умножение также соответствует умножению в $F_2$.\\
		\\
		2) Рассмотрим следующий базис: все множества из одного элементаю. Заметим, что это базис: он порождающий, так как любое множество представимо в виде объединения одноэлементных множеств, он линейно независим, так как пусть один (содержащий элемент $m_i$) выражается через остальные, но заметим, что во всех векторах i-тая координата нулевая, а сумма нулей равна $0$, при этом в $m_i$ эта координата равна единице, противоречие.\\
		Размерность этого базиса -- $|M|$.\\
		\\
		3) Заметим, что если множество не лежит в объединении других, то в этом множестве есть элемент, не принадлежащий объединению (очевидно, что если все элементы из множества принадлежат объединению, то само множество принадлежит объединению). Откуда множества $X_i$ линейно независимы.
		
		
		\subsection{17}
		Пусть нам даны точки $(x_1,\ y_1),\ (x_2,\ y_2),\ (x_3,\ y_3),\ (x_4,\ y_4),\ (x_5,\ y_5)$\\
		Тогда возьмем уравнение $a_{(2\ 0)} x^2 + a_{(1\ 1)} xy + a_{(0\ 2)} y^2 + a_{(1\ 0)} x + a_{(0\ 1)} y + a_{(0\ 0)} = 0$ и подставим в него наши точки, получим:\\
		\begin{gather}
			\det
			\begin{vmatrix}
				x^2 & xy & y^2 & x & y & 1 \\
				x_1^2 & x_1y_1 & y_1^2 & x_1 & y_1 & 1 \\
				x_2^2 & x_2y_2 & y_2^2 & x_2 & y_2 & 1 \\
				x_3^2 & x_3y_3 & y_3^2 & x_3 & y_3 & 1 \\
				x_4^2 & x_4y_4 & y_4^2 & x_4 & y_4 & 1 \\
				x_5^2 & x_5y_5 & y_5^2 & x_5 & y_5 & 1 
			\end{vmatrix}
			= 0
		\end{gather}
		И эта матрица задает искомую прямую.
		
		\subsection{18}
		A)\\
		Покажем, что да. Пусть даны матрицы $A$ содержащее $a_{i\ j}$ и $B$ содержащее $b_{i\ j}$ (считаем, что матрицы $\mathbb{N} x \mathbb{N}$) (первый индекс -- столбец, второй -- строка).\\ 
		Заметим, что сумма $l$ строки матрицы $A \cdot B$ равна сумме произведений всех возможных пар чисел $(a_{i_a\ l};\ b_{l\ i_b})$, где $i$ - параметры. Cумма произведений пар $(a_{\alpha\ l};\ b_{l\ j_b})$ равна $a_{\alpha\ l}$, так как сумма $b_{l\ j_b}$ равна $1$. Тогда сумма всех произведений равна сумме чисел вида $a_{i_a\ l} = 1 \cdot 1 = 1$. 
		Аналогично заметим, что сумма $h$ столбца матрицы $A \cdot B$ равна сумме произведений всех возможных пар чисел $(a_{h\ i_a};b_{i_b\ h})$, где $i$ - параметры. И аналогично эта сумма равна $1$.
		\\
		B)\\
		Пусть существует такие $A$ и $B$: $(E - A) \cdot B = E$. Заметим, что в $(E - A)$ сумма во всех столбцах и строках равна $0$, откуда следует, что $(E - A) \cdot B$ обладает таким же свойством, что нетрудно видеть из доказательнства пункта (а). Тогда $(E - A)*B \ne E$, так как $E$ не обладает таким свойтвом. Откуда ответ -- да\\
		
		\subsection{19}
		(1) Заметим, что в любой момент времени количество красок каждого типа в банках равно $\frac{p}{10^q}$, где $p,q \in N \cap {0}$.
		Заметим, что множество таких чисел замкнуто относительно сложения, вычитания и деления. При этом переливание - следующая операция над парой векторов $a = (a_1,\ a_2,\ ... ,\ a_7), b = (b_1,\ b_2,\ ... ,\ b_7)$: $a \to a - \frac{a}{1 - |b|};\ b \to b + \frac{a}{1 - |b|}$, где $|x| = x_1 + x_2 + ... + x_7$. Откуда следует утверждение (1).\\
		Предположим в какой-то момент появилась банка с одинаковым отношением красок. Заметим, что в момент, когда она появилась, её объём краски в ней равен 1, то есть каждая краска имеет объём $\frac{1}{7} \ne \frac{p}{10^q}$, откуда следует, что такого момента не может быть, поэтому ответ к задаче -- нет.
		
		\subsection{20}	
		Пусть:\\
		\begin{gather*}
			A = 
			\begin{vmatrix}
				a_{0\ 0} & a_{1\ 0} & \cdots & a_{n-1\ 0} & a_{n\ 0}\\
				a_{0\ 1} & a_{1\ 1} & \cdots & a_{n-1\ 1} & a_{n\ 0}\\
				\vdots & \vdots & \ddots & \vdots & \vdots\\
				a_{0\ n-1} & a_{1\ n-1} & \cdots & a_{n-1\ n-1} & a_{n\ 0}\\
				a_{0\ n} & a_{1\ n} & \cdots & a_{n-1\ n} & a_{n\ n}\\
			\end{vmatrix}
		\end{gather*}
		Тогда $(E + A)(E + B) = E + A + B + A \cdot B$\\
		$A + A \cdot B = A(E + B)$\\
		$A + A^{m-1} \cdot C = 0$
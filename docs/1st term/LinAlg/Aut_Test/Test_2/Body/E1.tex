\newpage		
	\section{Задачи для подготовки к экзамену}	
		\subsection{}
		Найдите вектор скорости линии пересечения плоскостей 
		$9x_{1} - 5x_{2} + x_{3} = 1$ и $4x_{1} + 2x_{2} - 8x_{3} = 9$
		и какую-нибудь точку на этой линии. 
		\\
		 Найдем пересечения этих плоскостей:
		\begin{equation*}
			\begin{cases}
				9x_{1} - 5x_{2} + x_{3} = 1\\
				4x_{1} + 2x_{2} - 8x_{3} = 9
			\end{cases}
		\end{equation*}
		Запишем расширенную матрицу системы и с помощью элементарных преобразований приведем ее к ступенчатому виду:\\
		
		\begin{gather*}
			\bigg( \begin{array}{c c c | c}
			9 & -5 & 1 & 1\\
			4 & 2 & -8 & 9\\
			\end{array} \bigg) 
			\to 
			\bigg(\begin{array}{c c c | c}
			1 & -9 & 17 & -17\\
			4 & 2 & -8 & 9\\
			\end{array} \bigg) 
			\to 
			\bigg(\begin{array}{c c c | c}
			1 & -9 & 17 & -17\\
			0 & 38 & -76 & 77\\
			\end{array}\bigg) 
			\to
			\\
			\Bigg(
			\begin{array}{c c c | c}
			1 & -9 & 17 & -17\\
			\; & \; & \; & \;\\
			0 & 1 & -2 & \dfrac{77}{38}\\
			\end{array}\Bigg)
			\to
			\Bigg( \begin{array}{c c c | c}
			1 & 0 & -1 & \dfrac{47}{38}\\
			\; & \; & \; & \;\\
			0 & 1 & -2 & \dfrac{77}{38}\\
			\end{array} \Bigg)
		\end{gather*}
		Таким образом, 
		\begin{equation*}
			\begin{cases}
				x_{1} = x_{3} + \dfrac{47}{38}\\
				\\
				x_{2} = 2x_{3} + \dfrac{77}{38}\\
			\end{cases}
		\end{equation*}
		Два частных решения: $\Big(\dfrac{47}{38}, \; \dfrac{77}{38}, \; 0\Big), \; \Big(\dfrac{85}{38}, \; \dfrac{153}{38}, \; 1\Big)$. Прямая пересечения плоскостей: $\dfrac{x - \dfrac{47}{38}}{1} = \dfrac{y - \dfrac{77}{38}}{2} = \dfrac{z}{1}$. Тогда также частное решение: $(0, -\dfrac{17}{38}, -\dfrac{47}{38})$, направляющий вектор: $(1, 2, 1) \sim (38, 76, 38)$.\\
		\\
		\textbf{Ответ:} (38, 76, 38), $(0, -\dfrac{17}{38}, -\dfrac{47}{38})$
		
		
		
		\subsection{}		
		Напишите уравнение плоскости в $\mathbb{Q}^{3}$, проходящей через точку $(3, -9, 7)$ параллельно векторам $(5, 6, 0)$ и $(-3, 14, -10)$\\
		\\
		Пусть 
		$V_{1} = 
		\Bigg(\begin{array}{c}
		5\\ 6\\ 0
		\end{array}\Bigg),
		V_{2} = \Bigg(\begin{array}{c}
		-3\\ 14\\ -10
		\end{array}\Bigg), 
		M = (3, -9, 7)$ 
		\\ 
		Тогда 
		\begin{gather*}
			\begin{vmatrix}
				x_{1} - 3 & 5 & -3\\
				x_{2} + 9 & 6 & 14\\
				x_{3} - 7 & 0 & -10\\
			\end{vmatrix} = 0
		\end{gather*}
		Разложим по первому столбцу: 
		\begin{gather*}
			(-1)^{1 + 1} \cdot (x_{1} - 3) \cdot 
			\begin{vmatrix}
				6 & 14\\
				0 & -10\\
			\end{vmatrix}
			+ (-1)^{2 + 1} \cdot (x_{2} + 9) \cdot
			\begin{vmatrix}
				5 & -3\\
				0 & -10\\
			\end{vmatrix} 
			+ (-1)^{3 + 1} \cdot (x_{3} - 7) \cdot
			\begin{vmatrix}
				5 & -3\\
				6 & 14\\
			\end{vmatrix} 
			\\
			= (x_{1} - 3) \cdot (-60) - (x_{2} + 9) \cdot (-50) + (x_{3} - 7) \cdot 88\\
			= -60x_{1} + 180 + 50x_{2} + 450 + 88x_{3} - 616 = -60x_{1} + 50x_{2} + 88x_{3} - 14
		\end{gather*}
		Так как вычисленный определитель равен нулю, $-60x_{1} + 50x_{2} + 88x_{3} - 14 = 0$, то есть $-60x_{1} + 50x_{2} + 88x_{3} = 14$ 
		\textbf{Ответ:} $-60x_{1} + 50x_{2} + 88x_{3} = 14$ 
		
		
		
		\subsection{}
		Напишите уравнение плоскости в $\mathbb{Q}^{3}$, проходящей через точки $(5, -9, -2)$, $(4, -3, 6)$, $(4, 2, 4)$.\\
		\\
		 Рассмотрим векторы 
		\begin{gather*}
		V_{1} = 
		\Bigg(\begin{array}{c}
			x - 5\\ x + 9\\ x + 2
		\end{array}\Bigg)
		\quad 
		V_{2} = 
		\Bigg(\begin{array}{c}
			4-5\\ -3- (-9)\\ 6- (-2)\\
		\end{array}\Bigg)
		= 
		\Bigg(\begin{array}{c}
			-1\\ 6\\ 8\\
		\end{array}\Bigg)
		\quad 
		V_{3} = 
		\Bigg(\begin{array}{c}
		4-5\\ 2- (-9)\\ 4- (-2)\\
		\end{array}\Bigg)
		= 
		\Bigg(\begin{array}{c}
			-1\\ 11\\ 6\\
		\end{array}\Bigg)
		\end{gather*}
		Задача равносильна предыдущей для данных векторов.\\
		\\
		Тогда 
		\begin{gather*}
			\begin{vmatrix}
				x_{1} - 5 & -1 & -1\\
				x_{2} + 9 & 6 & 11\\
				x_{3} + 2 & 8 & 6\\
			\end{vmatrix} = 0
		\end{gather*}
		Разложим по первому столбцу: 
		\begin{gather*}
		(-1)^{1 + 1} (x_{1} - 5) 
		\begin{vmatrix}
			6 & 11\\
			8 & 6\\
		\end{vmatrix} 
		+ (-1)^{2 + 1} (x_{2} + 9) 
		\begin{vmatrix}
			-1 & -1\\
			8 & 6\\
		\end{vmatrix} 
		+ (-1)^{3 + 1} (x_{3} + 2) 
		\begin{vmatrix}
			-1 & -1\\
			6 & 11\\
		\end{vmatrix}
		= 0
		\end{gather*}
		То есть $-52(x_{1} - 5) - 2(x_{2} + 9) - 5(x_{3} + 2) = 0$. Значит, $-52x_{1} - 2x_{2} - 5x_{3} = -232$\\
		\\
		\textbf{Ответ:} $-52x_{1} - 2x_{2} - 5x_{3} = -232$. 
		
		
		
		\subsection{}
		\begin{enumerate}
		\item 
			\begin{gather*}
				\begin{vmatrix}
					-4 & 0 & 0 & 2\\
					-3 & -3 & 4 & 0\\
					3 & -3 & -5 & 0\\
					-3 & 0 & 0 & -3\\
				\end{vmatrix}
				=
				\begin{vmatrix}
					-4 & 0 & 0 & 2\\
					0 & -3 & 4 & -\dfrac{3}{2}\\
					0 & -3 & -5 & \dfrac{3}{2}\\
					0 & 0 & 0 & \dfrac{-9}{2}\\
				\end{vmatrix}
				=
				\begin{vmatrix}
					-4 & 0 & 0 & 2\\
					0 & -3 & 4 & -\dfrac{3}{2}\\
					0 & 0 & -9 & 3\\
					0 & 0 & 0 & \dfrac{-9}{2}\\
				\end{vmatrix}
				= -4 \cdot (-3) \cdot (-9) \cdot \dfrac{-9}{2} = 486
			\end{gather*}
			
		
		\item 
			\begin{gather*}
				\begin{vmatrix}
					0 & 1 & 0 & 5\\
					0 & 1 & -3 & -5\\
					0 & 1 & 4 & 0\\
					-4 & 0 & -5 & 0\\
				\end{vmatrix}
				 = 
				\begin{vmatrix}
					0 & 1 & 0 & 5\\
					0 & 0 & -3 & -10\\
					0 & 0 & 4 & -5\\
					-4 & 0 & -5 & 0\\
				\end{vmatrix}
				 = 
				\begin{vmatrix}
					-4 & 0 & -5 & 0\\
					0 & 0 & -3 & -10\\
					0 & 0 & 4 & -5\\
					0 & 1 & 0 & 5\\
				\end{vmatrix}
				= -4
				\begin{vmatrix}
					0 & -3 & -10\\
					0 & 4 & -5\\
					1 & 0 & 5\\
				\end{vmatrix}
				= -4
				\begin{vmatrix}
					1 & 0 & 5\\
					0 & 4 & -5\\
					0 & -3 & -10\\
				\end{vmatrix}
				\\
				= -4
				\begin{vmatrix}
					4 & -5\\
					-3 & -10\\
				\end{vmatrix} 
				= -4(-40 - 15) = -4 \cdot (-55) = -220
			\end{gather*}
		\end{enumerate}		
		\textbf{Ответ:} а) -232, б) -220.
		
		 
		
		\subsection{}
		Во всех пунктах будем пользоваться тем, что $A^{-1} = \dfrac{1}{|A|} \cdot A_{\phi}^{T}$, где $A_{\phi}^{T}$ -- транспонированная матрица алгебраических дополнений соответствующих элементов матрицы $A$. 
		\begin{enumerate}
		\item
			$A = \Bigg(\begin{array}{c c c}
				6 & -3 & -1\\
				-6 & 5 & -5\\
				-6 & 3 & 0\\
			\end{array}\Bigg)$ найти $A^{-1}$. 
			\\
			1. Найдем определитель матрицы $A$:
			\begin{gather*}
				\begin{vmatrix}
					6 & -3 & -1\\
					-6 & 5 & -5\\
					-6 & 3 & 0\\
				\end{vmatrix}
				=
				\begin{vmatrix}
					6 & -3 & -1\\
					0 & 2 & -6\\
					0 & 0 & -1\\
				\end{vmatrix}
				= 6
				\begin{vmatrix}
					2 & -6\\
					0 & -1\\
				\end{vmatrix}
				= -12
			\end{gather*}
			\\
			2. Найдем матрицу миноров:
			\begin{gather*}
				\begin{vmatrix}
					5 & -5\\
					3 & 0\\
				\end{vmatrix}
				= 15\qquad 
				\begin{vmatrix}
					-6 & -5\\
					-6 & 0\\
				\end{vmatrix}
				= -30\qquad
				\begin{vmatrix}
					-6 & 5\\
					-6 & 3\\
				\end{vmatrix}
				= 12\qquad\\
				\begin{vmatrix}
					-3 & -1\\
					3 & 0\\
				\end{vmatrix}
				= 3\qquad
				\begin{vmatrix}
					6 & -1\\
					-6 & 0\\
				\end{vmatrix}
				= -6\qquad
				\begin{vmatrix}
					6 & -3\\
					-6 & 3\\
				\end{vmatrix}
				= 0\qquad\\
				\begin{vmatrix}
					-3 & -1\\
					5 & -5\\
				\end{vmatrix}
				= 20\qquad
				\begin{vmatrix}
					6 & -1\\
					-6 & -5\\
				\end{vmatrix}
				 = -36\qquad
				\begin{vmatrix}
					6 & -3\\
					-6 & 5\\
				\end{vmatrix}
				= 12
			\end{gather*}
			Тогда 
			\begin{gather*}
				M = 
				\begin{vmatrix}
					15 & -30 & 12\\
					3 & -6 & 0\\
					20 & -36 & 12\\
				\end{vmatrix}
			\end{gather*}
			\\
			3. Найдем матрицу алгебраических дополнений $A_{\phi}:$\\
			Домножим элементы $M$ на $(-1)^{i + j}$
			\begin{gather*}
			A_{\phi} = 
				\begin{vmatrix}
					15 & 30 & 12\\
					-3 & -6 & 0\\
					20 & 36 & 12\\
				\end{vmatrix}
			\end{gather*}
			\\
			4. Найдем матрицу алгебраических дополнений $A_{\phi}^{T}:$
			\begin{gather*}
				A_{\phi}^{T} = 
				\begin{vmatrix}
					15 & -3 & 20\\
					30 & -6 & 36\\
					12 & 0 & 12\\
				\end{vmatrix} 
			\end{gather*}
			\\
			5. Вспомним, что 
			\begin{gather*}
				A^{-1} = \dfrac{1}{|A|} \cdot A_{\phi}^{T} = -\dfrac{1}{12} 
				\begin{vmatrix}
					15 & -3 & 20\\
					30 & -6 & 36\\
					12 & 0 & 12\\
				\end{vmatrix}
				 = 
				\begin{vmatrix}
					-\dfrac{5}{3} & \dfrac{1}{4} & -\dfrac{5}{3}\\
					\\
					\dfrac{5}{2} & \dfrac{1}{2} & -3\\
					\\
					-1 & 0 & -1\\
				\end{vmatrix} 
			\end{gather*}
		
		
		\item
			\begin{gather*}
				A = 
				\begin{vmatrix}
					6 & 2 & 0\\
					-6 & -2 & 5\\
					-5 & 3 & 1\\
				\end{vmatrix}
			\end{gather*} 
			\\
			1. Найдем $|A|$:
			\begin{gather*}
				\begin{vmatrix}
					6 & 2 & 0\\
					-6 & -2 & 5\\
					-5 & 3 & 1\\
				\end{vmatrix}
				=
				\begin{vmatrix}
					6 & 2 & 0\\
					0 & 0 & 5\\
					-5 & 3 & 1\\
				\end{vmatrix}
				 = -50 - 15 \cdot 6 = -140
			\end{gather*}
			\\
			2. Найдем матрицу миноров: 
			\begin{gather*}
				\begin{vmatrix}
					-2 & 5\\
					3 & 1\\
				\end{vmatrix}
				= -17\qquad
				\begin{vmatrix}
					-6 & 5\\
					-5 & 1\\
				\end{vmatrix}
				= 19\qquad
				\begin{vmatrix}
					-6 & -2\\
					-5 & 3\\
				\end{vmatrix}
				= -28\qquad
				\\
				\begin{vmatrix}
					2 & 0\\
					3 & 1\\
				\end{vmatrix}
				= 2\qquad
				\begin{vmatrix}
					6 & 0\\
					-5 & 1\\
				\end{vmatrix}
				= 6\qquad
				\begin{vmatrix}
					6 & 2\\
					-5 & 3\\
				\end{vmatrix}
				= 28\qquad
				\\
				\begin{vmatrix}
					2 & 0\\
					-2 & 5\\
				\end{vmatrix}
				= 10\qquad
				\begin{vmatrix}
					6 & 0\\
					-6 & 5\\
				\end{vmatrix}
				= 30\qquad
				\begin{vmatrix}
					6 & 2\\
					-6 & -2\\
				\end{vmatrix}
				= 0	
			\end{gather*}
			Запишем матрицу: 
			\begin{gather*}
				M = 
				\begin{vmatrix}
					-17 & 19 & -28\\
					2 & 6 & 28\\
					10 & 30 & 0\\
				\end{vmatrix}
			\end{gather*}
			\\
			3. Теперь запишем матрицу алгебраических дополнений: 
			\begin{gather*}
				A_{\phi} = 
				\begin{vmatrix}
					-17 & -19 & -28\\
					-2 & 6 & -28\\
					10 & -30 & 0\\
				\end{vmatrix}
			\end{gather*}
			Где каждый элемент умножили на $(-1)^{i + j}$\\
			\\
			4. Транспонируем полученную матрицу: 
			\begin{gather*}
				A_{\phi}^{T} =
				\begin{vmatrix}
					-17 & -2 & 10\\
					-19 & 6 & -30\\
					-28 & -28 & 0\\
				\end{vmatrix} 
			\end{gather*}
			\\
			5. 
			\begin{gather*}
				A^{-1} = \dfrac{1}{|A|} \cdot A_{\phi}^{T} = -\dfrac{1}{140} \cdot 
				\begin{vmatrix}
					-17 & -2 & 10\\
					-19 & 6 & -30\\
					-28 & -28 & 0\\
				\end{vmatrix}
				=
				\begin{vmatrix}
					-\dfrac{17}{140} & -\dfrac{1}{70} & \dfrac{1}{14}\\
					\\
					-\dfrac{19}{140} & \dfrac{3}{70} & -\dfrac{3}{14}\\
					\\
					-\dfrac{1}{5} & -\dfrac{1}{5} & 0\\
				\end{vmatrix} 
			\end{gather*}
			
		\item
			\begin{gather*}
				A = 
				\begin{vmatrix}
					3 & 6 & 4\\
					6 & -2 & 0\\
					3 & 5 & 5\\
				\end{vmatrix}
			\end{gather*} 
			\\
			1. Найдем определитель $A$:
			\begin{gather*}
				\begin{vmatrix}
					3 & 6 & 4\\
					6 & -2 & 0\\
					3 & 5 & 5\\
				\end{vmatrix}
				=
				\begin{vmatrix}
					3 & 6 & 4\\
					0 & -14 & -8\\
					0 & -1 & 1\\
				\end{vmatrix}
				= 3
				\begin{vmatrix}
					-14 & -8\\
					-1 & 1\\
				\end{vmatrix}
				= 3(-14 - 8) = -66
			\end{gather*} 
			\\
			2. Найдем матрицу миноров: 
			\begin{gather*}
				\begin{vmatrix}
					-2 & 0\\
					5 & 5\\
				\end{vmatrix}
				= -10\qquad
				\begin{vmatrix}
					6 & 0\\
					3 & 5\\
				\end{vmatrix}
				= 30\qquad
				\begin{vmatrix}
					6 & -2\\
					3 & 5\\
				\end{vmatrix}
				= 36\qquad
				\\
				\begin{vmatrix}
					6 & 4\\
					5 & 5\\
				\end{vmatrix}
				= 10\qquad
				\begin{vmatrix}
					3 & 4\\
					3 & 5\\
				\end{vmatrix}
				= 3\qquad
				\begin{vmatrix}
					3 & 6\\
					3 & 5\\
				\end{vmatrix}
				= -3\qquad
				\\
				\begin{vmatrix}
					6 & 4\\
					-2 & 0\\
				\end{vmatrix}
				= 8\qquad
				\begin{vmatrix}
					3 & 4\\
					6 & 0\\
				\end{vmatrix}
				= -24\qquad
				\begin{vmatrix}
					3 & 6\\
					6 & -2\\
				\end{vmatrix}
				= -42\quad
			\end{gather*}
			Тогда 
			\begin{gather*}
				M = 
				\begin{vmatrix}
					-10 & 30 & 36\\
					10 & 3 & -3\\
					8 & -24 & -42\\
				\end{vmatrix}
			\end{gather*}
			\\
			3. Найдем матрицу алгебраических дополнений: 
			\begin{gather*}		
				A_{\phi} =
				\begin{vmatrix}
					-10 & -30 & 36\\
					-10 & 3 & 3\\
					8 & 24 & -42\\
				\end{vmatrix} 
			\end{gather*}
			\\
			4. Транспонируем полученную матрицу: 
			\begin{gather*}
				A_{\phi}^{T} =
				\begin{vmatrix}
					-10 & -10 & 8\\
					-30 & 3 & 24\\
					36 & 3 & -42\\
				\end{vmatrix}
			\end{gather*}
			\\
			5. Вычисляем 
			\begin{gather*}
				\dfrac{1}{|A|} A_{\phi}^{T} = -\dfrac{1}{66} 
				\begin{vmatrix}
					-10 & -10 & 8\\
					-30 & 3 & 24\\
					36 & 3 & -42\\
				\end{vmatrix}
				=
				\begin{vmatrix}
					\dfrac{5}{33} & \dfrac{5}{33} & -\dfrac{4}{33}\\
					\\
					\dfrac{5}{11} & -\dfrac{1}{22} & -\dfrac{4}{11}\\
					\\
					-\dfrac{6}{11} & -\dfrac{1}{22} & \dfrac{7}{11}\\
				\end{vmatrix}
			\end{gather*}
		\end{enumerate}
	
		\textbf{Ответ:}\\
		\begin{gather*}
			\text{а)}
			\begin{vmatrix}
				-\dfrac{5}{3} & \dfrac{1}{4} & -\dfrac{5}{3}\\
				\\
				\dfrac{5}{2} & \dfrac{1}{2} & -3\\
				\\
				-1 & 0 & -1\\
			\end{vmatrix}
			\quad
			\text{б)}
			\begin{vmatrix}
				-\dfrac{17}{140} & -\dfrac{1}{70} & \dfrac{1}{14}\\
				\\
				-\dfrac{19}{140} & \dfrac{3}{70} & -\dfrac{3}{14}\\
				\\
				-\dfrac{1}{5} & -\dfrac{1}{5} & 0\\
			\end{vmatrix}
			\quad
			\text{в)}
			\begin{vmatrix}
				\dfrac{5}{33} & \dfrac{5}{33} & -\dfrac{4}{33}\\
				\\
				\dfrac{5}{11} & -\dfrac{1}{22} & -\dfrac{4}{11}\\
				\\
				-\dfrac{6}{11} & -\dfrac{1}{22} & \dfrac{7}{11}\\
			\end{vmatrix}
		\end{gather*}
		
		
		
		\subsection{}
		Найдите собственные числа, укажите какие-нибудь базисы в собственных и корневых подпространствах и выясните, диагонализуемы ли линейные операторы $\mathbb{Q}^{3} \to \mathbb{Q}^{3}$, заданные в стандартном базисе матрицами: 
		\begin{enumerate}
		\item	
			\begin{gather*}
				\begin{vmatrix}
					13 & 75 & -21\\
					-16 & -108 & 31\\
					-48 & -330 & 95\\
				\end{vmatrix}
			\end{gather*}
			\\
			1. Собственные числа:\\
			Найдем характеристический многочлен: 
			\begin{gather*}
				|\lambda E - A| = 0 \Leftrightarrow 
				\begin{vmatrix}
					\lambda & 0 & 0\\
					0 & \lambda & 0\\
					0 & 0 & \lambda\\
				\end{vmatrix}
				-
				\begin{vmatrix}
					13 & 75 & -21\\
					-16 & -108 & 31\\
					-48 & -330 & 95\\
				\end{vmatrix}
				=
				\begin{vmatrix}
					\lambda - 13 & -75 & 21\\
					16 & \lambda + 108 & -31\\
					48 & 330 & \lambda - 95\\
				\end{vmatrix}
				= 0 
			\end{gather*}
			Преобразуем определитель: 
			\begin{gather*}
				\begin{vmatrix}
					\lambda - 13 & -75 & 21\\
					16 & \lambda + 108 & -31\\
					0 & 6 - 3\lambda & \lambda - 2\\
				\end{vmatrix}
				= 0 
			\end{gather*}
			Раскроем по первому столбцу:
			\begin{gather*}
				(\lambda - 13)
				\begin{vmatrix}
					\lambda + 108 & -31\\
					6 - 3\lambda & \lambda - 2\\
				\end{vmatrix}
				- 16 
				\begin{vmatrix}
					-75 & 21\\
					6 - 3\lambda & \lambda - 2\\
				\end{vmatrix}
				\\
				= (\lambda - 13)((\lambda + 108)(\lambda - 2) + 31(6 - 3 \lambda)) - 16(-75 \lambda + 150 - 126 + 63\lambda) = 0
			\end{gather*}
			\\
			то есть 
			\begin{gather*}
				(\lambda - 13)(\lambda^{2} + 13\lambda - 30) + 192\lambda - 384 =\\
				\lambda^{3} + 13 \lambda^{2} - 30\lambda - 13\lambda^{2} - 169\lambda + 390 + 192\lambda - 384 =\\
				\lambda^{3} - 7\lambda + 6 = 0
			\end{gather*}
			Собственные числа: $\lambda_{1} = 1, \lambda_{2} = -3, \lambda_{3} = 2$\\
			\\
			2. Найдем собственные векторы. Для этого будем подставлять найденные значения $\lambda_{i}$ в $(A - \lambda E)x = 0$. 
			\begin{enumerate}
			\item $\lambda = 1$\\
				\begin{gather*}
					\begin{vmatrix}
						12 & 75 & -21\\
						-16 & -109 & 31\\
						-48 & -330 & 94\\
					\end{vmatrix}
					\to
					\begin{vmatrix}
						12 & 75 & -21\\ 
						0 & -9 & 3\\
						0 & -3 & 1\\
					\end{vmatrix}
					\to
					\begin{vmatrix}
						12 & 75 & -21\\ 
						0 & -3 & 1\\
						0 & -3 & 1\\
					\end{vmatrix}
					\to
					\begin{vmatrix}
						1 & 0 & \dfrac{1}{3}\\ 
						0 & -1 & \dfrac{1}{3}\\
					\end{vmatrix}
				\end{gather*}
				Получается, $x_{1} = -\dfrac{1}{3} x_{3}$, $x_{2} = \dfrac{1}{3} x_{3}$, где $x_{3}$ -- свободная переменная.\\
				Тогда собственный вектор $\begin{vmatrix} 1\\ -1\\ -3\\ \end{vmatrix}$ 
				
			\item $\lambda = -3$\\
				\begin{gather*}
					\begin{vmatrix}
						16 & 75 & -21\\
						-16 & -105 & 31\\
						-48 & -330 & 98\\
					\end{vmatrix}
					\to
					\begin{vmatrix}
						16 & 75 & -21\\
						0 & 3 & -1\\
						0 & 3 & -1\\
					\end{vmatrix}
					\to
					\begin{vmatrix}
						16 & 75 & -21\\
						0 & 3 & -1\\
					\end{vmatrix}
					\to
					\begin{vmatrix}
						16 & 0 & 4\\
						0 & 3 & -1\\
					\end{vmatrix}
					\to
					\begin{vmatrix}
						4 & 0 & 1\\
						0 & 3 & -1\\
					\end{vmatrix}
					\to
					\begin{vmatrix}
						1 & 0 & \dfrac{1}{4}\\
						0 & 1 & -\dfrac{1}{3}\\
					\end{vmatrix} 
				\end{gather*}
				Таким образом, $x_{1} = -\dfrac{1}{4}x_{3}$, $x_{2} = \dfrac{1}{3}x_{3}$, где $x_{3}$ -- свободная переменная.\\
				Тогда собственный вектор $\begin{vmatrix} -3\\ 4\\ 12 \end{vmatrix}$ 
				
			\item $\lambda = 2$\\
				\begin{gather*}
					\begin{vmatrix}
						11 & 75 & -21\\
						-16 & -110 & 31\\
						-48 & -330 & 93\\
					\end{vmatrix}
					\to
					\begin{vmatrix}
						11 & 75 & -21\\
						-16 & -110 & 31\\ 
						0 & 0 & 0\\
					\end{vmatrix}
					\to
					\begin{vmatrix}
						11 & 75 & -21\\
						-16 & -110 & 31\\
					\end{vmatrix}
					\to
					\\
					\begin{vmatrix}
						1 & \dfrac{75}{11} & -\dfrac{21}{11}\\
						\\ 
						0 & \dfrac{-10}{11} & \dfrac{5}{11}\\
					\end{vmatrix}
					\to
					\begin{vmatrix}
						11 & 75 & -21\\
						0 & 1 & -\dfrac{1}{2}\\
					\end{vmatrix}
					\to
					\begin{vmatrix}
						11 & 0 & \dfrac{33}{2}\\
						\\
						0 & 1 & -\dfrac{1}{2}\\
					\end{vmatrix}
					\to
					\begin{vmatrix}
						1 & 0 & \dfrac{3}{2}\\
						\\
						0 & 1 & -\dfrac{1}{2}\\
					\end{vmatrix}
				\end{gather*}
				Таким образом, $x_{1} = -\dfrac{3}{2}x_{3}$, $x_{2} = \dfrac{1}{2}x_{3}$, а $x_{3}$ -- свободная переменная.\\
				Тогда собственный вектор, например, $\begin{vmatrix}
				3\\ -1\\ -2\\
				\end{vmatrix}$ 
			\end{enumerate}
		\end{enumerate}
		
		
		\subsection{}
		Напишите такую вещественную $2 \times 2$ матрицу $A$, что
		\begin{enumerate}
			\item
				\begin{gather*}
					A^{5} = 
					\begin{vmatrix}
						-31 & -16\\
						56 & 29
					\end{vmatrix}
				\end{gather*}
				
				1. Характеристический многочлен:
				\begin{gather*}
					\begin{vmatrix}
						t + 31 & -16\\ 
						56 & t - 29\\
					\end{vmatrix} 
					= 0 \Leftrightarrow 
					t^{2} + 2t - 3= 0 \Leftrightarrow 
					t = -3, t = 1
				\end{gather*}
				
				2. $\sqrt[5]{A^{5}} = aA + bE$, где $at + b$ в точках $-3, \; 1$ принимает те же значения, что и в $\sqrt[5]{t}:$
				\begin{gather*}
					\begin{cases}
						-3a + b = -\sqrt[5]{3}\\
						a + b = 1
					\end{cases}
				\\
					a = 
					\dfrac{
					\begin{vmatrix} 
						-\sqrt[5]{3} & 1\\ 
						1 & 1\\ 
					\end{vmatrix}
					}{
					\begin{vmatrix} 
						-3 & 1\\ 
						1 & 1\\ 
					\end{vmatrix}} 
					= \dfrac{\sqrt[5]{3} + 1}{4}
					\quad 
					b = 
					\dfrac{
					\begin{vmatrix} 
						-3 & -\sqrt[5]{3}\\
						1 & 1\\
					\end{vmatrix}
					}{
					\begin{vmatrix} 
						-3 & 1\\
						1 & 1\\
					\end{vmatrix}} 
					= \dfrac{3 - \sqrt[5]{3}}{4}
				\end{gather*}
				
				3. Помним, что $\sqrt[5]{A^{5}} = aA + bE$, то есть искомая матрица:
				\begin{gather*}
					\sqrt[5]{A^{5}} = 
					\dfrac{\sqrt[5]{3} + 1}{4} 
					\begin{vmatrix} 
						-31 & -16\\ 
						56 & 29\\ 
					\end{vmatrix} 
					+ \dfrac{3 - \sqrt[5]{3}}{4} 
					\begin{vmatrix} 
						1 & 0\\
						0 & 1\\ 
					\end{vmatrix}
					\\
					= 
					\begin{vmatrix} 
						\dfrac{-31 \sqrt[5]{3} - 31}{4} & -4\sqrt[5]{3} - 4\\
						14 \sqrt[5]{3} + 14 & \dfrac{29 \sqrt[5]{3} + 29}{4} 
					\end{vmatrix}
					+
					\begin{vmatrix} 
						\\ 
						\dfrac{3 - \sqrt[5]{3}}{4} & 0\\
						0 & \dfrac{3 - \sqrt[5]{3}}{4}\\
					\end{vmatrix}
					=
					\begin{vmatrix}
						-8\sqrt[5]{3} - 7 & -4\sqrt[5]{3} - 4\\
						14\sqrt[5]{3} + 14 & 7\sqrt[5]{3} + 8\\
					\end{vmatrix}
				\end{gather*}
				
			\item
				\begin{gather*}
					A^{4} = 
					\begin{vmatrix} 
						-3 & -4\\
						4 & 5\\
					\end{vmatrix}
				\end{gather*}
				1. Характеристический многочлен:
				\begin{gather*}
					\begin{vmatrix} 
						t + 3 & 4\\
						- 4 & t - 5\\
					\end{vmatrix}
					= 0 \Leftrightarrow 
					t^{2} + 3t - 5t - 15 + 16 = t^{2} - 2t + 1 = (t - 1)^{2} = 0 \Leftrightarrow 
					t = 1
				\end{gather*}
				
				2. $\sqrt[4]{A^{4}} = aE + b$, где $at + b$ принимает в $t = 1$ то же значение, что и в $\sqrt[4]{A^{4}}:$
				
				$a + b = 1$.
				
				Так как $1$ -- двукратное собственное число, рассмотрим производные от обеих частей равенства $at + b = \sqrt[4]{t}:$ $t = 1$ также должен быть удовлетворять этому равенству. Получается, $a = \dfrac{1}{4}\sqrt[3]{t} \overset{t = 1}{=} \dfrac{1}{4}$. 
				
				$a = \dfrac{1}{4}$. Из $a + b = 1$ имеем $b = \dfrac{3}{4}$. Таким образом, $aA + bE = \dfrac{1}{4}A + \dfrac{3}{4}E$, то есть 
				\begin{gather*}
					\dfrac{1}{4} 
					\begin{vmatrix} 
						-3 & -4\\ 
						4 & 5\\
					\end{vmatrix}
					+ \dfrac{3}{4} 
					\begin{vmatrix} 
						1 & 0\\ 
						0 & 1\\ 
					\end{vmatrix}
					= 
					\begin{vmatrix} 
						-\dfrac{3}{4} & -1\\
						1 & \dfrac{5}{4}\\
					\end{vmatrix} 
					+
					\begin{vmatrix}
						\dfrac{3}{4} & 0\\
						0 & \dfrac{3}{4}
					\end{vmatrix}
					= 
					\begin{vmatrix} 
						0 & -1\\
						1 & 2\\
					\end{vmatrix}
				\end{gather*}
			\item
				\begin{gather*}
					A^{3} = 
					\begin{vmatrix}
						-128 & 25\\
						-650 & 127\\
					\end{vmatrix}
				\end{gather*}
				
				1. Характеристический многочлен: 
				\begin{gather*}
					\begin{vmatrix} 
						t + 128 & -25\\
						-650 & t - 127
					\end{vmatrix}
					= 0 \Leftrightarrow
					t^{2} + 128t - 127t - 127 \cdot 128 - 650 \cdot 25 = t^{2} + t - 6 = 0 \Leftrightarrow
					t = -3, t = 2
				\end{gather*}
				
				2. $\sqrt[3]{A^{3}} = aA + bE$, где $at + b$ принимает в $t = -3, t = 2$ те же значения, что и в $\sqrt[3]{t}$. 
				\begin{gather*}
					\begin{cases}
						-3a + b = \sqrt[3]{-3},\\
						2a + b = \sqrt[3]{2};\\
					\end{cases}
				\\
					a = 
					\dfrac{
					\begin{vmatrix} 
						\sqrt[3]{-3} & 1\\
						\sqrt[3]{2} & 1 
					\end{vmatrix}
					}{
					\begin{vmatrix}
						-3 & 2\\ 
						1 & 1\\ 
					\end{vmatrix}}
					= \dfrac{\sqrt[3]{2} - \sqrt[3]{-3}}{5}
					\quad
					b = 
					\dfrac{
					\begin{vmatrix}
						-3 & \sqrt[3]{3}\\
						2 & \sqrt[3]{2}\\
					\end{vmatrix}
					}{
					\begin{vmatrix}
						-3 & 2\\
						1 & 1
					\end{vmatrix}} 
					= \dfrac{3\sqrt[3]{2} + 2\sqrt[3]{-3}}{5}
				\end{gather*}
				Тогда 
				\begin{gather*}
					aA + bE = \dfrac{\sqrt[3]{2} - \sqrt[3]{-3}}{5}
					\begin{vmatrix}
						-128 & 25\\
						-650 & 127\\
					\end{vmatrix}
					+ \dfrac{3\sqrt[3]{2} + 2\sqrt[3]{-3}}{5}
					\begin{vmatrix}
						1 & 0\\
						0 & 1\\
					\end{vmatrix}
					\\
					=
					\begin{vmatrix}
						-128\dfrac{\sqrt[3]{2} - \sqrt[3]{-3}}{5} & 25\dfrac{\sqrt[3]{2} - \sqrt[3]{-3}}{5}\\
						-650\dfrac{\sqrt[3]{2} - \sqrt[3]{-3}}{5} & 127\dfrac{\sqrt[3]{2} - \sqrt[3]{-3}}{5}\\
					\end{vmatrix}
					+
					\begin{vmatrix} 
						\dfrac{3\sqrt[3]{2} + 2\sqrt[3]{-3}}{5} & 0\\
						0 & \dfrac{3\sqrt[3]{2} + 2\sqrt[3]{-3}}{5}
					\end{vmatrix}
					\\
					=
					\begin{vmatrix}
						-25 \sqrt[3]{2} + 26 \sqrt[3]{-3} & 5\sqrt[3]{2} - 5\sqrt[3]{-3}\\
						-130 \sqrt[3]{2} + 130 \sqrt[3]{-3} & 26 \sqrt[3]{2} - 25\sqrt[3]{-3}\\
					\end{vmatrix}
				\end{gather*}
				
			\item
				\begin{gather*}
					A^{2} =
					\begin{vmatrix}
						-18 & -5\\
						80 & 22
					\end{vmatrix} 
				\end{gather*}
				
				1. Характеристический многочлен:
				\begin{gather*}
					\begin{vmatrix}
						t + 18 & 5\\
						-80 & t - 22
					\end{vmatrix}
					= 0 \Leftrightarrow
					t^{2} + 18t - 22t - 22 \cdot 18 + 5 \cdot 80 = t^{2} - 4t + 4 = (t - 2)^{2} = 0 \Leftrightarrow
					t = 2
				\end{gather*}
				
				2. $\sqrt{A^{2}} = aA^{2} + bE$, где $at + b$ принимает то же значение, что и $\sqrt{t}$, в $t = 2$. $t = 2$ -- двукратное собственное число, значит, если мы возьмем производную от обеих частей равенства $at + b = \sqrt{t}$, $t = 2$ по-прежнему будет ему удовлетворять. Запишем это: 
				\begin{gather*}
					\begin{cases}
						2a + b = \sqrt{2}\\
						a = \dfrac{1}{2\sqrt{2}}
					\end{cases}
				\end{gather*}
				Получается, $a = \dfrac{1}{2\sqrt{2}}, \; b = \sqrt{2} - \dfrac{1}{\sqrt{2}} = \dfrac{1}{\sqrt{2}}$.
				\begin{gather*}
					A = 
					\dfrac{1}{2\sqrt{2}}A^{2} + \dfrac{1}{\sqrt{2}}E =
					\dfrac{1}{2\sqrt{2}} 
					\begin{vmatrix} 
						-18 & -5\\
						80 & 22
					\end{vmatrix} 
					+ \dfrac{1}{\sqrt{2}}
					\begin{vmatrix}
						1 & 0\\
						0 & 1
					\end{vmatrix}
					\\
					= 
					\begin{vmatrix}
						-\dfrac{9}{\sqrt{2}} & -\dfrac{5}{2\sqrt{2}}\\
						\\
						\dfrac{40}{\sqrt{2}} & \frac{11}{\sqrt{2}}\\
					\end{vmatrix}
					+
					\begin{vmatrix} 
						\dfrac{1}{\sqrt{2}} & 0\\
						\\
						0 & \dfrac{1}{\sqrt{2}}\\
					\end{vmatrix}
					=
					\begin{vmatrix}
						-\dfrac{8}{\sqrt{2}} & -\dfrac{5}{2\sqrt{2}}\\
						\\
						\dfrac{40}{\sqrt{2}} & \dfrac{12}{\sqrt{2}}\\
					\end{vmatrix}
					=
					\begin{vmatrix}
						-4\sqrt{2} & -\dfrac{5 \sqrt{2}}{4}\\
						\\
						20 \sqrt{2} & 6 \sqrt{2}\\
					\end{vmatrix}
				\end{gather*}
		\end{enumerate}
		
		\textbf{Ответ.} 
		\begin{gather*}
			\text{а)}
				\begin{vmatrix}
					-8\sqrt[5]{3} - 7 & -4\sqrt[5]{3} - 4\\
					14\sqrt[5]{3} + 14 & 7\sqrt[5]{3} + 8
				\end{vmatrix}
			\\
			\text{б)}
				\begin{vmatrix}
					0 & -1\\
					1 & 2
				\end{vmatrix} 
			\\
			\text{в)}
				\begin{vmatrix}
					-25 \sqrt[3]{2} + 26 \sqrt[3]{-3} & 5\sqrt[3]{2} - 5\sqrt[3]{-3}\\
					-130 \sqrt[3]{2} + 130 \sqrt[3]{-3} & 26 \sqrt[3]{2} - 25\sqrt[3]{-3}
				\end{vmatrix}
			\\
			\text{г)}
				\begin{vmatrix}
					-4\sqrt{2} & -\dfrac{5 \sqrt{2}}{4}\\
					\\
					20 \sqrt{2} & 6 \sqrt{2}\\
				\end{vmatrix}
		\end{gather*}
	
		
		
		\subsection{}
		Над полем $\mathbb{Q}$ найдите минимальные многочлены следующих матриц и выясните, диагонализуемы ли они. 
		\\
		 \textbf{Способ:}\\
		берем удобный вектор $\to$ применяем оператор $\to \ldots \to$ линейная зависимость между $e, \; F_{v}, \; F_{v}^{2}, \; \ldots \to$ проверяем: $\nu_{F, \; v}(F) = 0?$ 
		
		$\bullet \; = 0: \nu_{F}(t) = \nu_{F, \; V}(t)$ -- минимальный многочлен;
		
		$\bullet \; \neq 0:$ повторить процедуру на $\text{im}\ \nu_{F, \; v}(F)$. Тогда $\nu_{F}(t) = \nu_{F, v}(t) \cdot \ldots$\\ 
		\\
		а) 
		\begin{gather*}
			\begin{vmatrix}
				1 & 0 & 2 & 2\\
				0 & 1 & 0 & 0\\
				0 & 0 & -2 & -1\\
				0 & 0 & 1 & 0
			\end{vmatrix}
		\end{gather*}
		$e_{3} = \begin{vmatrix} 0\\ 0\\ 1\\ 0\\ \end{vmatrix}$ -- прообраз $v = \begin{vmatrix} 2\\ 0\\ -2\\ 1 \end{vmatrix}$. Применяем $F_{e_{3}}$ еще раз: умножаем исходную матрицу на $F_{e_{3}}$\\
		\begin{gather*}
			F_{e_{3}}^{2} = 
			\begin{vmatrix}
				0\\ 0\\ 3\\ -2
			\end{vmatrix}\quad 
			F^{3}_{e_{3}} = 
			\begin{vmatrix}
				2\\ 0\\ -4\\ 3
			\end{vmatrix} 
		\end{gather*}
		
		Параллельно записываем по столбцам матрицы $e_{3},\ F_{e_{3}},\ F^{2}_{e_{3}},\ F^{3}_{e_{3}}$, чтобы увидеть линейную зависимость. Видим, что $F^{3}_{e_{3}}$ линейно выражается через другие столбцы матрицы
		\begin{gather*}
			\begin{vmatrix}
				0 & 2 & 0 & 2\\
				0 & 0 & 0 & 0\\
				1 & -2 & 3 & -4\\
				0 & 1 & -2 & 3
			\end{vmatrix}
		\end{gather*}
		
		Мы ищем линейную зависимость: ищем $\nu_{F, v}$, его коэффициенты -- коэффициенты линейной зависимости между записанными векторами.\\
		Заметим, что 
		\begin{gather*}
			F^{3}_{e_{3}} - F_{e_{3}} = 
			\begin{vmatrix} 
				0\\ 0\\ -2\\ 2
			\end{vmatrix}
			= -F^{2}_{e_{3}} + e_{3} \quad \Leftrightarrow \quad 
			F^{3}_{e_{3}} = -F^{2}_{e_{3}} + F_{e_{3}} + e_{3} \quad \Leftrightarrow \quad 
			F^{3}_{e_{3}} + F^{2}_{e_{3}} - F_{e_{3}} - e_{3}
		\end{gather*}	
		Таким образом, мы нашли многочлен: $t^{3} + t^{2} - t - 1$\\
		Единственный способ проверить минимальность: подставить в многочлен $F$. Найденный многочлен аннулирует прообраз третьего вектора (так как мы брали его), а также первого и второго, так как делится на $t - 1$, ведь $\nu_{F}(t) = \underset{v \in V}{\text{НОК}} \nu_{F, \; v}(t)$, а для первого и второго столбца $F_{e_{1}} - e_{1} = \begin{vmatrix} 0\\ 0\\ 0\\ 0\\ \end{vmatrix}, F_{e_{2}} - e_{2} = \begin{vmatrix} 0\\ 0\\ 0\\ 0\\ \end{vmatrix}$, то есть аннулирующий многочлен для них -- $t - 1$\\
		Поэтому нам остается проверить, что многочлен аннулирует прообраз четвертого вектора, то есть проделать все то же самое для него. 
		\begin{gather*}
			e_{4} = 
			\begin{vmatrix}
				0\\ 0\\ 0\\ 1
			\end{vmatrix}
		\quad
			F_{e_{4}} =
			\begin{vmatrix}
				2\\ 0\\ -1\\ 0
			\end{vmatrix}
		\quad
			F^{2}_{e_{4}} = 
			\begin{vmatrix} 
				0\\ 0\\ 2\\ -1
			\end{vmatrix}
		\quad
			F^{3}_{e_{4}} = 
			\begin{vmatrix}
				2\\ 0\\ -3\\ 2
			\end{vmatrix}
		\end{gather*}
		
		Заметим, что $F^{3}_{e_{4}} + F^{2}_{e_{4}} - F_{e_{4}} - e_{4} = \begin{vmatrix} 0\\ 0\\ 0\\ 0\\ \end{vmatrix}$. Значит, многочлен аннулирует и прообраз четвертого вектора. Таким образом, $t^{3} + t^{2} - t - 1$.\\
		\\\\
		\textbf{б)}
		\begin{gather*}
			\begin{vmatrix}
				1 & 0 & 0 & 0\\
				0 & 1 & 0 & 0\\
				-1 & 1 & 1 & 1\\
				-2 & 2 & -1 & 3
			\end{vmatrix}
		\end{gather*}
		
		Возьмем 
		\begin{gather*}
			e_{1} = 
			\begin{vmatrix}
				0\\ 0\\ 0\\ 1\\
			\end{vmatrix}
		\quad
			F_{e_{1}} =
			\begin{vmatrix}
				1\\ 0\\ -1\\ -2
			\end{vmatrix}
		\quad
			F^{2}_{e_{1}} =
			\begin{vmatrix}
				1\\ 0\\ -4\\ -7
			\end{vmatrix}
		\quad 
			F^{3}_{e_{1}} = 
			\begin{vmatrix}
				1\\ 0\\ -12\\ -19
			\end{vmatrix}
		\end{gather*}
		
		Параллельно записываем вычисленные значения по столбцам матрицы:
		\begin{gather*}
			\begin{vmatrix}
				0 & 1 & 1 & 1\\
				0 & 0 & 0 & 0\\ 
				0 & -1 & -4 & -12\\ 
				1 & -2 & -7 & -19
			\end{vmatrix}
		\end{gather*}
		Если векторы линейно зависимы (это нужно проверять после вычисления каждого нового применения оператора), то 
		\begin{gather*}
			\begin{vmatrix}
				0 & 1 & 1 & 1\\
				0 & -1 & -4 & -12\\
				1 & -2 & -7 & -19
			\end{vmatrix}
			\begin{vmatrix}
				\lambda_{1}\\
				\lambda_{2}\\
				\lambda_{3}\\
				\lambda_{4}
			\end{vmatrix}
			= 0
		\\
			\lambda_{1} = 
			\begin{vmatrix}
				1 & 1 & 1\\
				-1 & -4 & -12\\
				-2 & -7 & -19
			\end{vmatrix}
			= -4
		\quad 
			\lambda_{1} = -
			\begin{vmatrix}
				0 & 1 & 1\\
				0 & -4 & -12\\
				1 & -7 & -19
			\end{vmatrix}
			= -8
		\end{gather*}
		Аналогично, закрыв третий и четвертый столбцы с знаками $+$ и $-$ соответственно, вычисляем $\lambda_{3} = 5, \lambda_{4} = 1$.
		\begin{gather*}
			\begin{vmatrix} 
				0 & 1 & 1\\
				0 & -1 & -12\\
				1 & -2 & -19
			\end{vmatrix}
			=
			\begin{vmatrix}
				0 & 0 & -11\\
				0 & -1 & -12\\
				1 & 0 & 5
			\end{vmatrix}
			= 11
		\quad
			-
			\begin{vmatrix} 
				0 & 1 & 1\\
				0 & -1 & -4\\
				1 & -2 & -7
			\end{vmatrix}
			= -
			\begin{vmatrix}
				0 & 0 & -3\\
				0 & -1 & -4\\
				1 & 0 & 1
			\end{vmatrix}
			= -3
		\end{gather*}
		
		Получили, что $-8 F_{e_{1}} + 11 F_{e_{1}}^{2} - 3 F_{e_{1}}^{3} = 0$. Значит, многочлен -- $-3t^{3} + 11t^{2} - 8t$. 
		
		*не сходится ответ* 
		
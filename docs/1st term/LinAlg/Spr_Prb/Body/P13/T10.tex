\subsection*{ГЛ13 10*}
Докажем сперва вспомогательные факты:\\
1)\\
Из точки $O$ проведены касательные $OA$ и $OB$ к эллипсу с фокусами $F_1$ и $F_2$. Тогда $\angle AOF_1 = \angle BOF_2$ и $\angle AF_1O = \angle BF_1O$.\\
Пусть точки $G_1$ и $G_2$ симметричны $F_1$ и $F_2$ относительно прямых $OA$ и $OB$ соответственно. Точки $F_1$, $B$ и $G_2$ лежат на одной прямой и $F_1G_2 = F_1B + BG_2 = F_1B + BF_2$. Треугольники $G_2F_1O$ и $G_1F_2O$ имеют равные стороны. Поэтому $\angle G_1OF_1 = \angle  G_2OF_2$ и $\angle AF_1O = \angle AG_1O = \angle BF_1O$.\\
\\
2)\\
Длина диагонали прямоугольника описанного вокруг эллипса постоянна\\
Пусть $O$ -- вершина данного прямоугольника, $OA$ и $OB$ -- касательные к эллипсу. Тогда треугольник $F_1OG_2$ (из пункта $(1)$) прямоугольный. Следовательно, $F_1O^2 + F_2O^2 = F_1G_2^2$ -- константа. Если $M$ -- центр эллипса, то величина 
\begin{gather*}
	O M^{2}=\frac{1}{4}\left(2 F_{1} O^{2}+2 F_{2} O^{2}-F_{1} F_{2}^{2}\right)
\end{gather*}
Тоже константа\\
\\
Тогда заметим, что из вершин рассматриваемых прямоугольников эллипс виден под прямым углом. Тогда рассмотрим точку пересечения диагоналей этих прямоугольников, она совпадает с серединой отрезка, соединяющего фокусы эллипса, следовательно искомое ГМТ это окружность с центром в середине отрезка соединяющего фокусы и диаметром равным длине диагонали прямоугольника из пункта $(2)$
		
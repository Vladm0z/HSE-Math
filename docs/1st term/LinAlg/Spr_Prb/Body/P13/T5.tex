\subsection*{ГЛ13 5}
Рассмотрим параболу $ax^2 = y$ (все остальные параболы получаются сдвигом данной параболы), ее фокус имеет координаты $(0, \frac{1}{4a})$, а уравнение директрисы $y = -\frac{1}{4a}$.\\
Рассмотрим касательную в точке $(x_0,y_0)$, она имеет формулу $y = 2ax_0x - ax_0^2$, тогда точка симметричная фокусу относительно касательной имеет координаты
\begin{gather*}
	y = -\frac{1}{2ax_0}x + \frac{1}{4a}\\
	(x_0, -\frac{1}{4a})
\end{gather*}
Эта точка очевидно лежит на $y = -\frac{1}{4a}$, то есть на диретрисе 
	
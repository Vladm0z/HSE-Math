\subsection*{ГЛ10 2}
Вспомогательное утверждение\\
При любых $a_1, a_2 \in \mathbb{F}^*_q$ квадратичная форма $a_1x_1^2 + a_2x_2^2$ на двумерном координатном пространстве $\mathbb{F}^2_q$  принимает все значения из поля $\mathbb{F}_q$.
\vskip 0.2in
\noindent
Доказательство\\
При любых фиксированных $a_1, a_2 \in \mathbb{F}^*_q$ и $b \in \mathbb{F}_q$ чисел вида $a_1x_1^2$ и чисел вида $b - a_2x_2^2$ имеется ровно по
\begin{gather*}
	1+\frac{q-1}{2} = q+\frac{q-1}{2}
\end{gather*} 
штук, так как при отображении
\begin{gather*}
	f: \mathbb{F}^*_q \rightarrow \mathbb{F}^*_q\\ 
	f: x \rightarrow x^2
\end{gather*}
$x^2=1$ имеется 2 корня, следовательно $\displaystyle |\text{im} f|=\frac{q-1}{2}$. Значит, множества $a_1x_1^2$ и $b - a_2x_2^2$ имеют общий элемент, и $f(x_1, x_2)=b$

\vskip 0.2in
\noindent
Предложение\\
Всякая квадратичная форма в пространстве размерности $\geqslant 3$ над полем $\mathbb{F}^q$ имеет ненулевой изотропный вектор.
\vskip 0.2in
\noindent
Доказательство\\
по теореме 14.2 из лекции 15, квадратичная форма в подходящем базисе записывается как 
\begin{gather*}
	a_1x_1^2+a_2x_2^2+a_3x_1^3+ \ldots
\end{gather*}
Если $a_1=0$ или $a_2=0$, то вектор $(1, 0, 0, \ldots)$ или вектор $(0, 1, 0, \ldots)$ изотропен. Если $a_1a_2 \neq 0$, то по доказанной лемме найдутся такие $\lambda, \mu \in \mathbb{F}_{q},$ что $a_{1} \lambda^{2}+a_{2} \mu^{2}=-a_{3}$. Тогда вектор $(\lambda, \mu, 1,0, \ldots)$ изотропен.\\
\\
Из предложения вытекает следствие: анизотропные квадратичные формы имеются только в размерностях 1 и 2.\\ 
По предложению 15.4 из лекции 15, в размерности 2 квадратичная форма $x_{1}^{2}+x_{2}^{2}$ анизотропна если и только если $q \equiv-1\ (\bmod 4)$ а форма $x_{1}^{2}+\varepsilon x_{2}^{2}$ анизотропна если и только если $q \equiv 1\ (\bmod 4)$
		
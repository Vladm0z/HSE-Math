\subsection*{ГЛ7 3}
Очевидно, что в каждой размерности $\mathbb{R}^{n}$ есть хотя бы $n$ гиперплоскостей, относительно которых симметричен куб. Далее заметим, что плоскость относительно которой куб симметричен также может проходить через пары противоположных ребер (в размерности $\mathbb{R}^{n}$ это гиперплоскостей размерности $\mathbb{R}^{n-2}$ ровно $n \cdot (n-1) \cdot 2$).\\
Заметим, что для $\mathbb{R}^n$ есть ровно $\dim(\mathbb{R}^n) + \frac{n \cdot (n-1) \cdot 2}{2} = n + n \cdot (n-1) = n \cdot n$.\\
Тогда для 4: $4 \cdot 4 = 16$ 
		
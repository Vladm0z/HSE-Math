\subsection*{ГЛ7 4}
\textbf{(А)}\\
Существует $n+1$ точка вида $(0,\ldots, 0, 1, 0, \ldots, 0)$ ($1$ единица и $n$ нулей). В обоих подпространствах выберем базисные векторы. Каждое ребро будет соответствовать вектору $(0, \ldots, 0, 1, 0, \ldots, 0, -1, 0, \ldots, 0)$ -- одна $1$ и $-1$, а также $n-1$ ноль. Для $n-2$ грани векторы с фиксированной $1$ и нулями на местах, где у ребер $1$ или $-1$, а на оставшихся местах одна $-1$ и нули. Пример:
\begin{gather*}
	(0, \ldots, 0, 1, 0, \ldots, 0, -1, 0, \ldots, 0)\ \times\ (n-2)\\
	(0, \ldots, 0, 1, -1)
\end{gather*}
Скалярное произведение векторов из $(n-2)$ мерного пространства с $(0, \ldots, 0, 1, -1)$ будет равно $0$, а следовательно ребро перпендикулярно $(n-2)$ мерной грани.
\subsection*{ГЛ7 7}
\begin{enumerate}
\item[А] $\sigma_{\pi_{1}} \circ \sigma_{\pi_{2}}=\rho_{v, \varphi}$\\
	Две симметрии относительно плоскостей эквивалентны поороту когда они не параллельны $\pi_1 \not\parallel \pi_2$, $v$ -- вектор совпадающий с осью пересечения плоскостей $\pi_1, \pi_2$, $\varphi$ равен удвоенному углу между $\pi_1, \pi_2$
			
\item[Б] $\sigma_{\pi_{1}} \circ \sigma_{\pi_{2}}=\tau_{v}$\\
	Две симметрии относительно плоскостей эквивалентны сдвигу на вектор в случае если $\pi_1 \parallel \pi_2$, $v \perp \pi_1, \pi_2$ и равен удвоенному расстоянию между $\pi_1, \pi_2$
		
\item[В] $\sigma_{\pi} \circ \varrho_{u, \varphi} \circ \sigma_{\pi}=\varrho_{v, \psi}$\\
	Вектор $v$ симметричен вектору $u$ относительно плоскости $\pi$, $\varphi = \psi$
				
\item[Г] $\varrho_{u, \varphi} \circ \varrho_{w, \psi}=\tau_{v} \circ \varrho_{v, \vartheta}$
	\begin{gather*}
		\cos \frac{\vartheta}{2}=\cos \frac{\varphi}{2} \cos \frac{\psi}{2}-\vec{u} \cdot \vec{w} \sin \frac{\varphi}{2} \sin \frac{\psi}{2}\\
		\vec{v}=\vec{u} \sin \frac{\varphi}{2} \cos \frac{\psi}{2}+\vec{w} \sin \frac{\psi}{2} \cos \frac{\varphi}{2}+\vec{u} \times \vec{w} \sin \frac{\phi}{2} \sin \frac{\psi}{2}
	\end{gather*}
			
		
\item[Д] $\varrho_{u, \varphi} \circ \sigma_{\pi_{1}} \circ \varrho_{u,-\varphi}=\sigma_{\pi_{2}}$\\
		
\item[Е] $\varrho_{u, \varphi} \circ \sigma_{\pi_{1}}=\sigma_{\pi_{2}}$\\
		
\item[Ж] $\tau_{u_{2}} \circ \sigma_{\pi_{2}} \circ \tau_{u_{1}} \circ \sigma_{\pi_{1}}=\tau_{v} \circ \varrho_{v, \varphi}$\\
\end{enumerate}
		
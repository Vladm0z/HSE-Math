\subsection*{ГЛ7 1}
Назовем вектора $\vec{a}, \vec{b}, \vec{c}$, проведем через $\vec{a}, \vec{b}$ плоскость и перпендикуляр к ней (назовем его $l$). Спроецируем $\vec{c}$ на плоскость и $l$, получим $\vec{c} = \vec{c_1} + \vec{c_2}$. Заметим, что $(\vec{a}, \vec{c}) = (\vec{a}, \vec{c_1}) + (\vec{a}, \vec{c_2}) = (\vec{a}, \vec{c_1})$, так как $\vec{a} \perp \vec{c_2}$, аналогично $(\vec{b}, \vec{c}) = (\vec{b}, \vec{c_1})$.\\
Значит углы между всеми парами векторов $\leqslant \frac{\pi}{2}$. Рассмотрим наибольший угол (тогда заметим, что вектор, не являющийся стороной этого угла, лежит внутри угла). Тогда все 3 вектора $\vec{a}, \vec{b}$, $\vec{c}$ вписаны в какую-то четверть, а следовательно если выбрать еще один ортогональный вектор, то $\vec{a}, \vec{b}$, $\vec{c}$ будут вписаны в октант.
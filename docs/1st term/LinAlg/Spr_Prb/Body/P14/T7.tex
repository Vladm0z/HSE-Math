\subsection*{ГЛ14 7*}
Докажем задачу методом индукции.\\
Заметим, что при $n=2$ можно дополнить матрицу $D$ строкой единиц сверху, столбцом единиц слева и $0$ в верхнем левом углу и тогда определитель полученной матрицы будет эквивалентен формуле Герона.
\begin{gather*}
	\begin{vmatrix}
		0 & 1 & 1 & 1\\
		1 & 0 & d_{12}^2 & d_{13}^2\\
		1 & d_{12}^2 & 0 & d_{23}^2\\
		1 & d_{13}^2 & d_{23}^2 & 0
	\end{vmatrix}
	= d_{12}^4 + d_{13}^4 + d_{23}^4 + 2(d_{12}^2 d_{13}^2 + d_{12}^2 d_{23}^2 + d_{13}^2 d_{23}^2) =\\
	\\
	(-1)^{3}(d_{12}+d_{13}+d_{23})(d_{12}+d_{13}-d_{23})(d_{12}-d_{13}+d_{23})(-d_{12}+d_{13}+d_{23})\\
\end{gather*}
Откуда следует что он неотрицателен в случае когда треугольник с данными сторонами существует (в случае вырожденного треугольника определитель равен 0).\\
Тогда предположим что утверждение о существовании симплекса доказано для $n = 1, \ldots, k$, докажем тогда что утверждение выполнено и для $n = k+1$\\
Рассмотрим матрицу $D$ размера $(k+1) \times (k+1)$, тогда, по предположению индукции, мы знаем, что существуют $k$ мерные симплексы с вершинами $p_0, \ldots, p_{i-1}, p_{i+1}, \ldots, p_{k+1}$, так как для них выполнено утверждение. 
\subsection*{ГЛ9 3}
Рассмотрим набор $X=\{x_1,x_2,\dots,x_{d+2}\}\subset \mathbf{R}^d$ из $d+2$ точек в пространстве размерности $d$. Тогда существует множество множителей $a,\ldots,a_{d+2}$, не все из которых равны нулю, тогда решим систему линейных уравнений
\begin{gather*}
	\sum\limits_{i=1}^{d+2} a_i x_i=0,\quad \sum\limits_{i=1}^{d+2} a_i=0
\end{gather*}
Так как есть $d+2$ переменных и только $d+1$ уравнений которым они должны удовлетворять (по одному для каждой координаты точек, вместе с окончательным уравнением, требующим, чтобы сумма множителей была равна нулю). Зафиксируем некоторое ненулевое решение  $a_{1},\ldots,a_{d+2}$. Пусть $I\subseteq X$ Будет множеством точек с положительными множителями, и $J=X\setminus I$ с отрицательными или 0. Тогда $I$ и $J$ формируют требуемое разбиение точек на два подмножества с пересекающимися выпуклыми оболочками.\\		
выпуклые множества $I$ и $J$ должны пересекаться, так как у них есть общая точка
\begin{gather*}
	p= \sum\limits_{i\in I}\frac{a_i}{A} x_i=\sum\limits_{j\in J}\frac{-a_j}{A}x_j\\
	\text{где}\\
	A=\sum\limits_{i\in I} a_i=-\sum\limits_{j\in J} a_j.
\end{gather*}
Левая часть формулы для $p$ выражает эту точку как выпуклую комбинацию точек в $I$, а правая часть выражает ее как выпуклую комбинацию точек в $J$. Следовательно, $p$ принадлежит обеим выпуклым оболочкам.\\
		
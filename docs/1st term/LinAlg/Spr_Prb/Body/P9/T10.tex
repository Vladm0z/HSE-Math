\subsection*{ГЛ9 10}
Минимальная по включению грань -- это пересечение опорных гиперплоскостей. Заметим, что пересечение аффинных подпространств тоже аффинно. 
\vskip 0.2in
\noindent
Теперь покажем, что $\forall \, v \in \sigma \cap(-\sigma), \ v$ также лежит в этом аффинном подпространстве. Действительно, если $v \in \sigma \cap(-\sigma)$, то вся прямая, натянутая на $v$, лежит в $\sigma$. Она параллельна каждому опорному подпространству (так как иначе она бы пересекла его и не содержалась бы в $\sigma$). Следовательно $v$ лежит в аффинном подпространстве.
		
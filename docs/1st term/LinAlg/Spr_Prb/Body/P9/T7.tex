\subsection*{ГЛ9 7*}
Лемма Пала дает покрышку лучше, чем теорема Юнга, поэтому, доказав ее, мы докажем и теорему Юнга.\\
Проведем $4$ опорные прямые($2$ пары параллельных прямых, под углом в $60^{\circ}$), построим еще $2$ под углом $60^{\circ}$ ко всем предведущим и проведем к ним высоты из точек пересечения предведущих пар. Рассмотрим модуль разности длин этих высот, по-непрерывности при каком-то положении прямых он равен $0$ (так как при повороте на $180^{\circ}$ мы пройдем через $0$) -- при $0$ получим правильный шестиугольник с расстоянием между противоположными сторонами равном $1$, расстояние между противолежащими вершинами в нем $\frac{2}{\sqrt{3}}$ -- очевидно он вписывается в окружность такого же диаметра, что и требовалось	
	

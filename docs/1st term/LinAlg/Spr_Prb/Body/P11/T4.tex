\subsection*{ГЛ11 4}
Пусть прямые  $a_1,a_2,a_3$ проходят через точку A,  $a_1^{\prime},a_2^{\prime},a_3^{\prime}$ проходят через точку $A^{\prime}$.  $a_1$ пересекает  $a_2^{\prime}$ и  $a_3^{\prime}$ в точках B и C,  $a_2$ пересекает  $a_1^{\prime}$ и  $a_3^{\prime}$ в точках $C^{\prime}$ и $Z$,  $a_3$ пересекает  $a_1^{\prime}$ и  $a_2^{\prime}$ в точках $B^{\prime}$ и $X$. Тогда прямые $BC^{\prime}, B^{\prime}C$ и $XZ$ пересекаются в одной точке или параллельны.\\
\\
Двойственное утверждение равносильно самой теореме Паппа и доказывается по аналогии
		
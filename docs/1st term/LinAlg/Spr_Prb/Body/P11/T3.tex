\subsection*{ГЛ11 3}
Пусть точка $o$ -- точка пересечения прямых, на которых лежат точки $a_1$, $b_1$, $c_1$ и $a_2$, $b_2$, $c_2$.\\		
Рассмотрим пересечения прямых:\\		
\begin{gather*}
	a_1b_2\cap a_2b_1=x\\		
	a_1c_2\cap a_2c_1=y\\		
	c_1b_2\cap c_2b_1=z	
\end{gather*}
Теперь применим проективное отображение, переводящее прямую $xy$ на бесконечность. Тогда $a_1c_2\parallel a_2c_1$.\\		
Так как $x \to \infty:\ a_1b_2\parallel a_2b_1$. Теперь необходимо доказать, что $b_1c_2\parallel b_2c_1$.\\		
Рассмотрим подобные треугольники.
\begin{gather*}
	\bigtriangleup oa_1c_2\sim \ \bigtriangleup oc_1a_2 \Rightarrow \frac{oc_1}{oa_2}=\frac{oa_1}{oc_2} \Rightarrow oa_1\cdot oa_2=oc_1\cdot oc_2\\		
	\bigtriangleup oa_1b_2\sim \ \bigtriangleup ob_1a_2 \Rightarrow \frac{ob_1}{oa_2}=\frac{oa_1}{ob_2} \Rightarrow oa_1\cdot oa_2=ob_1\cdot ob_2
\end{gather*}
Отсюда следует, что $\frac{ob_1}{oc_1}=\frac{ob_2}{oc_2} \Rightarrow \bigtriangleup oc_1b_2\sim \bigtriangleup ob_1c_2$ (треугольники подобны) $\Rightarrow b_1c_2\parallel b_2c_1$.\\		
	Что и требовалось доказать.
		
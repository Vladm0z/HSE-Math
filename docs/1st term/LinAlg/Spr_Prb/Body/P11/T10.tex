\subsection*{ГЛ11 10}
Рассмотрим коническое сечение. Пусть каждая вершина треугольника $A_1B_1C_1$ является полюсом соответственной стороны треугольника $ABC$ (см. рис. ниже). Докажем, что прямые $AA_1$, $BB_1$, $CC_1$ пересекаются в одной точке $O$. Если фиксировать вершины $A_1$ и $B_1$, a $C_1$ перемещать по прямой $B_1C_1$, то поляра $AB$ этой вершины опишет пучок $(A)$ с центром $A$, высекая на $BC$ прямолинейный ряд $(BC)$ точек, проективный прямолинейному ряду $(B_1C_1)$, который пробегает $C_1$. Пучки прямых $(B_1)$ и $(C)$, соответственно перспективные проективным прямолинейным рядам $(BC)$ и $(B_1C_1)$, проективны. Кроме того, общая прямая $CB_1$ этих пучков сама себе соответствует. Поэтому пучки $(B_1)$ и $(C)$ перспективны. Прямая $AA_1$ является осью перспективы, так как прямая $CA$ пучка $(C)$ соответствует прямой $B_1A$ пучка $(B_1)$, а прямая $CA_1$ -- прямой $B_1A_1$. В самом деле, полярой точки $M = CA \cap B_1C_1$ является прямая $AB_1$, а поэтому в установленном с помощью пучка $(A)$ проективном соответствии рядов $(BC)$ и $(B_1C_1)$ точке $M$ ряда $(B_1C_1)$ соответствует точка $K = BC \cap AB_1$, а значит, прямой $CM = CA$ пучка $(C)$ соответствует прямая $B_1K = B_1A$ пучка $(B_1)$. Соответствие прямых $CA_1$ и $B_1A_1$ в этих пучках доказывается аналогично. Поскольку прямые $CC_1$ и $B_1B$ являются соответственными в перспективных пучках, то они пересекаются на оси перспективы $AA_1$.
		
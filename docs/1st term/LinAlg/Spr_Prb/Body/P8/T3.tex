\subsection*{ГЛ8 3}
\subsubsection*{\textbf{А}}
$(u_i, fu_i) = (fu_i)_i\ \Rightarrow Bu(f) = \sum\limits^{n}_{i=1} (fu_i)_i$\\
Выберем новый ортогональный базис\\
$v_j = \sum\limits^{n}_{i=1} x_{ij}u_i\qquad \sum\limits^{n}_{i=1} x^{2}_{ij} = 1$\\
$(v_1, fv_1) = \sum\limits^{n}_{i=1} x^2_{i1} (u_i, fu_i)$\\
$Bu(f) = \sum\limits^{n}_{j=1}x^2_{1j}(u_1,fu_1) + \sum\limits^{n}_{j=1} x^2_{2j} (u_2, fu_2) + \ldots + \sum\limits^{n}_{j=1} x^2_{nj} (u_n, fu_n)\ \Rightarrow$ не зависит
	
\subsubsection*{\textbf{Б}}
Сформулируем вспомогательную теорему:\\
$\forall$ самосопряженного оператора $F$ в $E^n$ существует ортонормальный базис, состоящий из собственных векторов $F$\\
(теорема о нормальном базисе), доказательство приведено в 12 лекции\\ %если найду время, то я ее сюда кину
\\
Согласно теореме и пункту (А) выберем в $V$ базис, состоящий из собственных векторов\\
$f:\ V \to V$ каждому собственному значению сопоставим собственный вектор\\
Откуда $fu_i = \lambda_{i} u_i\quad \forall i = 1, \ldots, n\qquad \Rightarrow (u_i, fu_i) = \lambda_{i}$, где $u_i$ имеет вид $(0, \ldots, \underset{i}{1}, \ldots, 0)$\\
$\max_u B_u(A) = \sum\limits^{r}_{j = 1} \alpha_j$ так как $\alpha_{1} \geqslant \ldots \geqslant \alpha_{n}$\\
		
		
\subsubsection*{\textbf{В}}
Единичная сфера $\sum\limits^{r}_{i=1} x^2_i = 1$\\
$x = \sum\limits^{r}_{i=1} x_i u_i$\\
$qf(x) = (x_1 u_1 + \ldots + x_r u_r, \lambda_1 x_1 u_1 + \ldots + \lambda_r x_r u_r) = \lambda_1 x^2_1 + \ldots + \lambda_r x^2_r$\\
$x^2_1 + \ldots + x^2_r = 1\ \Rightarrow\ x^2_i = 1 - (x_1^2 + \ldots + x^2_{i-1} + x^2_{i+1} + \ldots + x^2_r)$\\
Найдем экстремумы функции $(\frac{d}{dx_i}\quad i = 1, \ldots, r)$\\
$qf(x) = \lambda_1 + (\lambda_2 - \lambda_1)x^2_2 + \ldots + (\lambda_r - \lambda_1)x^2_r$
$\lambda_2 = \lambda_1$ или $x_2 = 0$ откуда $qf(x) = \lambda_1 (x^2_1 + x^2_2)$, а так как $x^2_1 = 1 - x^2_2$ откуда $qf(x) = \lambda_1$
\begin{gather*}
	\begin{cases}
		2(\lambda_2 - \lambda_1)x_2 = 0\\
		2(\lambda_3 - \lambda_1)x_3 = 0\\
		\ldots\\
		2(\lambda_r - \lambda_1)x_r = 0
	\end{cases}
\end{gather*}
Следовательно экстремум достигается при 
\begin{gather*}
	\begin{cases}
		x_i = 1\quad \forall i = 1, \ldots, r\\
		x_j = 0\quad \text{для всех остальных } x
	\end{cases}
\end{gather*}
и равен $qf(x) = \lambda_i$ откуда $m_u(f) = \lambda_{\min}$ из $\lambda_j$ -- собстенные значения собственных векторов подпространства $U$\\
$\max m_u(f) = \lambda_{n-r+1}$ (собственные значения в подпространстве $\lambda_n, \lambda_{n-1}, \ldots$)\\
$\min_W M_W (f) = \lambda_{n-r+1}$ ($\min$ собственные значения $\lambda_1, \ldots, \lambda_{n-r+1}$)
	
		
\subsubsection*{\textbf{Г}}
$u = \sum\limits^{n-1}_{i = 1} x_i u_i$\\
$(u, fu) = \lambda_1 x^2_1 + \ldots + \lambda_{n-1} x^2_{n-1}$\\
$(u, Fu) = (x_1 u _1 + \ldots + x_{n-1} u_{n-1},\ x_1 h(u_1) + \ldots + x_{n-1} h(u_{n-1}))$\\
$(u, fu) = (u, hu)$ откуда $h$ имеет те же собственные значения что и $f$\\
$\lambda_{n}$ значение у $h$: либо 0, либо $-\lambda_{n}$, так как у $h$ те же базисные вектора $e_1, \ldots, e_{n-1}$ что и у $f$ 
	
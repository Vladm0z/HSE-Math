\newpage
	\section{ЛИСТ 2}
		\subsection{1}
		Сделаем замену:
		$x - \frac{\pi}{2} = y$, тогда:
		\begin{gather*}
			\lim_{y \to 0} \biggl( y \cdot \tan (y + \frac{\pi}{2}) \biggl) = 
			\lim_{y \to 0} \biggl( y \cdot \frac{\sin(y + \frac{\pi}{2})}{\cos(y + \frac{\pi}{2})} \biggl) = 
			\lim_{y \to 0} \biggl( y \cdot \frac{\cos(y)}{ \sin(y)} \biggl) = \lim_{y \to 0} (\cos(y) ) = 
			1
		\end{gather*}
	
		\subsection{2}
		A)\\
		$\sum^{\infty}_{n = 1} n z^n$\\
		ряд знакочередуется, сходится, если $|z| < 1$\\
		$\lim_{n \to \infty} n z^n = 0$\\
		1) при $z \ne 0:\ |(n + 1)z^{n + 1}| > |n z^n| \ \Rightarrow$ возрастает, предел $\ne 0$\\
		2) при $z = 0:\ n z^n = 0 \ \Rightarrow \ \lim_{n \to \infty} nz^n = 0 \ \Rightarrow$ ряд сходится\\
		3) при $|z| < 1$ -- ряд сходится абсолютно и нет: $\lim_{n \to \infty} \frac{n+1}{n} = z \in (-1,\ 1) \ \Rightarrow$ сходится\\
		$z = 1:\ \sum^{\infty}_{n = 1} n$\\
		$z = -1: \ \sum^{\infty}_{n = 1} n(-1)^n$\\
		абсолютно:\\
		$\lim |n z^n| = 0$ 
		\\ \\
		B)\\
		$\sum^{\infty}_{n = 1} \frac{z^n}{n^2}$\\ 
		$\lim _ { n \rightarrow \infty } \bigm| \frac { \bigm( \frac { z^{n + 1} } { ( n + 1 ) ^ { 2 } } \bigm) } { \bigm( \frac { z ^ { n } } { n ^ { 2 } } \bigm) } \bigm| =
		\lim_{n \to \infty} \bigm| \frac{z}{(n + 1)^2} \cdot n^2 \bigm| = |z| \lim_{n \to \infty} \bigm| \frac{1}{(1 + \frac{1}{n})^2} \bigm| = |z|$\\
		Интервал сходимости: $|z| \leq 1$\\
		1)\\
		$z = -1:\ \sum^{\infty}_{n = 1} \frac{(-1)^n}{n^2}:\ \lim_{n \to \infty}|a_n| = \frac{1}{n^2} = 0$ -- члены ряда уменьшаются по модулю, предел равен $0$\\
		$\frac{1}{(n + 1)^2} < \frac{1}{n^2}\ \Rightarrow$ убывание монотонно. Следовательно, ряд сходится(признак Лейбница).\\
		$\sum^{\infty}_{n = 1} |a_n| \ = \ \sum^{\infty}_{n = 1} \frac{1}{n^2}$ -- сходится. И тогда $\sum^{\infty}_{n = 1} \frac{(-1)^n}{n^2}$ сходится абсолютно.\\
		2)\\
		$z = 1:\ \sum^{\infty}_{n = 1} \frac{1}{n^2}$ -- сходится (абсолютно) и ряд из модулей.\\
		\\
		Следовательно ряд сходится абсолютно при $z \in [-1,\ 1]$\\
		По признаку Лейбница: ряд знакочередуется и члены ряда убывают по модулю (так как $\lim \bigm| \frac{a_{n + 1}}{a_{n}} \bigm| = 0$ при $|z| < 1$)\\
		$z \geq 0 \ \Rightarrow$ ряд сходится (так как сходится абсолютно) при $|z| < 1$
		\\ \\
		\subsection{3}
		
		\subsection{4}
		$(a,\ b) \subset \mathbb{R}$\\
		$f:\ A \to C,\ a \in \mathbb{R}$\\
		\\
		A)\\
		$c \in (a,\ b)$
		$M = \{f(x)\ | \ x \in (c,\ b)) \}$. Множество непусто, так как $c < b$, и ограничено снизу, так как $\forall\ x > c:\ f(x) \geq f(c)$\\
		Пусть $\inf M = \gamma$. $\gamma$ -- правы предел. \\
		$\gamma = f(c + 0)$\\
		По определению точной нижней грани, $\exists \delta > 0:\ f(c + \delta) < \gamma + \varepsilon$\\
		$\forall \ x \in (c,\ b):\ \gamma \leq f(x) \ \Rightarrow \ \forall \ x \in (c,\ c + \delta):\ \gamma \leq f(x) < \gamma + \varepsilon \qquad |\gamma - f(x)| < \varepsilon$
		\\ \\
		B)\\
		\\ \\
		\subsection{5}
		
		\subsection{6}
		Заметим, что
		\begin{gather*}
			\sin(x) = \\
			2 \sin (\frac{x}{2}) \cos (\frac{x}{2}) = \\
			2^2 \sin (\frac{x}{4}) \cos (\frac{x}{4}) \cos (\frac{x}{2}) = \\
			2^3 \sin (\frac{x}{8}) \cos (\frac{x}{8}) \cos (\frac{x}{4}) \cos (\frac{x}{2}) = \\
			\vdots \\
			2^n \sin (\frac{x}{2^n}) \cos (\frac{x}{2^n}) ... \cos (\frac{x}{4}) \cos (\frac{x}{2})\\
			\Rightarrow
			\sin(x) = 2^n \sin (\frac{x}{2^n}) \cos (\frac{x}{2^n}) ... \cos (\frac{x}{4}) \cos (\frac{x}{2})\\
			\Rightarrow
			\frac{\sin(x)}{2^n \sin (\frac{x}{2^n})} = \cos (\frac{x}{2^n}) ... \cos (\frac{x}{4}) \cos (\frac{x}{2})\\
		\end{gather*}
		Тогда
		\begin{gather*}
		\lim_{n \to \infty} \biggl( \cos (\frac{x}{2^n}) ... \cos (\frac{x}{4}) \cos (\frac{x}{2}) \biggl) = 
		\lim_{n \to \infty} \biggl( \frac{\sin(x)}{2^n \sin (\frac{x}{2^n})} \biggl)
		\end{gather*}
		Откуда
		\begin{gather*}
			\lim_{n \to \infty} \biggl( \frac{\sin(x)}{2^n \sin (\frac{x}{2^n})} \biggl) = 
			\lim_{n \to \infty} \biggl( \frac{\sin(x)}{2^n \sin (\frac{x}{2^n}) \cdot \frac{\frac{x}{2^n}}{\frac{x}{2^n}}} \biggl) =
			\lim_{n \to \infty} \biggl( \frac{\sin(x)}{2^n \cdot \frac{x}{2^n} \frac{\sin (\frac{x}{2^n})}{\frac{x}{2^n}}} \biggl) =
			\lim_{n \to \infty} \biggl( \frac{\sin(x)}{\frac{2^n \cdot x}{2^n} \frac{\sin (\frac{x}{2^n})}{\frac{x}{2^n}}} \biggl) = \\
			\frac{\sin(x)}{x}
		\end{gather*}
		\subsection{7}
		1)\\
		\begin{gather*}
		\lim_{x \to +\infty} (1 + \frac{a}{x})^x \quad t = \frac{1}{x}\\
		\lim_{t \to 0} (1 + at)^{\frac{1}{t}} = \lim_{t \to 0} ((1 + at)^{\frac{1}{ta}})^{a} = \lim_{t \to 0} ((1 + at)^{\frac{1}{ta}})^{\lim_{t \to 0} a} = e^a 
		\end{gather*}
		\\
		2)\\
		Пусть $x = y \cdot a$.\\
		$\lim (1 + \frac{a}{x})^x = \lim (1 + \frac{1}{y})^{y \cdot a} = \exp(1)^a = \exp(a)$
		
		\subsection{8}
		\begin{comment}Заметим, что $\cos(x)$ при $x \to 0$, равен $1 - \frac{x^2}{2} + o(x^4)$, то есть при $n \to \inf: \ \cos(\frac{x}{\sqrt{n}}) = 1 - \frac{x^2}{2n} + o(\frac{1}{n^2})$\\
		Тогда $\cos^n(\frac{x}{\sqrt{n}}) = (1 - \frac{x^2}{2n} + o(\frac{1}{n^2}))^n = (1 - \frac{x^2}{2} + \frac{x^4}{2^2} - \frac{x^6}{2^3} + ... ) + o(\frac{1}{n})$\\ 
		Откуда $\cos^n(\frac{x}{\sqrt{n}}) \to (1 - \frac{x^2}{2} + \frac{x^4}{2^2} - \frac{x^6}{2^3} + ... ) = \frac{1}{1 + \frac{x^2}{2} }$
		\end{comment}
		
		A)\\
		$\pi(n - 1) < x_n < \pi n,\ n \in \mathbb{Z}$ Тогда $\pi(n-1),\ \pi n$ -- асимптоты, между которыми лежит часть графика, т.е. раз в период пересечение -- период равен $\pi$\\
		$\pi - \frac{\pi}{n} < \frac{x_n}{n} < \pi$\\
		$\lim_{n \to \infty} (\pi - \frac{\pi}{n}) = \pi,\ \lim_{n \to \infty} \pi = \pi$, по лемме о 2 милиционерах: $\lim_{n \to \infty} \frac{x_n}{n} = \pi$
		\\ \\
		B)\\
		$\pi(n - 1) < x_n < \pi n,\ n \in \mathbb{Z}$\\
		Сделаем более точную оценку сверху $y = x$ при $x > 0,\ y > 0\ \Rightarrow$ "ветви" пересекаются левее нуля координат, т.е. $x_n < \pi n - \frac{\pi}{2}$ (нули в $\frac{\pi}{2} + \pi k,\ k \in \mathbb{Z}$).
		\begin{gather*}
			\pi (n - 1) < x_n < \pi (n - \frac{1}{2})\\
			-\pi < x_n - \pi n < -\frac{\pi}{2}\\
			-\pi n < n(x_n - \pi n) < -\frac{\pi n}{2}
		\end{gather*}
		$\lim_{n \to \infty} -\pi n = -\infty,\ \lim_{n \to \infty} -\frac{\pi n}{2} = -\infty \quad \Rightarrow$ по лемме о 2 милиционерах: $\lim n(x_n - \pi n) = -\infty$
		\\ \\